% !TeX program = xelatex
\documentclass[12pt]{article}
\usepackage{standalone}

\usepackage[dvipsnames,svgnames,x11names]{xcolor}
\usepackage[a4paper,margin=1in]{geometry}
\usepackage{microtype}
\usepackage{amsmath}
\usepackage{amsthm}
\usepackage{mathtools}
\usepackage{mathrsfs}
\usepackage{stmaryrd}
\usepackage{extarrows}
\usepackage{enumerate}
\usepackage{tensor}
\usepackage{physics2}
  \usephysicsmodule{ab,xmat}
\usepackage{fixdif}
  \newcommand{\dd}{\d}
\usepackage{derivative}
  \newcommand{\dv}{\odv}
  \newcommand{\pd}[1]{\pdv{}{#1}}
  \newcommand{\eval}[1]{#1\big|}
\usepackage{graphicx}
\usepackage{subcaption}
\usepackage{tikz}
\usepackage{tikz-3dplot}
% \usepackage{tikz-cd}
% \usepackage{quiver}
% ========== Block quiver.sty ========== %
\usepackage{tikz-cd}
% \usepackage{amssymb}
\usetikzlibrary{calc}
\usetikzlibrary{decorations.pathmorphing}
\tikzset{curve/.style={settings={#1},to path={(\tikztostart)
    .. controls ($(\tikztostart)!\pv{pos}!(\tikztotarget)!\pv{height}!270:(\tikztotarget)$)
    and ($(\tikztostart)!1-\pv{pos}!(\tikztotarget)!\pv{height}!270:(\tikztotarget)$)
    .. (\tikztotarget)\tikztonodes}},
    settings/.code={\tikzset{quiver/.cd,#1}
        \def\pv##1{\pgfkeysvalueof{/tikz/quiver/##1}}},
    quiver/.cd,pos/.initial=0.35,height/.initial=0}
\tikzset{tail reversed/.code={\pgfsetarrowsstart{tikzcd to}}}
\tikzset{2tail/.code={\pgfsetarrowsstart{Implies[reversed]}}}
\tikzset{2tail reversed/.code={\pgfsetarrowsstart{Implies}}}
\tikzset{no body/.style={/tikz/dash pattern=on 0 off 1mm}}
% =========== End block ========== %
  \tikzset{every picture/.style={line width=0.75pt}}
\usepackage{pgfplots}
  \pgfplotsset{compat=newest}
\usepackage{tcolorbox}
  \tcbuselibrary{most}
\usepackage[colorlinks=true,linkcolor=blue]{hyperref}
\usepackage{cleveref}
% \usepackage[hyperref=true,backend=biber,style=alphabetic,backref=true,url=false]{biblatex}
\usepackage[warnings-off={mathtools-colon,mathtools-overbracket}]{unicode-math}
\usepackage[default,amsbb]{fontsetup}
  \setmathfont[StylisticSet=1,range=\mathscr]{NewCMMath-Book.otf}
\usepackage{fancyhdr}
\usepackage{import}

\newcommand{\Id}{\mathbb{1}}
\newcommand{\lap}{\increment}

\DeclareMathOperator{\sign}{sign}
\DeclareMathOperator{\dom}{dom}
\DeclareMathOperator{\ran}{ran}
\DeclareMathOperator{\ord}{ord}
\DeclareMathOperator{\Span}{span}
\DeclareMathOperator{\img}{Im}
\DeclareMathOperator{\Ric}{Ric}
\newcommand{\card}{\texttt{\#}}
\newcommand{\ie}{\emph{i.e.}}
\newcommand{\st}{\emph{s.t.}}
\newcommand{\eps}{\varepsilon}
\newcommand{\vphi}{\varphi}
\newcommand{\vthe}{\vartheta}
\newcommand{\II}{I\!I}
\renewcommand{\emptyset}{⌀}
\newcommand{\acts}{\curvearrowright}
\newcommand{\xrr}{\xlongrightarrow}
\newcommand{\lrr}{\longrightarrow}
\newcommand{\lmt}{\longmapsto}
\newcommand{\into}{\hookrightarrow}
\newcommand{\op}{\operatorname}

\let\originalleft\left
\let\originalright\right
\renewcommand{\left}{\mathopen{}\mathclose\bgroup\originalleft}
\renewcommand{\right}{\aftergroup\egroup\originalright}

\theoremstyle{plain}\newtheorem{theorem}{Theorem}
\theoremstyle{definition}\newtheorem{definition}[theorem]{Definition}
\theoremstyle{definition}\newtheorem{example}[theorem]{Example}
\theoremstyle{definition}\newtheorem{problem}[theorem]{Problem}
\theoremstyle{plain}\newtheorem{axiom}[theorem]{Axiom}
\theoremstyle{plain}\newtheorem{corollary}[theorem]{Corollary}
\theoremstyle{plain}\newtheorem{lemma}[theorem]{Lemma}
\theoremstyle{plain}\newtheorem{proposition}[theorem]{Proposition}
\theoremstyle{plain}\newtheorem{prop}[theorem]{Proposition}
\theoremstyle{plain}\newtheorem{conjecture}[theorem]{Conjecture}
\theoremstyle{plain}\newtheorem{conj}[theorem]{Conjecture}
\theoremstyle{remark}\newtheorem{notation}[theorem]{Notation}
\theoremstyle{definition}\newtheorem*{question}{Question}
\theoremstyle{definition}\newtheorem*{answer}{Answer}
\theoremstyle{definition}\newtheorem*{goal}{Goal}
\theoremstyle{definition}\newtheorem*{application}{Application}
\theoremstyle{plain}\newtheorem*{exercise}{Exercise}
\theoremstyle{remark}\newtheorem*{remark}{Remark}
\theoremstyle{remark}\newtheorem*{note}{\small{Note}}
\numberwithin{equation}{section}
\numberwithin{theorem}{section}
\numberwithin{figure}{section}

\usepackage{xeCJK}
\setCJKmainfont{FZShuSong-Z01}[BoldFont=FZXiaoBiaoSong-B05,ItalicFont=FZKai-Z03]
\setCJKsansfont{FZXiHeiI-Z08}[BoldFont=FZHei-B01]
\setCJKmonofont{FZFangSong-Z02}
\setCJKfamilyfont{zhsong}{FZShuSong-Z01}[BoldFont=FZXiaoBiaoSong-B05]
\setCJKfamilyfont{zhhei}{FZHei-B01}
\setCJKfamilyfont{zhkai}{FZKai-Z03}
\setCJKfamilyfont{zhfs}{FZFangSong-Z02}
\setCJKfamilyfont{zhli}{FZLiShu-S01}
\setCJKfamilyfont{zhyou}{FZXiYuan-M01}[BoldFont=FZZhunYuan-M02]

\allowdisplaybreaks{}

\newcommand{\isFullBook}[2]{
  \ifnum\pdfstrcmp{\FullBook}{True}=0
    \ifnum\pdfstrcmp{}{#1}=0\unskip\else#1\fi
  \else
    \ifnum\pdfstrcmp{}{#2}=0\unskip\else#2\fi
  \fi\ignorespaces{}
}

\counterwithout{theorem}{section}
\counterwithout{equation}{section}

\begin{document}
Now we go back to the Chern-Weil theory.

\(\det(I+\frac{\sqrt{-1}}{2\pi}R)\) is a differential form consisting different
degrees. We consider it as a section of \(\op{End}(\det E)\otimes\wedge^*M\) by
\begin{align*}
  \det(I+\frac{\sqrt{-1}}{2\pi}R)s_1\wedge\cdots\wedge s_r&:=
  (I+\frac{\sqrt{-1}}{2\pi}R)\wedge\cdots\wedge(I+\frac{\sqrt{-1}}{2\pi}R)
  s_1\wedge\cdots\wedge s_r \\ &:=((I+\frac{\sqrt{-1}}{2\pi}R)s_1)\wedge\cdots
  \wedge((I+\frac{\sqrt{-1}}{2\pi}R)s_r)
\end{align*} 
for any frame \(s_1,\ldots,s_r\).

In general, for a line bundle \(L\), any section of \(\op{End}(L)=L\otimes L^*\)
is expressed as \(ae\otimes e^*\) with respect to a frame \(e\) and its dual
frame. But this \(a\) is independent of the frame \(e\). Thus \[
  C^\infty(\op{End}(L))=C^\infty(M),\quad ae \otimes e^*\longleftrightarrow a
.\] Given a connection \(\nabla\), one can check that \[
  \dd^\nabla a=\dd{a}
.\] In fact if \(\omega\) is the connection form w.r.t. \(e\), \ie\ \(\nabla e
=e\omega\), then \(\nabla e^*=-\omega e^*\), and \[
  \nabla (ae\otimes e^*)=(\dd{a}+\omega-\omega)\otimes e\otimes e^*
  =\dd{a}\otimes e\otimes e^*
.\] Thus the induced connection on \(\op{End}(\det E)\) from \(\nabla\) is
just the exterior derivative \(\dd{}\).

Consider for \(t\in \mathbb{R}\), \[
  \det(I+\frac{\sqrt{-1}t}{2\pi}R)\in C^\infty(M,\op{End}(\det E)\otimes 
  \wedge^*M)=C^\infty(M,\wedge^*M)
\] which is a differential form on \(M\). So \[
  \dd^\nabla\det(I+\frac{\sqrt{-1}t}{2\pi}R)
  =\dd{\det(I+\frac{\sqrt{-1}t}{2\pi}R)}
.\] On the other hand for \(s_1,\ldots,s_r\in C^\infty(M,E)\), \[
  \det(I+\frac{\sqrt{-1}t}{2\pi}R)(s_1\wedge\cdots\wedge s_r)
  =(I+\frac{\sqrt{-1}t}{2\pi}R)s_1\wedge\cdots\wedge
  (I+\frac{\sqrt{-1}t}{2\pi}R)s_r
.\] By Bianchi identity,
\begin{align*}
  &(\dd\big(\det(I+\frac{\sqrt{-1}t}{2\pi}R)))(s_1\wedge\cdots\wedge s_r) \\
  =&(\dd^\nabla(\det(I+\frac{\sqrt{-1}t}{2\pi}R)))(s_1\wedge\cdots\wedge s_r) \\
  =&\dd^\nabla (I+\frac{\sqrt{-1}t}{2\pi}R)
  -\sum_{i=1}^r\det(I+\frac{\sqrt{-1}t}{2\pi}R)s_1\wedge\cdots \wedge\dd^\nabla 
  s_i\wedge\cdots \wedge s_r \\
  =&\sum_{i=1}^{r}(I+\frac{\sqrt{-1}t}{2\pi}R)s_1\wedge\dd^\nabla ((I+\frac{
  \sqrt{-1}t}{2\pi}R)s_i)\wedge\cdots \wedge(I+\frac{\sqrt{-1}t}{2\pi}R)s_r \\
  &-\sum_{i=1}^{r}(I+\frac{\sqrt{-1}t}{2\pi}R)s_1\wedge(I+\frac{\sqrt{-1}t}{2\pi}
  R)(\dd^\nabla s_i)\wedge\cdots \wedge(I+\frac{\sqrt{-1}t}{2\pi}R)s_r \\
  =&\sum_{i=1}^{r}(I+\frac{\sqrt{-1}t}{2\pi}R)s_1\wedge(\dd^\nabla (I+\frac{
  \sqrt{-1}t}{2\pi}R))s_i\wedge\cdots \wedge(I+\frac{\sqrt{-1}t}{2\pi}R)s_r \\
  =&0
.\end{align*}
Hence, putting \[
  \det(I+\frac{\sqrt{-1}t}{2\pi}R)=1+t c_1(\nabla)+\cdots +t^r c_r(\nabla)
.\] \(c_i(\nabla)\) are closed forms for \(i=1,\ldots,r\).

Next we will see if we choose another \(\nabla'\) then \(c_i(\nabla)\) and
\(c_i(\nabla')\) are cohomologous.

Let \(\alpha\) be the \(\op{End}(E)\) valued 1-form such that \[
  \nabla'-\nabla=\alpha
.\] Note that \[
  (\nabla'-\nabla)fs=(\dd{f}\otimes s+f\nabla's)-(\dd{f}\otimes s+f\nabla s)
  =f(\nabla'-\nabla)s
.\] We set
\begin{align*}
  \nabla_t&=\nabla +t\alpha \\
  R_t&=\dd^{\nabla_t}\circ \dd^{\nabla_t}=(\dd^\nabla+t\alpha)\circ
  (\dd^\nabla+t\alpha)
.\end{align*}
Then \[
  \dv{R_t}{t}=\alpha\circ\dd^{\nabla_t}+\dd^{\nabla_t}\circ\alpha
  =\dd^{\nabla_t}\alpha
.\] Put \[
  \vphi=\frac{\sqrt{-1}r}{2\pi}\int_{0}^{1}\alpha\wedge (I+\frac{\sqrt{-1}}{2\pi}
  R_t)\wedge\cdots\wedge (I+\frac{\sqrt{-1}}{2\pi}R_t)\dd{t}
.\] Here, the integrand is defined using a frame by
\begin{multline*}
  \alpha\wedge (I+\frac{\sqrt{-1}}{2\pi}R_t)\wedge\cdots\wedge
  (I+\frac{\sqrt{-1}}{2\pi}R_t)(s_1\wedge\cdots\wedge s_r)= \\
  \quad\frac{1}{r}\biggl(\alpha(s_1)\wedge(I+\frac{\sqrt{-1}}{2\pi}R_t)s_2\wedge
  \cdots\wedge(I+\frac{\sqrt{-1}}{2\pi}R_t)s_r)+\cdots \\
  \quad+(I+\frac{\sqrt{-1}}{2\pi}R_t)s_1
  \wedge\cdots\wedge(I+\frac{\sqrt{-1}}{2\pi}R_t)s_{r-1}\wedge\alpha(s_r)\biggr)
.\end{multline*} 
One can check this is independent of the choice of \(s_1,\ldots,s_r\).
Using Bianchi identity again we get 
\begin{align*}
  \dd{\vphi}&=\frac{\sqrt{-1}r}{2\pi}\int_{0}^{1}\dd\bigl(\alpha\wedge(I+\frac{
  \sqrt{-1}}{2\pi}R_t)\wedge\cdots\wedge(I+\frac{\sqrt{-1}}{2\pi}R_t)\bigr)
  \dd{t} \\
  &=\frac{\sqrt{-1}r}{2\pi}\int_{0}^{1}\dd^{\nabla_t}\bigl(\alpha\wedge(I+\frac{
  \sqrt{-1}}{2\pi}R_t)\wedge\cdots\wedge(I+\frac{\sqrt{-1}}{2\pi}R_t)\bigr)
  \dd{t} \\
  &=r\int_{0}^{1}(\dd^{\nabla_t}(\frac{\sqrt{-1}}{2\pi}\alpha))\wedge(I+\frac{
  \sqrt{-1}}{2\pi}R_t)\wedge\cdots\wedge(I+\frac{\sqrt{-1}}{2\pi}R_t)\dd{t} \\
  &=r\int_{0}^{1}\bigl(\dv{}{t}(I+\frac{\sqrt{-1}}{2\pi}R_t)\bigr)\wedge(I+\frac{
  \sqrt{-1}}{2\pi}R_t)\wedge\cdots\wedge(I+\frac{\sqrt{-1}}{2\pi}R_t)\dd{t} \\
  &=\int_{0}^{1}\dv{}{t}\biggl((I+\frac{\sqrt{-1}}{2\pi}R_t)\wedge(I+\frac{\sqrt{
  -1}}{2\pi}R_t)\wedge\cdots\wedge(I+\frac{\sqrt{-1}}{2\pi}R_t)\biggr)\dd{t} \\
  &=\det(I+\frac{\sqrt{-1}}{2\pi}R_1)-\det(I+\frac{\sqrt{-1}}{2\pi}R_0) \\
  &=\sum_{i=1}^{r}c_i(\nabla)-c_i(\nabla')
.\end{align*}
But \[
  \op{deg}(c_i(\nabla)-c_i(\nabla'))=2i
.\] Thus \[
  c_i(\nabla)-c_i(\nabla')=\dd{(\text{degree }(2i-1)\text{ term of }\vphi)}
.\] We have proved
\begin{theorem}[\CJKfamily{zhkai}陳省身]
  \([c_i(\nabla)]\in H^{2i}(M,\mathbb{C})\) is independent of the choice of
  connection \(\nabla\).
\end{theorem}

\begin{definition}
  \(c_i(E)=[c_i(\nabla)]\in H^{2i}(M,\mathbb{C})\) is called the \(i\)-th
  \textbf{Chern class}. (In fact it always belongs to \(H^{2i}(M,\mathbb{Z})\)
  as will be explained below). \[
    c(E)=1+c_1(E)+\cdots +c_r(E)\in H^*(M,\mathbb{C})
    \quad(=[\det(I+\frac{\sqrt{-1}}{2\pi}R)])
  \] is called the \textbf{total Chern class}.
\end{definition}

\begin{lemma}
  In fact \(c_i(E)\in H^{2i}(M,\mathbb{R})\).
\end{lemma}
\begin{proof}
  Choose a Hermitian metric \(h\) of \(E\) and a connection such that
  \(\nabla h=0\). (The existence of such a connection is left as exercise).
  Let \(e_1,\ldots,e_r\) be an orthonormal frame of \((E,h)\). Then using
  \(h_{\bar{i}j}=h(\bar{e}_i,e_j)=\delta_{ij}\), \[
    0=\dd{h_{\bar{i}j}}=h(\overline{\omega\indices{^k_i}e_k},e_j)
    +h(\bar{e}_i,\omega\indices{^k_j}e_k)=\overline{\omega_{\bar{j}i}}
    +\omega_{\bar{i}j}
  .\] And \[
    \overline{R_{\bar{j}i}}=\overline{\dd{\omega_{\bar{j}i}}+\omega_{\bar{j}k}
    \wedge\omega_{\bar{k}i}}=\dd{\omega_{\bar{i}j}}-\omega_{\bar{i}k}\wedge 
    \omega_{\bar{k}j}=-R_{\bar{i}j}
  .\] Where we raise or lower the indices using \(h\). Since \(R\) is
  skew-Hermitian, \(\frac{\sqrt{-1}}{2\pi}R\) is Hermitian and \[
    \overline{\det(I+\frac{\sqrt{-1}}{2\pi}R)}=\det(I+\frac{\sqrt{-1}}{2\pi}R)
  .\] 
\end{proof}

Another standard definition of Chern classes is given in the book
\emph{Milnor-Stasheff: Characteristic classes}. In their book it is shown that
given smooth complex vector bundle \(E\to M\) of rank \(r\), there is a 
cohomology class \(c(E)\in H^*(M,\mathbb{Z})\) with \[
  c(E)=\sum_{i=0}^{r},\quad c_i(E),c_i(E)\in H^{2i}(M,\mathbb{Z}),\quad c_0(E)=1
.\] And they are uniquely determined by the following three axioms:

\begin{axiom}[Naturality]
  For any manifold \(N\) and a map \(f\colon N\to M\), \[
    c(f^*E)=f^*c(E)
  .\] 
\end{axiom}
\begin{axiom}[Whitney sum formula]
  For two vector bundles \(E\to M\) and \(F\to M\) over \(M\), \[
    c(E\oplus F)=c(E)\cup c(F)
  .\] 
\end{axiom}
\begin{axiom}[Normalization]
   \[
     \left<c_1(\mathcal{O}_{\mathbb{P}^1}(-1)),[\mathbb{P}^1]\right> =-1
   .\] Here \(\mathbb{P}^1=\mathbb{P}^1(\mathbb{C})\).
\end{axiom}
It is easy to see the Chern classes defined using connection and curvature 
satisfy the three Axioms. Since the uniqueness holds as
\(\mathbb{R}\)-coefficient cohomology, they must coincide. Thus the Chern
classes defined using curvature are actually integral classes.

\end{document}
