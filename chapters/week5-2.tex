% !TeX program = xelatex
\documentclass[12pt]{article}
\usepackage{standalone}

\usepackage[dvipsnames,svgnames,x11names]{xcolor}
\usepackage[a4paper,margin=1in]{geometry}
\usepackage{microtype}
\usepackage{amsmath}
\usepackage{amsthm}
\usepackage{mathtools}
\usepackage{mathrsfs}
\usepackage{stmaryrd}
\usepackage{extarrows}
\usepackage{enumerate}
\usepackage{tensor}
\usepackage{physics2}
  \usephysicsmodule{ab,xmat}
\usepackage{fixdif}
  \newcommand{\dd}{\d}
\usepackage{derivative}
  \newcommand{\dv}{\odv}
  \newcommand{\pd}[1]{\pdv{}{#1}}
  \newcommand{\eval}[1]{#1\big|}
\usepackage{graphicx}
\usepackage{subcaption}
\usepackage{tikz}
\usepackage{tikz-3dplot}
% \usepackage{tikz-cd}
% \usepackage{quiver}
% ========== Block quiver.sty ========== %
\usepackage{tikz-cd}
% \usepackage{amssymb}
\usetikzlibrary{calc}
\usetikzlibrary{decorations.pathmorphing}
\tikzset{curve/.style={settings={#1},to path={(\tikztostart)
    .. controls ($(\tikztostart)!\pv{pos}!(\tikztotarget)!\pv{height}!270:(\tikztotarget)$)
    and ($(\tikztostart)!1-\pv{pos}!(\tikztotarget)!\pv{height}!270:(\tikztotarget)$)
    .. (\tikztotarget)\tikztonodes}},
    settings/.code={\tikzset{quiver/.cd,#1}
        \def\pv##1{\pgfkeysvalueof{/tikz/quiver/##1}}},
    quiver/.cd,pos/.initial=0.35,height/.initial=0}
\tikzset{tail reversed/.code={\pgfsetarrowsstart{tikzcd to}}}
\tikzset{2tail/.code={\pgfsetarrowsstart{Implies[reversed]}}}
\tikzset{2tail reversed/.code={\pgfsetarrowsstart{Implies}}}
\tikzset{no body/.style={/tikz/dash pattern=on 0 off 1mm}}
% =========== End block ========== %
  \tikzset{every picture/.style={line width=0.75pt}}
\usepackage{pgfplots}
  \pgfplotsset{compat=newest}
\usepackage{tcolorbox}
  \tcbuselibrary{most}
\usepackage[colorlinks=true,linkcolor=blue]{hyperref}
\usepackage{cleveref}
% \usepackage[hyperref=true,backend=biber,style=alphabetic,backref=true,url=false]{biblatex}
\usepackage[warnings-off={mathtools-colon,mathtools-overbracket}]{unicode-math}
\usepackage[default,amsbb]{fontsetup}
  \setmathfont[StylisticSet=1,range=\mathscr]{NewCMMath-Book.otf}
\usepackage{fancyhdr}
\usepackage{import}

\newcommand{\Id}{\mathbb{1}}
\newcommand{\lap}{\increment}

\DeclareMathOperator{\sign}{sign}
\DeclareMathOperator{\dom}{dom}
\DeclareMathOperator{\ran}{ran}
\DeclareMathOperator{\ord}{ord}
\DeclareMathOperator{\Span}{span}
\DeclareMathOperator{\img}{Im}
\DeclareMathOperator{\Ric}{Ric}
\newcommand{\card}{\texttt{\#}}
\newcommand{\ie}{\emph{i.e.}}
\newcommand{\st}{\emph{s.t.}}
\newcommand{\eps}{\varepsilon}
\newcommand{\vphi}{\varphi}
\newcommand{\vthe}{\vartheta}
\newcommand{\II}{I\!I}
\renewcommand{\emptyset}{⌀}
\newcommand{\acts}{\curvearrowright}
\newcommand{\xrr}{\xlongrightarrow}
\newcommand{\lrr}{\longrightarrow}
\newcommand{\lmt}{\longmapsto}
\newcommand{\into}{\hookrightarrow}
\newcommand{\op}{\operatorname}

\let\originalleft\left
\let\originalright\right
\renewcommand{\left}{\mathopen{}\mathclose\bgroup\originalleft}
\renewcommand{\right}{\aftergroup\egroup\originalright}

\theoremstyle{plain}\newtheorem{theorem}{Theorem}
\theoremstyle{definition}\newtheorem{definition}[theorem]{Definition}
\theoremstyle{definition}\newtheorem{example}[theorem]{Example}
\theoremstyle{definition}\newtheorem{problem}[theorem]{Problem}
\theoremstyle{plain}\newtheorem{axiom}[theorem]{Axiom}
\theoremstyle{plain}\newtheorem{corollary}[theorem]{Corollary}
\theoremstyle{plain}\newtheorem{lemma}[theorem]{Lemma}
\theoremstyle{plain}\newtheorem{proposition}[theorem]{Proposition}
\theoremstyle{plain}\newtheorem{prop}[theorem]{Proposition}
\theoremstyle{plain}\newtheorem{conjecture}[theorem]{Conjecture}
\theoremstyle{plain}\newtheorem{conj}[theorem]{Conjecture}
\theoremstyle{remark}\newtheorem{notation}[theorem]{Notation}
\theoremstyle{definition}\newtheorem*{question}{Question}
\theoremstyle{definition}\newtheorem*{answer}{Answer}
\theoremstyle{definition}\newtheorem*{goal}{Goal}
\theoremstyle{definition}\newtheorem*{application}{Application}
\theoremstyle{plain}\newtheorem*{exercise}{Exercise}
\theoremstyle{remark}\newtheorem*{remark}{Remark}
\theoremstyle{remark}\newtheorem*{note}{\small{Note}}
\numberwithin{equation}{section}
\numberwithin{theorem}{section}
\numberwithin{figure}{section}

\usepackage{xeCJK}
\setCJKmainfont{FZShuSong-Z01}[BoldFont=FZXiaoBiaoSong-B05,ItalicFont=FZKai-Z03]
\setCJKsansfont{FZXiHeiI-Z08}[BoldFont=FZHei-B01]
\setCJKmonofont{FZFangSong-Z02}
\setCJKfamilyfont{zhsong}{FZShuSong-Z01}[BoldFont=FZXiaoBiaoSong-B05]
\setCJKfamilyfont{zhhei}{FZHei-B01}
\setCJKfamilyfont{zhkai}{FZKai-Z03}
\setCJKfamilyfont{zhfs}{FZFangSong-Z02}
\setCJKfamilyfont{zhli}{FZLiShu-S01}
\setCJKfamilyfont{zhyou}{FZXiYuan-M01}[BoldFont=FZZhunYuan-M02]

\allowdisplaybreaks{}

\newcommand{\isFullBook}[2]{
  \ifnum\pdfstrcmp{\FullBook}{True}=0
    \ifnum\pdfstrcmp{}{#1}=0\unskip\else#1\fi
  \else
    \ifnum\pdfstrcmp{}{#2}=0\unskip\else#2\fi
  \fi\ignorespaces{}
}

\counterwithout{theorem}{section}
\counterwithout{equation}{section}

\begin{document}
\begin{application}\hfill
\begin{enumerate}[(1)]
\item \begin{align*}
    &\int_{M}T^i \nabla_i f\dd{V_g} \\
    =&\int_{M}\nabla_i (T^i f)\dd{V_g}-\int_{M}(\nabla_i T^i)f\dd{V_g} \\
    =&-\int_{M}(\nabla_i T^i)f \dd{V_g}
  .\end{align*}
  This is ``integration by parts''.
\item The Laplacian \(\lap f\) of \(f\in C^\infty(M)\) is defined by \[
    \lap  f=-\nabla^i \nabla_i f=-\nabla_i \nabla^i f
  .\] Here minus sign is used to adapt with Hodge theory. Then \[
    \int_{M}\lap f \dd{V_g}=0
  .\] 
\item If \(\lap f=\lambda f\) for \(f\in C^\infty(M)\), then \(\lambda\ge 0\). And
  \(\lambda=0\) implies \(f=\text{constant}\).
\end{enumerate}
\end{application}
\begin{proof}[Proof of (3)]
  \begin{align*}
    \lambda\int_{M}f^2\dd{V_g}&=\int_{M}\lap f\cdot f\dd{V_g}
    =-\int_{M}\nabla^i \nabla_i f\cdot f \\
    &=\int_{M}\nabla_i f \nabla^i f\dd{V_g}=\|\dd{f}\|_{L^2}^2\ge 0
  .\end{align*}
  Thus \(\lambda\ge 0\), and \(\lambda=0\) implies \(\dd{f}\equiv 0\).
\end{proof}

\section{Hodge theory}
We use the notation \[
  \Omega^p=C^\infty(M,\bigwedge^p T^*M)
\] to denote the space of all smooth \(p\)-forms on \(M\).

We have the exterior derivative \[
  \dd\colon \Omega^p\longrightarrow \Omega^{p+1}
.\] Since \(\dd^2=0\), we have the de Rham cohomology \[
  H_{\mathrm{dR}}^p(M)=\ker(\dd\colon\Omega^p\to \Omega^{p+1})/\img(\dd\colon
  \Omega^{p-1}\to \Omega^p)
.\] Let \(\dd^*\colon \Omega^p\to \Omega^{p-1}\) be the formal adjoint of
\(\dd\colon \Omega^p\to\Omega^{p+1}\) by \[
  (\dd{\alpha},\beta)_{L^2}=(\alpha,\dd^*\beta)_{L^2},\quad
  \alpha\in \Omega^{p-1},\beta\in \Omega^p
.\] The Hodge Laplacian \(\lap_{\mathrm{d}}\) is defined by \[
  \lap_{\mathrm{d}}=\dd^*\dd+\dd \dd^*\colon \Omega^p\longrightarrow\Omega^p
\] for each \(p=0,\ldots,n\). (Note \(\d^*=0\) on \(\Omega^0\)).

\begin{definition}
  \(\alpha\in \Omega^p\) is called a harmonic \(p\)-form if \[
    \lap_{\mathrm{d}} \alpha=0
  .\] 
\end{definition}
\begin{lemma}
  \(\alpha\) is harmonic \(\iff \dd{\alpha}=0\) and \(\dd^*\alpha=0\).
\end{lemma}
\begin{proof}
  \[
    (\lap \alpha,\alpha)_{L^2}=(\dd^*\dd{\alpha}+\dd \dd^*\alpha,\alpha)_{L^2}
    =(\dd{\alpha},\dd{\alpha})_{L^2}+(\dd^*\alpha,\dd^*\alpha)_{L^2}
  .\]
\end{proof}

\begin{theorem}[Hodge]
  Let \((M,g)\) be a compact oriented Riemannian manifold.
  \begin{enumerate}[(1)]
  \item The vector space \(\mathcal{H}_{\dd}^p(M)\) of all harmonic \(p\)-form is
    finite dimensional.
  \item Orthogonal projection \(H_{\dd }\colon \Omega^p\to \mathcal{H}_{\dd{}}^p(M)\)
    is well-defined.
  \item There is unique operator \(G_{\dd{}}\colon \Omega^p\to \Omega^p\) called the
    Green operator, such that
    \begin{gather*}
      G_{\dd} H_{\dd} =H_{\dd} G_{\dd} =0, \\
      \dd{G_{\dd{}}}=G_{\dd{}}\dd{},\quad \dd^*G_{\dd{}}=G_{\dd{}}\dd^*, \\
      \op{Id}_{\Omega^p}=H_{\dd{}}+\lap_{\dd{}}G_{\dd{}}
    .\end{gather*} 
  \end{enumerate}
\end{theorem}
We postpone the proof of this theorem to a later occasion, but I recommend to read
Frank Warner's book \emph{Foundations of Differentiable Manifolds and Lie Groups},
pages 220-250. Here you can study the standard theory of linear elliptic operators.

One consequence of Hodge theory is the following isomorphism:

\begin{theorem}
  \(\mathcal{H}_{\dd{}}^p(M)\cong H_{\mathrm{dR}}^p(M)\).
\end{theorem}
\begin{proof}
  Since \(\lap\alpha=0\) implies \(\dd{\alpha}=0\), we have a natural map \[
    \mathcal{H}_{\dd{}}^p(M)\longrightarrow H_{\mathrm{dR}}^p(M),\quad
    \alpha\longmapsto [\alpha]
  .\] To show this is injective, suppose \(\alpha\) is harmonic and \([\alpha]=0\).
  Then \(\alpha=\dd{\beta}\) for some \(\beta\in \Omega^{p-1}\). Since \(\dd^*\alpha=0
  \), we have \[
    0=(\dd^*\alpha,\beta)_{L^2}=(\alpha,\dd{\beta})_{L^2}=(\alpha,\alpha)_{L^2}
  .\] Hence \(\alpha=0\).

  To show subjectivity, suppose \(\dd{\alpha}=0\). Then using
  \(\op{Id}=H_{\dd{}}+\lap_{\dd{}}G_{\dd{}}\), we have \[
    \alpha=H_{\dd{}}\alpha+(\dd \dd^*+\dd^*\dd)G_{\dd{}}\alpha
  .\] Using the commutativity of \(G_{\dd{}}\) and \(\dd,\dd^*\) we have \[
    \alpha=H_{\dd{}}\alpha+\dd{(\dd^*G_{\dd{}}\alpha)}
  .\] So \([\alpha]=[H_{\dd{}}\alpha]\), \ie\ the harmonic form \(H_{\dd{}}\alpha\)
  represents \([\alpha]\).
\end{proof}
Note that in general, \[
  \alpha=H_{\dd{}}\alpha+\dd \dd^*G_{\dd{}}\alpha+\dd^*\dd G_{\dd{}}\alpha
\] gives a decomposition into \[
  \text{harmonic part}+\dd\text{-image part}+\dd^*\text{-image part}
.\] 

Next we consider a holomorphic vector bundle \(\pi\colon E\to M\) over a compact
complex manifold with a Hermitian (maybe not K\"ahler) metric \(h_{M}\) on \(M\).
The fundamental 2-form \(\gamma=\sqrt{-1}h^M_{i\bar{j}}\dd{z^i}\dd{\bar{z}^j}\) of
\(h_M\) gives a measure \[
  \gamma^m=\overbrace{\gamma\wedge \cdots \wedge \gamma}^m,\quad m=\dim_{\mathbb{C}}M
.\] We also give a Hermitian metric \(h_E\) on \(E\). Recall we have a
\(\bar{\partial}\) operator \[
  \bar{\partial}\colon \Omega^{p,q}(E)\longrightarrow\Omega^{p,q+1}(E)
\] where \(\Omega^{p,q}(E)=C^\infty(M,E\otimes\bigwedge^p T'M\wedge\bigwedge^q T''M)\)
is the space of all \(E\)-valued \((p,q)\)-forms on \(M\). Since
\(\bar{\partial}\circ \bar{\partial}=0\), we can consider \[
  H_{\bar{\partial}}^{p,q}(M,E)
  =\frac{\ker(\bar{\partial}\colon \Omega^{p,q}(E)\to \Omega^{p,q+1}(E))}
  {\img(\bar{\partial}\colon \Omega^{p,q-1}(E)\to \Omega^{p,q}(E))},
\] called the \textbf{Dolbeault cohomology}.

Using \(h_M\) and \(h_E\) we can consider \(L^2\)-inner product on \(\Omega^{p,q}(E)\).

\begin{definition}
  \[
    \lap_{\bar{\partial}}\bar{\partial}^*\bar{\partial}+\bar{\partial}\bar{\partial}^*
    \colon \Omega^{p,q}(E)\longrightarrow\Omega^{p,q}(E)
  \]  is called the \(\bar{\partial}\)-Laplacian.
\end{definition}
\begin{definition}
  \(\alpha\in \Omega^{p,q}(E)\) is called \(\bar{\partial}\)-harmonic if 
  \(\lap_{\bar{\partial}}\alpha=0\). This is equivalent to \(\bar{\partial}\alpha=0\)
  and \(\bar{\partial}^*\alpha=0\).
\end{definition}

\begin{theorem}[Hodge-Kodaira]
  Let \(E\to M\) be a holomorphic vector bundle with Hermitian metric \(h_E\) on \(E\)
  and \(h_M\) on \(M\).
  \begin{enumerate}[(1)]
  \item The space \(\mathcal{H}_{\bar{\partial}}^{p,q}(E)\) of all \(\bar{\partial}\)
    harmonic \(E\)-valued \((p,q)\)-forms is finite dimensional.
  \item There is a well-defined orthogonal projection \[
      H_{\bar{\partial}}\colon \Omega^{p,q}(E)\longrightarrow
      \mathcal{H}_{\bar{\partial}}^{p,q}(E)
    .\]
  \item There exists a unique operator \(G_{\bar{\partial}}\colon
    \Omega^{p,q}\to \Omega^{p,q}(E)\), called the Green operator, such that
    \begin{gather*}
      G_{\bar{\partial}}H_{\bar{\partial}}=0,\quad
      H_{\bar{\partial}}G_{\bar{\partial}}=0, \\ 
      \bar{\partial}G_{\bar{\partial}}=G_{\bar{\partial}}\bar{\partial},\quad
      \bar{\partial}^*G_{\bar{\partial}}=G_{\bar{\partial}}\bar{\partial}^*, \\
      \op{Id}_{\Omega^{p,q}(E)}=H_{\bar{\partial}}
      +\lap_{\bar{\partial}}G_{\bar{\partial}}
    \end{gather*}
  \end{enumerate}
\end{theorem}
Again, we postpone its proof in later occasion. See \emph{Griffiths-Harris}, pages
80-100.

\begin{corollary}
  \(\mathcal{H}_{\bar{\partial}}^{p,q}(E)\cong H_{\bar{\partial}}^{p,q}(E)\), \ie\ \[
    E\text{-valued harmonic forms}\cong \text{Dolbeault cohomology}
  .\] 
\end{corollary}

\end{document}
