% !TeX program = xelatex
\documentclass[12pt]{article}
\usepackage{standalone}

\usepackage[dvipsnames,svgnames,x11names]{xcolor}
\usepackage[a4paper,margin=1in]{geometry}
\usepackage{microtype}
\usepackage{amsmath}
\usepackage{amsthm}
\usepackage{mathtools}
\usepackage{mathrsfs}
\usepackage{stmaryrd}
\usepackage{extarrows}
\usepackage{enumerate}
\usepackage{tensor}
\usepackage{physics2}
  \usephysicsmodule{ab,xmat}
\usepackage{fixdif}
  \newcommand{\dd}{\d}
\usepackage{derivative}
  \newcommand{\dv}{\odv}
  \newcommand{\pd}[1]{\pdv{}{#1}}
  \newcommand{\eval}[1]{#1\big|}
\usepackage{graphicx}
\usepackage{subcaption}
\usepackage{tikz}
\usepackage{tikz-3dplot}
% \usepackage{tikz-cd}
% \usepackage{quiver}
% ========== Block quiver.sty ========== %
\usepackage{tikz-cd}
% \usepackage{amssymb}
\usetikzlibrary{calc}
\usetikzlibrary{decorations.pathmorphing}
\tikzset{curve/.style={settings={#1},to path={(\tikztostart)
    .. controls ($(\tikztostart)!\pv{pos}!(\tikztotarget)!\pv{height}!270:(\tikztotarget)$)
    and ($(\tikztostart)!1-\pv{pos}!(\tikztotarget)!\pv{height}!270:(\tikztotarget)$)
    .. (\tikztotarget)\tikztonodes}},
    settings/.code={\tikzset{quiver/.cd,#1}
        \def\pv##1{\pgfkeysvalueof{/tikz/quiver/##1}}},
    quiver/.cd,pos/.initial=0.35,height/.initial=0}
\tikzset{tail reversed/.code={\pgfsetarrowsstart{tikzcd to}}}
\tikzset{2tail/.code={\pgfsetarrowsstart{Implies[reversed]}}}
\tikzset{2tail reversed/.code={\pgfsetarrowsstart{Implies}}}
\tikzset{no body/.style={/tikz/dash pattern=on 0 off 1mm}}
% =========== End block ========== %
  \tikzset{every picture/.style={line width=0.75pt}}
\usepackage{pgfplots}
  \pgfplotsset{compat=newest}
\usepackage{tcolorbox}
  \tcbuselibrary{most}
\usepackage[colorlinks=true,linkcolor=blue]{hyperref}
\usepackage{cleveref}
% \usepackage[hyperref=true,backend=biber,style=alphabetic,backref=true,url=false]{biblatex}
\usepackage[warnings-off={mathtools-colon,mathtools-overbracket}]{unicode-math}
\usepackage[default,amsbb]{fontsetup}
  \setmathfont[StylisticSet=1,range=\mathscr]{NewCMMath-Book.otf}
\usepackage{fancyhdr}
\usepackage{import}

\newcommand{\Id}{\mathbb{1}}
\newcommand{\lap}{\increment}

\DeclareMathOperator{\sign}{sign}
\DeclareMathOperator{\dom}{dom}
\DeclareMathOperator{\ran}{ran}
\DeclareMathOperator{\ord}{ord}
\DeclareMathOperator{\Span}{span}
\DeclareMathOperator{\img}{Im}
\DeclareMathOperator{\Ric}{Ric}
\newcommand{\card}{\texttt{\#}}
\newcommand{\ie}{\emph{i.e.}}
\newcommand{\st}{\emph{s.t.}}
\newcommand{\eps}{\varepsilon}
\newcommand{\vphi}{\varphi}
\newcommand{\vthe}{\vartheta}
\newcommand{\II}{I\!I}
\renewcommand{\emptyset}{⌀}
\newcommand{\acts}{\curvearrowright}
\newcommand{\xrr}{\xlongrightarrow}
\newcommand{\lrr}{\longrightarrow}
\newcommand{\lmt}{\longmapsto}
\newcommand{\into}{\hookrightarrow}
\newcommand{\op}{\operatorname}

\let\originalleft\left
\let\originalright\right
\renewcommand{\left}{\mathopen{}\mathclose\bgroup\originalleft}
\renewcommand{\right}{\aftergroup\egroup\originalright}

\theoremstyle{plain}\newtheorem{theorem}{Theorem}
\theoremstyle{definition}\newtheorem{definition}[theorem]{Definition}
\theoremstyle{definition}\newtheorem{example}[theorem]{Example}
\theoremstyle{definition}\newtheorem{problem}[theorem]{Problem}
\theoremstyle{plain}\newtheorem{axiom}[theorem]{Axiom}
\theoremstyle{plain}\newtheorem{corollary}[theorem]{Corollary}
\theoremstyle{plain}\newtheorem{lemma}[theorem]{Lemma}
\theoremstyle{plain}\newtheorem{proposition}[theorem]{Proposition}
\theoremstyle{plain}\newtheorem{prop}[theorem]{Proposition}
\theoremstyle{plain}\newtheorem{conjecture}[theorem]{Conjecture}
\theoremstyle{plain}\newtheorem{conj}[theorem]{Conjecture}
\theoremstyle{remark}\newtheorem{notation}[theorem]{Notation}
\theoremstyle{definition}\newtheorem*{question}{Question}
\theoremstyle{definition}\newtheorem*{answer}{Answer}
\theoremstyle{definition}\newtheorem*{goal}{Goal}
\theoremstyle{definition}\newtheorem*{application}{Application}
\theoremstyle{plain}\newtheorem*{exercise}{Exercise}
\theoremstyle{remark}\newtheorem*{remark}{Remark}
\theoremstyle{remark}\newtheorem*{note}{\small{Note}}
\numberwithin{equation}{section}
\numberwithin{theorem}{section}
\numberwithin{figure}{section}

\usepackage{xeCJK}
\setCJKmainfont{FZShuSong-Z01}[BoldFont=FZXiaoBiaoSong-B05,ItalicFont=FZKai-Z03]
\setCJKsansfont{FZXiHeiI-Z08}[BoldFont=FZHei-B01]
\setCJKmonofont{FZFangSong-Z02}
\setCJKfamilyfont{zhsong}{FZShuSong-Z01}[BoldFont=FZXiaoBiaoSong-B05]
\setCJKfamilyfont{zhhei}{FZHei-B01}
\setCJKfamilyfont{zhkai}{FZKai-Z03}
\setCJKfamilyfont{zhfs}{FZFangSong-Z02}
\setCJKfamilyfont{zhli}{FZLiShu-S01}
\setCJKfamilyfont{zhyou}{FZXiYuan-M01}[BoldFont=FZZhunYuan-M02]

\allowdisplaybreaks{}

\newcommand{\isFullBook}[2]{
  \ifnum\pdfstrcmp{\FullBook}{True}=0
    \ifnum\pdfstrcmp{}{#1}=0\unskip\else#1\fi
  \else
    \ifnum\pdfstrcmp{}{#2}=0\unskip\else#2\fi
  \fi\ignorespaces{}
}

\counterwithout{theorem}{section}
\counterwithout{equation}{section}

\begin{document}
We wish to show the immersion \[
  \Phi_{L^\mu}\colon M\lrr \mathbb{P}^N(\mathbb{C})
\] is an embedding. What we need to show is \[
  \Phi_{L^\mu}(p_0)\neq \Phi_{L^\mu}(p_1),\forall\,p_0\neq p_1
.\] Let \(\pi\colon \tilde{M}=Q_{p_0p_1}(M)\to M\) be the blow-up of \(M\) at
\(p_0\) and \(p_1\). Note that \[
  Q_{p_0p_1}(M)=Q_{p_0}(Q_{p_1}(M))=Q_{p_1}(Q_{p_0}(M))
.\] Put \(E_0=\pi^{-1}(p_0)\), \(E_1=\pi^{-1}(p_1)\) and \(E=\pi^{-1}(\{p_0,p_1\}
)\). Of course we have \[
  [E]=[E_0]\otimes [E_1],\quad E=E_0+E_1
.\] 

We can argue exactly in the same way as the proof of the previous lemma to get:
\begin{lemma}
  Let \(L\to M\) be an ample line bundle, \(G\to M\) an arbitrary line bundle,
  and \(k\) be any positive integer. Then there exists \(n_0=n_0(L,G,k)>0\)
  depending only on \(L,G\) and \(k\) such that, for any integer \(n\ge n_0\),
  the line bundle \[
    \pi^*L^m\otimes \pi^*G\otimes [E]^{-k}\to \tilde{M}=Q_{p_0p_1}(M)
  \] is ample.
\end{lemma}

As before, we have commutative diagram of exact sequences
\[\begin{tikzcd}
	0 & {\Gamma(\mathcal{O}(\tilde{L})\otimes\ell_E)} & {\Gamma(\mathcal{O}(\tilde{L}))} & {\Gamma(\mathcal{O}({\tilde{L}})\otimes(\mathcal{O}_{\tilde{M}}/\ell_E))} & {H^1(\cdots} \\
	0 & {\Gamma(\mathcal{O}(L)\otimes\ell_{p_0,p_1})} & {\Gamma(\mathcal{O}(L))} & {\Gamma(\mathcal{O}(L)\otimes(\mathcal{O}_M/\ell_{p_0,p_1}))} & {H^1(\cdots}
	\arrow[from=1-1, to=1-2]
	\arrow[from=2-1, to=2-2]
	\arrow["{{\pi_1^*}}", from=2-2, to=1-2]
	\arrow[from=1-2, to=1-3]
	\arrow[from=2-2, to=2-3]
	\arrow["{{\pi^*}}", from=2-3, to=1-3]
	\arrow["{{\rho}}", from=2-3, to=2-4]
	\arrow["{{\pi_2^*}}"', from=2-4, to=1-4]
	\arrow["{{\tilde{\rho}}}", from=1-3, to=1-4]
	\arrow[from=1-4, to=1-5]
	\arrow[from=2-4, to=2-5]
\end{tikzcd}\]
with \(\pi_1^*,\pi^*\) isomorphism, and \(\pi_2^*\) injection. Thus we obtain:
\begin{lemma}
  If \(\Gamma(\mathcal{O}(L))\to \Gamma(\mathcal{O}(\tilde{L}))\otimes(
  \mathcal{O}_{\tilde{M}}/\ell_E)\) is surjective then \(\Gamma(\mathcal{O}(L))
  \to \Gamma(\mathcal{O}(L)\otimes \ell_{p_0,p_1})\) is also surjective.
\end{lemma}

Assume that the assumption of the above lemma is satisfied. Since \(\mathcal{O}_M
/\ell_{p_0,p_1}\) is a skyscraper sheaf, \[
  \Gamma(\mathcal{O}(L)\otimes (\mathcal{O}_M/\ell_{p_0,p_1}))\cong L_{p_0}
  \oplus L_{p_1}\cong \mathbb{C}\oplus \mathbb{C}
,\] and
\begin{align*}
  \Gamma(\mathcal{O}(L)) &\longrightarrow L_{p_0}\oplus L_{p_1} \\
  \vphi &\longmapsto (\vphi(p_0),\vphi(p_1))
.\end{align*}
is surjective. Thus there exists \(\vphi_0,\vphi_1\in \Gamma(\mathcal{O}(L))\)
such that \(\vphi_i(p_j)=\delta_{ij}\). Thus, to prove \(\Phi_{L^\mu}\) is an
embedding, it is sufficient to show \[
  \Gamma(\mathcal{O}(\tilde{L}^\mu))\lrr \Gamma(\mathcal{O}(\tilde{L}^\mu)
  \otimes (\mathcal{O}_{\tilde{M}}/\ell_E))
\] is surjective. This is true if and only if \[
  H^1(\tilde{M},\mathcal{O}(\tilde{L}^\mu)\otimes \ell_E)\cong 
  H^1(\tilde{M},\mathcal{O}(\tilde{L}^\mu)\otimes [E]^{-1})=0
,\] which vanishes if \(\tilde{L}^\mu\otimes[E]^{-1}\otimes K_{\tilde{M}}^{-1}\)
is ample by Kodaira vanishing theorem. It is ample if \(\mu\) is sufficiently
large since \[
  \tilde{L}^\mu\otimes [E]^{-1}\otimes K_{\tilde{M}}^{-1}
  =\tilde{L}^\mu\otimes [E]^{-m}\otimes \pi^*K_M^{-1}
\] apply lemma with \(G=K_M^{-1},k=m\). All in all we obtained:
\begin{theorem}[Kodaira embedding]
  Let \(M\) be a compact complex manifold, \(L\) an ample line bundle. Then
  there exists \(\mu_0>0\) such that for any integer \(\mu\ge \mu_0\), \[
    \Phi_{L^\mu}\colon M\lrr \mathbb{P}^N(\mathbb{C}),\quad
    N+1=\dim H^0(M,\mathcal{O}(L^\mu))
  \] is an embedding.
\end{theorem}

\begin{remark}
  When \(\Phi_L\) gives an embedding we say \(L\) is very ample. So the Kodaira
  embedding theorem say if \(L\) is ample then \(L^\mu\) is very ample for
  sufficiently large \(\mu\).
\end{remark}

The result can be extended as follows:
\begin{theorem}
  Let \(M\) be a compact complex manifold, \(L\to M\) an ample line bundle and
  \(G\to M\) arbitrary line bundle. Then \(\exists\,\mu_0>0\) such that for any
  integer \(\mu\ge \mu_0\), \(L^\mu\otimes G\) is very ample.
\end{theorem}
\begin{proof}
  One can repeat the proof replacing \(L^\mu\) by \(L^\mu\otimes G\) and
  \(\tilde{L}^\mu\) by \(\tilde{L}^\mu\otimes \tilde{G}\), where \(\tilde{G}
  =\pi^*G\). To show \(\Phi_{L^\mu\otimes G}\) is an immersion its is sufficient
  to show for \(E=\pi^{-1}(p_0)\), \[
    H^1(\tilde{M},\mathcal{O}(\tilde{L}^\mu\otimes\tilde{G})\otimes\ell_E^2)=0
  .\] But this is reduced to show \[
  H^1(\tilde{M},\mathcal{O}(\tilde{L}^\mu\otimes \tilde{G})\otimes [E]^{-2})=0
  ,\] which is satisfied if \(\mu\) is large, since \[
    \tilde{L}^{\mu}\otimes\tilde{G}\otimes [E]^{-2}\otimes K_{\tilde{M}}^{-1}
    =\tilde{L}^\mu\otimes[E]^{-m-1}\otimes \pi^*(G\otimes K_M^{-1})
  \] is ample for large \(\mu\).

  To show \(\Phi_{L^\mu}\) is an embedding it is sufficient to show for
  \(E=\pi^{-1}(\{p_0,p_1\})\), \[
    H^1(\tilde{M},\mathcal{O}(\tilde{L}^\mu\otimes \tilde{G})\otimes \ell_E)=0
  .\] And similarly, \[
    \tilde{L}^{\mu}\otimes\tilde{G}\otimes [E]^{-1}\otimes K_{\tilde{M}}^{-1}
    =\tilde{L}^\mu\otimes[E]^{-m}\otimes \pi^*(G\otimes K_M^{-1})
  \] is ample for large \(\mu\).
\end{proof}

\section{Deformations of complex structure}
Before we leave \emph{Morrow-Kodaira}'s book, we take up the infinitesimal
deformation of complex structure because this is one of the main themes
of their book.
\begin{definition}
  A complex analytic family of compact complex manifolds is a holomorphic
  map \(\varpi\colon\mathscr{M}\to B\) between connected complex
  manifolds \(\mathscr{M}\) and \(B\) satisfying the following:
  \begin{enumerate}[i)]
  \item \(\varpi\) is proper, \ie\ for any compact subset \(K\subset B\),
    \(\varpi^{-1}(K)\) is compact. In particular, \(\varpi^{-1}(t)\) is compact
    for any \(t\in B\).
  \item \(M_t=\varpi^{-1}(t)\) is a compact submanifold of \(\mathscr{M}\).
  \item The rank of the Jacobian of \(\varpi\) is equal to \(n=\dim B\)
    everywhere. (Thus all \(M_t\) are diffeomorphic each other).
  \end{enumerate}
\end{definition}

In this case, we regard \((\mathscr{M},B,\varpi)\) expresses the family
\(\{M_t:t\in B\}\) of compact complex manifolds \(M_t=\varpi^{-1}(t)\), and we
say \(M_t\) depends on \(t\) holomorphically. We also say \(M_s=\varpi^{-1}(s)\)
is a deformation of \(M_t=\varpi^{-1}(t)\). We wish to define the infinitesimal
deformation \(\pdv{M_t}{t}\). We regard the change of complex structure as
coming from the change of local coordinate patches. To describe this we take
the local coordinates of \(\mathscr{M}\) as follows:

First take a point \(0\in B\) and choose a local coordinate \[
  N=\{(t_1,\ldots,t_n):|t_\lambda|<\eps\}
\] in \(B\) around \(0\) so that \((0,\ldots,0)=0\in N\subset B\). Next we take
local coordinate of \(\varpi^{-1}(N)\) as \[
\varpi^{-1}(N)=\bigcup_{\alpha\in A}\mathcal{U}_\alpha
.\] Where \[
  \mathcal{U}_{\alpha}=\{(z_\alpha,t):z_\alpha=(z^1_\alpha,\ldots,z^m_\alpha),
  |z^i_\alpha|<1,t\in N\}
.\] If we put \[
  U_\alpha=\{z_\alpha:|z_\alpha|<1\}
\] then \(\mathcal{U}_\alpha=U_\alpha\times N\). Strictly speaking,
\(t=(t_1,\ldots,t_n)\) should be written as \[
  \varpi^* t=(\varpi^*t_1,\ldots,\varpi^*t_n)
.\] But we avoid this clumsy notation by omitting \(\varpi^*\).

Over \(\mathcal{U}_\alpha\cap \mathcal{U}_\beta\), if \((z_\alpha,t)\in
\mathcal{U}_\alpha\) and \((z_\beta,t)\in\mathcal{U}_\beta\) are the same point
then they are related by \[
  z_\alpha^i=f_{\alpha\beta}^i(z_\beta,t),\quad i=1,\ldots,m
\] for some holomorphic functions \(f_{\alpha\beta}^i\). We consider the change
of complex structure as the change of \(f_{\alpha\beta}^i(z_\beta,t)\) in \(t\).

\end{document}
