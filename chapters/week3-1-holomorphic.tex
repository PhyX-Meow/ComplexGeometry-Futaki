\section{Canonical connection for holomorphic vector bundle with Hermitian metrics}
Let \(M\) be a complex manifold of complex dimension \(m\). Let \(z^1,\ldots,z^m\)
be local holomorphic coordinates on \(U\subset M\). Set \(z^i=x^i+\sqrt{-1}y^i\) with
real part \(x^i\) and imaginary part \(y^i\). Then \[
  (x^1,y^1,\ldots,x^m,y^m)
\] are local coordinates of \(M\) as a smooth manifold. Since \[
  \pdv{z^i}=\frac{1}{2}\left(\pdv{x^i}-\sqrt{-1}\pdv{y^i}\right),\quad
  \pdv{\overline{z}^i}=\frac{1}{2}\left(\pdv{x^i}+\sqrt{-1}\pdv{y^i}\right),
\] \[
  \pdv{x^i}=\pdv{z^i}+\pdv{\overline{z}^i},\quad
  \pdv{y^i}=\sqrt{-1}(\pdv{z^i}-\pdv{\overline{z}^i}),
\] we have a decomposition \[
  TM\otimes \mathbb{C}=T'M\oplus T''M
,\] where
\begin{align*}
  T'M&=\Span\{\pdv{z^1},\ldots,\pdv{z^m}\}, \\
  T''M&=\Span\{\pdv{\overline{z}^1},\ldots,\pdv{\overline{z}^m}\}
.\end{align*}
On the coordinate neighborhood \((U;z^1,\ldots,z^m)\),
\begin{align*}
  T'M\Big|_U&=\{\xi^1\pdv{z^1}+\cdots +\xi^m\pdv{z^m}:
  \xi^1,\ldots,\xi^m\in \mathbb{C}\} \\
  &\xrightarrow[\vphi_U]{\sim}\{(z^1,\ldots,z^m,\xi^1,\ldots,\xi^m)\in
  U\times \mathbb{C}^n\}
\end{align*}
gives a local trivialization.

On another coordinate neighbourhood \((V,w^1,\ldots,w^n)\) with \(U\cap V\neq
\emptyset\),
\begin{align*}
  \vphi_U\circ \vphi_V^{-1}(w^1,\ldots,w^m,\eta^1,\ldots,\eta^m)
  &=\vphi_U(\eta^1\pdv{w^1}+\cdots+\eta^m\pdv{w^m}) \\
  &=\vphi_U(\sum \eta^j\pdv{z^i}{w^j}\pdv{z^i}) \\
  &=(z^1,\ldots,z^m,\pdv{z^1}{w^j}\eta^j,\ldots,\pdv{z^m}{w^j}\eta^j)
.\end{align*}
Thus the transition function \(\vphi_{UV}(p)\) is given by \[
  \vphi_{UV}(p)=\left(\pdv{z^i}{w^j}(p)\right)
,\] which is holomorphic. So \(T'M\) is a holomorphic vector bundle.

The transition function of \(T''M\) is \(\left(\overline{\pdv{z^i}{w^j}}(p)\right)\),
which is not holomorphic. So \(T''M\) is not a holomorphic vector bundle.

Similarly, the cotangent bundle has the decomposition \[
  T^*M\otimes \mathbb{C}=T^{*\prime}M\oplus T^{*\prime\prime}M
.\] Where
\begin{align*}
  T^{*\prime}M&=\Span\{\dd{z^1},\ldots,\dd{z^m}\}, \\
  T^{*\prime\prime}M&=\Span\{\dd{\overline{z}^1},\ldots,\dd{\overline{z}^m}\}
.\end{align*}
The transition function of \(T^{*\prime}M\) is \(\tensor[^t]{\left(\pdv{z^i}{w^j}
\right)}{^{-1}}\), which is holomorphic.

\(T'M\) is called the holomorphic tangent bundle or tangent bundle of type \((1,0)\),
and \(T^{*\prime}M\) is called the holomorphic cotangent bundle.

\begin{example}
  Tensor product \[
    (\otimes^p T'M)\otimes (\otimes^q T^{*\prime}M)
  \] is also a holomorphic vector bundle.
\end{example}

\begin{definition}
  \(K_M=\wedge^m T^{*\prime}M\), \(m=\dim_{\mathbb{C}}M\) is a line bundle, called the
  \textbf{canonical line bundle} of \(M\).
  
  This line bundle plays an important role in the classification theory of complex 
  manifolds.
\end{definition}

Let \(D\) be a complex submanifold of codimension 1, which is also called a
non-singular \textbf{divisor}. For a divisor \(D\) there is a corresponding line
bundle \([D]\) defined as follows:

Let \(\{U_\lambda\}\) be an open covering by coordinate neighbourhoods such that there 
exist holomorphic functions \(f_\lambda\) vanishing along \(D\) with first order: \[
  D\cap U_\lambda=\{f_\lambda=0\}
.\] When \(D\cap U_\lambda=\emptyset\) we take \(f_\lambda\equiv 1\) (for example).
Then \[
  f_{\lambda\mu}=\frac{f_\lambda}{f_\mu}
\] is a nowhere vanishing holomorphic function on \(U_\lambda\cap U_\mu\). So \[
  f_{\lambda\mu}\colon U_\lambda\cap U_\mu\longrightarrow GL(1,\mathbb{C})\cong
  \mathbb{C}^*
,\] and \(f_{\lambda\mu}\cdot f_{\mu\nu}\cdot f_{\nu\lambda}=1\).

So \(\{f_{\lambda\mu}\}_{\lambda,\mu\in \Lambda}\) defines a holomorphic line bundle,
denoted by \([D]\), called the line bundle associated with the divisor \(D\).

\begin{remark}
  It is possible to consider singular divisors and also their formal sums.
  Assume \(D\rightsquigarrow L_D=[D]\), \(D'\rightsquigarrow L_{D'}=[D']\), then
  \begin{align*}
    D+D'&\rightsquigarrow L_{D+D'}=[D]\otimes [D']=L_D\otimes L_{D'} \\
    -D&\rightsquigarrow L_{-D}=L_D^{-1}
  .\end{align*}
  Where the transition function of \(L_D^{-1}\) is the inverse of the transition
  functions of \(L_D\). In particular, \[
    2D\rightsquigarrow L_{2D}=L_D\otimes L_D
  ,\] denoted by \(L_D^{\otimes 2}\) or \(L_D^2\).
\end{remark}
\begin{example}
  Let \[
    H=\{[0:z^1:\cdots :z^n]\in \mathbb{P}^m(\mathbb{C}):(z^1,\ldots,z^n)\neq 0\}
  \] be the hyperplane in \(\mathbb{P}^m(\mathbb{C})\). The associated line bundle
  \([H]\) is called the hyperplane bundle, denoted by \(\mathcal{O}(1)\).
  \(L_{-H}\) is denoted by \(\mathcal{O}(-1)\), and is isomorphic to the tautological
  line bundle (see homework 5).
\end{example}
\begin{example}
  Consider the case \(m=1\), \[
    \mathbb{P}^1(\mathbb{C})=\{[z^0:z^1]\},\quad H=\{[0:1]\}
  .\] The charts are
  \begin{gather*}
    U_0=\{z^0\neq 0\},\quad s=\frac{z^1}{z^0} \\
    U_1=\{z^1\neq 0\},\quad t=\frac{z^0}{z^1}
  \end{gather*}
  \(U_0\cap H=\emptyset\), so \(f_0\equiv 1\), \(U_1\cap H=\{t=0\}\), so \(f_1=t\).
  Then \[
    f_{01}=\frac{1}{t}=s,\quad f_{10}=t.
  \] We will use this example later.
\end{example}
