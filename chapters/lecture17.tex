% !TeX program = xelatex
\documentclass[12pt]{article}
\usepackage{standalone}

\usepackage[dvipsnames,svgnames,x11names]{xcolor}
\usepackage[a4paper,margin=1in]{geometry}
\usepackage{microtype}
\usepackage{amsmath}
\usepackage{amsthm}
\usepackage{mathtools}
\usepackage{mathrsfs}
\usepackage{stmaryrd}
\usepackage{extarrows}
\usepackage{enumerate}
\usepackage{tensor}
\usepackage{physics2}
  \usephysicsmodule{ab,xmat}
\usepackage{fixdif}
  \newcommand{\dd}{\d}
\usepackage{derivative}
  \newcommand{\dv}{\odv}
  \newcommand{\pd}[1]{\pdv{}{#1}}
  \newcommand{\eval}[1]{#1\big|}
\usepackage{graphicx}
\usepackage{subcaption}
\usepackage{tikz}
\usepackage{tikz-3dplot}
% \usepackage{tikz-cd}
% \usepackage{quiver}
% ========== Block quiver.sty ========== %
\usepackage{tikz-cd}
% \usepackage{amssymb}
\usetikzlibrary{calc}
\usetikzlibrary{decorations.pathmorphing}
\tikzset{curve/.style={settings={#1},to path={(\tikztostart)
    .. controls ($(\tikztostart)!\pv{pos}!(\tikztotarget)!\pv{height}!270:(\tikztotarget)$)
    and ($(\tikztostart)!1-\pv{pos}!(\tikztotarget)!\pv{height}!270:(\tikztotarget)$)
    .. (\tikztotarget)\tikztonodes}},
    settings/.code={\tikzset{quiver/.cd,#1}
        \def\pv##1{\pgfkeysvalueof{/tikz/quiver/##1}}},
    quiver/.cd,pos/.initial=0.35,height/.initial=0}
\tikzset{tail reversed/.code={\pgfsetarrowsstart{tikzcd to}}}
\tikzset{2tail/.code={\pgfsetarrowsstart{Implies[reversed]}}}
\tikzset{2tail reversed/.code={\pgfsetarrowsstart{Implies}}}
\tikzset{no body/.style={/tikz/dash pattern=on 0 off 1mm}}
% =========== End block ========== %
  \tikzset{every picture/.style={line width=0.75pt}}
\usepackage{pgfplots}
  \pgfplotsset{compat=newest}
\usepackage{tcolorbox}
  \tcbuselibrary{most}
\usepackage[colorlinks=true,linkcolor=blue]{hyperref}
\usepackage{cleveref}
% \usepackage[hyperref=true,backend=biber,style=alphabetic,backref=true,url=false]{biblatex}
\usepackage[warnings-off={mathtools-colon,mathtools-overbracket}]{unicode-math}
\usepackage[default,amsbb]{fontsetup}
  \setmathfont[StylisticSet=1,range=\mathscr]{NewCMMath-Book.otf}
\usepackage{fancyhdr}
\usepackage{import}

\newcommand{\Id}{\mathbb{1}}
\newcommand{\lap}{\increment}

\DeclareMathOperator{\sign}{sign}
\DeclareMathOperator{\dom}{dom}
\DeclareMathOperator{\ran}{ran}
\DeclareMathOperator{\ord}{ord}
\DeclareMathOperator{\Span}{span}
\DeclareMathOperator{\img}{Im}
\DeclareMathOperator{\Ric}{Ric}
\newcommand{\card}{\texttt{\#}}
\newcommand{\ie}{\emph{i.e.}}
\newcommand{\st}{\emph{s.t.}}
\newcommand{\eps}{\varepsilon}
\newcommand{\vphi}{\varphi}
\newcommand{\vthe}{\vartheta}
\newcommand{\II}{I\!I}
\renewcommand{\emptyset}{⌀}
\newcommand{\acts}{\curvearrowright}
\newcommand{\xrr}{\xlongrightarrow}
\newcommand{\lrr}{\longrightarrow}
\newcommand{\lmt}{\longmapsto}
\newcommand{\into}{\hookrightarrow}
\newcommand{\op}{\operatorname}

\let\originalleft\left
\let\originalright\right
\renewcommand{\left}{\mathopen{}\mathclose\bgroup\originalleft}
\renewcommand{\right}{\aftergroup\egroup\originalright}

\theoremstyle{plain}\newtheorem{theorem}{Theorem}
\theoremstyle{definition}\newtheorem{definition}[theorem]{Definition}
\theoremstyle{definition}\newtheorem{example}[theorem]{Example}
\theoremstyle{definition}\newtheorem{problem}[theorem]{Problem}
\theoremstyle{plain}\newtheorem{axiom}[theorem]{Axiom}
\theoremstyle{plain}\newtheorem{corollary}[theorem]{Corollary}
\theoremstyle{plain}\newtheorem{lemma}[theorem]{Lemma}
\theoremstyle{plain}\newtheorem{proposition}[theorem]{Proposition}
\theoremstyle{plain}\newtheorem{prop}[theorem]{Proposition}
\theoremstyle{plain}\newtheorem{conjecture}[theorem]{Conjecture}
\theoremstyle{plain}\newtheorem{conj}[theorem]{Conjecture}
\theoremstyle{remark}\newtheorem{notation}[theorem]{Notation}
\theoremstyle{definition}\newtheorem*{question}{Question}
\theoremstyle{definition}\newtheorem*{answer}{Answer}
\theoremstyle{definition}\newtheorem*{goal}{Goal}
\theoremstyle{definition}\newtheorem*{application}{Application}
\theoremstyle{plain}\newtheorem*{exercise}{Exercise}
\theoremstyle{remark}\newtheorem*{remark}{Remark}
\theoremstyle{remark}\newtheorem*{note}{\small{Note}}
\numberwithin{equation}{section}
\numberwithin{theorem}{section}
\numberwithin{figure}{section}

\usepackage{xeCJK}
\setCJKmainfont{FZShuSong-Z01}[BoldFont=FZXiaoBiaoSong-B05,ItalicFont=FZKai-Z03]
\setCJKsansfont{FZXiHeiI-Z08}[BoldFont=FZHei-B01]
\setCJKmonofont{FZFangSong-Z02}
\setCJKfamilyfont{zhsong}{FZShuSong-Z01}[BoldFont=FZXiaoBiaoSong-B05]
\setCJKfamilyfont{zhhei}{FZHei-B01}
\setCJKfamilyfont{zhkai}{FZKai-Z03}
\setCJKfamilyfont{zhfs}{FZFangSong-Z02}
\setCJKfamilyfont{zhli}{FZLiShu-S01}
\setCJKfamilyfont{zhyou}{FZXiYuan-M01}[BoldFont=FZZhunYuan-M02]

\allowdisplaybreaks{}

\newcommand{\isFullBook}[2]{
  \ifnum\pdfstrcmp{\FullBook}{True}=0
    \ifnum\pdfstrcmp{}{#1}=0\unskip\else#1\fi
  \else
    \ifnum\pdfstrcmp{}{#2}=0\unskip\else#2\fi
  \fi\ignorespaces{}
}

\counterwithout{theorem}{section}
\counterwithout{equation}{section}

\begin{document}

In the note of lecture 10, we studied the divergence theorem:
For compact Riemannian manifold \((M,g)\), \[
  \int_{M}\op{div}(X)\dd{V_g}=0
,\] where \[
  \op{div}(X)=\nabla_i X^i=\pdv{X^i}{x^i}+\Gamma_{ik}^i X^i
,\] and its application, the integration by parts formula, \[
  \int_{M}T^i \nabla_i f\dd{V_g}=-\int_{M}(\nabla_i T^i)f\dd{V_g}
.\] We will use this to the very basic due to Bochner.
\begin{theorem}
  On a compact Riemannian manifold with negative Ricci curvature, there is no
  non-zero Killing vector field, in particular, the group of isometries is
  a finite group. (It is well-known that the isometry group is compact on a 
  compact Riemannian manifold).
\end{theorem}
\begin{proof}
  As we discussed in the previous notes, if \(X\) is a Killing vector field,
  then \[
    \mathcal{L}_X g=0\iff g(\nabla_Y X,Z)+g(Y,\nabla_Z X)=0,\forall\,Y,Z
  .\] Taking \(Y=\pd{x^i},Z=\pd{x^j}\), we obtain \[
    g_{kj}\nabla_i X^k+g_{ik}\nabla_j X^k=0
  .\] Thus \[
    \nabla_i X_j+\nabla_j X_i=0
  .\] \ie\ \(\nabla X\) is skew-symmetric.

  One computes
  \begin{align*}
    0&\le \int_{M}|\nabla X|^2\dd{V_g} \\
    &=\int_{M}g^{ij}g^{kl}\nabla_i X_k\cdot \nabla_j X_l
    &(X_k=g_{kp}X^p) \\
    &=-\int_{M}g^{ij}g^{kl}(\nabla_j \nabla_i X_k)\cdot X_l \dd{V_g}
    &(\text{integrate by parts}) \\
    &=\int_{M}g^{ij}g^{kl}(\nabla_j \nabla_k X_i)\cdot X_l \dd{V_g}
    &(\nabla X\text{ is skew}) \\
    &=\int_{M}g^{ij}g^{kl}(\nabla_k \nabla_j X_i-R\indices{_{jk}^p_i}X_p)
    \cdot X_l
  .\end{align*} Note that \(\nabla_i X_j+\nabla_j X_i=0\) implies \(g^{ij}
  \nabla_i X_j=0\) by multiplying \(g^{ij}\). Thus the above computes \[
    0\le\int_{M}g^{ij}g^{kl}R\indices{_{jki}^p}X_p X_l
    =\int_{M}R_{kq}X^k X^q
  .\] Here \(R_{kq}=g^{ij}R_{jkiq}\) is the Ricci curvature. If \((R_{ij})\) is
  negative definite and \(X\neq 0\), then \(\int R_{kq}X^k X^q<0\). This is
  a contradiction.
\end{proof}

For a 1-form \(\alpha\), \(\dd^*\alpha\) was defined by \[
  (\dd{}^*\alpha,f)_{L^2}=(\alpha,\dd{f})_{L^2},\quad\forall\,f\in C^\infty(M)
.\] But
\begin{align*}
  (\alpha,\dd{f})_{L^2}&=\int_{M}g^{ij}\alpha_i \pdv{f}{x^j} \\
  &=\int_{M}\nabla_j (\underbrace{g^{ij}\alpha_i f}_{\mathclap{
  \text{regard as a v.f.}}})-\int_{M}\nabla_j (g^{ij}\alpha_i)\cdot f \\
  &=-\int_{M}\nabla_j \alpha^j f &(\text{div thm})
.\end{align*} 
From this we see \[
  \dd{}^*\alpha=-\nabla_j \alpha^j=-\op{div}(\sharp \alpha)
.\] Here \(\sharp\) is the musical isomorphism.

Recall that \(\alpha\) is a harmonic 1-form if and only if
\begin{align*}
  \lap_{\dd{}}\alpha=0&\iff \dd{\alpha}=0\text{ and }\dd{}^*\alpha=0 \\
  &\iff\pdv{\alpha_i}{x^j}-\pdv{\alpha_j}{x^i}=0\text{ and }\nabla_i\alpha^i=0 \\
  &\iff \nabla_j\alpha_i=\nabla_i \alpha_j\text{ and }\nabla_i \alpha^i=0
  \quad (\text{since }\Gamma_{ji}^k=\Gamma_{ij}^k)
.\end{align*} 
Now we prove another very basic theorem by Bochner:
\begin{theorem}
  Let \(M\) be a compact Riemannian manifold with positive Ricci curvature, \ie\ 
  \(R_{ij}\) is positive definite. Then the first Betti number \(b_1\) is zero,
  \ie\ there is no non-trivial harmonic 1-form.
\end{theorem}
\begin{proof}
  As in the case of Killing vector field, we compute
  \begin{align*}
    0&\le\int_{M}|\nabla \alpha|^2\dd{V_g}=\int_{M}g^{ij}g^{kl}\nabla_i \alpha_k
    \cdot \nabla_j \alpha_l \\
    &=-\int_{M}g^{ij}g^{kl}(\nabla_j \nabla_i \alpha_k)\cdot \alpha_l
    &(\text{integrate by parts}) \\
    &=-\int_{M}g^{ij}g^{kl}(\nabla_j \nabla_k \alpha_i) \cdot \alpha_l
    &(\text{since }\nabla_i\alpha_k=\nabla_k\alpha_i) \\
    &=-\int_{M}g^{ij}g^{kl}(\nabla_k\nabla_j-R\indices{_{jk}^p_i}\alpha_p)
    \alpha_l & (\text{Ricci identity}) \\
    &=-\int_{M}R_{kl}\alpha^k \alpha^l
    &(g^{ij}\nabla_j \alpha^i=\nabla_i \alpha^i=0)
  .\end{align*}
  If \(R_{ij}\) is positive definite and \(\alpha\neq 0\), then the last
  expression is negative and we have a contradiction. 
\end{proof}

In the future, if you forget the sign convention and you are not certain
whether you are using the right sign convention. It is recommend to prove
Bochner's theorems. If you can prove then you are using the right convention.

Next, we consider the Kähler case. Recall that the Kähler metric is of the
form \[
  g_{i\bar{j}}=g\left(\pd{z^i},\pd{\bar{z}^j}\right)=g\left(\pd{\bar{z}^j}
  \pd{z^i}\right)=g_{\bar{j}i}
\] since we regard \(g\) as the \(\mathbb{C}\)-linear extension of the Riemannian
metric. Also, \(g\) is \(J\)-invariant, \ie\ \(g(JX,JY)=g(X,Y)\). So \[
  g\left(\pd{z^i},\pd{z^j}\right)=g\left(J\pd{z^i},J\pd{z^j}\right)
  =g\left(\sqrt{-1}\pd{z^i},\sqrt{-1}\pd{z^j}\right)
  =-g\left(\pd{z^i},\pd{z^j}\right)
.\] So \(g_{ij}=0\), similarly \(g_{\bar{i}\bar{j}}=0\). And only \(g_{i\bar{j}}
=g_{\bar{j}i}\) can be non-zero.

Thus the musical isomorphism are defined as follows:
\begin{align*}
  \sharp\colon T^{\prime*}M &\longrightarrow T''M=\overline{T'M}
  & \sharp\colon T^{\prime\prime*}M &\longrightarrow T'M \\
  \alpha_i \dd{z^i} &\longmapsto g^{i\bar{j}}\alpha_i \pd{\bar{z}^j}
  =\alpha^{\bar{j}}\pd{\bar{z}^j}
  &\alpha_{\bar{i}} \dd{\bar{z}^i} &\longmapsto g^{j\bar{i}}\alpha_{\bar{i}}
  \pd{z^j}=\alpha^{j}\pd{z^j}
.\end{align*}
\begin{align*}
  \flat\colon T'M &\longrightarrow T^{\prime\prime*}M
  &\flat\colon T''M &\longrightarrow T^{\prime*}M \\
  X^i\pd{z^i} &\longmapsto g_{i\bar{j}}x^i\dd{\bar{z}^j}
  =X_{\bar{j}}\dd{\bar{z}^j}
  &X^{\bar{i}}\pd{\bar{z}^i} &\longmapsto g_{j\bar{i}}x^{\bar{i}}\dd{z^j}
  =X_{j}\dd{z^j}
.\end{align*}

\section{Vanishing theorems on complex manifold}
We move on to vanishing theorems, to state them, we use the notation
\(\mathcal{O}(L)\) to be the sheaf of germs of holomorphic sections of a
holomorphic line bundle \(L\). Thus \[
  H^q(M,\mathcal{O}(L))
\] is the \(q\)-th sheaf cohomology of \(\mathcal{O}(L)\). But this is
isomorphic to the space of harmonic \((0,q)\)-forms with values in \(L\).

On a compact Kähler manifold, the volume element is given for the Kähler
form \(\omega=\sqrt{-1}g_{i\bar{j}}\dd{z^i}\dd{\bar{z}^j}\) by \[
  \dd{V_g}=\frac{\omega^m}{m!}=\frac{\overbrace{\omega\wedge\cdots\wedge\omega}
  ^m}{m!},\quad m=\dim_{\mathbb{C}}M
.\] But here we omit the \(m!\), and take \(\omega^m\) to be the volume form.
(We change the notation for Kähler form form \(\gamma\) to \(\omega\) since
this is more standard).

Let \(M\) be a compact Kähler manifold, for a \((0,1)\)-form \(\alpha\),
\(\bar{\partial}^*\alpha\) was defined by \[
  (\bar{\partial}^*\alpha,f)_{L^2}=(\alpha,\bar{\partial}f)_{L^2},
  \quad\forall\,f\in C^\infty(M)\otimes \mathbb{C}
.\] But
\begin{align*}
  (\alpha,\bar{\partial}f)_{L^2}&=\int_{M}g^{i\bar{j}}\alpha_{\bar{j}}
  \overline{\pdv{f}{\bar{z}^i}}\omega^m \\
  &=\int_{M}g^{i\bar{j}}\alpha_{\bar{j}}\nabla_i \bar{f}\omega^m \\
  &=\int_{M}\nabla_i (g^{i\bar{j}}\alpha_{\bar{j}}\bar{f})\omega^m
  -\int_{M}\nabla_i \alpha^i\cdot f\omega^m \\
  &=-\int_{M}\nabla_i \alpha^i \cdot f\omega^m & (\text{divergence thm})
.\end{align*}
Thus \[
  \bar{\partial}^*\alpha=-\nabla_i \alpha^i
  =-\pdv{\alpha^i}{z^i}-\Gamma_{ik}^i\alpha^k
\] Note that \(i,k\) are holomorphic indices. Then \[
  \bar{\partial}^*\alpha=-\op{div}(\sharp\alpha)
\] where \(\op{div}\) is with respect to holomorphic indices. This suggests
us that for higher degree differential forms, \(\bar{\partial}^*\) is better
expressed if we use the musical isomorphism \(\sharp\).

Let \(\alpha\in \mathcal{A}^{p,q+1}\) be a \(C^\infty\) differential form of type
\((p,q+1)\): \[
  \alpha=\alpha_{i_1\cdots i_p \bar{j}1\cdots \bar{j}_{q+1}}
  \dd{z^1}\wedge\cdots\wedge\dd{z^p}\wedge\dd{\bar{z}^1}\wedge\cdots\wedge
  \dd{\bar{z}^{q+1}}
.\] Then
\begin{align*}
  \sharp\alpha=&g^{i_1\bar{k}1}\cdots g^{i_p\bar{k}_p}g^{l_1\bar{j}_1}\cdots
  g^{l_{q+1}\bar{j}_{q+1}}\\
  &\cdot\alpha_{i_1\cdots i_p \bar{j}1\cdots \bar{j}_{q+1}}
  \pd{\bar{z}^{k_1}}\wedge\cdots\wedge\pd{\bar{z}^{k_p}}\wedge\pd{z^{l_1}}\wedge
  \cdots\wedge\pd{z^{l_{q+1}}} \\
  =&\alpha^{\bar{k}_1\cdots \bar{k}_p l_1\cdots l_{q+1}}\pd{\bar{z}^{k_1}}
  \wedge\cdots 
.\end{align*} 
We write \(\bar{K}=\bar{k}_1\cdots \bar{k}_p\) for short. Then as in the (1,0)
case, we obtain \[
  (\bar{\partial}^*\alpha)^{\bar{K}j_1\cdots j_q}
  =-\nabla_j \alpha^{j\bar{K}j_1\cdots j_q}
  =(-1)^{p+1}\nabla_j \alpha^{\bar{K}jj_1\cdots j_q}
.\] The following is a direct analogue for Kähler manifolds of Bochner's
Killing vector field vanishing theorem.
\begin{theorem}
  Let \(M\) be a compact Kähler manifold with negative Ricci curvature. Then
  there is no non-trivial holomorphic vector field, and thus the group of
  bi-holomorphic automorphisms is discrete.
\end{theorem}
\begin{proof}
  The proof is similar to the Killing case:
  \begin{align*}
    0&\le\int_{M}g^{i\bar{k}}g_{j\bar{l}}\nabla_i X^j \nabla_{\bar{k}}
    \overline{X^l}\omega^m \\
    &=-\int_{M}g^{i\bar{k}}g_{j\bar{l}}(\nabla_i \nabla_{\bar{k}}X^j+
    R\indices{_{\bar{k}i}^j_p}X^p)X^{\bar{l}} \\
    &=+\int_{M}R\indices{^i_{ip\bar{l}}}X^p X^{\bar{l}} \\
    &=\int_{M}R_{p\bar{l}}X^p X^{\bar{l}}
  .\end{align*}
  If \(X\) is non-zero and the Ricci curvature \(R_{p\bar{l}}\) is negative
  definite, then the last term is strictly negative. This is a contradiction.
\end{proof}
Note that the Lie algebra of all holomorphic vector fields is the Lie algebra
of the automorphism group.

\end{document}
