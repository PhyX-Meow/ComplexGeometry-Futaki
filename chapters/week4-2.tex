% !TeX program = xelatex
\documentclass[12pt]{article}
\usepackage{standalone}

\usepackage[dvipsnames,svgnames,x11names]{xcolor}
\usepackage[a4paper,margin=1in]{geometry}
\usepackage{microtype}
\usepackage{amsmath}
\usepackage{amsthm}
\usepackage{mathtools}
\usepackage{mathrsfs}
\usepackage{stmaryrd}
\usepackage{extarrows}
\usepackage{enumerate}
\usepackage{tensor}
\usepackage{physics2}
  \usephysicsmodule{ab,xmat}
\usepackage{fixdif}
  \newcommand{\dd}{\d}
\usepackage{derivative}
  \newcommand{\dv}{\odv}
  \newcommand{\pd}[1]{\pdv{}{#1}}
  \newcommand{\eval}[1]{#1\big|}
\usepackage{graphicx}
\usepackage{subcaption}
\usepackage{tikz}
\usepackage{tikz-3dplot}
% \usepackage{tikz-cd}
% \usepackage{quiver}
% ========== Block quiver.sty ========== %
\usepackage{tikz-cd}
% \usepackage{amssymb}
\usetikzlibrary{calc}
\usetikzlibrary{decorations.pathmorphing}
\tikzset{curve/.style={settings={#1},to path={(\tikztostart)
    .. controls ($(\tikztostart)!\pv{pos}!(\tikztotarget)!\pv{height}!270:(\tikztotarget)$)
    and ($(\tikztostart)!1-\pv{pos}!(\tikztotarget)!\pv{height}!270:(\tikztotarget)$)
    .. (\tikztotarget)\tikztonodes}},
    settings/.code={\tikzset{quiver/.cd,#1}
        \def\pv##1{\pgfkeysvalueof{/tikz/quiver/##1}}},
    quiver/.cd,pos/.initial=0.35,height/.initial=0}
\tikzset{tail reversed/.code={\pgfsetarrowsstart{tikzcd to}}}
\tikzset{2tail/.code={\pgfsetarrowsstart{Implies[reversed]}}}
\tikzset{2tail reversed/.code={\pgfsetarrowsstart{Implies}}}
\tikzset{no body/.style={/tikz/dash pattern=on 0 off 1mm}}
% =========== End block ========== %
  \tikzset{every picture/.style={line width=0.75pt}}
\usepackage{pgfplots}
  \pgfplotsset{compat=newest}
\usepackage{tcolorbox}
  \tcbuselibrary{most}
\usepackage[colorlinks=true,linkcolor=blue]{hyperref}
\usepackage{cleveref}
% \usepackage[hyperref=true,backend=biber,style=alphabetic,backref=true,url=false]{biblatex}
\usepackage[warnings-off={mathtools-colon,mathtools-overbracket}]{unicode-math}
\usepackage[default,amsbb]{fontsetup}
  \setmathfont[StylisticSet=1,range=\mathscr]{NewCMMath-Book.otf}
\usepackage{fancyhdr}
\usepackage{import}

\newcommand{\Id}{\mathbb{1}}
\newcommand{\lap}{\increment}

\DeclareMathOperator{\sign}{sign}
\DeclareMathOperator{\dom}{dom}
\DeclareMathOperator{\ran}{ran}
\DeclareMathOperator{\ord}{ord}
\DeclareMathOperator{\Span}{span}
\DeclareMathOperator{\img}{Im}
\DeclareMathOperator{\Ric}{Ric}
\newcommand{\card}{\texttt{\#}}
\newcommand{\ie}{\emph{i.e.}}
\newcommand{\st}{\emph{s.t.}}
\newcommand{\eps}{\varepsilon}
\newcommand{\vphi}{\varphi}
\newcommand{\vthe}{\vartheta}
\newcommand{\II}{I\!I}
\renewcommand{\emptyset}{⌀}
\newcommand{\acts}{\curvearrowright}
\newcommand{\xrr}{\xlongrightarrow}
\newcommand{\lrr}{\longrightarrow}
\newcommand{\lmt}{\longmapsto}
\newcommand{\into}{\hookrightarrow}
\newcommand{\op}{\operatorname}

\let\originalleft\left
\let\originalright\right
\renewcommand{\left}{\mathopen{}\mathclose\bgroup\originalleft}
\renewcommand{\right}{\aftergroup\egroup\originalright}

\theoremstyle{plain}\newtheorem{theorem}{Theorem}
\theoremstyle{definition}\newtheorem{definition}[theorem]{Definition}
\theoremstyle{definition}\newtheorem{example}[theorem]{Example}
\theoremstyle{definition}\newtheorem{problem}[theorem]{Problem}
\theoremstyle{plain}\newtheorem{axiom}[theorem]{Axiom}
\theoremstyle{plain}\newtheorem{corollary}[theorem]{Corollary}
\theoremstyle{plain}\newtheorem{lemma}[theorem]{Lemma}
\theoremstyle{plain}\newtheorem{proposition}[theorem]{Proposition}
\theoremstyle{plain}\newtheorem{prop}[theorem]{Proposition}
\theoremstyle{plain}\newtheorem{conjecture}[theorem]{Conjecture}
\theoremstyle{plain}\newtheorem{conj}[theorem]{Conjecture}
\theoremstyle{remark}\newtheorem{notation}[theorem]{Notation}
\theoremstyle{definition}\newtheorem*{question}{Question}
\theoremstyle{definition}\newtheorem*{answer}{Answer}
\theoremstyle{definition}\newtheorem*{goal}{Goal}
\theoremstyle{definition}\newtheorem*{application}{Application}
\theoremstyle{plain}\newtheorem*{exercise}{Exercise}
\theoremstyle{remark}\newtheorem*{remark}{Remark}
\theoremstyle{remark}\newtheorem*{note}{\small{Note}}
\numberwithin{equation}{section}
\numberwithin{theorem}{section}
\numberwithin{figure}{section}

\usepackage{xeCJK}
\setCJKmainfont{FZShuSong-Z01}[BoldFont=FZXiaoBiaoSong-B05,ItalicFont=FZKai-Z03]
\setCJKsansfont{FZXiHeiI-Z08}[BoldFont=FZHei-B01]
\setCJKmonofont{FZFangSong-Z02}
\setCJKfamilyfont{zhsong}{FZShuSong-Z01}[BoldFont=FZXiaoBiaoSong-B05]
\setCJKfamilyfont{zhhei}{FZHei-B01}
\setCJKfamilyfont{zhkai}{FZKai-Z03}
\setCJKfamilyfont{zhfs}{FZFangSong-Z02}
\setCJKfamilyfont{zhli}{FZLiShu-S01}
\setCJKfamilyfont{zhyou}{FZXiYuan-M01}[BoldFont=FZZhunYuan-M02]

\allowdisplaybreaks{}

\newcommand{\isFullBook}[2]{
  \ifnum\pdfstrcmp{\FullBook}{True}=0
    \ifnum\pdfstrcmp{}{#1}=0\unskip\else#1\fi
  \else
    \ifnum\pdfstrcmp{}{#2}=0\unskip\else#2\fi
  \fi\ignorespaces{}
}

\counterwithout{theorem}{section}
\counterwithout{equation}{section}

\begin{document}
Let \(M\) be a compact complex surface \((m=2)\), and \(D\) be any divisor. Then by
\ifdefined\FullBook{} \cref{thm:4-1:A}
\else Theorem A.
\fi \[
  c_1(L_D)\cdot [D]=[D]\circ [D]
\] where \(L_D\) is the line bundle associated to \(D\).

\begin{exercise}
  \(L_D\Big|_{D}\cong\) the normal bundle of \(D\).
\end{exercise}

Consider the blow-up \(\pi\colon \tilde{M}\to M\) of \(M\) at \(p\in M\). (c.f. homework
5). \(E=\pi^{-1}(\{p\})\) is a divisor of \(\tilde{M}\), called the \textbf{
exceptional divisor} (of the first kind).

For \(m=2\), we can show using
\ifdefined\FullBook{} \cref{thm:4-1:A}
\else Theorem A.
\fi \[
  E\cdot E=-1
\] \ie\ the self-intersection of the exceptional divisor on a compact complex surface
is \(-1\).

(Hint: The normal bundle of \(E\cong \mathcal{O}_{\mathbb{P}^1}(-1)\), then use the
exercise above.)

Conversely, if \(E\) is a rational curve (A rational curve is a smooth Riemann surface
homeomorphic to \(\mathbb{S}^2\)) with \(E\cdot E=-1\), then \(E\) can be
``blown down'' if \(M\) is an algebraic surface. (Castelnuovo-Enriques criterion,
see \emph{Griffiths-Harris}, page 476). 

Let \(M\) be a complex manifold with \(\dim_{\mathbb{C}}M=m\), \(g\) a Hermitian metric
of \(T'M\) so that \[
  g\in C^\infty(M,T^{*\prime}M\otimes \overline{T^{*\prime}M})
.\] We exchange the holomorphic part and the anti-holomorphic part following the
standard notation: \[
  g_{i\bar{j}}:=g\left(\pd{z^i},\pd{\bar{z}^j}\right),\quad
  g=g_{i\bar{j}}\dd{z^i}\otimes \dd{\bar{z}^j}
.\] If we write \(z^i=x^i+\sqrt{-1}y^i\), then the real part \(\Re(g)\) is 
\begin{align*}
  \Re(g)&=\frac{1}{2}(g_{i\bar{j}}\dd{z^i}\otimes \dd{\bar{z}^j}
  +\overline{g_{i\bar{j}}}\dd{\bar{z}^i}\otimes \dd{z^j}) \\
  &=\frac{1}{2}(g_{i\bar{j}}\dd{z^i}\otimes \dd{\bar{z}^j}+g_{j\bar{i}}\dd{\bar{z}^i}
  \otimes \dd{z^j}) \\
  &=g_{i\bar{j}}\cdot \frac{1}{2}(\dd{z^i}\otimes \dd{\bar{z}^j}
  +\dd{\bar{z}^j}\otimes \dd{z^i})
.\end{align*}
Here we used the Hermitian property \(g_{j\bar{i}}=\overline{g_{i\bar{j}}}\).

Fix a point \(p\in M\), and take local coordinates \(z^1,\ldots,z^m\) so that
\(\pd{z^1}\Big|_p,\ldots,\pd{z^m}\Big|_p\) are orthonormal. \ie\ 
\(g_{i\bar{j}}=\delta_{ij}\) at \(p\). Then at \(p\),
\begin{align*}
  \Re(g)&=\frac{1}{2}\sum_{i=1}^{m}(\dd{z^i}\otimes \dd{\bar{z}^i}
  +\dd{\bar{z}^i}\otimes \dd{z^i}) \\
  &=\Re\sum_i (\dd{x^i}+\sqrt{-1}\dd{y^i})\otimes (\dd{x^i}-\sqrt{-1}\dd{y^i}) \\
  &=\sum_i(\dd{x^i}\otimes \dd{x^i}+\dd{y^i}\otimes \dd{y^i})
.\end{align*}
That is to say, \(\Re(g)\) is a Riemannian metric on the underlying real manifold
\(M\).

\begin{question}
  When do the Levi-Civita connection of the Riemannian metric \(\Re(g)\) and the
  Chern connection of the Hermitian metric \(g\) coincide?
\end{question}
\textbf{Answer:} They coincide when the \textbf{K\"ahler condition} described
below is satisfied.

\begin{definition}
  The following 2-form \(\gamma\) is called the \textbf{fundamental 2-form}.
  \begin{align*}
    \gamma&=-2\Im(g)=-2\text{ times the imaginary part of }g \\
    &=\sqrt{-1}g_{i\bar{j}}(\dd{z^i}\dd{\bar{z}^j}-\dd{z^j}\otimes \dd{z^i}) \\
    &=\sqrt{-1}g_{i\bar{j}}\dd{z^i}\dd{\bar{z}^j}
  .\end{align*}
\end{definition}
\begin{remark}
  \(\gamma\) is a real 2-form.
\end{remark}
\begin{definition}
\begin{enumerate}[(1)]
\item A Hermitian metric \(g\) is said to satisfy the \textbf{K\"ahler condition}
  if \(\dd{\gamma}=0\).
\item \(g\) is called a \textbf{K\"ahler metric} iff \(\dd{\gamma}=0\).
\item The pair \((M,g)\) is called a \textbf{K\"ahler manifold} when \(g\) is
  K\"ahler. 
\end{enumerate}
\end{definition}
Sometimes, a complex manifold which admits a K\"ahler metric is called a K\"ahler
manifold and a complex manifold which never admits a K\"ahler metric is called a
non-K\"ahler manifold.

\begin{theorem}
  For a Hermitian metric \(g\), the Levi-Civita connection (after extending in
  \(\mathbb{C}\)-linear manner) coincides with the Chern connection if and only if
  \(g\) is K\"ahler.
\end{theorem}
\begin{proof}
  Let \(\nabla\) be the Chern connection, \ie\ \(\nabla g=0\) and \(\nabla =\nabla'
  +\bar{\partial}\). We only need to show \(\nabla\) is torsion free (or symmetric) if
  and only if \(\dd{\gamma}=0\).

  Recall \(\omega=g^{-1}\partial g=g^{i\bar{k}}\partial g_{j\bar{k}}\), so \[
    \nabla_{\pd{z^i}}\pd{z^j}=\omega\indices{^p_j}\left(\pd{z^i}\right)
    \pd{z^p}=g^{p\bar{k}}\pdv{g_{j\bar{k}}}{z^i}\pd{z^p}
  .\] Thus \[
    \left[\pd{z^i},\pd{z^j}\right]=\nabla_{\pd{z^i}}\pd{z^j}-\nabla_{\pd{z^j}}\pd{z^i}
    \iff \pdv{g_{j\bar{k}}}{z^i}=\pdv{g_{i\bar{k}}}{z^j}
  .\] On the other hand,
  \begin{align*}
    \dd{\gamma}&=\dd(\sqrt{-1}g_{i\bar{k}}\dd{z^i}\wedge\dd{\bar{z}^k}) \\
    &=\sqrt{-1}\left(\pdv{g_{i\bar{k}}}{z^j}\dd{z^j}\wedge\dd{z^i}\wedge\dd{\bar{z}^k}
    +\pdv{g_{i\bar{k}}}{\bar{z}^j}\dd{\bar{z}^j}\wedge\dd{z^i}\wedge
    \dd{\bar{z}^k}\right)
  .\end{align*}
  Thus \[
    \dd{\gamma}=0\iff \pdv{g_{i\bar{k}}}{z^j}=\pdv{g_{j\bar{k}}}{z^i}
  .\] Hence \[
    \nabla_X Y-\nabla_Y X=[X,Y]\iff \dd{\gamma}=0
  .\] 
\end{proof}

\end{document}
