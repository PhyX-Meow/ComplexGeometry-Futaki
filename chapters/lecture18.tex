% !TeX program = xelatex
\documentclass[12pt]{article}
\usepackage{standalone}

\usepackage[dvipsnames,svgnames,x11names]{xcolor}
\usepackage[a4paper,margin=1in]{geometry}
\usepackage{microtype}
\usepackage{amsmath}
\usepackage{amsthm}
\usepackage{mathtools}
\usepackage{mathrsfs}
\usepackage{stmaryrd}
\usepackage{extarrows}
\usepackage{enumerate}
\usepackage{tensor}
\usepackage{physics}
\usepackage{graphicx}
\usepackage{subcaption}
\usepackage{tikz}
\usepackage{tikz-3dplot}
% \usepackage{tikz-cd}
\usepackage{quiver}
  \tikzset{every picture/.style={line width=0.75pt}}
\usepackage{pgfplots}
  \pgfplotsset{compat=newest}
\usepackage{tcolorbox}
  \tcbuselibrary{most}
\usepackage[colorlinks=true,linkcolor=blue]{hyperref}
\usepackage{cleveref}
\usepackage[hyperref=true,backend=biber,style=alphabetic,backref=true,url=false]{biblatex}
\usepackage[warnings-off={mathtools-colon,mathtools-overbracket}]{unicode-math}
\usepackage[default]{fontsetup}
\usepackage{fancyhdr}
\usepackage{import}

\newcommand{\Id}{\mathbb{1}}
\newcommand{\lap}{\increment}

\DeclareMathOperator{\sign}{sign}
\DeclareMathOperator{\dom}{dom}
\DeclareMathOperator{\ran}{ran}
\DeclareMathOperator{\ord}{ord}
\DeclareMathOperator{\Span}{span}
\DeclareMathOperator{\img}{Im}
\DeclareMathOperator{\Ric}{Ric}
\newcommand{\card}{\texttt{\#}}
\newcommand{\ie}{\emph{i.e.}}
\newcommand{\st}{\emph{s.t.}}
\newcommand{\eps}{\varepsilon}
\newcommand{\vphi}{\varphi}
\newcommand{\vthe}{\vartheta}
\newcommand{\II}{I\!I}
\renewcommand{\emptyset}{\varnothing}
\newcommand{\acts}{\curvearrowright}
\newcommand{\xrr}{\xlongrightarrow}
\newcommand{\into}{\hookrightarrow}
\newcommand{\pdif}[2]{\frac{\partial #1}{\partial #2}}
\renewcommand{\op}{\operatorname}

\theoremstyle{plain}\newtheorem{theorem}{Theorem}
\theoremstyle{definition}\newtheorem{definition}[theorem]{Definition}
\theoremstyle{definition}\newtheorem{example}[theorem]{Example}
\theoremstyle{plain}\newtheorem{axiom}[theorem]{Axiom}
\theoremstyle{plain}\newtheorem{corollary}[theorem]{Corollary}
\theoremstyle{plain}\newtheorem{lemma}[theorem]{Lemma}
\theoremstyle{plain}\newtheorem{proposition}[theorem]{Proposition}
\theoremstyle{plain}\newtheorem{prop}[theorem]{Proposition}
\theoremstyle{plain}\newtheorem{conjecture}[theorem]{Conjecture}
\theoremstyle{plain}\newtheorem{conj}[theorem]{Conjecture}
\theoremstyle{plain}\newtheorem{problem}[theorem]{Problem}
\theoremstyle{remark}\newtheorem{notation}[theorem]{Notation}
\theoremstyle{definition}\newtheorem*{question}{Question}
\theoremstyle{definition}\newtheorem*{answer}{Answer}
\theoremstyle{definition}\newtheorem*{goal}{Goal}
\theoremstyle{plain}\newtheorem*{application}{Application}
\theoremstyle{plain}\newtheorem*{exercise}{Exercise}
\theoremstyle{remark}\newtheorem*{remark}{Remark}
\theoremstyle{remark}\newtheorem*{note}{\small{Note}}
\numberwithin{equation}{section}
\numberwithin{theorem}{section}
\numberwithin{figure}{section}

\usepackage{xeCJK}
\setCJKmainfont{FZShuSong-Z01}[BoldFont=FZXiaoBiaoSong-B05,ItalicFont=FZKai-Z03]
\setCJKsansfont{FZXiHeiI-Z08}[BoldFont=FZHei-B01]
\setCJKmonofont{FZFangSong-Z02}
\setCJKfamilyfont{zhsong}{FZShuSong-Z01}[BoldFont=FZXiaoBiaoSong-B05]
\setCJKfamilyfont{zhhei}{FZHei-B01}
\setCJKfamilyfont{zhkai}{FZKai-Z03}
\setCJKfamilyfont{zhfs}{FZFangSong-Z02}
\setCJKfamilyfont{zhli}{FZLiShu-S01}
\setCJKfamilyfont{zhyou}{FZXiYuan-M01}[BoldFont=FZZhunYuan-M02]

\geometry{a4paper,margin=1in}
\allowdisplaybreaks{}

\counterwithout{theorem}{section}
\counterwithout{equation}{section}

\begin{document}
\begin{theorem}
  Let \(L\to M\) be a holomorphic line bundle over a compact manifold with
  negative \(c_1(L)\), which means for a metric \(h\) on \(L\), its first
  Chern form \(-\frac{\sqrt{-1}}{2\pi}\partial \bar{\partial}h\) is negative
  definite. Then \(H^0(M,\mathcal{O}(L))=0\), \ie\ there is no non-zero 
  holomorphic section of \(L\).
\end{theorem}
\begin{proof}
  Let \(\nabla\) be the connection for \(L,L\otimes T^*M,\ldots\) whenever
  we want to take derivative. Let \(s\) be a holomorphic section of \(L\).
  We choose a local holomorphic frame \(e\) of \(L\), and use the same notation
  \(h\) for \(h(\bar{e},e)\). (This was already used in the statement of the
  theorem). We write \[
    -\partial\bar{\partial}\log h=r_{i\bar{j}}\dd{z^i}\wedge \dd{\bar{z}^j}
  .\] So that \((r_{\bar{j}})\) is negative definite. Write \(s=s^1 e\). Then
  note that \(\nabla\) makes \(h\) and \(\omega\) parallel,
  \begin{align*}
    0\le \int_{M}|\nabla s|^2\omega^m&=\int_{M}hg^{i\bar{j}}\nabla_i s^1
    \cdot \overline{\nabla_j s^1}\omega^m \\
    &=-\int_{M}h g^{i\bar{j}}(\nabla_{\bar{j}}\nabla_i s)\cdot \bar{s}\omega^m \\
    &=-\int_{M}hg^{i\bar{j}}(\nabla_i\underbracket{\nabla_{\bar{j}}s}_
    {\mathclap{0\text{ since }s\text{ holo}}}+r_{\bar{j}i}s)\bar{s}\omega^m \\
    &=\int_{M}g^{i\bar{j}}r_{i\bar{j}}|s|^2\omega^m<0
  .\end{align*}
  If \(s\neq 0\), this gives a contradiction.
\end{proof}

Kodaira vanishing theorem is a harmonic form version of the above theorem for
positive line bundles. Before doing it, we will consider the \(\bar{\partial}
\)-Laplacian \(\lap_{\bar{\partial}}\) in the case without line bundle.

Recall the curvature tensor on Kähler manifolds satisfy
\begin{gather*}
  R_{ijAB}=R_{ABij}=0, \\
  R_{i\bar{j}k\bar{l}}=-R_{\bar{j}ik\bar{l}}=R_{k\bar{l}i\bar{j}}
  =-R_{k\bar{l}\bar{j}i}
.\end{gather*} 
Also by the 1st Bianchi identity \[
  R_{i\bar{j}k\bar{l}}+\underbracket{R_{ik\bar{l}\bar{j}}}_{=0}
  +R_{i\bar{l}\bar{j}k}=0
.\] So this means it is symmetric in \(\bar{j}\) and \(\bar{l}\), and also in
\(i\) and \(k\). Recall also the Ricci identity is \[
  (\nabla_i \nabla_{\bar{j}}-\nabla_{\bar{j}}\nabla_i)\alpha_k 
  =-R\indices{_{i\bar{j}}^p_k}\alpha_p
.\] Similarly, for a tensor field \[
  T\indices{^i_{j\bar{k}}}\pd{z^i}\otimes \dd{z^j}\otimes \dd{\bar{z}^k}
,\] we have \[
  (\nabla_k\nabla_{\bar{l}}-\nabla_{\bar{l}}\nabla_k)T\indices{^i_{j\bar{k}}}
  =R\indices{_{k\bar{l}}^i_p}T\indices{^p_{j\bar{k}}}-R\indices{_{k\bar{l}}^p_j}
  T\indices{^i_{p\bar{k}}}-R\indices{_{k\bar{l}}^{\bar{p}}_{\bar{k}}}
  T\indices{^i_{j\bar{p}}}
.\] 
\begin{remark}
  In \emph{Morrow-Kodaira}, page 119, there is a similar formula. But it shows \[
    [\nabla_\lambda,\nabla_{\bar{\nu}}]\xi\indices{^\alpha_{\beta\bar{\gamma}}}
    =-\sum_{\tau}R_{\tau\bar{\lambda}\nu}^\alpha
    \xi\indices{^\tau_{\beta\bar{\gamma}}}
    +\sum_{\tau}R_{\beta\bar{\lambda}\nu}^\tau
    \xi\indices{^\alpha_{\tau\bar{\gamma}}}
    +\sum_{\tau}R_{\bar{\gamma}\bar{\lambda}\nu}^{\bar{\tau}}
    \xi\indices{^\alpha_{\beta\bar{\tau}}}
  .\] It appears that \(\bar{\lambda}\nu\) are typos, and for example \[
    R_{\beta\lambda\bar{\nu}}^\tau=R\indices{_\beta^\tau_{\lambda\bar{\nu}}}
    =-R\indices{^\tau_{\beta\lambda\bar{\nu}}}
  .\] As I wrote before, it is very bad to write in the form \[
    R_{\tau\lambda\bar{\nu}}^{\beta}
  \] because the sign is missing.

  In view of  Proposition 6.2 and Proposition 6.6, the sign convention of
  \emph{Morrow-Kodaira} is the same as these notes.
\end{remark}
Let \(\vphi=\frac{1}{p!q!}\vphi_{i_1\cdots i_p\bar{j}_1\cdots \bar{j}_q}
\dd{z}^{i_1}\wedge\cdots\wedge\dd{z^{i_p}}\wedge\dd{\bar{z}^{j_1}}\wedge\cdots
\wedge\dd{\bar{z}^{j_q}}\) be a type \((p,q)\)-form on a compact Kähler manifold.

\begin{theorem}[Morrow-Kodaira, Theorem 6.1]
  \begin{align*}
    (\lap_{\bar{\partial}}\vphi)_{i_1\cdots i_p\bar{j}_1\cdots\bar{j}_q}
    &=-g^{i\bar{j}}\nabla_i \nabla_{\bar{j}}\vphi_{i_1\cdots i_p\bar{j}_q\cdots
    \bar{j}_q}+\sum_{\beta=1}^q R\indices{^{\bar{k}}_{\bar{j}_\beta}}
    \vphi_{i_1\cdots i_p\bar{j}_1\cdots \bar{j}_{\beta-1}\bar{k}\bar{j}_{\beta+1}
    \cdots \bar{j}_q} \\
    &\phantom{=}+\sum_{\alpha=1}^{p}\sum_{\beta=1}^{q}R\indices{^k_{i_\alpha
    \bar{j}_\beta}^{\bar{l}}}\vphi_{i_1\cdots i_{\alpha-1}ki_{\alpha+1}\cdots
    i_p\bar{j}_1\cdots \bar{j}_{\beta-1}\bar{l}\bar{j}_{\beta+1}\cdots\bar{j}_q}
  .\end{align*}
\end{theorem}
\begin{proof}
  Since \(\Gamma_{\bar{i}\bar{j}}^{\bar{k}}=\Gamma_{\bar{j}\bar{i}}^{\bar{k}}\),
  using \(I=(i_1\cdots i_p)\), we have
  \begin{align*}
    (\bar{\partial}\vphi)_{I\bar{j}_0\cdots\bar{j}_q}&=(-1)^p\left(
    \pd{\bar{z}^{j_0}}\vphi_{I\bar{j}_1\cdots\bar{j}_q}-\pd{\bar{z}^{j_1}}
    \vphi_{I\bar{j}_0\bar{j}_2\cdots \bar{j}_p}+\cdots\right) \\
    &=(-1)^p\left(\nabla_{\bar{j}_0}\vphi_{I\bar{j}_1\cdots \bar{j}_q}-
    \nabla_{\bar{j}_1}\vphi_{I\bar{j}_0\bar{j}_2\cdots\bar{j}_q}+\cdots\right) \\
    &=(-1)^p \sum_{\beta=0}^{q}(-1)^\beta \nabla_{\bar{j}_\beta}
    \vphi_{I\bar{j}_0\cdots\widehat{\bar{j}_\beta}\cdots \bar{j}_q}
  .\end{align*}
  And then
  \begin{align*}
    (\bar{\partial}^*\bar{\partial}\vphi)_{I\bar{j}_1\cdots \bar{j}_q}
    &=-(-1)^p g^{k\bar{l}}\nabla_k (\bar{\partial}\vphi)_{I\bar{l}\bar{j}_1\cdots
    \bar{j}_q} \\
    &=g^{i\bar{j}}\nabla_i \nabla_{\bar{j}}\vphi_{I\bar{j}_q\cdots \bar{j}_q}
    +\sum_{\beta=1}^{q}(-1)^{\beta+1}g^{i\bar{j}}\nabla_i \nabla_{\bar{j}_\beta}
    \vphi_{I\bar{j}\bar{j}_q\cdots\widehat{\bar{j}_\beta}\cdots \bar{j}_q}
  .\end{align*}
  (The sign of \emph{M-K}, page 119, last is typo, \((-1)^{\lambda+1}\) should
  be \((-1)^\lambda\)).

  Also \[
    (\bar{\partial}^*\vphi)_{I\bar{j}_2\cdots \bar{j}_q}
    =-(-1)^q g^{i\bar{j}}\nabla_i \vphi_{I\bar{j}\bar{j}_2\cdots \bar{j}_q}
  .\] So \[
    (\bar{\partial}\bar{\partial}^*\vphi)_{I\bar{j}_1\cdots \bar{j}_q}
    =-\sum_{\beta=1}^{q}(-1)^{\beta+1}\nabla_{\bar{j}_\beta}g^{i\bar{j}}\nabla_i
    \vphi_{I\bar{j}\bar{j}_1\cdots\widehat{\bar{j}_\beta}\cdots \bar{j}_q}
  .\] Thus \[
    (\lap_{\bar{\partial}}\vphi)_{I\bar{j}_1\cdots \bar{j}_q}
    =-g^{i\bar{j}}\nabla_i \nabla_{\bar{j}}\vphi_{I\bar{j}_1\cdots \bar{j}_q}
    -\sum_{\beta=1}^{q}(-1)^\beta g^{i\bar{j}}[\nabla_i,\nabla_{\bar{j}_\beta}]
    \vphi_{I\bar{j}\bar{j}_1\cdots\widehat{\bar{j}_\beta}\cdots \bar{j}_q}
  .\] The first term is identical to the first term in desired result. The
  third term in the theorem is obtained by apply the Ricci identity to
  \(I\)-indices. So we apply the Ricci identity to \(\bar{j}\bar{j}_1\cdots
  \widehat{\bar{j}_\beta}\cdots \bar{j}_q\), and obtain \[
    +\sum_{\beta=1}^{q}(-1)^\beta g^{i\bar{j}}R\indices{_{i\bar{j}_\beta}^{
    \bar{p}}_{\bar{j}}}\vphi_{I\bar{p}\bar{j}_1\cdots\widehat{\bar{j}_\beta}
    \cdots \bar{j}_q}+\sum_{\beta=1}^{q}(-1)^\beta g^{i\bar{j}}
    \sum_{\gamma\neq\beta}R\indices{_{i\bar{j}_\beta}^{\bar{p}}_{\bar{j}_\gamma}}
    \vphi_{I\bar{j}\bar{j}_1\cdots\underset{\mathclap{\substack{\uparrow\\\gamma
    \text{-th}}}}{\bar{p}}\cdots\widehat{\bar{j}_\beta}\cdots \bar{j}_q}
  .\] Using the symmetry of curvature tensor, we see \[
    g^{i\bar{j}}R\indices{_{i\bar{j}_\beta}^{\bar{p}}_{\bar{j}}}
    =g^{i\bar{j}}R\indices{_{i\bar{j}}^{\bar{p}}_{\bar{j}_\beta}}
    =-g^{i\bar{j}}R\indices{_{\bar{j}i}^{\bar{p}}_{\bar{j}_\beta}}
    =-R\indices{^{\bar{p}}_{\bar{j}_\beta}}
  .\] So
  \begin{align*}
    \sum_{\beta=1}^{q}(-1)^\beta g^{i\bar{j}}R\indices{_{i\bar{j}_\beta}^{\bar{p}
    }_{\bar{j}}}
    \vphi_{I\bar{p}\bar{j}_1\cdots\widehat{\bar{j}_\beta}\cdots\bar{j}_q}
    &=-\sum_{\beta=1}^{q}(-1)^\beta R\indices{^{\bar{p}}_{\bar{j}_\beta}}
    \vphi_{I\bar{p}\bar{j}_1\cdots\widehat{\bar{j}_\beta}\cdots\bar{j}_q} \\
    &=\sum_{j=1}^{q}R\indices{^{\bar{p}}_{\bar{j}_\beta}}\vphi_{I\bar{j}_1\cdots
    \bar{j}_{\beta-1}\bar{p}\bar{j}_{\beta+1}\cdots \bar{j}_q}
  .\end{align*} 
  In \emph{Morrow-Kodaira}, the coefficient of this term is
  \(-R\indices{_{\bar{j}_\beta}^{\bar{p}}}\). But in view of Proposition 6.4,
  their Ricci \[
    R_{\bar{j}i}^{\text{M-K}}=\partial_i\partial_{\bar{j}}\lap\det g
    =-(-\partial_i\partial_{\bar{j}}\log\det g)=-R_{i\bar{j}}
  .\] So our computation coincide with theirs. (Note that the majority of
  Kähler geometers use \(R_{i\bar{j}}=-\partial_i\partial_{\bar{j}}
  \log\det g\)).

  Lastly, \[
    \sum_{\beta=1}^{q}(-1)^\beta g^{i\bar{j}}
    \sum_{\gamma\neq\beta}R\indices{_{i\bar{j}_\beta}^{\bar{p}}_{\bar{j}_\gamma}}
    \vphi_{I\bar{j}\bar{j}_1\cdots\underset{\mathclap{\substack{\uparrow\\\gamma
    \text{-th}}}}{\bar{p}}\cdots\widehat{\bar{j}_\beta}\cdots \bar{j}_q}
    =\sum_{\beta=1}^{q}(-1)^\beta \sum_{\gamma\neq \beta}R\indices{^{\bar{j}}
    _{\bar{j}_\beta}^{\bar{p}}_{\bar{j}_\gamma}}\vphi_{I\bar{j}\bar{j}_1\cdots
    \bar{p}\cdots\widehat{\bar{j}_\beta}\cdots \bar{j}_q}
  .\] But \(R\) is symmetric on 1st and 3rd indices, while \(\vphi\) is
  skew-symmetric. Thus this term is 0.

  All in all, we proved the theorem.
\end{proof}

\begin{remark}
  You may worry that \emph{Morrow-Kodaira} is too ambiguous about the sign
  of Ricci. But it does not affect Kodaira vanishing, as wee will see later.
\end{remark}

Let \(L\) be a holomorphic line bundle over a compact Kähler manifold \((M,g)\).
Take a Hermitian metric \(h\) and on an open set \(U\), we take a local
nowhere zero holomorphic section \(e\). We denote by the same letter \(h\) to
express function \(h(e,e)\). So the curvature with respect to \(e\) is \[
  \psi_{i\bar{j}}:=-\partial_i\partial_{\bar{j}}\log h
.\] Curvature 2-form of \((L,h)\) is \[
  -\sqrt{-1}\partial\bar{\partial}\log h
\] and this is an \(1\times 1\) matrix valued 2-form.

We can consider \(\bar{\partial}\)-Laplacian on \(L\)-valued \((p,q)\)-forms
using the Kähler metric \(g\) and the Hermitian metric \(h\). As in the previous
computations we see for \(\vphi\in \mathcal{A}^{p,q}(L)\), an \(L\)-valued
\((p,q)\)-form,
\begin{align*}
  (\lap_{\bar{\partial}}\vphi)_{i_1\cdots i_p\bar{j}_1\cdots \bar{j}_q}
  &=-g^{i\bar{j}}\nabla_i\nabla_{\bar{j}}\vphi_{i_1\cdots i_p\bar{j}_1\cdots
  \bar{j}_q} \\
  &\phantom{=}-\sum_{\beta=1}^q (-1)^\beta g^{i\bar{j}}[\nabla_i,\nabla_{\bar{j}
  _\beta}]\vphi_{I\bar{j}\bar{j}_1\cdots\widehat{\bar{j}_\beta}\cdots\bar{j}_q}\\
  &=-g^{i\bar{j}}\nabla_i\nabla_{\bar{j}}\vphi_{i_1\cdots i_p\bar{j}_1\cdots
  \bar{j}_q} \\ &\phantom{=}+\sum_{\alpha=1}^p
  \sum_{\beta=1}^q R\indices{^k_{j_\alpha\bar{j}_{\beta}}^{\bar{l}}}
  \vphi_{i_1\cdots i_{\alpha-1}k i_{\alpha+1}\cdots i_p\bar{j}_1\cdots
  \bar{j}_{\beta-1}\bar{l}\bar{j}_{\beta+1}\cdots \bar{j}_q} \\
  &\phantom{=}+\sum_{\beta=1}^{q}(R\indices{^{\bar{k}}_{\bar{j}_\beta}}
  +\psi\indices{^{\bar{k}}_{\bar{j}_\beta}})\vphi_{i_1\cdots i_p
  \bar{j}_1\cdots \bar{j}_{\beta-1}\bar{k}\bar{j}_{\beta+1}\cdots \bar{j}_q}
.\end{align*}
To show this we simply consider a local expression \[
  \vphi=\vphi_{i_1\cdots i_p\bar{j}_1\cdots \bar{j}_q}
  e\otimes\dd{z^{i_1}}\wedge\cdots\wedge\dd{z^{i_p}}\wedge\dd{\bar{z}^{j_1}}
  \wedge\cdots\wedge\dd{\bar{z}^{j_q}}
\] and use \[
  [\nabla_i,\nabla_{\bar{j}}]e=e\psi_{i\bar{j}}
.\] We apply the previous formula of \(\lap_{\bar{\partial}}\) for line bundle
\(L\).
\begin{theorem}
  Let \((L,h)\to (M,g)\) be a Hermitian line bundle over a compact Kähler
  manifold. Let \((\psi_{i\bar{j}})\) be the curvature of \((L,h)\) w.r.t.
  some local holomorphic frame and \((R_{i\bar{j}})\) be the Ricci curvature
  of \((M,g)\). If \((\psi_{i\bar{j}}+R_{i\bar{j}})\) is positive definite
  then \[
    H^q(M,\mathcal{O}(L))=0\quad \text{for }q\ge 1
  .\] 
\end{theorem}
\begin{proof}
  By Dolbeault theorem and Hodge-Kodaira theorem, \[
    H^q(M,\mathcal{O}(L))\cong \mathcal{H}^{0,q}_{\bar{\partial}}(M,L)
  \] where the right hand side is the space of \(\bar{\partial}\)-harmonic
  \((0,q)\)-forms. Let \(\vphi\in \mathcal{H}_{\bar{\partial}}^{0,q}(M,L)\),
  we may apply the previous formula of \(\lap_{\bar{\partial}}\vphi\) with
  \(I=\emptyset\). Thus  the second term on the right hand side is zero, and
  we obtain \[
    (\lap_{\bar{\partial}}\vphi)_{j_1\cdots j_q}
    =-g^{i\bar{j}}\nabla_i \nabla_{\bar{j}}\vphi_{\bar{j}_1\cdots \bar{j}_q}
    +\sum_{\beta=1}^q (R\indices{^{\bar{k}}_{\bar{j}_\beta}}
    +\psi\indices{^{\bar{k}}_{\bar{j}_\beta}})
    \vphi_{\bar{j}_1\cdots\underset{\mathclap{\substack{\uparrow\\\beta\text{-th}
    }}}{\bar{k}}\cdots \bar{j}_q}
  .\] From this, by integration by parts on the first term, we get
  \begin{align*}
    0=(\lap_{\bar{\partial}}\vphi,\vphi)_{L^2}&=|\bar{\partial}\vphi|^2
    +\sum_{\beta=1}^{q}\int_{M}g^{l\bar{k}g^{i_1\bar{j}_1}}\cdots
    g^{i_q\bar{j}_q}(\psi_{l\bar{j}_\beta}+R_{l\bar{j}_\beta})
    \vphi_{\bar{j}_1\cdots \bar{k}\cdots \bar{j}_q}\overline{\vphi_{\bar{i}_1
    \cdots\bar{i}_q}} \\
    &=|\bar{\partial}\vphi|^2+\sum_{\beta=1}^{q}\int_{M}g^{i_1\bar{j}_1}\cdots
    (\psi^{j_\beta\bar{k}}+R^{j_\beta\bar{k}})\cdots g^{i_q \bar{j}_q}
    \vphi_{\bar{j}_1\cdots\underset{\mathclap{\substack{\uparrow\\\beta\text{-th}
    }}}{\bar{k}}\cdots \bar{j}_q}\overline{\vphi_{\bar{i}_1\cdots\bar{j}_\beta
    \cdots\bar{i}_{q}}}
  .\end{align*} 
  If \(\vphi\neq 0\), the right hand side is strictly positive. This is a
  contradiction.
\end{proof}

\end{document}
