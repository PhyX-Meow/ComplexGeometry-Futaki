% !TeX program = xelatex
\documentclass[12pt]{article}
\usepackage{standalone}

\usepackage[dvipsnames,svgnames,x11names]{xcolor}
\usepackage[a4paper,margin=1in]{geometry}
\usepackage{microtype}
\usepackage{amsmath}
\usepackage{amsthm}
\usepackage{mathtools}
\usepackage{mathrsfs}
\usepackage{stmaryrd}
\usepackage{extarrows}
\usepackage{enumerate}
\usepackage{tensor}
\usepackage{physics2}
  \usephysicsmodule{ab,xmat}
\usepackage{fixdif}
  \newcommand{\dd}{\d}
\usepackage{derivative}
  \newcommand{\dv}{\odv}
  \newcommand{\pd}[1]{\pdv{}{#1}}
  \newcommand{\eval}[1]{#1\big|}
\usepackage{graphicx}
\usepackage{subcaption}
\usepackage{tikz}
\usepackage{tikz-3dplot}
% \usepackage{tikz-cd}
% \usepackage{quiver}
% ========== Block quiver.sty ========== %
\usepackage{tikz-cd}
% \usepackage{amssymb}
\usetikzlibrary{calc}
\usetikzlibrary{decorations.pathmorphing}
\tikzset{curve/.style={settings={#1},to path={(\tikztostart)
    .. controls ($(\tikztostart)!\pv{pos}!(\tikztotarget)!\pv{height}!270:(\tikztotarget)$)
    and ($(\tikztostart)!1-\pv{pos}!(\tikztotarget)!\pv{height}!270:(\tikztotarget)$)
    .. (\tikztotarget)\tikztonodes}},
    settings/.code={\tikzset{quiver/.cd,#1}
        \def\pv##1{\pgfkeysvalueof{/tikz/quiver/##1}}},
    quiver/.cd,pos/.initial=0.35,height/.initial=0}
\tikzset{tail reversed/.code={\pgfsetarrowsstart{tikzcd to}}}
\tikzset{2tail/.code={\pgfsetarrowsstart{Implies[reversed]}}}
\tikzset{2tail reversed/.code={\pgfsetarrowsstart{Implies}}}
\tikzset{no body/.style={/tikz/dash pattern=on 0 off 1mm}}
% =========== End block ========== %
  \tikzset{every picture/.style={line width=0.75pt}}
\usepackage{pgfplots}
  \pgfplotsset{compat=newest}
\usepackage{tcolorbox}
  \tcbuselibrary{most}
\usepackage[colorlinks=true,linkcolor=blue]{hyperref}
\usepackage{cleveref}
% \usepackage[hyperref=true,backend=biber,style=alphabetic,backref=true,url=false]{biblatex}
\usepackage[warnings-off={mathtools-colon,mathtools-overbracket}]{unicode-math}
\usepackage[default,amsbb]{fontsetup}
  \setmathfont[StylisticSet=1,range=\mathscr]{NewCMMath-Book.otf}
\usepackage{fancyhdr}
\usepackage{import}

\newcommand{\Id}{\mathbb{1}}
\newcommand{\lap}{\increment}

\DeclareMathOperator{\sign}{sign}
\DeclareMathOperator{\dom}{dom}
\DeclareMathOperator{\ran}{ran}
\DeclareMathOperator{\ord}{ord}
\DeclareMathOperator{\Span}{span}
\DeclareMathOperator{\img}{Im}
\DeclareMathOperator{\Ric}{Ric}
\newcommand{\card}{\texttt{\#}}
\newcommand{\ie}{\emph{i.e.}}
\newcommand{\st}{\emph{s.t.}}
\newcommand{\eps}{\varepsilon}
\newcommand{\vphi}{\varphi}
\newcommand{\vthe}{\vartheta}
\newcommand{\II}{I\!I}
\renewcommand{\emptyset}{⌀}
\newcommand{\acts}{\curvearrowright}
\newcommand{\xrr}{\xlongrightarrow}
\newcommand{\lrr}{\longrightarrow}
\newcommand{\lmt}{\longmapsto}
\newcommand{\into}{\hookrightarrow}
\newcommand{\op}{\operatorname}

\let\originalleft\left
\let\originalright\right
\renewcommand{\left}{\mathopen{}\mathclose\bgroup\originalleft}
\renewcommand{\right}{\aftergroup\egroup\originalright}

\theoremstyle{plain}\newtheorem{theorem}{Theorem}
\theoremstyle{definition}\newtheorem{definition}[theorem]{Definition}
\theoremstyle{definition}\newtheorem{example}[theorem]{Example}
\theoremstyle{definition}\newtheorem{problem}[theorem]{Problem}
\theoremstyle{plain}\newtheorem{axiom}[theorem]{Axiom}
\theoremstyle{plain}\newtheorem{corollary}[theorem]{Corollary}
\theoremstyle{plain}\newtheorem{lemma}[theorem]{Lemma}
\theoremstyle{plain}\newtheorem{proposition}[theorem]{Proposition}
\theoremstyle{plain}\newtheorem{prop}[theorem]{Proposition}
\theoremstyle{plain}\newtheorem{conjecture}[theorem]{Conjecture}
\theoremstyle{plain}\newtheorem{conj}[theorem]{Conjecture}
\theoremstyle{remark}\newtheorem{notation}[theorem]{Notation}
\theoremstyle{definition}\newtheorem*{question}{Question}
\theoremstyle{definition}\newtheorem*{answer}{Answer}
\theoremstyle{definition}\newtheorem*{goal}{Goal}
\theoremstyle{definition}\newtheorem*{application}{Application}
\theoremstyle{plain}\newtheorem*{exercise}{Exercise}
\theoremstyle{remark}\newtheorem*{remark}{Remark}
\theoremstyle{remark}\newtheorem*{note}{\small{Note}}
\numberwithin{equation}{section}
\numberwithin{theorem}{section}
\numberwithin{figure}{section}

\usepackage{xeCJK}
\setCJKmainfont{FZShuSong-Z01}[BoldFont=FZXiaoBiaoSong-B05,ItalicFont=FZKai-Z03]
\setCJKsansfont{FZXiHeiI-Z08}[BoldFont=FZHei-B01]
\setCJKmonofont{FZFangSong-Z02}
\setCJKfamilyfont{zhsong}{FZShuSong-Z01}[BoldFont=FZXiaoBiaoSong-B05]
\setCJKfamilyfont{zhhei}{FZHei-B01}
\setCJKfamilyfont{zhkai}{FZKai-Z03}
\setCJKfamilyfont{zhfs}{FZFangSong-Z02}
\setCJKfamilyfont{zhli}{FZLiShu-S01}
\setCJKfamilyfont{zhyou}{FZXiYuan-M01}[BoldFont=FZZhunYuan-M02]

\allowdisplaybreaks{}

\newcommand{\isFullBook}[2]{
  \ifnum\pdfstrcmp{\FullBook}{True}=0
    \ifnum\pdfstrcmp{}{#1}=0\unskip\else#1\fi
  \else
    \ifnum\pdfstrcmp{}{#2}=0\unskip\else#2\fi
  \fi\ignorespaces{}
}

\begin{document}

\section{Convention of vectors}

There are two traditions of vector notations:
\begin{itemize}
\item \textbf{Western Style.}\[
    \mathbb{y}=A\mathbb{x}\qquad \begin{pmatrix}
        y_1 \\ \vdots \\ y_n
    \end{pmatrix}=A\begin{pmatrix}
        x_1 \\ \vdots \\ x_n
    \end{pmatrix}\qquad
    \text{\CJKfamily{zhkai}清華園}
.\] 
\item \textbf{Eastern Style.}\[
    \mathbb{x}A=\mathbb{y}\qquad \begin{pmatrix}
        x_1 & \cdots & x_n
    \end{pmatrix}A=\begin{pmatrix}
        y_1 & \cdots & y_n
    \end{pmatrix}\qquad
    \text{\CJKfamily{zhkai}園華清}
.\] 
\end{itemize}
Throughout this Notes, we use the western style, and write vectors as column vectors.

Let \(V\) be a vector space over \(\mathbb{R}\) or \(\mathbb{C}\) of \(\dim=n\). Let
\(e_1,\ldots,e_n\) be a basis. Then any \(\mathbb{x}\in V\) is expressed as \[
    \mathbb{x}=x^1 e_1+\cdots +x^n e_n
.\] Since we want to identify \(V\cong \mathbb{R}^n (\mathbb{C}^n)\), we want to
identify \(\mathbb{x}\) with a column vector. So \[
    \mathbb{x}=(e_1,\ldots,e_n)\begin{pmatrix}
        x^1 \\ \vdots \\ x^n
    \end{pmatrix}
.\] Note we used upper indices  for the coefficients \(x^i\), and lower indices for
the basis \(e_i\). This is a standard notation in physics and differential geometry.

Vector \(\mathbb{x}\) is also expressed as \[
    \mathbb{x}=x^i e_i
\] by omitting \(\sum_{i=1}^{n}\). When upper and lower indices appear at the same
time, it is understood that the sum is taken. This convention is called the
\textbf{Einstein summation convection}.

Let \(e^1,\ldots,e^n\in V^*\) be the dual basis of \(e_1,\ldots,e_n\), so that \[
    \left<e^i,e_j\right> =\delta_j^i\quad\text{(Kronecker delta)}
.\] Then any \(\alpha\in V^*\) is expressed as \[
    \alpha=\alpha_i e^i\quad \Big(=\sum_{i=1}^{n}\alpha_i e^i\Big)
.\] Let \(f_1,\ldots,f_n\) be another basis of \(V\). Then \(\exists\,P\) non-singular
matrix \st\ \[
    (f_1,\ldots,f_n)=(e_1,\ldots,e_n)P
,\] \ie\ \[
    f_j=\sum_{i=1}^{n}P_j^i e_i\ \left(=P_j^i e_i=e_i p^i_j\right)
.\] Here \(P\) is considered as an endomorphism of \(V\), so \(P\in V\otimes V^*\),
\begin{align*}
    P&=P_j^i e_i\otimes e^j &&\Big(\sum\text{ is omitted}\Big) \\
    &=e_i P_j^i e^j &&(\otimes \text{ omitted})
.\end{align*}

Let \(T\colon V\to V\) be a linear map, \[
    T(e_j)=\sum_{i=1}^{n}a_j^i e_i=\sum_{i=1}^{n}e_i a_j^i
.\] Let \(A=(a_j^i)\), then \[
    T(e_1,\ldots,e_n)=(Te_1,\ldots,Te_n)=(e_1,\ldots,e_n)A
.\] \(A\) is called the matrix expressing \(T\) in terms of \(e_1,\ldots,e_n\). 

Let \(B\) be the matrix expressing \(T\) in terms of \(f_1,\ldots,f_n\), so that \[
    T(f_1,\ldots,f_n)=(Tf_1,\ldots,Tf_n)=(f_1,\ldots,f_n)B
.\] Then 
\begin{align*}
    T(f_1,\ldots,f_n)&=T((e_1,\ldots,e_n)P) \\
    &=T(e_1,\ldots,e_n)P \\
    &=(e_1,\ldots,e_n)AP
.\end{align*}
On the other hand, \[
    T(f_1,\ldots,f_n)=(f_1,\ldots,f_n)B=(e_1,\ldots,e_n)PB
.\] Thus \(AP=PB\), and \(\boxed{B=P^{-1}AP}\).

Following backward, if \(B=P^{-1}AP\) for a non-singular \(P\), then \(A\) and \(B\)
express the same \(T\) for two different bases. So we could conclude that if
\(B=P^{-1}AP\), then \(A\) and \(B\) are essentially the same, and express the same
endomorphism.

\section{Geometry of vector bundles}
Let \(M\) be a smooth manifold of \(\dim=n\).
\begin{definition}
    \(\pi\colon E\to M\) is said to be a real (or complex) vector bundle of rank \(r\)
    if
    \begin{enumerate}[(i)]
    \item \(E\) is a smooth manifold of dim \(n+r\) (or \(n+2r\));
    \item \(\pi\) is a smooth map and the rank of \(\dd{\pi}\) is \(n\) everywhere
        (maximal rank);
    \item There is an open covering \(\{U_\lambda\}_{\lambda\in \Lambda}\) of \(M\)
        with diffeomorphism \(\vphi_\lambda\colon \pi^{-1}(U_\lambda)\to U_\lambda
        \times \mathbb{R}^r\) (or \(\mathbb{C}^r\)), such that
        \begin{enumerate}[(a)]
        \item For the projection \(p_\lambda\colon U_\lambda\times \mathbb{R}^n\to 
            U_\lambda\), \(\pi=p_\lambda\circ \vphi_\lambda\). \ie\ 
            \[\begin{tikzcd}
                {\pi^{-1}(U_\lambda)} & {U_\lambda\times\mathbb{R}^r} \\
                {U_\lambda}
                \arrow["{\varphi_\lambda}", from=1-1, to=1-2]
                \arrow["\pi"', from=1-1, to=2-1]
                \arrow["{p_\lambda}", from=1-2, to=2-1]
            \end{tikzcd}\]
            commutes. Via \(\vphi_\lambda\), \(\pi^{-1}(U_\lambda)\) is identified with
            \(U_\lambda\times \mathbb{R}^r\).
        \item When \(U_\lambda\cap U_\mu\neq \emptyset\), \[
            \vphi_\lambda\circ \vphi_\mu^{-1}\colon (U_\lambda\cap U_\mu)\times
            \mathbb{R}^r\longrightarrow (U_\lambda\cap U_\mu)\times \mathbb{R}^r
        \] is expressed as \[
            \vphi_\lambda\circ \vphi_\mu^{-1}(p,\mathbb{x})=(p,\vphi_{\lambda\mu}(p)
            \mathbb{x})
        \] with \[
            \vphi_{\lambda\mu}\colon U_\lambda\cap U_\mu\longmapsto
            GL(r,\mathbb{R})
        .\] 
        \end{enumerate}
    \end{enumerate}
\end{definition}
\begin{remark}\hfill
\begin{enumerate}[1.]
\item \(\{\vphi_{\lambda\mu}\}_{\lambda,\mu\in \Lambda}\) are called the
    \textbf{transition functions}.
\item \(\pi^{-1}(p)\) is called the \textbf{fiber}, and has the structure of
    vector space since, if we give it a vector space structure by \[
        \left(\vphi_\lambda\Big|_{\pi^{-1}(p)}\right)^{-1}\colon p\times
        \mathbb{R}^r\longrightarrow \pi^{-1}(p)
    .\] Then  this is well-defined independent of \(\lambda\) by (iii) (b).
\item Replacing \(\mathbb{R}^r\) by \(\mathbb{C}^r\) and assuming
    \begin{enumerate}[(a)]
    \item \(M\) is a complex manifold of \(\dim_{\mathbb{C}}M=m\);
    \item \(E\) is also a complex manifold of \(\dim_{\mathbb{C}}=m+r\) ;
    \item All maps are holomorphic, including \[
        \vphi_{\lambda\mu}\colon U_\lambda\times U_\mu\longrightarrow
        GL(r,\mathbb{C})
    .\] 
    \end{enumerate}
    Then \(E\) is called a \textbf{holomorphic vector bundle}.
\item Typical example of vector bundles are tangent bundle, cotangent bundle
    and their tensor products. If \(U_\lambda,x^1,\ldots,x^n\) and \(U_\mu,y^1,\ldots
    y^n\) are two coordinate neighbourhoods and \(U_\lambda\cap U_\mu\neq \emptyset\),
    write \[
        \pd{y^j}=\pdv{x^i}{y^j}\pd{x^i},\quad
        X^i\pd{x^i}=Y^j\pd{y^j}
    ,\] then \[
        X^i\pd{x^i}=Y^j\pdv{x^i}{y^j}\pd{x^i}
    .\] Hence \(X^i=\displaystyle\pdv{x^i}{y^j}Y^j\). And \(\vphi_{\lambda\mu}=\left(
    \displaystyle\pdv{x^i}{y^j}\right)\) is the transition function of \(TM\).
\item When the rank \(r=1\), \(E\) is called a \textbf{line bundle},
    often denoted by \(L\).
\item Let \(\pi\colon E\to M\) be a vector bundle and \(f\colon N\to M\) be a smooth 
    map. Then the pull-back bundle \(f^*E\) is \[
        f^*E=\{(x,v)\in N\times E:v\in \pi^{-1}(f(x))\}
    .\] 
\item \(E\) is called a trivial bundle if it is a pull-back bundle of of the product
    bundle \(M\times \mathbb{R}^r\) (or \(M\times \mathbb{C}^r\)).
\end{enumerate}
\end{remark}

\end{document}
