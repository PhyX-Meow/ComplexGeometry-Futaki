Further remarks on Riemannian curvature:
\begin{enumerate}[1.]
\item If there is a constant  \(\lambda\in \mathbb{R}\) such that
  \begin{equation*}
    R_{ij}=\lambda g_{ij} \tag{\ast}
  .\end{equation*}
  Then the Riemannian metric \(g\) is called an Einstein metric.

  Notice that the curvature tensor \(R\) and Ricci tensor \(\Ric\) are expressed by
  second derivative of the metric \(g\).\ (\ast) is a differential equation of \(g\),
  called the Einstein (field) equation. This is an equation satisfied by the
    gravity (with Lorentzian metric).
\item Ricci curvature \(R_{ij}\) and the Riemannian metric are both symmetric tensors.
  Let \(g(t)\) be a smooth family of Riemannian metric with parameter \(t\in
  \mathbb{R}\). \(\{g(t)\}_{t\in \mathbb{R}}\) is called the Ricci flow if it
  satisfies \[
    \pdv{g}{t}=-2\Ric(g(t))
  ,\] where \(\Ric(g(t))\) is the Ricci curvature of the metric \(g(t)\).

  Ricci flow was used to prove Poincaré conjecture.
\item For compact surface \(M\subset \mathbb{R}^3\), the Gaussian curvature coincide
  with \(R(e_1,e_2,e_1,e_2)\). I leave this fundamental fact to other lectures, or
  your own study.

  A very important theorem on the Gaussian curvature is that its integral over \(M\)
  is equal to \(2\pi\) times the Euler characteristic \(\chi(M)\). \[
    \frac{1}{2\pi}\int_{M}K\dd{V_g}=\chi(M).
    \quad\text{\underline{Gauss-Bonnet theorem}}
  \] Where \(K\) is the Gaussian curvature and \(\dd{V_g}\) is the Riemannian volume
  form \[
    \dd{V_g}=\sqrt{\det(g_{ij})}\dd{x^1}\wedge \dd{x^2}
  .\] The de Rham class \(\left[\frac{1}{2\pi}K\dd{V_g}\right]\) is an invariant
  of \(M\), called the Euler class.
\item For complex vector bundle \(E\) there are Chern classes \(c_1(E),\ldots,
  c_r(E)\). For the tangent bundle \(TM\) of a compact complex manifold \(M\), the 
  first Chern class \(c_1(TM)\) is expressed in terms of the Ricci curvature.
  So, the Ricci curvature knows the topology of \(M\).
\end{enumerate}
We will study 1\(\sim\)4 in more detail in following chapters.
