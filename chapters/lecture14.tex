% !TeX program = xelatex
\documentclass[12pt]{article}
\usepackage{standalone}

\usepackage[dvipsnames,svgnames,x11names]{xcolor}
\usepackage[a4paper,margin=1in]{geometry}
\usepackage{microtype}
\usepackage{amsmath}
\usepackage{amsthm}
\usepackage{mathtools}
\usepackage{mathrsfs}
\usepackage{stmaryrd}
\usepackage{extarrows}
\usepackage{enumerate}
\usepackage{tensor}
\usepackage{physics2}
  \usephysicsmodule{ab,xmat}
\usepackage{fixdif}
  \newcommand{\dd}{\d}
\usepackage{derivative}
  \newcommand{\dv}{\odv}
  \newcommand{\pd}[1]{\pdv{}{#1}}
  \newcommand{\eval}[1]{#1\big|}
\usepackage{graphicx}
\usepackage{subcaption}
\usepackage{tikz}
\usepackage{tikz-3dplot}
% \usepackage{tikz-cd}
% \usepackage{quiver}
% ========== Block quiver.sty ========== %
\usepackage{tikz-cd}
% \usepackage{amssymb}
\usetikzlibrary{calc}
\usetikzlibrary{decorations.pathmorphing}
\tikzset{curve/.style={settings={#1},to path={(\tikztostart)
    .. controls ($(\tikztostart)!\pv{pos}!(\tikztotarget)!\pv{height}!270:(\tikztotarget)$)
    and ($(\tikztostart)!1-\pv{pos}!(\tikztotarget)!\pv{height}!270:(\tikztotarget)$)
    .. (\tikztotarget)\tikztonodes}},
    settings/.code={\tikzset{quiver/.cd,#1}
        \def\pv##1{\pgfkeysvalueof{/tikz/quiver/##1}}},
    quiver/.cd,pos/.initial=0.35,height/.initial=0}
\tikzset{tail reversed/.code={\pgfsetarrowsstart{tikzcd to}}}
\tikzset{2tail/.code={\pgfsetarrowsstart{Implies[reversed]}}}
\tikzset{2tail reversed/.code={\pgfsetarrowsstart{Implies}}}
\tikzset{no body/.style={/tikz/dash pattern=on 0 off 1mm}}
% =========== End block ========== %
  \tikzset{every picture/.style={line width=0.75pt}}
\usepackage{pgfplots}
  \pgfplotsset{compat=newest}
\usepackage{tcolorbox}
  \tcbuselibrary{most}
\usepackage[colorlinks=true,linkcolor=blue]{hyperref}
\usepackage{cleveref}
% \usepackage[hyperref=true,backend=biber,style=alphabetic,backref=true,url=false]{biblatex}
\usepackage[warnings-off={mathtools-colon,mathtools-overbracket}]{unicode-math}
\usepackage[default,amsbb]{fontsetup}
  \setmathfont[StylisticSet=1,range=\mathscr]{NewCMMath-Book.otf}
\usepackage{fancyhdr}
\usepackage{import}

\newcommand{\Id}{\mathbb{1}}
\newcommand{\lap}{\increment}

\DeclareMathOperator{\sign}{sign}
\DeclareMathOperator{\dom}{dom}
\DeclareMathOperator{\ran}{ran}
\DeclareMathOperator{\ord}{ord}
\DeclareMathOperator{\Span}{span}
\DeclareMathOperator{\img}{Im}
\DeclareMathOperator{\Ric}{Ric}
\newcommand{\card}{\texttt{\#}}
\newcommand{\ie}{\emph{i.e.}}
\newcommand{\st}{\emph{s.t.}}
\newcommand{\eps}{\varepsilon}
\newcommand{\vphi}{\varphi}
\newcommand{\vthe}{\vartheta}
\newcommand{\II}{I\!I}
\renewcommand{\emptyset}{⌀}
\newcommand{\acts}{\curvearrowright}
\newcommand{\xrr}{\xlongrightarrow}
\newcommand{\lrr}{\longrightarrow}
\newcommand{\lmt}{\longmapsto}
\newcommand{\into}{\hookrightarrow}
\newcommand{\op}{\operatorname}

\let\originalleft\left
\let\originalright\right
\renewcommand{\left}{\mathopen{}\mathclose\bgroup\originalleft}
\renewcommand{\right}{\aftergroup\egroup\originalright}

\theoremstyle{plain}\newtheorem{theorem}{Theorem}
\theoremstyle{definition}\newtheorem{definition}[theorem]{Definition}
\theoremstyle{definition}\newtheorem{example}[theorem]{Example}
\theoremstyle{definition}\newtheorem{problem}[theorem]{Problem}
\theoremstyle{plain}\newtheorem{axiom}[theorem]{Axiom}
\theoremstyle{plain}\newtheorem{corollary}[theorem]{Corollary}
\theoremstyle{plain}\newtheorem{lemma}[theorem]{Lemma}
\theoremstyle{plain}\newtheorem{proposition}[theorem]{Proposition}
\theoremstyle{plain}\newtheorem{prop}[theorem]{Proposition}
\theoremstyle{plain}\newtheorem{conjecture}[theorem]{Conjecture}
\theoremstyle{plain}\newtheorem{conj}[theorem]{Conjecture}
\theoremstyle{remark}\newtheorem{notation}[theorem]{Notation}
\theoremstyle{definition}\newtheorem*{question}{Question}
\theoremstyle{definition}\newtheorem*{answer}{Answer}
\theoremstyle{definition}\newtheorem*{goal}{Goal}
\theoremstyle{definition}\newtheorem*{application}{Application}
\theoremstyle{plain}\newtheorem*{exercise}{Exercise}
\theoremstyle{remark}\newtheorem*{remark}{Remark}
\theoremstyle{remark}\newtheorem*{note}{\small{Note}}
\numberwithin{equation}{section}
\numberwithin{theorem}{section}
\numberwithin{figure}{section}

\usepackage{xeCJK}
\setCJKmainfont{FZShuSong-Z01}[BoldFont=FZXiaoBiaoSong-B05,ItalicFont=FZKai-Z03]
\setCJKsansfont{FZXiHeiI-Z08}[BoldFont=FZHei-B01]
\setCJKmonofont{FZFangSong-Z02}
\setCJKfamilyfont{zhsong}{FZShuSong-Z01}[BoldFont=FZXiaoBiaoSong-B05]
\setCJKfamilyfont{zhhei}{FZHei-B01}
\setCJKfamilyfont{zhkai}{FZKai-Z03}
\setCJKfamilyfont{zhfs}{FZFangSong-Z02}
\setCJKfamilyfont{zhli}{FZLiShu-S01}
\setCJKfamilyfont{zhyou}{FZXiYuan-M01}[BoldFont=FZZhunYuan-M02]

\allowdisplaybreaks{}

\newcommand{\isFullBook}[2]{
  \ifnum\pdfstrcmp{\FullBook}{True}=0
    \ifnum\pdfstrcmp{}{#1}=0\unskip\else#1\fi
  \else
    \ifnum\pdfstrcmp{}{#2}=0\unskip\else#2\fi
  \fi\ignorespaces{}
}

\counterwithout{theorem}{section}
\counterwithout{equation}{section}

\begin{document}

We apply the above theorem to prove the de Rham theorem and Dolbeault theorem.
\begin{theorem}[de Rham]
  Let \(X\) be a real manifold of \(\dim m\). Then \[
    H^p(X,\mathbb{R})\cong H_{\mathrm{dR}}^p(X)
  .\] Where the left hand side is the cohomology of the constant sheaf \(\mathbb{R}\)
  and the right hand side is the de Rham cohomology.
\end{theorem}
\begin{proof}
  Let \(\mathcal{A}^p\) be the sheaf of genus of smooth \(p\)-forms on \(X\), which
  is a fine sheaf as we saw. By the Poincaré lemma, we have the fine resolution
  of the constant sheaf \(\mathbb{R}\): \[
    0\lrr \mathbb{R}\xrr{\iota} \mathcal{A}^0\xrr{\dd} \mathcal{A}^1\lrr \cdots
    \lrr \mathcal{A}^n \lrr 0
  .\] Then by the above theorem, \[
    H^p(X,\mathbb{R})\cong \frac{\ker\left(\dd\colon\Gamma(X,\mathcal{A}^p)\to 
    \Gamma(X,\mathcal{A}^{p+1})\right)}{\img\left(\dd\colon\Gamma(X,\mathcal{A}^{p-1})
    \to \Gamma(X,\mathcal{A}^{p})\right)}=H_{\mathrm{dR}}^{p}(M)
  .\] 
\end{proof}

In \emph{Frank Warner}'s text book, isomorphisms of de Rham cohomology with other
classical cohomology theories such Alexander-Spanier, singular, differentiable
singular, Cěch cohomology through fine resolutions of the constant sheaf. I recommend
you read it when you have time.

Combining the Hodge theory, for compact orientable manifolds, these cohomology
groups in dimension \(p\) are all isomorphic to the space \(\mathcal{H}_{\dd}^p(M)\)
of harmonic \(p\)-forms.

For holomorphic bundle \(E\to M\), there is a parallel story. The sheaf \(\mathcal{A}
^{p,q}\) of germs of smooth \((p,q)\)-forms with value in \(E\) are fine sheaves.
(The germs are those of local restrictions of \(E\otimes \bigwedge^{p,q}(M)\).)

By the Dolbeault lemma (exercise), we have a fine resolution of the sheaf \(\Omega^p
(E)\) of germs of holomorphic sections of \(E\). \[
  0\lrr \Omega^p(E)\xrr{\iota} \mathcal{A}^{p,0}(E)\xrr{\bar{\partial}}
  \mathcal{A}^{p,1}(E)\xrr{\bar{\partial}}\cdots \xrr{\bar{\partial}}
  \mathcal{A}^{p,m}(E)\lrr 0
.\] Where \(m=\dim_{\mathbb{C}}M\). Using
\ifdefined\FullBook{}
  \cref{thm:7-1:fine},
\else
  the theorem in last note,
\fi
we obtain
\begin{theorem}[Dolbeault]
  \[
    H^{q}(M,\Omega^p(E))\cong \frac{\ker\left(\bar{\partial}\colon \Gamma(M,\mathcal{A}
    ^{p,q}(E))\to \Gamma(M,\mathcal{A}^{p,q+1}(E))\right)}{\img\left(\bar{\partial}
    \colon\Gamma(M,\mathcal{A}^{p,q-1}(E))\to \Gamma(M,\mathcal{A}^{p,q}(E))\right)}
    =H^{p,q}_{\bar{\partial}}(M,E)
  .\] 
\end{theorem}

For compact complex manifold \(M\), this is isomorphic to the space \(\mathcal{H}^{p,q}
_{\bar{\partial}}(M,E)\) of harmonic \((p,q)\)-forms with values in \(E\).

\section{Hodge identities on K\"ahler manifolds}

We continue to study fundamental K\"ahler geometry.
\begin{prop}
  Let \((M,g)\) be a K\"ahler manifold and \(\gamma\) be its K\"ahler form. Then at
  any point \(p\in M\), there is a neighborhood \(U\) of \(p\) and a real smooth
  function \(f\) on \(U\) such that \[
    \gamma=\sqrt{-1}\partial\bar{\partial}f
  .\] This \(f\) is called the \textbf{K\"ahler potential} on \(U\).
\end{prop}
\begin{proof}
  Since the K\"ahler form is a real closed 2-form, by Poincaré lemma, there is a
  neighborhood \(U\) of \(p\) and a real smooth 1-form \(\psi\) such that \[
    \gamma=\dd{\psi}
  .\] Since \(\psi\) is a real 1-form there is a smooth type \((1,0)\)-form \(\vphi\)
  such that \(\psi=\vphi+\bar{\vphi}\). Thus \[
    \gamma=\dd{\psi}=(\partial +\bar{\partial})(\vphi+\bar{\vphi})=
    \partial\vphi+(\bar{\partial}\vphi+\partial\bar{\vphi})+\bar{\partial}\bar{\vphi}
  .\] Since \(\gamma\) is type \((1,1)\) we have \[
    \partial\vphi=0,\quad\bar{\partial}\bar{\vphi}=0
  .\] By Dolbeault lemma, there is a smooth complex valued function \(\eta\)
  such that \[
    \bar{\vphi}=\bar{\partial}\bar{\eta}\text{ and hence }\vphi=\partial\eta
  .\] Therefore \[
    \gamma=\bar{\partial}\vphi+\partial\bar{\vphi}=\bar{\partial}\partial\eta
    +\partial\bar{\partial}\bar{\eta}=\partial\bar{\partial}(-\eta+\bar{\eta})
    =\sqrt{-1}\partial\bar{\partial}f
  .\] By putting \(f=\sqrt{-1}(\eta-\bar{\eta})\), which is real-valued.
\end{proof}
\begin{remark}
  In this situation \[
    g_{i\bar{j}}=\pdv{f}{z^i,\bar{z}^j}
  .\]
\end{remark}

\begin{definition}
  Let \(p\in M\), local holomorphic coordinates \(z^1,\ldots,z^m\) around \(p\) is
  said to be \textbf{adapted coordinates} if
  \begin{enumerate}[(i)]
  \item \(z^1(p)=\cdots =z^m(p)=0\),
  \item \(g_{i\bar{j}}(p)=\delta_{ij}\),
  \item \(\Gamma_{jk}^i(p)=0\).
  \end{enumerate}
\end{definition}
\begin{remark}
  This is similar to ``normal coordinates'' in Riemannian geometry. But normal
  coordinates are no in general holomorphic coordinates.
\end{remark}
\begin{prop}
  A Hermitian manifold (\ie\ complex manifold endowed with a Hermitian metric on
  \(T'M\)) is a K\"ahler manifold if and only if there are adapted coordinate
  around any point \(p\in M\).
\end{prop}
\begin{proof}
  ``If'' part: If  (iii) is satisfied for Hermitian connection the connection is
  symmetric, \(\Gamma_{jk}^i=\Gamma_{kj}^i\), so it is K\"ahler.

  ``Only if'' part: Take any local holomorphic coordinates \(z^1,\ldots,z^m\)
  satisfying (i),(ii). This is possible by Schmidt's method. We consider change
  of coordinates \[
    z^j=w^j+\frac{1}{2}c_{ik}^j w^i w^k,\quad c_{ik}^j=c_{ki}^j
  .\] Here \(c_{ik}^j\) are constants to be determined later. \\
  For \(h_{i\bar{j}}=g(\pd{w^i},\pd{w^j})\), we have expression \[
    h_{i\bar{j}}=\pdv{z^p}{w^i}g_{p\bar{q}}\pdv{\bar{z}^q}{\bar{w}^i}
  \] where \(g_{p\bar{q}}=g(\pd{z^p},\pd{\bar{z}^q})\). \\
  From this, \[
    \pdv{h_{i\bar{j}}}{w^k}=\pdv{z^p}{w^k,w^i}g_{p\bar{q}}\pdv{\bar{z}^q}{\bar{w}^j}
    +\pdv{z^p}{w^i}\pdv{g_{p\bar{q}}}{z^r}\pdv{z^r}{w^k}\pdv{\bar{z}^q}{\bar{w}^j}
  .\] Evaluating the right hand side at \(p\) we get \[
    \text{RHS}\big|_{p}=c_{ik}^j+\pdv{g_{i\bar{j}}}{z^k}\bigg|_{p}
  .\] So we only need to take \[
    c_{ik}^j=-\pdv{g_{i\bar{j}}}{z^k}\bigg|_p
  .\] By the K\"ahler condition, \(\pdv{g_{i\bar{j}}}{z^k}=\pdv{g_{k\bar{j}}}{z^i}\),
  so \(c_{ik^j}=c_{kj}^i\).
\end{proof}

Recall that \(\mathcal{A}^{p,q}=\Gamma(M,\wedge^p T^{*\prime}M\otimes \wedge^q
T^{*\prime\prime}M)\), the set of all smooth \((p,q)\)-forms.

\begin{definition}
  We define two operators,
  \begin{align*}
    L\colon \mathcal{A}^{p,q}&\lrr\mathcal{A}^{p+1,q+1} \\
    \alpha &\longmapsto \alpha\wedge \gamma\ (=\gamma\wedge \alpha)
  .\end{align*}
  \begin{align*}
    \Lambda\colon \mathcal{A}^{p,q} &\longrightarrow \mathcal{A}^{p-1,q-1} \\
    \left<L\alpha,\beta\right> &=\left<\alpha,\Lambda\beta\right> \text{ pointwise}
  .\end{align*}
  Here \(\left<\cdot,\cdot\right> \) is the inner product by the K\"ahler metric. If
  \begin{align*}
    \alpha &=\sum_{\substack{i_1<\cdots<i_p\\ j_1<\cdots<j_q}}
    \alpha_{i_1,\ldots,i_p,\bar{j}_1,\ldots,\bar{j}_q}\dd{z^{i_1}}\wedge\cdots 
    \wedge\dd{z^{i_p}}\wedge\dd{\bar{z}^{j_1}}\wedge\cdots\wedge\dd{\bar{z}^{j_q}}, \\
    \beta &=\sum_{\substack{i_1<\cdots<i_p\\ j_1<\cdots<j_q}}
    \beta_{i_1,\ldots,i_p,\bar{j}_1,\ldots,\bar{j}_q}\dd{z^{i_1}}\wedge\cdots 
    \wedge\dd{z^{i_p}}\wedge\dd{\bar{z}^{j_1}}\wedge\cdots\wedge\dd{\bar{z}^{j_q}}
  \end{align*}
  for adapted coordinates, then \[
    \left<\alpha,\beta\right> =\sum_{\substack{i_1<\cdots <i_p\\ j_1<\cdots <j_q}}
    \alpha_{i_1,\ldots,i_p,\bar{j}_1,\ldots,\bar{j}_q}\overline{\beta_{i_1,\ldots,i_p,
    \bar{j}_1,\ldots,\bar{j}_q}}
  .\] 
\end{definition}

\begin{theorem}[Hodge identities]
  On K\"ahler manifolds, we have
  \begin{enumerate}[(i)]
    \item \([L,\partial]=[L,\bar{\partial}]=0\), \([\Lambda,\partial^*]=
      [\Lambda,\bar{\partial}^*]=0\).
    \item \([L,\partial^*]=\sqrt{-1}\bar{\partial}\), \([L,\bar{\partial}^*]=-\sqrt{-1}
      \partial\).
    \item \([\Lambda,\partial]=\sqrt{-1}\bar{\partial}^*\), \([\Lambda,\bar{\partial}]
      =-\sqrt{-1}\partial^*\).
  \end{enumerate}
\end{theorem}
\begin{proof}
  Left as exercise. See \emph{Griffiths-Harris}, page 111-114, \emph{Morrow-Kodaira},
  Proposition 5.4.
\end{proof}
\begin{remark}
  \(\Lambda\) is just the contraction by \(\gamma\), thus \(\Lambda\beta=0\) if
  \(\beta\) is a \((p,0)\)-form.
\end{remark}
\begin{corollary}
  Let \(M\) be a compact K\"ahler manifold. If \(\alpha\) is a holomorphic form then
  \(\dd{\alpha}=0\).
\end{corollary}
\begin{proof}
  Let \(\alpha\) be a holomorphic \(p\)-form, then \(\partial\alpha\) is also a
  holomorphic form. Thus \[
    \bar{\partial}\partial\alpha=0\text{ and }\Lambda\partial\alpha=0
  .\] It follows that \[
    (\partial\alpha,\partial\alpha)_{L^2}=(\alpha,\partial^*\partial\alpha)_{L^2}
    =-\sqrt{-1}(\alpha,(\Lambda\bar{\partial}-\bar{\partial}\Lambda)\partial\alpha)=0
  .\] Hence \(\dd{\alpha}=\partial\alpha=0\).
\end{proof}


\end{document}
