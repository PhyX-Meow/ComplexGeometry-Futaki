% !TeX program = xelatex
\documentclass[12pt]{article}
\usepackage{standalone}

\usepackage[dvipsnames,svgnames,x11names]{xcolor}
\usepackage[a4paper,margin=1in]{geometry}
\usepackage{microtype}
\usepackage{amsmath}
\usepackage{amsthm}
\usepackage{mathtools}
\usepackage{mathrsfs}
\usepackage{stmaryrd}
\usepackage{extarrows}
\usepackage{enumerate}
\usepackage{tensor}
\usepackage{physics}
\usepackage{graphicx}
\usepackage{subcaption}
\usepackage{tikz}
\usepackage{tikz-3dplot}
% \usepackage{tikz-cd}
\usepackage{quiver}
  \tikzset{every picture/.style={line width=0.75pt}}
\usepackage{pgfplots}
  \pgfplotsset{compat=newest}
\usepackage{tcolorbox}
  \tcbuselibrary{most}
\usepackage[colorlinks=true,linkcolor=blue]{hyperref}
\usepackage{cleveref}
\usepackage[hyperref=true,backend=biber,style=alphabetic,backref=true,url=false]{biblatex}
\usepackage[warnings-off={mathtools-colon,mathtools-overbracket}]{unicode-math}
\usepackage[default]{fontsetup}
\usepackage{fancyhdr}
\usepackage{import}

\newcommand{\Id}{\mathbb{1}}
\newcommand{\lap}{\increment}

\DeclareMathOperator{\sign}{sign}
\DeclareMathOperator{\dom}{dom}
\DeclareMathOperator{\ran}{ran}
\DeclareMathOperator{\ord}{ord}
\DeclareMathOperator{\Span}{span}
\DeclareMathOperator{\img}{Im}
\DeclareMathOperator{\Ric}{Ric}
\newcommand{\card}{\texttt{\#}}
\newcommand{\ie}{\emph{i.e.}}
\newcommand{\st}{\emph{s.t.}}
\newcommand{\eps}{\varepsilon}
\newcommand{\vphi}{\varphi}
\newcommand{\vthe}{\vartheta}
\newcommand{\II}{I\!I}
\renewcommand{\emptyset}{\varnothing}
\newcommand{\acts}{\curvearrowright}
\newcommand{\xrr}{\xlongrightarrow}
\newcommand{\into}{\hookrightarrow}
\newcommand{\pdif}[2]{\frac{\partial #1}{\partial #2}}
\renewcommand{\op}{\operatorname}

\theoremstyle{plain}\newtheorem{theorem}{Theorem}
\theoremstyle{definition}\newtheorem{definition}[theorem]{Definition}
\theoremstyle{definition}\newtheorem{example}[theorem]{Example}
\theoremstyle{plain}\newtheorem{axiom}[theorem]{Axiom}
\theoremstyle{plain}\newtheorem{corollary}[theorem]{Corollary}
\theoremstyle{plain}\newtheorem{lemma}[theorem]{Lemma}
\theoremstyle{plain}\newtheorem{proposition}[theorem]{Proposition}
\theoremstyle{plain}\newtheorem{prop}[theorem]{Proposition}
\theoremstyle{plain}\newtheorem{conjecture}[theorem]{Conjecture}
\theoremstyle{plain}\newtheorem{conj}[theorem]{Conjecture}
\theoremstyle{plain}\newtheorem{problem}[theorem]{Problem}
\theoremstyle{remark}\newtheorem{notation}[theorem]{Notation}
\theoremstyle{definition}\newtheorem*{question}{Question}
\theoremstyle{definition}\newtheorem*{answer}{Answer}
\theoremstyle{definition}\newtheorem*{goal}{Goal}
\theoremstyle{plain}\newtheorem*{application}{Application}
\theoremstyle{plain}\newtheorem*{exercise}{Exercise}
\theoremstyle{remark}\newtheorem*{remark}{Remark}
\theoremstyle{remark}\newtheorem*{note}{\small{Note}}
\numberwithin{equation}{section}
\numberwithin{theorem}{section}
\numberwithin{figure}{section}

\usepackage{xeCJK}
\setCJKmainfont{FZShuSong-Z01}[BoldFont=FZXiaoBiaoSong-B05,ItalicFont=FZKai-Z03]
\setCJKsansfont{FZXiHeiI-Z08}[BoldFont=FZHei-B01]
\setCJKmonofont{FZFangSong-Z02}
\setCJKfamilyfont{zhsong}{FZShuSong-Z01}[BoldFont=FZXiaoBiaoSong-B05]
\setCJKfamilyfont{zhhei}{FZHei-B01}
\setCJKfamilyfont{zhkai}{FZKai-Z03}
\setCJKfamilyfont{zhfs}{FZFangSong-Z02}
\setCJKfamilyfont{zhli}{FZLiShu-S01}
\setCJKfamilyfont{zhyou}{FZXiYuan-M01}[BoldFont=FZZhunYuan-M02]

\geometry{a4paper,margin=1in}
\allowdisplaybreaks{}

\counterwithout{theorem}{section}
\counterwithout{equation}{section}

\begin{document}
\begin{definition}
  We call \(h\in C^\infty(M,\bar{E}^*\otimes E^*)\) a Hermitian metric on \(E\) iff \[
    h:\bar{E}_p\times E_p\longrightarrow \mathbb{C}
  \] is a positive definite Hermitian inner product at any \(p\).
  Equivalently, for \(e_1,\ldots,e_r\) a local holomorphic frame, \[
    \big(h(\bar{e}_i,e_j)\big)_{r\times r}
  \] is a positive definite Hermitian matrix. We write \[
    h_{\bar{i}j}=h(\bar{e}_i,e_j)
  .\] 
\end{definition}

\begin{definition}
  For a complex manifold \(M\), a Hermitian metric \(g\) of the holomorphic tangent
  bundle is called a \underline{Hermitian metric of \(M\)}. We write \[
    g_{\bar{i}j}=g\left(\pd{\bar{z}^i},\pd{z^j}\right)
  .\]  
\end{definition}
\begin{remark}
  It is more convenient to have the convention of anti-holomorphic for the first
  term and holomorphic for the second term. But many mathematicians use the opposite
  convention. Therefore we will change the convention later to adapt to most
  people's convention.

  But for the time being, we use the convention of \(g_{\bar{i}j}\). This is because
  we employed the western culture and endomorphisms are expressed as \[
    (a\indices{^i_j}) \text{ for } \op{End}(E)\cong E\otimes E^*
  .\] I believe you will feel more comfortable with this convention in the 
  computations below.
\end{remark}

\begin{example}
  Consider the hyperplane bundle \(H=\mathcal{O}(1)\to \mathbb{P}^1(\mathbb{C})\).
  On \(V_0=\{z^0\neq 0\}\), \(s=\frac{z^1}{z^0}\) is the local coordinate, We put \[
    h_0=\frac{1}{1+|s|^2}
  .\] On \(U_1=\{z^1\neq 0\}\), \(t=\frac{z^0}{z^1}\) is the coordinate, we put \[
    h_1=\frac{1}{1+|t|^2}
  .\] Then \(h_0\) on \(U_0\) and \(h_1\) on \(U_1\) define a well-defined 
  Hermitian metric on \(H=\mathcal{O}(1)\).
\end{example}
\begin{proof}
  It is sufficient to check \[
    h_1=\tensor[^t]{\bar{f}}{_{01}}h_0 f_{01}\quad
    \text{(Remember }f_{01}=s=\frac{1}{t}\text{)}
  .\] The RHS (right hand side) is \[
    \bar{s}\frac{1}{1+|s|^2}s=\frac{1}{1+|t|^2}\quad \text{ since }s=\frac{1}{t}
  .\] 
\end{proof}

\begin{example}
  \(T'\mathbb{P}^1(\mathbb{C})=H\otimes H=\mathcal{O}(2)\).
\end{example}
\begin{proof}
  On \(U_0\), \(\pd{s}\) is a basis of \(T'\mathbb{P}^1(\mathbb{C})\). On \(U_1\),
  we take \(-\pd{t}\) as a basis. The minus sign is just for convenience.

  Over \(U_0\cap U_1\), \(s=\frac{1}{t}\) and \[
    -\pd{t}=-\pdv{s}{t}\pd{s}=\frac{1}{t^2}\pd{s}=s^2\pd{s}
  .\] If \(a_0\pd{s}=a_1(-\pd{t})\), then \(a_0=a_1s^2\). Hence \(f_{01}=s^2\).
\end{proof}

\begin{example}
  By putting \[
    g\left(\pd{\bar{s}},\pd{s}\right)=\frac{1}{(1+|s|^2)^2}
  \] on \(U_0\) and \[
    g\left(\pd{\bar{t}},\pd{t}\right)=\frac{1}{(1+|t|^2)^2}
  \] on \(U_1\), we obtain a Hermitian metric.
\end{example}
\begin{proof}
  Left as exercise.
\end{proof}

\begin{example}[Fubini-Study metric]
  On \(U_\lambda\subset \mathbb{P}^m(\mathbb{C})\) we have local coordinates \[
    t_{\lambda}^j=\frac{z^j}{z^\lambda},\ j=0,\ldots,\hat{\lambda},\ldots,m
  .\] We define on \(U_\lambda=\{z^\lambda\neq 0\}\), \[
    g_{\bar{i}j}^{(\lambda)}=\pdv{}{\bar{t}_\lambda^i,\bar{t}_\lambda^j}
    \log\Big(|t_\lambda^0|^2+\cdots +\overset{\mathclap\lambda\text{ th}}{1}+\cdots
    +|t_\lambda^m|^2\Big)
  .\] This is well-defined independent of \(\lambda\) and is called the 
  \textbf{Fubini-Study} metric.
\end{example}
\begin{proof}
  Left as exercise.
\end{proof}

\begin{lemma}
  On a holomorphic vector bundle \(E\to M\), the \(\bar{\partial}\) operator \[
    \bar{\partial}\colon C^\infty(M,E)\longrightarrow C^\infty(M,E\otimes T^{*\prime
    \prime}M)
  \] is well-defined.
\end{lemma}
\begin{proof}
  The point is that the transition functions are holomorphic, so that the
  \(\bar{\partial}\) operator defined for local trivialization is well-defined.
  More precisely, if \(M=\bigcup_{\lambda\in \Lambda}U_\lambda\), and 
  \(e_{\lambda 1},\ldots,e_{\lambda r}\) is a local holomorphic frame. Any \(C^\infty\)
  section of \(E\) can be expressed as \[
    s=s^1 e_{\lambda 1}+\cdots +s^r e_{\lambda r}, s^j \in C^\infty(U_\lambda)
  .\] Then we define \[
    \bar{\partial}s=e_{\lambda 1}\otimes \bar{\partial}s^1+\cdots 
    +e_{\lambda r}\otimes \bar{\partial}s^r
  .\] Because the transition functions are holomorphic, \ie\ 
  \begin{gather*}
    e_\mu=e_\lambda\vphi_{\lambda\mu}\qquad \bar{\partial}\vphi_{\lambda\mu}=0 \\
    e_\mu t=e_\lambda s\iff s=\vphi_{\lambda\mu}t
  .\end{gather*} 
  Then \[
    e_{\lambda i}\bar{\partial}s^i=e_{\lambda i}\bar{\partial}(\vphi\indices{_\lambda
    _\mu^i_j}t^i)=e_{\lambda i}\vphi\indices{_\lambda_\mu^i_j}\bar{\partial}t^j
    =e_{\mu j}\bar{\partial}t^j
  .\] 
\end{proof}

Just as the Levi-Civita connection for a Riemannian metric, there is a unique
connection, called the \textbf{Chern connection}, for a Hermitian metric on a
holomorphic vector bundle, as in the following theorem.
\begin{theorem}
  Let \(E\to M\) be a holomorphic vector bundle, \(h\) a Hermitian metric on \(E\).
  Then there exists a unique connection \(\nabla\) on \(E\) such that
  \begin{enumerate}[(1)]
  \item If we decompose \(\nabla\) as \[
      \nabla =\nabla'+\nabla''\colon C^\infty(M,E)\longrightarrow
      C^\infty(M,E\otimes T^{*\prime}M)\oplus C^\infty(M,E\otimes T^{*\prime\prime}M)
    ,\] then \(\nabla''=\bar{\partial}\).
  \item \(\nabla h=0\), \ie\ \[
      \dd (h(\bar{s},t))=h(\overline{\nabla s},t)+h(s,\nabla t)
    .\] 
  \end{enumerate}
\end{theorem}
\begin{proof}
  Suppose such a connection exists. Let \(e_1,\ldots,e_r\) be a local holomorphic
  frame, then by (1), \[
    \nabla e_j=(\nabla'+\bar{\partial})e_j=\nabla'e_j 
    \quad \text{(since \(\bar{\partial}1=0\))}
  .\] So the connection form \(\omega\) is type \((1,0)\), \[
    \nabla e_j=e_i \otimes \omega\indices{^i_j}
    \quad\text{\ie\ }\nabla e=e\otimes \omega
  .\] On the other hand, by (2),
  \begin{align*}
    \dd{h_{\bar{i}j}}&=\dd{(h(\bar{e}_i,e_j))}=h(\overline{\nabla e_i},e_j)
    +h(\bar{e}_i,\nabla e_j) \\
    &=h(\overline{e_k\otimes \omega\indices{^k_i}},e_j)+h(\bar{e}_i,e_k\otimes 
    \omega\indices{^k_j}) \\
    &=h_{\bar{k}j}\overline{\omega\indices{^k_i}}+h_{\bar{i}k}\omega\indices{^k_j}
  .\end{align*}
  Taking (1,0)-part of both sides, we have \[
    \partial h_{\bar{i}j}=h_{\bar{i}k}\omega\indices{^k_j}
  .\] Hence \[
    \omega\indices{^i_j}=h^{i\bar{k}}\partial h_{\bar{k}j},\quad
    \text{where }h^{i\bar{k}}h_{\bar{k}j}=\delta_j^i
  .\] Thus \[
    \boxed{\omega=h^{-1}\partial h}
  .\] Over \(U_\lambda\cap U_\mu\neq \emptyset\), one have 
  \begin{align*}
    \omega_\lambda&=h_{\lambda}^{-1}\partial h_{\lambda}\text{ on }U_\lambda, \\
    \omega_\mu&=h_{\mu}^{-1}\partial h_{\mu}\text{ on }U_\mu, \\
    h_\mu=\tensor[^t]{\overline{\vphi}}{_\lambda_\mu}h_{\lambda}\vphi_{\lambda\mu}
  .\end{align*}
  One can show \[
    \omega_{\mu}=\vphi_{\lambda\mu}^{-1}\omega_{\lambda}\vphi_{\lambda\mu}
    +\vphi_{\lambda\mu}^{-1}\dd{\vphi_{\lambda\mu}}
  .\] By \ifdefined\FullBook{}
    \cref{thm:1-2:prop2},
  \else
  Proposition 5 in notes of week 1-2,
  \fi \(\{\omega_{\lambda}\}_{\lambda\in \Lambda}\) defines a globally well-defined
  connection on \(E\).
\end{proof}

\begin{definition}
  We call this connection the \textbf{Chern connection} of \((E,h)\). (Chern is of
  course the English name of {\CJKfamily{zhkai}陈省身}).
\end{definition}

\begin{theorem}
  Given a Hermitian metric \(h\) on a holomorphic vector bundle \(E\), the connection
  form \(\omega\) of the Chern connection is given by \[
    \omega=h^{-1}\partial h=\left(h^{i\bar{k}}\partial h_{\bar{k}j}\right)
  ,\] and the curvature form is given by \[
    \boxed{\Omega=\bar{\partial}\omega=\bar{\partial}(h^{-1}\partial h)}
  \] with respect to a local frame \(e_1,\ldots,e_r\), where \[
    h_{\bar{i}j}=h(\bar{e}_i,e_j),h^{i\bar{k}}h_{\bar{k}j}=\delta^i_j
  .\] 
\end{theorem}
\begin{proof}
  We have seen already \(\omega=h^{-1}\partial h\). For curvature,
  \begin{align*}
    \Omega&=\dd{\omega}+\omega\wedge \omega \\ 
    &=(\partial+\bar{\partial})h^{-1}\partial h
    +h^{-1}\partial h\wedge h^{-1}\partial h \\
    &=\underline{-h^{-1}\partial h\cdot h^{-1}\wedge \partial h}
    -h^{-1}\bar{\partial}h\cdot h^{-1}\wedge \partial h \\
    &\phantom{=\ } +h^{-1}\bar{\partial}\partial h
    +\underline{h^{-1}\partial h\wedge h^{-1}\partial h} \\
    &=\bar{\partial}(h^{-1}\partial h)
  .\end{align*}
  Note that \(\partial h^{-1}=-h^{-1}\partial h\cdot h^{-1}\)
\end{proof}

\end{document}
