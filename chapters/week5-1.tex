% !TeX program = xelatex
\documentclass[12pt]{article}
\usepackage{standalone}

\usepackage[dvipsnames,svgnames,x11names]{xcolor}
\usepackage[a4paper,margin=1in]{geometry}
\usepackage{microtype}
\usepackage{amsmath}
\usepackage{amsthm}
\usepackage{mathtools}
\usepackage{mathrsfs}
\usepackage{stmaryrd}
\usepackage{extarrows}
\usepackage{enumerate}
\usepackage{tensor}
\usepackage{physics}
\usepackage{graphicx}
\usepackage{subcaption}
\usepackage{tikz}
\usepackage{tikz-3dplot}
% \usepackage{tikz-cd}
\usepackage{quiver}
  \tikzset{every picture/.style={line width=0.75pt}}
\usepackage{pgfplots}
  \pgfplotsset{compat=newest}
\usepackage{tcolorbox}
  \tcbuselibrary{most}
\usepackage[colorlinks=true,linkcolor=blue]{hyperref}
\usepackage{cleveref}
\usepackage[hyperref=true,backend=biber,style=alphabetic,backref=true,url=false]{biblatex}
\usepackage[warnings-off={mathtools-colon,mathtools-overbracket}]{unicode-math}
\usepackage[default]{fontsetup}
\usepackage{fancyhdr}
\usepackage{import}

\newcommand{\Id}{\mathbb{1}}
\newcommand{\lap}{\increment}

\DeclareMathOperator{\sign}{sign}
\DeclareMathOperator{\dom}{dom}
\DeclareMathOperator{\ran}{ran}
\DeclareMathOperator{\ord}{ord}
\DeclareMathOperator{\Span}{span}
\DeclareMathOperator{\img}{Im}
\DeclareMathOperator{\Ric}{Ric}
\newcommand{\card}{\texttt{\#}}
\newcommand{\ie}{\emph{i.e.}}
\newcommand{\st}{\emph{s.t.}}
\newcommand{\eps}{\varepsilon}
\newcommand{\vphi}{\varphi}
\newcommand{\vthe}{\vartheta}
\newcommand{\II}{I\!I}
\renewcommand{\emptyset}{\varnothing}
\newcommand{\acts}{\curvearrowright}
\newcommand{\xrr}{\xlongrightarrow}
\newcommand{\into}{\hookrightarrow}
\newcommand{\pdif}[2]{\frac{\partial #1}{\partial #2}}
\renewcommand{\op}{\operatorname}

\theoremstyle{plain}\newtheorem{theorem}{Theorem}
\theoremstyle{definition}\newtheorem{definition}[theorem]{Definition}
\theoremstyle{definition}\newtheorem{example}[theorem]{Example}
\theoremstyle{plain}\newtheorem{axiom}[theorem]{Axiom}
\theoremstyle{plain}\newtheorem{corollary}[theorem]{Corollary}
\theoremstyle{plain}\newtheorem{lemma}[theorem]{Lemma}
\theoremstyle{plain}\newtheorem{proposition}[theorem]{Proposition}
\theoremstyle{plain}\newtheorem{prop}[theorem]{Proposition}
\theoremstyle{plain}\newtheorem{conjecture}[theorem]{Conjecture}
\theoremstyle{plain}\newtheorem{conj}[theorem]{Conjecture}
\theoremstyle{plain}\newtheorem{problem}[theorem]{Problem}
\theoremstyle{remark}\newtheorem{notation}[theorem]{Notation}
\theoremstyle{definition}\newtheorem*{question}{Question}
\theoremstyle{definition}\newtheorem*{answer}{Answer}
\theoremstyle{definition}\newtheorem*{goal}{Goal}
\theoremstyle{plain}\newtheorem*{application}{Application}
\theoremstyle{plain}\newtheorem*{exercise}{Exercise}
\theoremstyle{remark}\newtheorem*{remark}{Remark}
\theoremstyle{remark}\newtheorem*{note}{\small{Note}}
\numberwithin{equation}{section}
\numberwithin{theorem}{section}
\numberwithin{figure}{section}

\usepackage{xeCJK}
\setCJKmainfont{FZShuSong-Z01}[BoldFont=FZXiaoBiaoSong-B05,ItalicFont=FZKai-Z03]
\setCJKsansfont{FZXiHeiI-Z08}[BoldFont=FZHei-B01]
\setCJKmonofont{FZFangSong-Z02}
\setCJKfamilyfont{zhsong}{FZShuSong-Z01}[BoldFont=FZXiaoBiaoSong-B05]
\setCJKfamilyfont{zhhei}{FZHei-B01}
\setCJKfamilyfont{zhkai}{FZKai-Z03}
\setCJKfamilyfont{zhfs}{FZFangSong-Z02}
\setCJKfamilyfont{zhli}{FZLiShu-S01}
\setCJKfamilyfont{zhyou}{FZXiYuan-M01}[BoldFont=FZZhunYuan-M02]

\geometry{a4paper,margin=1in}
\allowdisplaybreaks{}

\counterwithout{theorem}{section}
\counterwithout{equation}{section}

\begin{document}

\begin{lemma}
  \((M,g)\) is compact K\"ahler implies \(H^2(M)\neq 0\).
\end{lemma}
\begin{proof}
  Since \([\gamma]\in H^2(M;\mathbb{R})\) and \[
    \Big<\underbrace{[\gamma]\cup\cdots \cup [\gamma]}_{m},[M]\Big> 
    =\int_{M}\overbrace{\gamma\wedge\cdots \wedge \gamma}^m\neq 0
  .\] 
\end{proof}

\begin{corollary}
  Let \(M\) be compact complex manifold with \(H^2(M)=0\), then \(M\) is
  non-K\"ahler.
\end{corollary}

\begin{example}
  \(\mathbb{P}^m(\mathbb{C})\) is K\"ahler with respect to the Fubini-Study metric.
\end{example}
\begin{proof}
  We have \[
    g_{i\bar{j}}=\pd{t^i,\bar{t}^j}\log(1+\sum_k |t^k|^2)
  \] on \(U_0=\{z^0\neq 0\},t^i=\frac{z^i}{z^0}\). Then \[
    \gamma_{FS}=\sqrt{-1} g_{i\bar{j}}\dd{t^i}\dd{\bar{t}^j}
    =\sqrt{-1}\partial\bar{\partial}\log(1+\sum_k |t^k|^2)
  .\] Hence \[
    \dd{\gamma_{FS}}=(\partial +\bar{\partial})\Big(\sqrt{-1}\partial\bar{\partial}
    \log(1+\sum_k|t^k|^2)\Big)=0
  .\] 
\end{proof}

\begin{definition}
  A compact complex manifold \(M\) is called a non-singular algebraic variety if there
  is a sufficiently large integer \(N>0\) such that there is an embedding \[
    f\colon M\longrightarrow \mathbb{P}^N(\mathbb{C})
  .\] Simply stated, a compact complex submanifold of \(\mathbb{P}^N(\mathbb{C})\),
  \(N\) large.
\end{definition}
Note that a submanifold of \(\mathbb{P}^N(\mathbb{C})\) is algebraic
by the following well-known theorem:
\begin{theorem}[Chow]
  Any compact complex submanifold of \(\mathbb{P}^N(\mathbb{C})\) is expressed as a
  common zero set of finite number of polynomials.
\end{theorem}
(See \emph{Griffiths-Harris}, page 167. Not so difficult)

\begin{prop}
  Any submanifold of a K\"ahler manifold is K\"ahler with respect to the induced
  metric.
\end{prop}
\begin{proof}
  Let \(f\colon N\into(M,g)\) be an embedding of a complex submanifold \(N\) into
  a K\"ahler manifold \((M,g)\). In local coordinates \((z^1,\ldots,z^n)\) in \(N\)
  and \((w^1,\ldots,w^n)\) in \(M\), \(f\) may be expressed as \[
    f^i(z^1,\ldots,z^n)=w^i(f(z^1,\ldots,z^n)),\quad i=1,\ldots,m
  .\] The induced metric \(f^*g\) is expressed as \[
  (f^*g)_{pq}=g_{i\bar{j}}\pdv{f^i}{z^p}\overline{\pdv{f^j}{z^q}}
  .\] Hence \[
    \gamma_{f^*g}=\sqrt{-1}(f^*g)_{pq}\dd{z^p}\dd{\bar{z}^q}=f^*\gamma_g
  .\] The pull-back and exterior derivative commutes, so \[
    \dd{\gamma_{f^*g}}=f^* \dd{\gamma_g}=0
  .\] 
\end{proof}

Since \(\mathbb{P}^N(\mathbb{C})\) is K\"ahler with respect to the Fubini-Study metric,
a non-singular algebraic varieties are K\"ahler. We have \[
  \{\text{algebraic manifolds}\}\subset \{\text{K\"ahler manifolds}\}
  \subset \{\text{complex manifolds}\}
.\] Recall we have a hyperplane line bundle \(H\to \mathbb{P}^N(\mathbb{C})\) whose
metric is given by \[
  h_{U_0}=\frac{1}{1+|t^1|^2+\cdots +|t^N|^2}\text{ on }U_0=\{z^0\neq 0\},
  t^i=\frac{z^i}{z^0}
.\] And \[
  c_1(H)=-\frac{\sqrt{-1}}{2\pi}\partial\bar{\partial}\log \frac{1}{1+\sum |t^i|^2}
  =\frac{\sqrt{-1}}{2\pi}\partial\bar{\partial}\log(1+\sum |t^i|^2)
.\] This is the K\"ahler form of the Fubini-Study metric (up to a constant).

For any non-singular algebraic variety \(M\), embedded by \(f\colon M\into
\mathbb{P}^N(\mathbb{C})\), there is a line bundle \(L=f^*H\) with the property that
\(c_1(L)\) is represented by a K\"ahler form (or often referred as a positive form).

There are terminologies in complex geometry to express these situations.
\begin{definition}\hfill
\begin{enumerate}[(1)]
\item Let \(M\) be a compact complex manifold. A holomorphic line bundle \(L\to M\)
  is said to be ample (or positive) if \(c_1(L)\) is represented by a K\"ahler
  form (positive form).
\item A pair \((M,L)\) of a compact complex manifold \(M\) and an ample line bundle
  \(L\to M\) over it is called a polarized manifold.
\end{enumerate}
\end{definition}

Hence a non-singular algebraic variety \(M\) and \(L=f^*H\) give a polarized
manifold. The following \underline{Kodaria embedding theorem} asserts the converse.

\begin{theorem}[Kodaira embedding]
For any polarized manifold \((M,L)\), \(\exists\,k_0>0\) \st\ for any \(k\ge k_0\),
the map
\begin{align*}
  \Phi_k\colon M &\longrightarrow \mathbb{P}^N(\mathbb{C}) \\
  p &\longmapsto [s_0(p):s_1(p):\cdots :s_N(p)]
\end{align*}
is an embedding. Where \[
  N+1=\dim H^0(M,L^k)=\text{the dimension of all holomorphic sections of }L^k
\] and \(s_0,\ldots,s_n\) are basis of \(H^0*(M,L^k)\).
\end{theorem}

Let \((M,g)\) be a compact Riemannian manifold (without boundary). For local
coordinates \(x^1,\ldots,x^n\), the form \[
  \dd{V_g}=\sqrt{\det g}\dd{x^1}\wedge \cdots \wedge \dd{x^n}
\] is called the volume element. On the overlap of two coordinate neighborhoods 
one can check \(\dd{V_g}\) coincides (left as exercise) and defines a measure globally.

Let \(X=X^i\pd{x^i}\) be a vector field on \(M\). We define the \textbf{divergence}
\(\op{div}(X)\in C^\infty(M)\) by \[
  \dd{(\iota_X \dd{V_g})}=\op{div}(X)\dd{V_g}
\] where \(\iota_X\) denotes the inner product (contraction by \(X\)).

One can show (exercise) \[
  \op{div}(X)=\nabla_i X^i=\pdv{X^i}{x^i}+\Gamma_{ik}^i X^k
.\] By the Stokes theorem, we get
\begin{theorem}[Divergence theorem]
  \[
    \int_{M}\op{div}(X)\dd{V_g}=0
  .\] 
\end{theorem}

\end{document}
