% !TeX program = xelatex
\documentclass[12pt]{article}
\usepackage{standalone}

\usepackage[dvipsnames,svgnames,x11names]{xcolor}
\usepackage[a4paper,margin=1in]{geometry}
\usepackage{microtype}
\usepackage{amsmath}
\usepackage{amsthm}
\usepackage{mathtools}
\usepackage{mathrsfs}
\usepackage{stmaryrd}
\usepackage{extarrows}
\usepackage{enumerate}
\usepackage{tensor}
\usepackage{physics2}
  \usephysicsmodule{ab,xmat}
\usepackage{fixdif}
  \newcommand{\dd}{\d}
\usepackage{derivative}
  \newcommand{\dv}{\odv}
  \newcommand{\pd}[1]{\pdv{}{#1}}
  \newcommand{\eval}[1]{#1\big|}
\usepackage{graphicx}
\usepackage{subcaption}
\usepackage{tikz}
\usepackage{tikz-3dplot}
% \usepackage{tikz-cd}
% \usepackage{quiver}
% ========== Block quiver.sty ========== %
\usepackage{tikz-cd}
% \usepackage{amssymb}
\usetikzlibrary{calc}
\usetikzlibrary{decorations.pathmorphing}
\tikzset{curve/.style={settings={#1},to path={(\tikztostart)
    .. controls ($(\tikztostart)!\pv{pos}!(\tikztotarget)!\pv{height}!270:(\tikztotarget)$)
    and ($(\tikztostart)!1-\pv{pos}!(\tikztotarget)!\pv{height}!270:(\tikztotarget)$)
    .. (\tikztotarget)\tikztonodes}},
    settings/.code={\tikzset{quiver/.cd,#1}
        \def\pv##1{\pgfkeysvalueof{/tikz/quiver/##1}}},
    quiver/.cd,pos/.initial=0.35,height/.initial=0}
\tikzset{tail reversed/.code={\pgfsetarrowsstart{tikzcd to}}}
\tikzset{2tail/.code={\pgfsetarrowsstart{Implies[reversed]}}}
\tikzset{2tail reversed/.code={\pgfsetarrowsstart{Implies}}}
\tikzset{no body/.style={/tikz/dash pattern=on 0 off 1mm}}
% =========== End block ========== %
  \tikzset{every picture/.style={line width=0.75pt}}
\usepackage{pgfplots}
  \pgfplotsset{compat=newest}
\usepackage{tcolorbox}
  \tcbuselibrary{most}
\usepackage[colorlinks=true,linkcolor=blue]{hyperref}
\usepackage{cleveref}
% \usepackage[hyperref=true,backend=biber,style=alphabetic,backref=true,url=false]{biblatex}
\usepackage[warnings-off={mathtools-colon,mathtools-overbracket}]{unicode-math}
\usepackage[default,amsbb]{fontsetup}
  \setmathfont[StylisticSet=1,range=\mathscr]{NewCMMath-Book.otf}
\usepackage{fancyhdr}
\usepackage{import}

\newcommand{\Id}{\mathbb{1}}
\newcommand{\lap}{\increment}

\DeclareMathOperator{\sign}{sign}
\DeclareMathOperator{\dom}{dom}
\DeclareMathOperator{\ran}{ran}
\DeclareMathOperator{\ord}{ord}
\DeclareMathOperator{\Span}{span}
\DeclareMathOperator{\img}{Im}
\DeclareMathOperator{\Ric}{Ric}
\newcommand{\card}{\texttt{\#}}
\newcommand{\ie}{\emph{i.e.}}
\newcommand{\st}{\emph{s.t.}}
\newcommand{\eps}{\varepsilon}
\newcommand{\vphi}{\varphi}
\newcommand{\vthe}{\vartheta}
\newcommand{\II}{I\!I}
\renewcommand{\emptyset}{⌀}
\newcommand{\acts}{\curvearrowright}
\newcommand{\xrr}{\xlongrightarrow}
\newcommand{\lrr}{\longrightarrow}
\newcommand{\lmt}{\longmapsto}
\newcommand{\into}{\hookrightarrow}
\newcommand{\op}{\operatorname}

\let\originalleft\left
\let\originalright\right
\renewcommand{\left}{\mathopen{}\mathclose\bgroup\originalleft}
\renewcommand{\right}{\aftergroup\egroup\originalright}

\theoremstyle{plain}\newtheorem{theorem}{Theorem}
\theoremstyle{definition}\newtheorem{definition}[theorem]{Definition}
\theoremstyle{definition}\newtheorem{example}[theorem]{Example}
\theoremstyle{definition}\newtheorem{problem}[theorem]{Problem}
\theoremstyle{plain}\newtheorem{axiom}[theorem]{Axiom}
\theoremstyle{plain}\newtheorem{corollary}[theorem]{Corollary}
\theoremstyle{plain}\newtheorem{lemma}[theorem]{Lemma}
\theoremstyle{plain}\newtheorem{proposition}[theorem]{Proposition}
\theoremstyle{plain}\newtheorem{prop}[theorem]{Proposition}
\theoremstyle{plain}\newtheorem{conjecture}[theorem]{Conjecture}
\theoremstyle{plain}\newtheorem{conj}[theorem]{Conjecture}
\theoremstyle{remark}\newtheorem{notation}[theorem]{Notation}
\theoremstyle{definition}\newtheorem*{question}{Question}
\theoremstyle{definition}\newtheorem*{answer}{Answer}
\theoremstyle{definition}\newtheorem*{goal}{Goal}
\theoremstyle{definition}\newtheorem*{application}{Application}
\theoremstyle{plain}\newtheorem*{exercise}{Exercise}
\theoremstyle{remark}\newtheorem*{remark}{Remark}
\theoremstyle{remark}\newtheorem*{note}{\small{Note}}
\numberwithin{equation}{section}
\numberwithin{theorem}{section}
\numberwithin{figure}{section}

\usepackage{xeCJK}
\setCJKmainfont{FZShuSong-Z01}[BoldFont=FZXiaoBiaoSong-B05,ItalicFont=FZKai-Z03]
\setCJKsansfont{FZXiHeiI-Z08}[BoldFont=FZHei-B01]
\setCJKmonofont{FZFangSong-Z02}
\setCJKfamilyfont{zhsong}{FZShuSong-Z01}[BoldFont=FZXiaoBiaoSong-B05]
\setCJKfamilyfont{zhhei}{FZHei-B01}
\setCJKfamilyfont{zhkai}{FZKai-Z03}
\setCJKfamilyfont{zhfs}{FZFangSong-Z02}
\setCJKfamilyfont{zhli}{FZLiShu-S01}
\setCJKfamilyfont{zhyou}{FZXiYuan-M01}[BoldFont=FZZhunYuan-M02]

\allowdisplaybreaks{}

\newcommand{\isFullBook}[2]{
  \ifnum\pdfstrcmp{\FullBook}{True}=0
    \ifnum\pdfstrcmp{}{#1}=0\unskip\else#1\fi
  \else
    \ifnum\pdfstrcmp{}{#2}=0\unskip\else#2\fi
  \fi\ignorespaces{}
}

\counterwithout{theorem}{section}
\counterwithout{equation}{section}

\begin{document}
\begin{proof}
  Let \(L_1\) and \(L_2\) be isomorphic line bundles, and suppose the transition
  functions are respectively given by \(\{a_{\lambda\mu}\}\) and \(\{a'_{\lambda\mu}\}
  \) with respect to local frames \(\{s_{\lambda}\}_{\lambda\in \Lambda}\) and
  \(\{t_{\lambda}\}_{\lambda\in\Lambda}\) on an open covering \(\{U_\lambda\}_{\lambda
  \in \Lambda}\). Thus \[
    s_{\mu}=s_{\lambda}a_{\lambda\mu},\quad
    t_{\mu}=t_{\lambda}a'_{\lambda\mu}
  .\] The bundle isomorphism \(h\colon L_1\to L_2\) is expressed as \[
    h(s_\lambda)=t_\lambda h_\lambda,\quad
    h(s_\mu)=t_\mu h_\mu
  .\] Thus \[
    h(s_\mu)=h(s_\lambda a_{\lambda\mu})=h(s_\lambda)a_{\lambda\mu}
    =t_\lambda h_\lambda a_{\lambda\mu}
  .\] But also \[
    h(s_\mu)=t_\mu h_\mu=t_\lambda a'_{\lambda\mu}h_\mu
  .\] So \[
    a'_{\lambda\mu}a_{\lambda\mu}^{-1}=h_\lambda h_\mu^{-1}
  .\] Hence \(\{a'_{\lambda\mu}\}\{a_{\lambda\mu}^{-1}\}=\delta\{h_{\lambda}\}\).
  That is to say, isomorphic line bundles assign cohomologous cocyle.
  (Note that \(\mathscr{A}^*\) is multiplicatively Abelian).
  
  Conversely, following backward the above arguments, we see \(\{a_{\lambda\mu}\}\)
  and \(\{a'_{\lambda\mu}\}\) are cohomologous. Then they define isomorphic line
  bundle. More precisely, if we define \(h\) by \(h(s_\mu)=t_\mu h_\mu\). Then
  using \(a'_{\lambda\mu}a_{\lambda\mu}^{-1}=h_\lambda h_\mu^{-1}\) we see
  \(h(s_\lambda)=t_\lambda h_\lambda\), so \(h\) is well defined.
\end{proof}

Let \(\mathscr{A}\) (resp. \(\mathscr{A}^*\)) be the sheaf of germs of smooth
(resp. non-vanishing smooth) complex-valued functions on a compact smooth manifold
\(M\). Denote by \(\mathbb{Z}\) the constant sheaf with integer value. Then we have
a short exact sequence 
\[\begin{tikzcd}[row sep=tiny]
	0 & {\mathbb{Z}} & {\mathscr{A}} & {\mathscr{A}^*} & 0 \\
	&& a & {exp(2\pi ia)}
	\arrow[from=1-1, to=1-2]
	\arrow["j", from=1-2, to=1-3]
	\arrow["e", from=1-3, to=1-4]
	\arrow[from=1-4, to=1-5]
	\arrow[maps to, from=2-3, to=2-4]
\end{tikzcd}\]
Where \(j\) is the natural inclusion and \(e=\exp(2\pi i \cdot )\). Since
\(\mathscr{A}\) is a fine sheaf and \[
  H^i(M,\mathscr{A})=0\quad\text{ for }i\ge 1
,\] we obtain from the long exact sequence induced from the above short exact
sequence \[
  \delta^*\colon H^i(M,\mathscr{A}^*)\cong H^{i+1}(M,\mathbb{Z})
.\] In particular, for \(i=1\), \[
  \delta^*\colon H^1(M,\mathscr{A}^*)\cong H^2(M,\mathbb{Z})
.\] Recall in the last proposition we saw \(H^1(M,\mathscr{A}^*)\) is isomorphic
to the Abelian group of isomorphism classes of complex line bundle over \(M\).
\begin{definition}[A refined definition of Chern class of line bundles]
  Let \(L\to M\) be a complex line bundle, and denote by \([L]\in H^1(M,\mathscr{A}^*)
  \) the isomorphism class of \(L\). We define \[
    c_1(L)=-\delta^*([L])\in H^2(M,\mathbb{Z})
  .\] 
\end{definition}
This definition is consistent with the earlier definition using the curvature as
we will see in the next proposition. However this new definition is more sensitive
because it is in \(\mathbb{Z}\)-coefficient cohomology.
\begin{prop}
  Under \(H^2(M,\mathbb{Z})\to H^2(M,\mathbb{R})\cong H_{\mathrm{dR}}^2(M)\),
  \(c_1(L)\) is sent to \(\left[\frac{\sqrt{-1}}{2\pi}\Omega\right]\) for some
  connection \(\omega\) and its curvature \(\Omega\). Here \(\mathbb{R}\) denotes the
  constant sheaf of \(\mathbb{R}\)-valued locally constant functions and \[
    H^2(M,\mathbb{Z})\lrr H^2(M,\mathbb{R})
  \] is induced by the natural inclusion \(\mathbb{Z}\to \mathbb{R}\) and 
  \(H^2(M,\mathbb{R})\cong H^2_{\mathrm{dR}}(M)\) is the de Rham theorem.
\end{prop}
\begin{proof}
  Let \(\{a_{\lambda\mu}\}\) be the transition function of \(L\) with respect to an
  open covering \(\mathcal{U}=\{U\}_{\lambda}\). \(\{a_{\lambda\mu}\}\) satisfies the
  cocycle condition \[
    a_{\lambda\mu}a_{\mu\nu}a_{\nu\lambda}=1
  .\] Taking logarithm we obtain \[
    \log a_{\lambda\mu}+\log a_{\mu\nu}+\log a_{\nu\lambda}=0 \bmod{2\pi\sqrt{-1}}
  .\] Here the logarithm is defined up to mod \(2\pi\sqrt{-1}\mathbb{Z}\) but we may
  choose the branch of \(\log\) so that \[
    \log a_{\mu\lambda}=-\log a_{\lambda\mu}
  .\] Then we obtain \(c=\{c_{\lambda\mu\nu}\}\), \[
    2\pi\sqrt{-1}c_{\lambda\mu\nu}=\log a_{\lambda\mu}
    +\log a_{\mu\nu}+\log a_{\nu\lambda}
  .\] One can check the cocycle condition \[
    c_{\mu\nu\tau}-c_{\lambda\nu\tau}+c_{\lambda\mu\tau}-c_{\lambda\mu\nu}=0
  .\] The connecting homomorphism \[
    f^*\colon H^1(M,\mathscr{A}^*)\xrr{\sim}H^2(M,\mathbb{Z})
  \] is given by \[
  \delta^*([L])=\delta^*[\{a_{\lambda\mu}\}]=[\{c_{\lambda\mu\nu}\}]
  .\] By the long exact sequence \[
    H^2(M,\mathbb{Z})\lrr H^2(M,\mathbb{R})\lrr H^2(M,\mathscr{A})=0
  \] induced by \(\mathbb{Z}\to \mathbb{R}\to \mathscr{A}\),
  \([\{c_{\lambda\mu\nu}\}]\) should be a coboundary in \(H^2(M,\mathscr{A})\).
  Indeed, putting \[
    f_{\lambda\mu}=\log a_{\lambda\mu}
  ,\] we have \[
    2\pi i c_{\lambda\mu\nu}=f_{\mu\nu}-f_{\lambda\nu}+f_{\lambda\mu}
  ,\] \ie\ \[
    \delta\{\frac{f_{\lambda\mu}}{2\pi i}\}=\{c_{\lambda\mu\nu}\}
  .\] Since \(c_{\lambda\mu\nu}\) are (locally) constant functions, \[
    \dd{f_{\mu\nu}}-\dd{f_{\lambda\nu}}+\dd{f_{\lambda\mu}}=0
  .\] Thus \(\{\dd{f_{\lambda\mu}}\}\in Z^1(\mathcal{U},\mathscr{A}^1)\), but since 
  \(H^1(M,\mathscr{A}^1)=0\), there should exists \(\{\theta_\lambda\}\in
  C^0(\mathcal{U},\mathscr{A}^1)\) such that \[
    \frac{1}{2\pi i}\dd{f_{\lambda\mu}}=\theta_{\mu}-\theta_{\lambda}
  .\] Thus \(\dd{\theta_\mu}=\dd{\theta_\lambda}\) and \(\{\dd{\theta_\lambda}\}\)
  defines a global 2-form. Its de Rham class corresponds to \[
    \delta^*[L]=-c \in H^2(M,\mathbb{Z})
  .\] But recall that if we have a connection \(\omega=\{\omega_\lambda\}\), then
  the condition for it to be globally defined is 
  \begin{align*}
    \omega_\mu&=a_{\lambda\mu}^{-1}\omega_\lambda a_{\lambda\mu}+a_{\lambda\mu}^{-1}
    \dd{a_{\lambda\mu}} \\
    &=\omega_\lambda+\dd{\log a_{\lambda\mu}}
  .\end{align*}
  Thus we can take \[
    \theta_\lambda=\frac{1}{2\pi\sqrt{-1}}\omega_\lambda
    =-\frac{\sqrt{-1}}{2\pi}\omega_\lambda
  .\] Then \[
    -\dd{\theta_\lambda}=\frac{\sqrt{-1}}{2\pi}\dd{\omega_\lambda}=
    \frac{\sqrt{-1}}{2\pi}\Omega
  \] defines a global 2-form, and is exactly the Chern form with respect to the
  connection \(\{\omega_\lambda\}\).
\end{proof}

Next we consider the holomorphic category. \ie\ the holomorphic line bundles over
a compact manifold \(M\).
\begin{definition}
  Denote by \(\op{Pic}(M)\) the group of all isomorphism classes of holomorphic
  line bundles over \(M\), called \textbf{Picard group}. Multiplication is given
  by the tensor product.
\end{definition}
Denote by \(\mathcal{O}_M\) (resp. \(\mathcal{O}_M^*\)) the sheaf of germs of
holomorphic (resp. non-vanishing holomorphic) functions on \(M\). As in the \(C^\infty
\)-case, we have a short exact sequence \[
  0\lrr \mathbb{Z}\xrr{j}\mathcal{O}_M\xrr{e}\mathcal{O}_M^*\lrr 0
\] and the induced long exact sequence \[
  H^0(M,\mathcal{O}_M^*)\xrr{\delta^*}H^1(M,\mathbb{Z})\xrr{j^*}H^1(M,\mathcal{O}_M)
  \xrr{e^*}H^1(M,\mathcal{O}_M^*)\xrr{\delta^*}H^2(M,\mathbb{Z})
.\] Just as in the \(C^\infty\) case we have
\begin{prop}
  \(\op{Pic}(M)\cong H^1(M,\mathcal{O}_M^*)\).
\end{prop}
The natural inclusion \(\mathcal{O}_M^*\to \mathscr{A}^*\) induces the commutative
diagram 
\[\begin{tikzcd}
  {H^1(M,\mathcal{O}_M^*)} & {H^2(M,\mathbb{Z})} \\
  {H^1(M,\mathscr{A}^*)} & {H^2(M,\mathbb{Z})}
	\arrow["{\delta^*}", from=1-1, to=1-2]
	\arrow[from=1-1, to=2-1]
	\arrow["\cong", from=2-1, to=2-2]
	\arrow["{=}"{marking, allow upside down}, draw=none, from=1-2, to=2-2]
\end{tikzcd}\]
Thus \(\delta^*([L])=-c_1(L)\).

\begin{definition}
  The \textbf{Picard variety} \(\op{Pic}^0(M)\) of \(M\) is defined by \[
    \op{Pic}^0(M)=\{L\in \op{Pic}(M):c_1(L)=0\}
  .\]
\end{definition}

From the long exact sequence above there is an injection \[
  \op{Pic}(M)/\op{Pic}^0(M)\lrr H^2(M,\mathbb{Z})
\] and an isomorphism \[
\op{Pic}^0(M)=H^1(M,\mathcal{O}_M)/j^*H^1(M,\mathbb{Z})
.\] But as we are assuming \(M\) is compact, \[
  H^0(M,\mathcal{O}^*)=\mathbb{C}^*\quad (\text{constant functions})
.\] Thus \[
  \delta^*\colon H^0(M,\mathcal{O}^*)\lrr H^1(M,\mathbb{Z})
\] is a zero map by the definition of connecting homomorphism. \ie\ for an open
covering \(\{U_\lambda\}\), \(c\in \mathbb{C}^*=H^0(M,\mathcal{O}^*)\) gives \[
  c_{\lambda}=c\quad\text{ on }U_\lambda,\quad c=\{c_\lambda\}
.\] Then \[
  \delta(c)_{\lambda\mu}=c_\mu-c_\lambda=c-c=0
.\] The long exact sequence shows \[
  j^*\colon H^1(M,\mathbb{Z})\lrr H^1(M,\mathcal{O})
\] is injective, and thus \[
  \op{Pic}^0(M)=H^1(M,\mathcal{O}_M)/H^1(M,\mathbb{Z})
.\] 

\end{document}
