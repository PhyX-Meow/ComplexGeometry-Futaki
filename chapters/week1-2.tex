% !TeX program = xelatex
\documentclass[12pt]{article}
\usepackage{standalone}

\usepackage[dvipsnames,svgnames,x11names]{xcolor}
\usepackage[a4paper,margin=1in]{geometry}
\usepackage{microtype}
\usepackage{amsmath}
\usepackage{amsthm}
\usepackage{mathtools}
\usepackage{mathrsfs}
\usepackage{stmaryrd}
\usepackage{extarrows}
\usepackage{enumerate}
\usepackage{tensor}
\usepackage{physics2}
  \usephysicsmodule{ab,xmat}
\usepackage{fixdif}
  \newcommand{\dd}{\d}
\usepackage{derivative}
  \newcommand{\dv}{\odv}
  \newcommand{\pd}[1]{\pdv{}{#1}}
  \newcommand{\eval}[1]{#1\big|}
\usepackage{graphicx}
\usepackage{subcaption}
\usepackage{tikz}
\usepackage{tikz-3dplot}
% \usepackage{tikz-cd}
% \usepackage{quiver}
% ========== Block quiver.sty ========== %
\usepackage{tikz-cd}
% \usepackage{amssymb}
\usetikzlibrary{calc}
\usetikzlibrary{decorations.pathmorphing}
\tikzset{curve/.style={settings={#1},to path={(\tikztostart)
    .. controls ($(\tikztostart)!\pv{pos}!(\tikztotarget)!\pv{height}!270:(\tikztotarget)$)
    and ($(\tikztostart)!1-\pv{pos}!(\tikztotarget)!\pv{height}!270:(\tikztotarget)$)
    .. (\tikztotarget)\tikztonodes}},
    settings/.code={\tikzset{quiver/.cd,#1}
        \def\pv##1{\pgfkeysvalueof{/tikz/quiver/##1}}},
    quiver/.cd,pos/.initial=0.35,height/.initial=0}
\tikzset{tail reversed/.code={\pgfsetarrowsstart{tikzcd to}}}
\tikzset{2tail/.code={\pgfsetarrowsstart{Implies[reversed]}}}
\tikzset{2tail reversed/.code={\pgfsetarrowsstart{Implies}}}
\tikzset{no body/.style={/tikz/dash pattern=on 0 off 1mm}}
% =========== End block ========== %
  \tikzset{every picture/.style={line width=0.75pt}}
\usepackage{pgfplots}
  \pgfplotsset{compat=newest}
\usepackage{tcolorbox}
  \tcbuselibrary{most}
\usepackage[colorlinks=true,linkcolor=blue]{hyperref}
\usepackage{cleveref}
% \usepackage[hyperref=true,backend=biber,style=alphabetic,backref=true,url=false]{biblatex}
\usepackage[warnings-off={mathtools-colon,mathtools-overbracket}]{unicode-math}
\usepackage[default,amsbb]{fontsetup}
  \setmathfont[StylisticSet=1,range=\mathscr]{NewCMMath-Book.otf}
\usepackage{fancyhdr}
\usepackage{import}

\newcommand{\Id}{\mathbb{1}}
\newcommand{\lap}{\increment}

\DeclareMathOperator{\sign}{sign}
\DeclareMathOperator{\dom}{dom}
\DeclareMathOperator{\ran}{ran}
\DeclareMathOperator{\ord}{ord}
\DeclareMathOperator{\Span}{span}
\DeclareMathOperator{\img}{Im}
\DeclareMathOperator{\Ric}{Ric}
\newcommand{\card}{\texttt{\#}}
\newcommand{\ie}{\emph{i.e.}}
\newcommand{\st}{\emph{s.t.}}
\newcommand{\eps}{\varepsilon}
\newcommand{\vphi}{\varphi}
\newcommand{\vthe}{\vartheta}
\newcommand{\II}{I\!I}
\renewcommand{\emptyset}{⌀}
\newcommand{\acts}{\curvearrowright}
\newcommand{\xrr}{\xlongrightarrow}
\newcommand{\lrr}{\longrightarrow}
\newcommand{\lmt}{\longmapsto}
\newcommand{\into}{\hookrightarrow}
\newcommand{\op}{\operatorname}

\let\originalleft\left
\let\originalright\right
\renewcommand{\left}{\mathopen{}\mathclose\bgroup\originalleft}
\renewcommand{\right}{\aftergroup\egroup\originalright}

\theoremstyle{plain}\newtheorem{theorem}{Theorem}
\theoremstyle{definition}\newtheorem{definition}[theorem]{Definition}
\theoremstyle{definition}\newtheorem{example}[theorem]{Example}
\theoremstyle{definition}\newtheorem{problem}[theorem]{Problem}
\theoremstyle{plain}\newtheorem{axiom}[theorem]{Axiom}
\theoremstyle{plain}\newtheorem{corollary}[theorem]{Corollary}
\theoremstyle{plain}\newtheorem{lemma}[theorem]{Lemma}
\theoremstyle{plain}\newtheorem{proposition}[theorem]{Proposition}
\theoremstyle{plain}\newtheorem{prop}[theorem]{Proposition}
\theoremstyle{plain}\newtheorem{conjecture}[theorem]{Conjecture}
\theoremstyle{plain}\newtheorem{conj}[theorem]{Conjecture}
\theoremstyle{remark}\newtheorem{notation}[theorem]{Notation}
\theoremstyle{definition}\newtheorem*{question}{Question}
\theoremstyle{definition}\newtheorem*{answer}{Answer}
\theoremstyle{definition}\newtheorem*{goal}{Goal}
\theoremstyle{definition}\newtheorem*{application}{Application}
\theoremstyle{plain}\newtheorem*{exercise}{Exercise}
\theoremstyle{remark}\newtheorem*{remark}{Remark}
\theoremstyle{remark}\newtheorem*{note}{\small{Note}}
\numberwithin{equation}{section}
\numberwithin{theorem}{section}
\numberwithin{figure}{section}

\usepackage{xeCJK}
\setCJKmainfont{FZShuSong-Z01}[BoldFont=FZXiaoBiaoSong-B05,ItalicFont=FZKai-Z03]
\setCJKsansfont{FZXiHeiI-Z08}[BoldFont=FZHei-B01]
\setCJKmonofont{FZFangSong-Z02}
\setCJKfamilyfont{zhsong}{FZShuSong-Z01}[BoldFont=FZXiaoBiaoSong-B05]
\setCJKfamilyfont{zhhei}{FZHei-B01}
\setCJKfamilyfont{zhkai}{FZKai-Z03}
\setCJKfamilyfont{zhfs}{FZFangSong-Z02}
\setCJKfamilyfont{zhli}{FZLiShu-S01}
\setCJKfamilyfont{zhyou}{FZXiYuan-M01}[BoldFont=FZZhunYuan-M02]

\allowdisplaybreaks{}

\newcommand{\isFullBook}[2]{
  \ifnum\pdfstrcmp{\FullBook}{True}=0
    \ifnum\pdfstrcmp{}{#1}=0\unskip\else#1\fi
  \else
    \ifnum\pdfstrcmp{}{#2}=0\unskip\else#2\fi
  \fi\ignorespaces{}
}

\counterwithout{theorem}{section}
\counterwithout{equation}{section}

\begin{document}
Let \(\pi\colon E\to M\) be a smooth vector bundle of rank \(r\) (over \(\mathbb{R}\)
or \(\mathbb{C}\)). We put \[
    C^\infty(M,E):=\{s\colon M\to E:\pi\circ s=\Id_M\}
,\] the set of all \(C^\infty\)-sections of \(E\).

\begin{center}
\includestandalone{../figures/1-2_vector_bundle}
\end{center}

\(C^\infty(M,E)\) is an infinity
dimensional vector space, and is also a module over the ring \(C^\infty(M)\) of 
all smooth functions. (For \(\mathbb{C}\) we denote by \(C_{\mathbb{C}}^\infty(M)\) 
the ring of all complex valued smooth functions).

\begin{definition}
    A linear map \[
        \nabla\colon C^\infty(M,E)\longrightarrow C^\infty(M,E\otimes T^*M)
    \] is called a \textbf{linear connection} or simply \textbf{connection} if for
    any \(f\in C^\infty(M)\) and \(s \in C^\infty(M,E)\), \[
        \nabla (fs)=f\nabla s+s\otimes \dd{f}
    .\] 
\end{definition}
\begin{definition}
    We define \(\nabla_X s\) for a vector field \(X\in C^\infty(M,TM)\) by the
    contraction of \(T^*M\) and \(TM\),
    \[\begin{tikzcd}[row sep=tiny]
        C^\infty(M,E\otimes T^*M)\times C^\infty(M,TM) & C^\infty(M,E) \\
        (\nabla s,x) & \nabla_X s 
        \arrow[from=1-1, to=1-2]
        \arrow[maps to, from=2-1, to=2-2]
    \end{tikzcd}\]
    \(\nabla\) is also called the \textbf{covariant derivative} and \(\nabla_X s\) the
    covariant derivative in the direction of \(X\).
\end{definition}
\begin{example}
    Let \(E=M\times \mathbb{R}\to M\) the trivial line bundle of rank 1. Then
    \(C^\infty(M,E)=C^\infty(M)\). For any section \(s\), \ie\ a smooth function
    \(s\), the exterior derivative \[
        \dd{s}\in C^\infty(M,E\otimes T^*M)=C^\infty(M,T^*M)
    \] defines a connection \(\nabla\) be putting \[
        \nabla s=\dd{s}
    .\] In this case, \[
        \nabla_X s=\left<\dd{s},X\right> =Xs
    \] is the usual derivative of \(s\) by the vector field \(X\).
\end{example}

For a general vector bundle \(E\), there is no simple covariant
derivative such as the exterior derivative \(\dd\) since \(E\) is 
not a trivial bundle in general.

But recall \(E\) has ``local triviality'': \[
    \vphi_\lambda\colon E\Big|_{U_\lambda}=\pi^{-1}(U_\lambda)
    \xlongrightarrow{\sim} U_\lambda\times \mathbb{R}^r
.\] Let \(e_1,\ldots,e_r\) be a basis of \(\mathbb{R}^r\) (or
\(\mathbb{C}^r\)), and consider the sections \(e_{\lambda,i},i=1,\ldots,
r\) defined by \begin{align*}
    \vphi_\lambda^{-1}\colon U_\lambda\times \mathbb{R}^r
    &\longrightarrow E\Big|_{U_\lambda} \\
    (p,e_i) &\longmapsto e_{\lambda,i}(p)=\vphi_\lambda^{-1}(p,e_i)
.\end{align*}
Such \(e_{\lambda 1},\ldots,e_{\lambda r}\) are called local frame
or local frame field.

At each point \(p\in U_\lambda\),
\(e_{\lambda 1}(p),\ldots,e_{\lambda r}(p)\) form a basis of
\(E_p=\pi^{-1}(p)\), the fiber over \(p\).
Any section \(s \in C^\infty(M,E)\) is expressed over \(U_\lambda\) as
\begin{align*}
    s&=s^1e_{\lambda 1}+\cdots +s^r e_{\lambda r} \\
    &= (e_{\lambda 1},\ldots,e_{\lambda r})\begin{pmatrix}
        s^1 \\ \vdots \\ s^r
    \end{pmatrix}
\end{align*}
where \(s^1,\ldots,s^r\in C^\infty(U_\lambda)\).

By the definition of the connection \[
    \nabla s=s^1\nabla e_{\lambda 1}+\cdots +s^r\nabla e_{\lambda r}
    +e_{\lambda 1}\otimes \dd{s^1}+\cdots +e_{\lambda r}\otimes \dd{s^r}
.\] For another open set \(U_\mu,\mu\neq \lambda\), we have a local frame
\(e_{\mu 1},\ldots,e_{\mu r}\), and similarly \[
    s=t^1 e_{\mu 1}+\cdots +t^r e_{\mu r}\quad
    \text{with }t^1,\ldots,t^r\in C^\infty(U_\mu)
.\] And \[
    \nabla s=t^1\nabla e_{\mu 1}+\cdots +t^r\nabla e_{\mu r}
    +e_{\mu 1}\otimes \dd{t^1}+\cdots +e_{\mu r}\otimes \dd{t^r}
.\] When \(U_\lambda \cap U_\mu\neq \emptyset\), we have \(\tensor{f}
{_\lambda_\mu^i_j}\in C^\infty(U_\lambda \cap U_\mu)\), such that \[
    e_{\mu j}=\tensor{f}{_\lambda_\mu^1_j} e_{\lambda 1}+\cdots +
    \tensor{f}{_\lambda_\mu^r_j}e_{\lambda r}
,\] so that
\begin{equation}\label{eq:1-2-1}
(e_{\mu 1},\ldots,e_{\mu r})=(e_{\lambda 1},\ldots,e_{\lambda r})
f_{\lambda \mu}
.\end{equation}
That is, \[
    f_{\lambda \mu}=(\tensor{f}{_\lambda_\mu^i_j})
\] is the non-singular matrix of ``change of basis''.

By \cref{eq:1-2-1} we have \[
    \nabla e_{\mu j}=\sum_{i=1}^{r}(\tensor{f}{_\lambda_\mu^i_j}\nabla 
    e_{\lambda i}+e_{\lambda i}\otimes \dd{\tensor{f}{_\lambda_\mu^i_j}})
.\] \ie\ 
\begin{equation}\label{eq:1-2-2}
    \nabla e_\mu=\nabla e_\lambda\cdot f_{\lambda\mu}+e_\lambda\otimes 
    \dd{f_{\lambda\mu}}
.\end{equation}

\begin{definition}
    On \(U_\lambda\), \(\nabla e_{\lambda i} \in C^\infty(U_\lambda,
    E\Big|_{U_\lambda}\otimes T^*M)\), so that there are 1-forms
    \(\tensor{\omega}{_\lambda^i_j}, i,j=1,\ldots,r\) such that \[
        \nabla e_{\lambda j}=\sum_{i=1}^{r}e_{\lambda i}\otimes 
        \tensor{\omega}{_\lambda^i_j}
    .\] Hence \[
        \nabla e_\lambda=e_\lambda\otimes \omega_\lambda, \quad
        \text{ with }\omega_\lambda=(\tensor{\omega}{_\lambda^i_j})
    .\] We call \(\omega_\lambda=(\tensor{\omega}{_\lambda^i_j})\) the
    \textbf{connector 1-form} or \textbf{connection form}, or
    \textbf{connection matrix}.
\end{definition}
The connection form is also referred as ``matrix-valued 1-form''.

\begin{prop}
    Over \(U_\lambda\cap U_\mu\), \(\omega_\lambda\) and \(\omega_\mu\)
    are related by
    \begin{equation}\label{eq:1-2-3}
        \omega_\mu=f_{\lambda\mu}^{-1}\omega_\mu f_{\lambda\mu}
        +f_{\lambda\mu}^{-1}\dd{f_{\lambda\mu}}
    .\end{equation}
\end{prop}
\begin{proof}
    By \cref{eq:1-2-2},
    \begin{equation}\label{eq:1-2-4}
        \nabla e_\mu=e_\lambda\otimes \omega_\lambda f_{\lambda\mu}
        +e_{\lambda}\otimes f_{\lambda\mu}
    .\end{equation}
    On the other hand by \cref{eq:1-2-1}, we also have
    \begin{equation}\label{eq:1-2-5}
        \nabla_{e_\mu}=e_{\mu}\otimes \omega_\mu
        =e_\lambda f_{\lambda\mu}\otimes \omega_\mu
    .\end{equation}
    Comparing \cref{eq:1-2-4,eq:1-2-5}, we obtain \[
        f_{\lambda\mu}\omega_\mu=\omega_\lambda f_{\lambda\mu}
        +\dd{f_{\lambda\mu}}
    .\] This implies \cref{eq:1-2-3}
\end{proof}

Tracing back the proof above, we obtain the converse
\begin{prop}
    Suppose we are given matrix-valued 1-forms \(\omega_\lambda\) on
    \(U_\lambda\), \(\lambda\in \Lambda\) satisfying \cref{eq:1-2-3}.
    If we define \(\nabla\) by \[
        \nabla e_{\lambda j}=\sum_{i=1}^{r} e_{\lambda i}\otimes 
        \tensor{\omega}{_\lambda^i_j}
    ,\] \ie\ \[
        \nabla e_\lambda=e_\lambda\otimes\omega_\lambda
    ,\] then \(\nabla\) defines a well-defined connection on \(E\).
\end{prop}

\begin{definition}
    For a given connection \(\nabla\) with connection form \(\omega
    _\lambda\) on \(U_\lambda\), we define the \textbf{curvature form}
    \(\Omega_\lambda\) on \(U_\lambda\) by \[
        \Omega_\lambda=\dd{\omega_\lambda}+\omega_\lambda\wedge
        \omega_\lambda
    .\] Here \(\omega_\lambda\wedge \omega_\lambda\) is computed by the
    multiplication of matrices and wedge product is taken as differential
    forms.
\end{definition}
Note that the definition depends on the choice of local frame
\(e_{\lambda 1},\ldots,e_{\lambda r}\) on \(U_\lambda\).

On \(U_\lambda\), omitting \(\lambda\) in formula, we have \[
    \tensor{\Omega}{^i_j}=\dd{\tensor{\omega}{^i_j}}
    +\sum_{k=1}^{r}\tensor{\omega}{^i_k}\wedge \tensor{\omega}{^k_j}
.\] 

\begin{prop}
    On \(U_\lambda\cap U_\mu\neq \emptyset\), we have \[
        \Omega_\mu=f_{\lambda\mu}^{-1}\Omega_{\lambda}f_{\lambda\mu}
    .\] 
\end{prop}
\begin{proof}
    Recall \[
        \omega_\mu=f_{\lambda\mu}^{-1}\omega_\mu f_{\lambda\mu}
        +f_{\lambda\mu}^{-1}\dd{f_{\lambda\mu}}
    \] and since \(f_{\lambda\mu}f_{\lambda\mu}^{-1}=1\), \[
        \dd{f_{\lambda\mu}}^{-1}=-f_{\lambda\mu}^{-1}\cdot
        \dd{f_{\lambda\mu}}\cdot f_{\lambda\mu}^{-1}
    .\] The proposition follows easily from these.
\end{proof}

Thus, at \(p\in U_\lambda\cap U_\mu\), \(\Omega_\lambda\) and
\(\Omega_\mu\) define the same endomorphism of the fiber \(E_p\).
So \(\{\Omega_\lambda\}_{\lambda\in \Lambda}\) defines a section of \[
    \operatorname{End}(E)\otimes \wedge^2T^*M\cong E\otimes E^*\otimes 
    \wedge^2T^*M
.\] 

Another description of the curvature is given by \[
    R(X,Y)=\nabla_X\nabla_Y-\nabla_Y\nabla_X-\nabla_{[X,Y]}
\] for any vector field \(X\) and \(Y\).
\begin{exercise}
    One can check \[
        R(fX,gY)(hs)=fghR(X,Y)s
    \] for any section \(s \in C^\infty(M,E)\) and smooth functions
    \(f,g,h\).
\end{exercise}
Then we see \(R(\cdot ,\cdot )\) is a tensor in
\(\operatorname{End}(E)\otimes \wedge^2M\).

\begin{prop}
    Let \(e_{\lambda 1},\ldots,e_{\lambda r}\) be a local frame on
    \(U_\lambda\), then \(\Omega_\lambda\) is the matrix representation
    of \(R\) with respect to \(e_{\lambda 1},\ldots,e_{\lambda r}\).
\end{prop}
\begin{proof}
    We omit \(\lambda\) in subscript, and write \(e=(e_1,\ldots,e_r)\).
    We have
    \begin{align*}
        &(\nabla_X\nabla_Y-\nabla_Y\nabla_X-\nabla_{[X,Y]})e \\
        =& \nabla_X(e\omega(Y))-\nabla_Y(e\omega(X))-e\omega(X)\omega(Y)
        -e(Y\omega(X)-e\omega([X,Y])) \\
        =&e((\dd{\omega}+\omega\wedge \omega)(X,Y)) \\
        =&e\Omega(X,Y)
    .\end{align*}
\end{proof}
\begin{definition}
    \(R\) is called the \textbf{curvature tensor}
\end{definition}

\end{document}
