% !TeX program = xelatex
\documentclass[12pt]{article}
\usepackage{standalone}

\usepackage[dvipsnames,svgnames,x11names]{xcolor}
\usepackage[a4paper,margin=1in]{geometry}
\usepackage{microtype}
\usepackage{amsmath}
\usepackage{amsthm}
\usepackage{mathtools}
\usepackage{mathrsfs}
\usepackage{stmaryrd}
\usepackage{extarrows}
\usepackage{enumerate}
\usepackage{tensor}
\usepackage{physics}
\usepackage{graphicx}
\usepackage{subcaption}
\usepackage{tikz}
\usepackage{tikz-3dplot}
% \usepackage{tikz-cd}
\usepackage{quiver}
  \tikzset{every picture/.style={line width=0.75pt}}
\usepackage{pgfplots}
  \pgfplotsset{compat=newest}
\usepackage{tcolorbox}
  \tcbuselibrary{most}
\usepackage[colorlinks=true,linkcolor=blue]{hyperref}
\usepackage{cleveref}
\usepackage[hyperref=true,backend=biber,style=alphabetic,backref=true,url=false]{biblatex}
\usepackage[warnings-off={mathtools-colon,mathtools-overbracket}]{unicode-math}
\usepackage[default]{fontsetup}
\usepackage{fancyhdr}
\usepackage{import}

\newcommand{\Id}{\mathbb{1}}
\newcommand{\lap}{\increment}

\DeclareMathOperator{\sign}{sign}
\DeclareMathOperator{\dom}{dom}
\DeclareMathOperator{\ran}{ran}
\DeclareMathOperator{\ord}{ord}
\DeclareMathOperator{\Span}{span}
\DeclareMathOperator{\img}{Im}
\DeclareMathOperator{\Ric}{Ric}
\newcommand{\card}{\texttt{\#}}
\newcommand{\ie}{\emph{i.e.}}
\newcommand{\st}{\emph{s.t.}}
\newcommand{\eps}{\varepsilon}
\newcommand{\vphi}{\varphi}
\newcommand{\vthe}{\vartheta}
\newcommand{\II}{I\!I}
\renewcommand{\emptyset}{\varnothing}
\newcommand{\acts}{\curvearrowright}
\newcommand{\xrr}{\xlongrightarrow}
\newcommand{\into}{\hookrightarrow}
\newcommand{\pdif}[2]{\frac{\partial #1}{\partial #2}}
\renewcommand{\op}{\operatorname}

\theoremstyle{plain}\newtheorem{theorem}{Theorem}
\theoremstyle{definition}\newtheorem{definition}[theorem]{Definition}
\theoremstyle{definition}\newtheorem{example}[theorem]{Example}
\theoremstyle{plain}\newtheorem{axiom}[theorem]{Axiom}
\theoremstyle{plain}\newtheorem{corollary}[theorem]{Corollary}
\theoremstyle{plain}\newtheorem{lemma}[theorem]{Lemma}
\theoremstyle{plain}\newtheorem{proposition}[theorem]{Proposition}
\theoremstyle{plain}\newtheorem{prop}[theorem]{Proposition}
\theoremstyle{plain}\newtheorem{conjecture}[theorem]{Conjecture}
\theoremstyle{plain}\newtheorem{conj}[theorem]{Conjecture}
\theoremstyle{plain}\newtheorem{problem}[theorem]{Problem}
\theoremstyle{remark}\newtheorem{notation}[theorem]{Notation}
\theoremstyle{definition}\newtheorem*{question}{Question}
\theoremstyle{definition}\newtheorem*{answer}{Answer}
\theoremstyle{definition}\newtheorem*{goal}{Goal}
\theoremstyle{plain}\newtheorem*{application}{Application}
\theoremstyle{plain}\newtheorem*{exercise}{Exercise}
\theoremstyle{remark}\newtheorem*{remark}{Remark}
\theoremstyle{remark}\newtheorem*{note}{\small{Note}}
\numberwithin{equation}{section}
\numberwithin{theorem}{section}
\numberwithin{figure}{section}

\usepackage{xeCJK}
\setCJKmainfont{FZShuSong-Z01}[BoldFont=FZXiaoBiaoSong-B05,ItalicFont=FZKai-Z03]
\setCJKsansfont{FZXiHeiI-Z08}[BoldFont=FZHei-B01]
\setCJKmonofont{FZFangSong-Z02}
\setCJKfamilyfont{zhsong}{FZShuSong-Z01}[BoldFont=FZXiaoBiaoSong-B05]
\setCJKfamilyfont{zhhei}{FZHei-B01}
\setCJKfamilyfont{zhkai}{FZKai-Z03}
\setCJKfamilyfont{zhfs}{FZFangSong-Z02}
\setCJKfamilyfont{zhli}{FZLiShu-S01}
\setCJKfamilyfont{zhyou}{FZXiYuan-M01}[BoldFont=FZZhunYuan-M02]

\geometry{a4paper,margin=1in}
\allowdisplaybreaks{}

\counterwithout{theorem}{section}
\counterwithout{equation}{section}

\begin{document}
So we define the tensor \(R_{i\bar{j}}\) by
\begin{align*}
  R_{i\bar{j}}&=\op{Kähler-Ric}\left(\pd{z^i},\pd{\bar{z}^j}\right) \\
  &=g^{k\bar{l}}R_{i\bar{l}\bar{j}k}=-g^{k\bar{l}}R_{i\bar{j}k\bar{l}} \\
  &=R\indices{_{i\bar{j}}^k_k}=R\indices{^k_{ki\bar{j}}} \\
  &=(\op{tr}R)_{i\bar{j}}
.\end{align*}

In local coordinates,
\begin{align*}
  R_{i\bar{j}}&=-g^{k\bar{l}}R_{i\bar{j}k\bar{l}}
  =-g^{k\bar{l}}R_{k\bar{l}i\bar{j}} \\
  &=g^{k\bar{l}}\pdv{g_{k\bar{l}}}{z^i,\bar{z}^j}+g^{k\bar{l}}g^{p\bar{q}}
  \pdv{g_{k\bar{q}}}{z^i}\pdv{g_{p\bar{l}}}{\bar{z}^j} \\
  &=-\pdv{}{z^i,\bar{z}^j}\log \det(g_{k\bar{l}})
.\end{align*}
As a peculiar terminology in K\"ahler geometry, ``K\"ahler-Ricci'' curvature
is referred to as the \textbf{Ricci curvature}. As we saw before, \[
  \frac{\sqrt{-1}}{2\pi}R_{i\bar{j}}\dd{z^i}\wedge \dd{\bar{z}^j}
  =-\frac{\sqrt{-1}}{2\pi}\partial\bar{\partial}\log\det(g_{k\bar{l}})
\] represents the first Chern class of \(T'M\), or equivalently, \(c_1(T'M)=
-c_1(K_M)\) since \(h:=\det(g_{i\bar{j}})\) defines a Hermitian metric on
\(K_M^{-1}=\bigwedge^m T'M\).

Warning: The majority of Kähler geometers use the following conventions:
{\color{Red}
\begin{align*}
  R_{i\bar{j}k\bar{l}}&=-\pdv{g_{i\bar{j}}}{z^k,\bar{z}^l}+g^{p\bar{q}}
  \pdv{g_{i\bar{q}}}{z^k}\pdv{g_{p\bar{j}}}{\bar{z}^l} \\
  &\text{(This is opposite sign compared to ours)} \\
  R_{i\bar{j}}&=R_{i\bar{j}}\underbracket{\indices{_k^k}}_{\mathclap{
  \text{opposite}}}=-\pdv{}{z^i,\bar{z}^j}
.\end{align*}}
It is up to you which convention you use.

\section{Bochner technique}
Now we turn to Bochner technique. The most primitive form of Bochner technique is
of the Killing vector fields on compact Riemannian manifolds of negative Ricci
curvature.

An isometry of a Riemannian manifold \((M,g)\) is a diffeomorphism \(f\colon M
\to M\) such that \[
  f^*g=g
.\] If a vector field \(X\) generates a flow \(f_t=\exp(tX)\) consisting of
isometries so that \[
  f_t^*g=g,\quad \forall\,t,
\] we say that \(X\) is a \textbf{Killing vector field}.

Taking the derivative of above equation with respect to \(t\) we get \[
  \mathcal{L}_X g=0
.\] This implies
\begin{align*}
  X(g(Y,Z))&=(\mathcal{L}_X g)(Y,Z)+g(\mathcal{L}_X Y,Z)+g(Y,\mathcal{L}_X Z) \\
  &=g(\nabla_X Y-\nabla_Y X,Z)+g(Y,\nabla_X Z-\nabla_Z X)
.\end{align*}
But \[
  X(g(Y,Z))=g(\nabla_X Y,Z)+g(Y,\nabla_X Z)
.\] Thus \(X\) is Killing if and only if \[
  g(\nabla_Y X,Z)+g(Y,\nabla_Z X)=0,\quad \forall\,Y,Z
.\]

Now we spend some time on the standard notation in physics and differential
geometry (or tensor calculus).

Let \(T\) be a tensor field in \(C^\infty(M,TM^{\otimes p}\otimes T^*M^{\otimes
q})\). Then we express \(T\) locally using Einstein convention as \[
  T=T\indices{^{i_1\cdots i_p}_{j_1\cdots j_q}}\pd{x^{i_1}}\otimes\cdots\otimes 
  \pd{x^{i_p}}\otimes \dd{x^{j_1}}\otimes \cdots \otimes \dd{x^{j_q}}
.\] If \(\alpha\) is a \(q\)-form, it is expressed as either \[
  \alpha=\sum_{j_1<\cdots <j_q}\alpha_{j_1\cdots j_q}\dd{x^{j_1}}\wedge \cdots 
  \wedge \dd{x^{j_q}},
\] or \[
  \alpha=\frac{1}{q!}\sum_{j_1,\ldots,j_q}\alpha_{j_1\cdots j_q}\dd{x^{j_1}}
  \wedge\cdots\wedge\dd{x^{j_q}}
\] where \[
  \alpha_{j_1'\cdots j'_q}=\sign\begin{pmatrix}
    j_1' & \cdots & j_q' \\
    j_1 & \cdots & j_q
  \end{pmatrix}\alpha_{j_1,\ldots,j_q}\text{ whenever }\{j_1',\ldots,j_q'\}
  =\{j_1,\ldots,j_q\}
.\] If we take the covariant derivative \(\nabla T\) of \(T\) we express \[
  \nabla T=\nabla_k T\indices{^{i_1\cdots i_p}_{j_1\cdots j_q}}\pd{x^{i_1}}
  \otimes\cdots\otimes\pd{x^{i_p}}\otimes \dd{x^k}\dd{x^{j_1}}\otimes
  \cdots \otimes \dd{x^{j_q}}
.\] Note that \(\nabla_k T\indices{^{i_1\cdots i_p}_{j_1\cdots j_q}}\) is 
the notation of just the coefficient of \(\nabla T\). It looks like the \(k\)-th
partial derivative of smooth function \(T\indices{^{i_1\cdots i_p}_{j_1\cdots
j_q}}\), but the meaning is not so.

Recall one defined Riemannian curvature tensor \(R_{ijkl}\) by \[
  R_{ijkl}=g\left(\pd{x^k},\left(\nabla_{\pd{x^i}}\nabla_{\pd{x^j}}-
  \nabla_{\pd{x^j}}\nabla_{\pd{x^i}}\right)\pd{x^k}\right)
.\] Putting \[
  R\indices{_{ij}^p_l}=R_{ijkl}g^{kp}
,\] we have \[
  \left(\nabla_{\pd{x^i}}\nabla_{\pd{x^j}}-\nabla_{\pd{x^j}}\nabla_{\pd{x^i}}
  \right)\nabla_{\pd{x^l}}=R\indices{_{ij}^p_l}\pd{x^p}=R\indices{^p_{lij}}
  \pd{x^p}
.\] In the last term I used the symmetry of Riemannian curvature tensor. I did
this because I have been using the convention of the curvature 2-form as a
section of \(C^\infty(\op{End}(TM)\otimes \bigwedge^2T^*M)\). Then for
\(Z=Z^i \pd{x^i}\), \[
  \nabla_{\pd{x^i}}\nabla_{\pd{x^j}}Z-\nabla_{\pd{x^j}}\nabla_{\pd{x^i}}Z
  =Z^k R\indices{^l_{kij}}\pd{x^l}
.\] This means \[
  \nabla_i \nabla_j Z^l-\nabla_j \nabla_i Z^l=R\indices{^l_{kij}}Z^k=
  R\indices{_{ij}^l_k}Z^k
.\] This equality is called the \textbf{Ricci identity}. This is the same as the
definition of the curvature tensor. This says that if you want to change the
order of covariant derivative, curvature term appears.

For a 1-form \(\alpha=\alpha_i \dd{x^i}\) one can compute
\begin{align*}
  &\phantom{=}\left<\nabla_{\pd{x^i}}\nabla_{\pd{x^j}}\alpha-\nabla_{\pd{x^j}}
  \nabla_{\pd{x^i}}\alpha,\pd{x^k}\right> \\
  &=\pd{x^i}\left<\nabla_{\pd{x^j}}\alpha,\pd{x^k}\right>
  -\left<\nabla_{\pd{x^i}}\alpha,\nabla_{\pd{x^i}}\pd{x^k}\right> \\
  &\phantom{=}-\pd{x^j}\left<\nabla_{\pd{x^i}}\alpha,\pd{x^k}\right>
  +\left<\nabla_{\pd{x^i}}\alpha,\nabla_{\pd{x^j}}\pd{x^k}\right> \\
  &=\cdots \\
  &=-\left<\alpha,\left(\nabla_{\pd{x^i}}\nabla_{\pd{x^j}}
  -\nabla_{\pd{x^j}}\nabla_{\pd{x^i}}\right),\pd{x^k}\right> \\
  &=-\left<\alpha,R\indices{_{ij}^p_k}\pd{x^p}\right> =-R\indices{_{ij}^p_k}
  \alpha_p
.\end{align*}
Thus \[
  \nabla_i \nabla_j \alpha_k-\nabla_j \nabla_i \alpha_k=-R\indices{_{ij}^p_k}
  \alpha_p
.\] This is the Ricci identity for \(1\)-forms. but this can be obtained from 
the Ricci identity for vector fields as follows:

First note that \(\nabla g=0\). This is because \[
  (\nabla_X g)(Y,Z)=X(g(Y,Z))-g(\nabla_X Y,Z)-g(Y,\nabla_X Z)=0
.\] So \(\nabla_k g_{ij}=0\) and also \(\nabla_k g^{ij}=0\). There is an
isomorphism between \(TM\) and \(T^*M\), called the musical isomorphism,
\begin{align*}
  \flat\colon TM &\longrightarrow T^*M \\
  X^i\pd{x^i} &\longmapsto g_{ki}X^i\dd{x^k}=X_k\dd{x^k}
.\end{align*}
\begin{align*}
  \sharp\colon T^*M &\longrightarrow TM \\
  \alpha^i\dd{x^i} &\longmapsto g^{ki}\alpha_i\pd{x^k}=\alpha^k\pd{x^k}
.\end{align*}
Here we have put \(X_k=g_{ki}X^i\) and \(\alpha^k=g^{ki}\alpha_i\). Then
since \(\nabla_k g_{ij}=\nabla_k g^{ij}=0\), we obtain 
\begin{align*}
  (\nabla_i \nabla_j-\nabla_j \nabla_i)\alpha_k
  &=(\nabla_i \nabla_j-\nabla_j \nabla_i)(g_{kl}\alpha^l) \\
  &=g_{kl}(\nabla_i \nabla_j-\nabla_j \nabla_i)\alpha^l \\
  &=g_{kl}R\indices{_{ij}^l_p}\alpha^p=R_{ijkp}\alpha^p=R\indices{_{ijk}^l}
  \alpha_l \\
  &=-R\indices{_{ij}^l_k}\alpha_l
.\end{align*}
For tensor product \(X\otimes \alpha\) we have \[
  (\nabla_i \nabla_j-\nabla_j\nabla_i)(X^k\alpha_l)=R\indices{_{ij}^k_p}X^p
  \alpha_l-R\indices{_{ij}^p_l}X^l\alpha_p
.\] So, for \(T=T\indices{^k_l}\pd{x^k}\otimes \dd{x^l}\), \[
  (\nabla_i \nabla_j-\nabla_j\nabla_i)(T\indices{^k_l})
  =R\indices{_{ij}^k_p}T\indices{^p_l}-R\indices{_{ij}^p_l}T\indices{^k_p}
.\] For higher order tensors, \[
  T=T\indices{^{i_1\cdots i_p}_{j_1\cdots j_q}}\pd{x^{i_1}}\otimes\cdots\otimes 
  \pd{x^{i_p}}\otimes \dd{x^{j_1}}\otimes \cdots \otimes \dd{x^{j_q}}
.\] The Ricci identity is similar, with \(p+q\) terms on the right hand side.

Here is one warning: It is not recommended to write upper indices in this way: \[
  R_{jkl}^i
,\] because this is unclear whether \[
  g_{pi}R_{jkl}^i=R_{pjkl}\text{ or }R_{jpkl}
\] The have opposite signs. You should write \(R\indices{^i_{jkl}}\) and
\(R\indices{_j^i_{kl}}\). In tex, you can use the \emph{tensor} package, and
write
\begin{verbatim}
 R\indices{^j_i_k_l}
 R\indices{_j^i_k_l}
\end{verbatim}



\end{document}
