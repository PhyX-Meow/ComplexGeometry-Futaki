% !TeX program = xelatex
\documentclass[12pt]{article}
\usepackage{standalone}

\usepackage[dvipsnames,svgnames,x11names]{xcolor}
\usepackage[a4paper,margin=1in]{geometry}
\usepackage{microtype}
\usepackage{amsmath}
\usepackage{amsthm}
\usepackage{mathtools}
\usepackage{mathrsfs}
\usepackage{stmaryrd}
\usepackage{extarrows}
\usepackage{enumerate}
\usepackage{tensor}
\usepackage{physics2}
  \usephysicsmodule{ab,xmat}
\usepackage{fixdif}
  \newcommand{\dd}{\d}
\usepackage{derivative}
  \newcommand{\dv}{\odv}
  \newcommand{\pd}[1]{\pdv{}{#1}}
  \newcommand{\eval}[1]{#1\big|}
\usepackage{graphicx}
\usepackage{subcaption}
\usepackage{tikz}
\usepackage{tikz-3dplot}
% \usepackage{tikz-cd}
% \usepackage{quiver}
% ========== Block quiver.sty ========== %
\usepackage{tikz-cd}
% \usepackage{amssymb}
\usetikzlibrary{calc}
\usetikzlibrary{decorations.pathmorphing}
\tikzset{curve/.style={settings={#1},to path={(\tikztostart)
    .. controls ($(\tikztostart)!\pv{pos}!(\tikztotarget)!\pv{height}!270:(\tikztotarget)$)
    and ($(\tikztostart)!1-\pv{pos}!(\tikztotarget)!\pv{height}!270:(\tikztotarget)$)
    .. (\tikztotarget)\tikztonodes}},
    settings/.code={\tikzset{quiver/.cd,#1}
        \def\pv##1{\pgfkeysvalueof{/tikz/quiver/##1}}},
    quiver/.cd,pos/.initial=0.35,height/.initial=0}
\tikzset{tail reversed/.code={\pgfsetarrowsstart{tikzcd to}}}
\tikzset{2tail/.code={\pgfsetarrowsstart{Implies[reversed]}}}
\tikzset{2tail reversed/.code={\pgfsetarrowsstart{Implies}}}
\tikzset{no body/.style={/tikz/dash pattern=on 0 off 1mm}}
% =========== End block ========== %
  \tikzset{every picture/.style={line width=0.75pt}}
\usepackage{pgfplots}
  \pgfplotsset{compat=newest}
\usepackage{tcolorbox}
  \tcbuselibrary{most}
\usepackage[colorlinks=true,linkcolor=blue]{hyperref}
\usepackage{cleveref}
% \usepackage[hyperref=true,backend=biber,style=alphabetic,backref=true,url=false]{biblatex}
\usepackage[warnings-off={mathtools-colon,mathtools-overbracket}]{unicode-math}
\usepackage[default,amsbb]{fontsetup}
  \setmathfont[StylisticSet=1,range=\mathscr]{NewCMMath-Book.otf}
\usepackage{fancyhdr}
\usepackage{import}

\newcommand{\Id}{\mathbb{1}}
\newcommand{\lap}{\increment}

\DeclareMathOperator{\sign}{sign}
\DeclareMathOperator{\dom}{dom}
\DeclareMathOperator{\ran}{ran}
\DeclareMathOperator{\ord}{ord}
\DeclareMathOperator{\Span}{span}
\DeclareMathOperator{\img}{Im}
\DeclareMathOperator{\Ric}{Ric}
\newcommand{\card}{\texttt{\#}}
\newcommand{\ie}{\emph{i.e.}}
\newcommand{\st}{\emph{s.t.}}
\newcommand{\eps}{\varepsilon}
\newcommand{\vphi}{\varphi}
\newcommand{\vthe}{\vartheta}
\newcommand{\II}{I\!I}
\renewcommand{\emptyset}{⌀}
\newcommand{\acts}{\curvearrowright}
\newcommand{\xrr}{\xlongrightarrow}
\newcommand{\lrr}{\longrightarrow}
\newcommand{\lmt}{\longmapsto}
\newcommand{\into}{\hookrightarrow}
\newcommand{\op}{\operatorname}

\let\originalleft\left
\let\originalright\right
\renewcommand{\left}{\mathopen{}\mathclose\bgroup\originalleft}
\renewcommand{\right}{\aftergroup\egroup\originalright}

\theoremstyle{plain}\newtheorem{theorem}{Theorem}
\theoremstyle{definition}\newtheorem{definition}[theorem]{Definition}
\theoremstyle{definition}\newtheorem{example}[theorem]{Example}
\theoremstyle{definition}\newtheorem{problem}[theorem]{Problem}
\theoremstyle{plain}\newtheorem{axiom}[theorem]{Axiom}
\theoremstyle{plain}\newtheorem{corollary}[theorem]{Corollary}
\theoremstyle{plain}\newtheorem{lemma}[theorem]{Lemma}
\theoremstyle{plain}\newtheorem{proposition}[theorem]{Proposition}
\theoremstyle{plain}\newtheorem{prop}[theorem]{Proposition}
\theoremstyle{plain}\newtheorem{conjecture}[theorem]{Conjecture}
\theoremstyle{plain}\newtheorem{conj}[theorem]{Conjecture}
\theoremstyle{remark}\newtheorem{notation}[theorem]{Notation}
\theoremstyle{definition}\newtheorem*{question}{Question}
\theoremstyle{definition}\newtheorem*{answer}{Answer}
\theoremstyle{definition}\newtheorem*{goal}{Goal}
\theoremstyle{definition}\newtheorem*{application}{Application}
\theoremstyle{plain}\newtheorem*{exercise}{Exercise}
\theoremstyle{remark}\newtheorem*{remark}{Remark}
\theoremstyle{remark}\newtheorem*{note}{\small{Note}}
\numberwithin{equation}{section}
\numberwithin{theorem}{section}
\numberwithin{figure}{section}

\usepackage{xeCJK}
\setCJKmainfont{FZShuSong-Z01}[BoldFont=FZXiaoBiaoSong-B05,ItalicFont=FZKai-Z03]
\setCJKsansfont{FZXiHeiI-Z08}[BoldFont=FZHei-B01]
\setCJKmonofont{FZFangSong-Z02}
\setCJKfamilyfont{zhsong}{FZShuSong-Z01}[BoldFont=FZXiaoBiaoSong-B05]
\setCJKfamilyfont{zhhei}{FZHei-B01}
\setCJKfamilyfont{zhkai}{FZKai-Z03}
\setCJKfamilyfont{zhfs}{FZFangSong-Z02}
\setCJKfamilyfont{zhli}{FZLiShu-S01}
\setCJKfamilyfont{zhyou}{FZXiYuan-M01}[BoldFont=FZZhunYuan-M02]

\allowdisplaybreaks{}

\newcommand{\isFullBook}[2]{
  \ifnum\pdfstrcmp{\FullBook}{True}=0
    \ifnum\pdfstrcmp{}{#1}=0\unskip\else#1\fi
  \else
    \ifnum\pdfstrcmp{}{#2}=0\unskip\else#2\fi
  \fi\ignorespaces{}
}

\counterwithout{theorem}{section}
\counterwithout{equation}{section}

\begin{document}
If we express \(g\) as \[
    g=\sum_{i,j=1}^{n}g_{ij}\dd{x^i}\otimes \dd{x^j},\quad
    g_{ij}=g\left(\pd{x^i},\pd{x^j}\right)
.\] Then the matrix \((g_{ij})\) is a positive definite symmetric matrix at each point
\(p\in M\).

\(g\) is  also expressed as \[
    g=g_{ij}\dd{x^i}\dd{x^j}
.\] In this expression, \(\dd{x^i}\dd{x^j}\) should be understood as the symmetric
product \[
    \dd{x^i}\dd{x^j}=\frac{1}{2}(\dd{x^i}\otimes \dd{x^j}+\dd{x^j}\otimes \dd{x^i})
.\] This expression lacks accuracy, but is convenient.

\begin{definition}
    The pair \((M,g)\) is called a \textbf{Riemannian manifold}.
\end{definition}
There are many Riemannian metrics, one can construct many using partition of unity.

\begin{theorem}
    On a Riemannian manifold \((M,g)\) there is a unique connection \(\nabla\) on
    the tangent bundle \(TM\), called the \textbf{Levi-Civita connection} or \textbf{
    Riemannian connection}, such that
    \begin{enumerate}[(1)]
    \item For any vector fields \(X\) and \(Y\), \[
        \nabla_X Y-\nabla_Y X=[X,Y]
    .\] 
    \item For any vector fields \(X,Y,Z\), \[
        Xg(Y,Z)=g(\nabla_X Y,Z)+g(Y,\nabla_X,Z)
    .\] 
    \end{enumerate}
\end{theorem}
\begin{remark}\hfill
\begin{enumerate}[(1)]
    \item is referred as ``torsion-free'' or ``symmetric''.
    \item is referred as ``metric compatible''.
\end{enumerate}
\end{remark}

We can prove the existence and uniqueness of the Levi-Civita connection as follows
\begin{proof}
    Suppose \(\nabla\) exists, then by (2) \(\nabla\) should satisfy
    \begin{align}
        Xg(Y,Z)&=g(\nabla_X Y,Z)+g(Y,\nabla_X,Z) \tag{3} \\
        Yg(Z,X)&=g(\nabla_Y Z,X)+g(Z,\nabla_Y,X) \tag{4} \\
        Zg(X,Y)&=g(\nabla_Z X,Y)+g(X,\nabla_Z,Y) \tag{5}
    .\end{align}
    Compute \(\text{(3)}+\text{(4)}-\text{(5)}\) using (1) we get
    \begin{align*}
        &Xg(Y,Z)+Yg(Z,X)-Zg(X,Y) \\
        =&g(\nabla_X Y+\nabla_Y X,Z)+g([Y,Z],X)+g([X,Z],Y) \\
        =&2g(\nabla_X Y,Z)-g([X,Y],Z)+g([Y,Z],X)+g([X,Z],Y)
    .\end{align*}
    Thus 
    \begin{align*}
        g(\nabla_X Y,Z)=&\frac{1}{2}\big(Xg(Y,Z)+Yg(Z,X)-Zg(X,Y) \\
        &+g([X,Y],Z)-g([Y,Z],X)-g([X,Z],Y)\big)
    .\end{align*}
    \(\nabla_X,Y\) is defined by this equation. From this expression one get
    uniqueness. One can check this \(\nabla\) satisfies all the conditions of
    Levi-Civita connection (Exercise), which gives existence.
\end{proof}

For local coordinates \(x^1,\ldots,x^n\) we have \[
    [\pd{x^i},\pd{x^j}]=0
.\] So
\begin{equation}\label{eq:2-2:6}
    g\left(\nabla_{\pd{x^i}}\pd{x^j},\pd{x^k}\right)
    =\frac{1}{2}\left(\pdv{g_{jk}}{x^i}+\pdv{g_{ik}}{x^j}-\pdv{g_{ij}}{x^k}\right)
.\end{equation}
We set \[
    \nabla_{\pd{x^i}}\pd{x^j}=\Gamma_{ij}^k\pd{x^k}
.\] These \(\Gamma_{ij}^k\) are called the \textbf{Christoffel symbols}.

If we set \((g^{ij})=(g_{ij})^{-1}\), the inverse matrix, then \cref{eq:2-2:6} shows \[
    \Gamma_{ij}^k=\frac{1}{2}g^{jp}\left(\pdv{g_{jp}}{x^i}+\pdv{g_{ip}}{x^j}
    -\pdv{g_{ij}}{x^p}\right)
.\] Of course \[
    \nabla\pd{x^j}=\Gamma_{ij}^k\pd{x^k}\otimes \dd{x^i}
.\] So \[
    \omega=(\tensor{\omega}{^i_j})=(\Gamma_{kj}^i\dd{x^k})
\] is the connection matrix with respect to \(\pd{x^1},\ldots,\pd{x^k}\).

Note that the torsion-freeness is equivalent to \(\Gamma_{ij}^k=\Gamma_{ji}^k\). This
is the reason why torsion-free is referred as ``symmetric''.

Recall that the curvature tensor \(R\) was defined by \[
    R(X,Y)=\nabla_X\nabla_Y-\nabla_Y\nabla_X-\nabla_{[X,Y]}
.\] In Riemannian geometry, another \(R\) is defined by \[
    R(X,Y,Z,W)=g(Z,R(X,Y)W)
.\] This new \(R\) is a tensor field in \(C^\infty(M,\otimes^4 T^*M)\), and called
the \textbf{Riemannian curvature tensor}. (There are many different conventions of
\(R\), but our convention gives the standard sign convention).
\begin{prop}
    Show the following:
    \begin{enumerate}[(a)]
    \item \(R(X,Y,Z,W)=-R(Y,X,Z,W)\).
    \item \(R(X,Y,Z,W)=R(Z,W,X,Y)\).
    \item \(R(X,Y,Z,W)+R(Y,W,Z,X)+R(W,X,Z,Y)=0\), (The 1st Bianchi identity).
    \item \((\nabla_X R)(Y,Z)=(\nabla_Y R)(Z,X)+(\nabla_Z R)(X,Y)=0\),
        (The 2nd Bianchi identity).
    \end{enumerate}
\end{prop}
\begin{proof}
    Left as exercise.
\end{proof}

\begin{definition}
    Let \(X,Y\in T_p M\) be orthonormal, that is \(\|X\|=\|Y\|=1\) and \(g(X,Y)=0\),
    (\(\|X\|=\sqrt{g(X,X)}\)). Then \(R(X,Y,X,Y)\) is called the \textbf{sectional
    curvature} of the plane spanned by \(X\) and \(Y\).
\end{definition}
It is easy to check if  \[
    X'=X\cos\theta+Y\sin\theta,\quad Y'=-X\sin\theta+Y\cos\theta
,\] then \[
    R(X',Y',X',Y')=R(X,Y,X,Y)
.\] Thus the sectional curvature is a function on the Grassmannian bundle
\(\op{Gr}(2,TM)\) of 2-planes.

\begin{definition}
    Let \(e_1,\ldots,e_n\) be an orthonormal frame of \(T_p M\), we define
    \(\Ric\in C^\infty(M,\otimes^2 T^*M)\) by \[
        \Ric(X,Y)=\sum_{i=1}^{n}R(X,e_i,Y,e_i)
    .\] \(\Ric\) is a symmetric 2-tensor, and is called the \textbf{Ricci curvature}.
    It is usually expressed as \[
        \Ric=R_{ij}\dd{x^i}\otimes \dd{x^j}
    ,\] and \(R_{ij}\) is also called the Ricci curvature.

    \[
        S:=\sum_{i,j=1}^{n}R(e_i,e_j,e_i,e_j)=g^{ij}R_{ij}=g^{ik}g^{jl}R_{ijkl}
    ,\] is called the \textbf{scalar curvature}. For \(n=2\), that is, for surfaces,
    sectional curvature \(R_{1212}\) and the Ricci curvature \(R_{11}=R_{22}\) and
    the scalar curvature \(S\) coincide up to multiple constants, to classical
    Gaussian curvature.
\end{definition}

\end{document}
