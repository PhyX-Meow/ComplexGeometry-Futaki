% !TeX program = xelatex
\documentclass[12pt]{article}
\usepackage{standalone}

\usepackage[dvipsnames,svgnames,x11names]{xcolor}
\usepackage[a4paper,margin=1in]{geometry}
\usepackage{microtype}
\usepackage{amsmath}
\usepackage{amsthm}
\usepackage{mathtools}
\usepackage{mathrsfs}
\usepackage{stmaryrd}
\usepackage{extarrows}
\usepackage{enumerate}
\usepackage{tensor}
\usepackage{physics2}
  \usephysicsmodule{ab,xmat}
\usepackage{fixdif}
  \newcommand{\dd}{\d}
\usepackage{derivative}
  \newcommand{\dv}{\odv}
  \newcommand{\pd}[1]{\pdv{}{#1}}
  \newcommand{\eval}[1]{#1\big|}
\usepackage{graphicx}
\usepackage{subcaption}
\usepackage{tikz}
\usepackage{tikz-3dplot}
% \usepackage{tikz-cd}
% \usepackage{quiver}
% ========== Block quiver.sty ========== %
\usepackage{tikz-cd}
% \usepackage{amssymb}
\usetikzlibrary{calc}
\usetikzlibrary{decorations.pathmorphing}
\tikzset{curve/.style={settings={#1},to path={(\tikztostart)
    .. controls ($(\tikztostart)!\pv{pos}!(\tikztotarget)!\pv{height}!270:(\tikztotarget)$)
    and ($(\tikztostart)!1-\pv{pos}!(\tikztotarget)!\pv{height}!270:(\tikztotarget)$)
    .. (\tikztotarget)\tikztonodes}},
    settings/.code={\tikzset{quiver/.cd,#1}
        \def\pv##1{\pgfkeysvalueof{/tikz/quiver/##1}}},
    quiver/.cd,pos/.initial=0.35,height/.initial=0}
\tikzset{tail reversed/.code={\pgfsetarrowsstart{tikzcd to}}}
\tikzset{2tail/.code={\pgfsetarrowsstart{Implies[reversed]}}}
\tikzset{2tail reversed/.code={\pgfsetarrowsstart{Implies}}}
\tikzset{no body/.style={/tikz/dash pattern=on 0 off 1mm}}
% =========== End block ========== %
  \tikzset{every picture/.style={line width=0.75pt}}
\usepackage{pgfplots}
  \pgfplotsset{compat=newest}
\usepackage{tcolorbox}
  \tcbuselibrary{most}
\usepackage[colorlinks=true,linkcolor=blue]{hyperref}
\usepackage{cleveref}
% \usepackage[hyperref=true,backend=biber,style=alphabetic,backref=true,url=false]{biblatex}
\usepackage[warnings-off={mathtools-colon,mathtools-overbracket}]{unicode-math}
\usepackage[default,amsbb]{fontsetup}
  \setmathfont[StylisticSet=1,range=\mathscr]{NewCMMath-Book.otf}
\usepackage{fancyhdr}
\usepackage{import}

\newcommand{\Id}{\mathbb{1}}
\newcommand{\lap}{\increment}

\DeclareMathOperator{\sign}{sign}
\DeclareMathOperator{\dom}{dom}
\DeclareMathOperator{\ran}{ran}
\DeclareMathOperator{\ord}{ord}
\DeclareMathOperator{\Span}{span}
\DeclareMathOperator{\img}{Im}
\DeclareMathOperator{\Ric}{Ric}
\newcommand{\card}{\texttt{\#}}
\newcommand{\ie}{\emph{i.e.}}
\newcommand{\st}{\emph{s.t.}}
\newcommand{\eps}{\varepsilon}
\newcommand{\vphi}{\varphi}
\newcommand{\vthe}{\vartheta}
\newcommand{\II}{I\!I}
\renewcommand{\emptyset}{⌀}
\newcommand{\acts}{\curvearrowright}
\newcommand{\xrr}{\xlongrightarrow}
\newcommand{\lrr}{\longrightarrow}
\newcommand{\lmt}{\longmapsto}
\newcommand{\into}{\hookrightarrow}
\newcommand{\op}{\operatorname}

\let\originalleft\left
\let\originalright\right
\renewcommand{\left}{\mathopen{}\mathclose\bgroup\originalleft}
\renewcommand{\right}{\aftergroup\egroup\originalright}

\theoremstyle{plain}\newtheorem{theorem}{Theorem}
\theoremstyle{definition}\newtheorem{definition}[theorem]{Definition}
\theoremstyle{definition}\newtheorem{example}[theorem]{Example}
\theoremstyle{definition}\newtheorem{problem}[theorem]{Problem}
\theoremstyle{plain}\newtheorem{axiom}[theorem]{Axiom}
\theoremstyle{plain}\newtheorem{corollary}[theorem]{Corollary}
\theoremstyle{plain}\newtheorem{lemma}[theorem]{Lemma}
\theoremstyle{plain}\newtheorem{proposition}[theorem]{Proposition}
\theoremstyle{plain}\newtheorem{prop}[theorem]{Proposition}
\theoremstyle{plain}\newtheorem{conjecture}[theorem]{Conjecture}
\theoremstyle{plain}\newtheorem{conj}[theorem]{Conjecture}
\theoremstyle{remark}\newtheorem{notation}[theorem]{Notation}
\theoremstyle{definition}\newtheorem*{question}{Question}
\theoremstyle{definition}\newtheorem*{answer}{Answer}
\theoremstyle{definition}\newtheorem*{goal}{Goal}
\theoremstyle{definition}\newtheorem*{application}{Application}
\theoremstyle{plain}\newtheorem*{exercise}{Exercise}
\theoremstyle{remark}\newtheorem*{remark}{Remark}
\theoremstyle{remark}\newtheorem*{note}{\small{Note}}
\numberwithin{equation}{section}
\numberwithin{theorem}{section}
\numberwithin{figure}{section}

\usepackage{xeCJK}
\setCJKmainfont{FZShuSong-Z01}[BoldFont=FZXiaoBiaoSong-B05,ItalicFont=FZKai-Z03]
\setCJKsansfont{FZXiHeiI-Z08}[BoldFont=FZHei-B01]
\setCJKmonofont{FZFangSong-Z02}
\setCJKfamilyfont{zhsong}{FZShuSong-Z01}[BoldFont=FZXiaoBiaoSong-B05]
\setCJKfamilyfont{zhhei}{FZHei-B01}
\setCJKfamilyfont{zhkai}{FZKai-Z03}
\setCJKfamilyfont{zhfs}{FZFangSong-Z02}
\setCJKfamilyfont{zhli}{FZLiShu-S01}
\setCJKfamilyfont{zhyou}{FZXiYuan-M01}[BoldFont=FZZhunYuan-M02]

\allowdisplaybreaks{}

\newcommand{\isFullBook}[2]{
  \ifnum\pdfstrcmp{\FullBook}{True}=0
    \ifnum\pdfstrcmp{}{#1}=0\unskip\else#1\fi
  \else
    \ifnum\pdfstrcmp{}{#2}=0\unskip\else#2\fi
  \fi\ignorespaces{}
}

\counterwithout{theorem}{section}
\counterwithout{equation}{section}

\begin{document}

\ifdefined\FullBook{}\else
\begin{theorem}
  Given a Hermitian metric \(h\) on a holomorphic vector bundle \(E\), the connection
  form \(\omega\) of the Chern connection is given by \[
    \omega=h^{-1}\partial h=\left(h^{i\bar{k}}\partial h_{\bar{k}j}\right)
  ,\] and the curvature form is given by \[
    \boxed{\Omega=\bar{\partial}\omega=\bar{\partial}(h^{-1}\partial h)}
  \] with respect to a local frame \(e_1,\ldots,e_r\), where \[
    h_{\bar{i}j}=h(\bar{e}_i,e_j),h^{i\bar{k}}h_{\bar{k}j}=\delta^i_j
  .\] 
\end{theorem}
\begin{proof}
  We have seen already \(\omega=h^{-1}\partial h\). For curvature,
  \begin{align*}
    \Omega&=\dd{\omega}+\omega\wedge \omega \\ 
    &=(\partial+\bar{\partial})h^{-1}\partial h
    +h^{-1}\partial h\wedge h^{-1}\partial h \\
    &=\underline{-h^{-1}\partial h\cdot h^{-1}\wedge \partial h}
    -h^{-1}\bar{\partial}h\cdot h^{-1}\wedge \partial h \\
    &\phantom{=\ } +h^{-1}\bar{\partial}\partial h
    +\underline{h^{-1}\partial h\wedge h^{-1}\partial h} \\
    &=\bar{\partial}(h^{-1}\partial h)
  .\end{align*}
  Note that \(\partial h^{-1}=-h^{-1}\partial h\cdot h^{-1}\)
\end{proof}
\fi

\begin{corollary}
  Let \(L\to M\) be a holomorphic line bundle with a Hermitian metric \(h\), so that
  \(h\) is \(1\times 1\) matrix with respect to a nowhere zero holomorphic section
  \(s\) on an open set \(U\): \[
    h_U=h(\bar{s},s)
  .\] Then the curvature 2-form is given by \[
    \Omega=-\partial\bar{\partial}\log h_U
  .\] 
\end{corollary}

\begin{lemma}
  \(\partial\bar{\partial}\log h_U\) is independent of the local section \(s\) and
  defines a global 2-form.
\end{lemma}
\begin{proof}
  If \(t\) is another nowhere zero holomorphic section on \(V\), over \(U\cap V\) there
  is a nowhere zero holomorphic function \(\vphi\) such that \[
    t=\vphi s
  .\] Then \[
    h_V=h(\bar{t},t)=h(\overline{\vphi s},\vphi s)=|\vphi|^2 h_U
  .\] Thus
  \begin{align*}
    \partial\bar{\partial}\log h_V&=\partial\bar{\partial}\log\vphi+
    \partial\bar{\partial}\log \bar{\vphi}+\partial\bar{\partial}h_U \\
    &=\partial\bar{\partial}\log h_U
  .\end{align*} 
\end{proof}

By this lemma, there is no confusion even if we write \(\Omega=-\partial\bar{\partial}
\log h\) instead of \(h_U\).

\begin{definition}
  \[
    c_1(L,h)=\frac{\sqrt{-1}}{2\pi}\Omega=-\frac{\sqrt{-1}}{2\pi}\partial\bar{\partial}
    \log h
  \] is called the first Chern form of \((L,h)\).
\end{definition}
\begin{theorem}\hfill
\begin{enumerate}[(1)]
\item \(c_1(L,h)\) is a real closed \((1,1)\)-form.
\item The de Rham class \([c_1(L,h)]\) is independent of choice of the metric \(h\).
\end{enumerate}
\end{theorem}
\begin{proof}
\begin{enumerate}[(1)]
\item It is clear that \(c_1(L,h)\) is closed and of type \((1,1)\). It is real since
  \[
    \overline{\sqrt{-1}\partial\bar{\partial}\log h}=-\sqrt{-1}\bar{\partial}\partial 
    \log h=\sqrt{-1}\partial\bar{\partial}\log h
  .\] 
\item Let \(h'\) be another Hermitian metric, choose \(s\) and \(t\) as in the proof
  of previous lemma. Then \[
    h_V=|\vphi|^2 h_U,\quad h'_V=|\vphi|^2h'_U
  .\] So \[
    \frac{h_V}{h'_V}=\frac{h_U}{h'_U}
  .\] This implies that \(f:=\frac{h_U}{h'_U}\) is a globally defined positive
  smooth function on \(M\). Thus 
  \begin{align*}
    c_1(L,h)-c_1(L,h')&=-\frac{\sqrt{-1}}{2\pi}\partial\bar{\partial}\log
    \frac{h_U}{h'_U}=-\frac{\sqrt{-1}}{2\pi}\partial\bar{\partial}\log f\\
    &=\dd\left(-\frac{\sqrt{-1}}{2\pi}\bar{\partial}\log f\right)
  .\end{align*}
\end{enumerate}
\end{proof}

\begin{definition}
  The de Rham class \[
    c_1(L):=[c_1(L,h)] \in H^2(M;\mathbb{R})
  \] is called the \textbf{first Chern class} of \(L\).
\end{definition}
\begin{remark}
  This class is in fact an integral class, \ie\ lie in the image of \[
    H^2(M;\mathbb{Z})\longrightarrow H^2(M;\mathbb{R})
  \] coming from the short exact sequence \[
    0\to \mathbb{Z}\to \mathbb{R}\to \mathbb{R}/\mathbb{Z}\to 0
  .\] There are other formulations of Chern classes, see for example
  \emph{Milnor-Stasheff, Bott-Tu, \ldots}.
\end{remark}

\begin{example}
  Consider \[
    \mathcal{O}_{\mathbb{P}^m}(k)=[H]^{\otimes k}=[kH]\to \mathbb{P}^m(\mathbb{C})
  \] with \(m=1\). As divisors and line bundles are identified, we will remove
  \([\ ]\) and write \(H\) instead of \([H]\), and call hyperplane bundle. As in the
  homework \[
    H^{-1}=L=\mathcal{O}(-1)
  .\] This line bundle seems to have many names, but we call \(L=\mathcal{O}(-1)\)
  the tautological line bundle after \emph{Milnor-Stasheff}.

  As was seen in the note 2 of week 3, \[
    h_{U_0}=\frac{1}{(1+|s|^2)^k}
  \] defines a Hermitian metric \(h\) on \(\mathcal{O}_{\mathbb{P}^1}(k)\). So \[
    c_1(\mathcal{O}_{\mathbb{P}^1}(k),h)=-\frac{\sqrt{-1}}{2\pi}\partial\bar{\partial}
    \log \frac{1}{(1+|s|^2)^k}
    =\frac{k\sqrt{-1}}{2\pi}\cdot \frac{\dd{s}\wedge \dd{\bar{s}}}{(1+|s|^2)^2}
  .\] \[
    \left<c_1(\mathcal{O}_{\mathbb{P}^1}(k)),[\mathbb{P}^1(\mathbb{C})]\right> 
    =\int_{U_0}\frac{k\sqrt{-1}}{2\pi}\cdot\frac{\dd{s}\wedge\dd{\bar{s}}}{(1+|s|^2)^2}
    =k
  .\]
  The verification of last integral is left as exercise.
\end{example}

\begin{corollary}
  \(\left<c_1(T'\mathbb{P}^1(\mathbb{C})),[\mathbb{P}^1(\mathbb{C})]\right> =2\)
  since \(T'\mathbb{P}^1(\mathbb{C})=\mathcal{O}(2)\).
\end{corollary}

\begin{remark}
  This ``2'' is the Euler characteristic of \(\mathbb{P}^1(\mathbb{C})\cong
  \mathbb{S}^2\). It is a general fact that for \(\dim_{\mathbb{C}}M=m\), \[
    c_m(M)=c_m(T'M)=\chi(M)
  .\] See \textbf{Milnor-Stasheff}.
\end{remark}

\begin{definition}
  Let \(M\) be an \(m\)-dimensional compact complex manifold. Recall \[
    K_M=\bigwedge^m T^{*\prime}M=\Big(\bigwedge^m T'M\Big)^{-1}
  .\] We define the first Chern class \(c_1(M)\) of \(M\) by \[
    c_1(M)=-c_1(K_M)=+c_1\Big(\bigwedge^m T'M\Big)
  .\] 
\end{definition}

\begin{definition}
  Let \(E\to M\) be a holomorphic vector bundle of rank \(r\), \(h\) be a Hermitian
  metric so that the curvature 2-form is expressed as \[
    \Omega=\bar{\partial}(h^{-1}\partial h)
  \] for a local holomorphic frame. Define \(2k\)-forms \(c_k(h)\) be \[
    \det(I+\frac{t\sqrt{-1}}{2\pi}\Omega)=1+tc_1(h)+\cdots +t^r c_r(h)
  .\] Where \(I=(\delta^i_j)\) is the identity matrix.
  
  Since \(c_k(h)\) is closed \(2k\)-form and the de Rham class \([c_k(h)]\) is
  independent of \(h\), we define \[
    c_k(E)=[c_k(h)]
  \] and call it the \textbf{\(k\)-th Chern class} of \(E\).
\end{definition}

\begin{remark}
  We postpone the proof of the fact that the de Rham class \([c_k(h)]\) is independent
  of \(h\). This fact is a part of Chern-Weil theory.
\end{remark}

\begin{definition}
  \(c_k(M):=c_k(T'M)\) is called the \(k\)-th Chern class of \(M\).  
\end{definition}
\begin{lemma}\hfill
\begin{enumerate}[(1)]
\item \[
    c_1(E)=\frac{\sqrt{-1}}{2\pi}\bar{\partial}h(h^{-1}\partial h)
    =-\frac{\sqrt{-1}}{2\pi}\partial\bar{\partial}\log\det h
  .\] 
\item \(\det h\) gives a Hermitian metric of \(\bigwedge^r E\).
\item The two definitions \(c_1(M)=c_1(T'M)\) and \(c_1(M)=-c_1(K_M)\) coincide.
\end{enumerate}
\end{lemma}
\begin{proof}
  Left as exercise.
\end{proof}

\ifdefined\FullBook{}
\begin{theorem}\label{thm:4-1:A}
\else
\begin{theorem}[A]
\fi
  Let \(L_D\) be the line bundle associated with a divisor \(D\). Then \(c_1(L_D)\) is
  the Poincaré dual of the homology class \([D]\) represented by \(D\). That is to say,
  for any 2-dimensional homology class \(\alpha\), we have \[
    \left<c_1(L_D),\alpha\right> =[D]\cdot\alpha
  .\] Note \([D]\in H_{2m-2}(M)\), \(\alpha\in H_2(M)\), and the ``\(\cdot\)'' means
  the homological intersection.
\end{theorem}

We prove this for \(m=1\) (so \(M\) is a compact Riemann surface). You may consider
how you can extend the proof to general dimension (exercise).
\begin{theorem}[1 dim case of theorem A]
  Let \(M\) be a compact 1-dimensional complex manifold. Let \(D=\{p_1,\ldots,p_l\}\)
  be distinct \(l\) points on \(M\), \ie\ a divisor. Then \[
    \left<c_1(L_D),[M]\right> =l\quad(=[D]\cdot M)
  .\] 
\end{theorem}
\begin{proof}
  Take an open covering \(M=\bigcup_{\lambda\in \Lambda}U_\lambda\) such that
  \(p_1\in U_1,\ldots,p_l\in U_l\) and that no other \(U_\lambda\) contains
  \(p_1,\ldots,p_l\).
  \begin{center}
    \includestandalone{../figures/4-1_divisor}
  \end{center}
  There exists holomorphic function \(f_\lambda\) on \(U_\lambda\) such that \[
    D\cap U_\lambda=\{f_\lambda=0\},\ j=1,\ldots,l
  \] and \(f_j\) vanishes at \(p_j\) with first order. The transition function
  of \(L_D\) is \[
    f_{\lambda\mu}=\frac{f_\lambda}{f_\mu}
  .\] Thus \[
    f_{\lambda}=f_{\lambda\mu}f_{\mu}
  \] and \(\{f_{\lambda}\}_{\lambda\in \Lambda}\) gives a holomorphic section \(s\)
  of \(L_D\). We also have that zero set of \(s\) is exactly \(\{p_1,\ldots,p_l\}\).
  Next we take a Hermitian metric of \(h\) so that \[
    h_{U_{\lambda}}\equiv 1,\ \lambda=1,\ldots,l
  \] with respect to local trivialization of \(L_{D}\Big|_{U_{\lambda}}\).

  Note this is allowed because \(c_1(L_D)\) is independent of \(h\) and for any
  positive function on \(M\), \(fh\) is again a Hermitian metric.

  Then \(\frac{\sqrt{-1}}{2\pi}\log h(\bar{s},s)\) is a smooth function on
  \(M\setminus\{p_1,\ldots,p_l\}\). Further, over \(U_\lambda\setminus\{p_\lambda\}\),
  \begin{align*}
    \partial\bar{\partial}\frac{\sqrt{-1}}{2\pi}\log h(\bar{s},s)
    &=\frac{\sqrt{-1}}{2\pi}\partial\bar{\partial}\log(h_U \bar{s}_U s_U)
    =\frac{\sqrt{-1}}{2\pi}\partial\bar{\partial}\log h_U \\
    &=-c_1(L,h)
  .\end{align*} Thus
  \begin{align*}
    \left<c_1(L),[M]\right> &=\lim_{\eps \to 0} \int_{M\setminus\bigcup_{i=1}^l
    B(p_i,\eps)}-\frac{\sqrt{-1}}{2\pi}\partial\bar{\partial}\log h(\bar{s},s) \\
    &\text{(where }B(p_i,\eps)\text{ is the disk of radius }\eps\text{ around }p_i
    \text{)} \\
    &=\lim_{\eps \to 0} \int_{M\setminus\bigcup_{i=1}^l B(p_i,\eps)}
    -\frac{1}{2\pi\sqrt{-1}}\dd\partial\log h(\bar{s},s) \\
    &\overset{\mathclap{\text{Stokes}}}{=}\hspace{.5em}\lim_{\eps \to 0} \sum_{i=1}^{l}
    \int_{\partial B(p_i,\eps)}\frac{1}{2\pi\sqrt{-1}}\partial\log h(\bar{s},s) \\
    &=\lim_{\eps \to 0} \sum_{i}\int_{|s|=\eps}\frac{1}{2\pi\sqrt{-1}}\partial\log s \\
    &\text{(here we used }h_U\equiv 1,s=s_U\text{)} \\
    &=\frac{l}{2\pi\sqrt{-1}}\lim_{\eps\to 0} \int_{|s|=\eps}\frac{\dd{s}}{s}=l
  .\end{align*} 
\end{proof}

\begin{remark}
  This is a variant of Gauss-Bonnet theorem.
\end{remark}

\end{document}
