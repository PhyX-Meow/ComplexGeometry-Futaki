% !TeX program = xelatex
\documentclass[12pt]{article}
\usepackage{standalone}

\usepackage[dvipsnames,svgnames,x11names]{xcolor}
\usepackage[a4paper,margin=1in]{geometry}
\usepackage{microtype}
\usepackage{amsmath}
\usepackage{amsthm}
\usepackage{mathtools}
\usepackage{mathrsfs}
\usepackage{stmaryrd}
\usepackage{extarrows}
\usepackage{enumerate}
\usepackage{tensor}
\usepackage{physics2}
  \usephysicsmodule{ab,xmat}
\usepackage{fixdif}
  \newcommand{\dd}{\d}
\usepackage{derivative}
  \newcommand{\dv}{\odv}
  \newcommand{\pd}[1]{\pdv{}{#1}}
  \newcommand{\eval}[1]{#1\big|}
\usepackage{graphicx}
\usepackage{subcaption}
\usepackage{tikz}
\usepackage{tikz-3dplot}
% \usepackage{tikz-cd}
% \usepackage{quiver}
% ========== Block quiver.sty ========== %
\usepackage{tikz-cd}
% \usepackage{amssymb}
\usetikzlibrary{calc}
\usetikzlibrary{decorations.pathmorphing}
\tikzset{curve/.style={settings={#1},to path={(\tikztostart)
    .. controls ($(\tikztostart)!\pv{pos}!(\tikztotarget)!\pv{height}!270:(\tikztotarget)$)
    and ($(\tikztostart)!1-\pv{pos}!(\tikztotarget)!\pv{height}!270:(\tikztotarget)$)
    .. (\tikztotarget)\tikztonodes}},
    settings/.code={\tikzset{quiver/.cd,#1}
        \def\pv##1{\pgfkeysvalueof{/tikz/quiver/##1}}},
    quiver/.cd,pos/.initial=0.35,height/.initial=0}
\tikzset{tail reversed/.code={\pgfsetarrowsstart{tikzcd to}}}
\tikzset{2tail/.code={\pgfsetarrowsstart{Implies[reversed]}}}
\tikzset{2tail reversed/.code={\pgfsetarrowsstart{Implies}}}
\tikzset{no body/.style={/tikz/dash pattern=on 0 off 1mm}}
% =========== End block ========== %
  \tikzset{every picture/.style={line width=0.75pt}}
\usepackage{pgfplots}
  \pgfplotsset{compat=newest}
\usepackage{tcolorbox}
  \tcbuselibrary{most}
\usepackage[colorlinks=true,linkcolor=blue]{hyperref}
\usepackage{cleveref}
% \usepackage[hyperref=true,backend=biber,style=alphabetic,backref=true,url=false]{biblatex}
\usepackage[warnings-off={mathtools-colon,mathtools-overbracket}]{unicode-math}
\usepackage[default,amsbb]{fontsetup}
  \setmathfont[StylisticSet=1,range=\mathscr]{NewCMMath-Book.otf}
\usepackage{fancyhdr}
\usepackage{import}

\newcommand{\Id}{\mathbb{1}}
\newcommand{\lap}{\increment}

\DeclareMathOperator{\sign}{sign}
\DeclareMathOperator{\dom}{dom}
\DeclareMathOperator{\ran}{ran}
\DeclareMathOperator{\ord}{ord}
\DeclareMathOperator{\Span}{span}
\DeclareMathOperator{\img}{Im}
\DeclareMathOperator{\Ric}{Ric}
\newcommand{\card}{\texttt{\#}}
\newcommand{\ie}{\emph{i.e.}}
\newcommand{\st}{\emph{s.t.}}
\newcommand{\eps}{\varepsilon}
\newcommand{\vphi}{\varphi}
\newcommand{\vthe}{\vartheta}
\newcommand{\II}{I\!I}
\renewcommand{\emptyset}{⌀}
\newcommand{\acts}{\curvearrowright}
\newcommand{\xrr}{\xlongrightarrow}
\newcommand{\lrr}{\longrightarrow}
\newcommand{\lmt}{\longmapsto}
\newcommand{\into}{\hookrightarrow}
\newcommand{\op}{\operatorname}

\let\originalleft\left
\let\originalright\right
\renewcommand{\left}{\mathopen{}\mathclose\bgroup\originalleft}
\renewcommand{\right}{\aftergroup\egroup\originalright}

\theoremstyle{plain}\newtheorem{theorem}{Theorem}
\theoremstyle{definition}\newtheorem{definition}[theorem]{Definition}
\theoremstyle{definition}\newtheorem{example}[theorem]{Example}
\theoremstyle{definition}\newtheorem{problem}[theorem]{Problem}
\theoremstyle{plain}\newtheorem{axiom}[theorem]{Axiom}
\theoremstyle{plain}\newtheorem{corollary}[theorem]{Corollary}
\theoremstyle{plain}\newtheorem{lemma}[theorem]{Lemma}
\theoremstyle{plain}\newtheorem{proposition}[theorem]{Proposition}
\theoremstyle{plain}\newtheorem{prop}[theorem]{Proposition}
\theoremstyle{plain}\newtheorem{conjecture}[theorem]{Conjecture}
\theoremstyle{plain}\newtheorem{conj}[theorem]{Conjecture}
\theoremstyle{remark}\newtheorem{notation}[theorem]{Notation}
\theoremstyle{definition}\newtheorem*{question}{Question}
\theoremstyle{definition}\newtheorem*{answer}{Answer}
\theoremstyle{definition}\newtheorem*{goal}{Goal}
\theoremstyle{definition}\newtheorem*{application}{Application}
\theoremstyle{plain}\newtheorem*{exercise}{Exercise}
\theoremstyle{remark}\newtheorem*{remark}{Remark}
\theoremstyle{remark}\newtheorem*{note}{\small{Note}}
\numberwithin{equation}{section}
\numberwithin{theorem}{section}
\numberwithin{figure}{section}

\usepackage{xeCJK}
\setCJKmainfont{FZShuSong-Z01}[BoldFont=FZXiaoBiaoSong-B05,ItalicFont=FZKai-Z03]
\setCJKsansfont{FZXiHeiI-Z08}[BoldFont=FZHei-B01]
\setCJKmonofont{FZFangSong-Z02}
\setCJKfamilyfont{zhsong}{FZShuSong-Z01}[BoldFont=FZXiaoBiaoSong-B05]
\setCJKfamilyfont{zhhei}{FZHei-B01}
\setCJKfamilyfont{zhkai}{FZKai-Z03}
\setCJKfamilyfont{zhfs}{FZFangSong-Z02}
\setCJKfamilyfont{zhli}{FZLiShu-S01}
\setCJKfamilyfont{zhyou}{FZXiYuan-M01}[BoldFont=FZZhunYuan-M02]

\allowdisplaybreaks{}

\newcommand{\isFullBook}[2]{
  \ifnum\pdfstrcmp{\FullBook}{True}=0
    \ifnum\pdfstrcmp{}{#1}=0\unskip\else#1\fi
  \else
    \ifnum\pdfstrcmp{}{#2}=0\unskip\else#2\fi
  \fi\ignorespaces{}
}

\counterwithout{theorem}{section}
\counterwithout{equation}{section}

\begin{document}
Here we go back to the difference between western and eastern culture,
{\CJKfamily{zhkai}清華園} or {\CJKfamily{zhkai}園華清}. If we choose to write
\(\mathbb{x}A=\mathbb{y}\), \(\mathbb{x}\) should be a row vector. Thus we should
write \[
    \mathbb{x}=(x^1,\ldots,x^n)\begin{pmatrix}
        e_1 \\ \vdots \\ e_n
    \end{pmatrix}
.\] So \[
    e=\begin{pmatrix} e_1\\ \vdots\\ e_n \end{pmatrix}
\] and \[
    (x^1,\ldots,x^n)A^{-1}A\begin{pmatrix} e_1\\ \vdots\\ e_n \end{pmatrix}
    =(x^1,\ldots,x^n)\begin{pmatrix} e^1\\ \vdots\\ e^n \end{pmatrix}
.\] The matrices multiply on \(e\) on the left. Thus connection matrix acts on \(e\)
from the left, \ie\ \[
    \nabla e=\omega e
.\] It follows that
\begin{align*}
    &(\nabla_X\nabla_Y-\nabla_Y\nabla_X-\nabla_{[X,Y]})e \\
    =&\nabla_X(\omega(Y)e)-\nabla_Y(\omega(X)e)-\omega([X,Y])e \\
    =&(X\omega(Y)+\omega(Y)\omega(X)-Y\omega(X)-\omega(X)\omega(Y)
    -\omega([X,Y]))e \\
    =&((\dd{\omega}-\omega\wedge \omega)(X,Y))e
.\end{align*}
So for eastern culture, \(\Omega\) should be \[
    \Omega=\dd{\omega}-\omega\wedge \omega
.\] 

\begin{remark}
    Professor Shiing-Shen Chern ({\CJKfamily{zhkai}陈省身})'s book
    ``Complex manifold without potential theory'' uses
    \(\Omega=\dd{\omega}-\omega\wedge \omega\). Also this book uses the convention of
    principal bundle with left action of the structure group.

    This lecture notes employs, as most other books, western culture and principal
    bundles have right action of the structure group and \(\Omega=\dd{\omega}+\omega
    \wedge\omega\). (Note that typical principal bundles are the frame bundles.)
\end{remark}

\begin{definition}
    Given a connection \(\nabla\) on a vector bundle \(E\), we define the
    \textbf{exterior covariant derivative} \[
        \dd^\nabla\colon \mathbb{C}^\infty(M,E\otimes \wedge^k M)
        \longrightarrow C^\infty(M,E\otimes \wedge^{k+1}M)
    \] by \[
        \dd^\nabla (s\otimes \alpha)=\nabla s\wedge \alpha+s\otimes \dd{\alpha}
    .\] Note that for \(k=0\), we have \(\dd^\nabla =\nabla\).
\end{definition}
\begin{prop}\hfill
\begin{enumerate}[(1)]
\item \(R=\dd^\nabla\circ \dd^\nabla\).
\item \(\dd^\nabla R=0\). (2nd Bianchi identity)
\end{enumerate}
\end{prop}
\begin{proof}
    Left as exercise.
\end{proof}

\begin{definition}
    \[
        \op{Aut}(E)=\{\vphi\in \op{End}(E):(\vphi\colon E_p\to E_p)
        \text{ is linear isomorphism for }p=\pi(\vphi)\}
    \] is called the \textbf{automorphism bundle}, and an element of
    \(C^\infty(M,\op{Aut(M)})\) is called a \textbf{gauge transformation}.
\end{definition}
Obviously the gauge group is a group, and acts on the space \(C^\infty(M,E)\) of 
sections of \(E\). Let \(\mathscr{C}(E)\) be the space of connections. Then
\(C^\infty(M,\op{Aut}(E))\) acts on \(\mathscr{C}(E)\) by 
\[\begin{tikzcd}[row sep=small]
	{C^\infty(M,\op{Aut}(E))\times \mathscr{C}} & {\mathscr{C}(E)} \\
	{(\alpha,\nabla)} & {\alpha\circ\nabla\circ\alpha^{-1}=:\alpha(\nabla)}
	\arrow[maps to, from=2-1, to=2-2]
	\arrow[from=1-1, to=1-2]
	\arrow["\in"{marking}, draw=none, from=2-1, to=1-1]
\end{tikzcd}\]
If a connection on \(E\) is given, then we can define a connection on the dual vector
bundle \(E^*\) as follows:

For \(r\in C^\infty(M,E^*)\) and \(s\in C^\infty(M,E)\), \[
    \left<\nabla r,s\right> =\dd{\left<r,s\right>}-\left<r,\nabla s\right> 
.\] Then we can define a connection on \(E\otimes E^*\) be derivation: \[
    \nabla (s\otimes r)=\nabla s\otimes r+s\otimes \nabla r
.\] 

\begin{prop}
    \(\alpha(\nabla)=\nabla -\nabla\alpha\cdot \alpha^{-1}\)
\end{prop}
\begin{proof}
    For \(s\in \)
\begin{align*}
    \alpha(\nabla)&=\alpha \nabla(\alpha^{-1}s) \\
    &=\alpha(-\alpha^{-1}\nabla\alpha\cdot\alpha^{-1}s+\alpha^{-1}\nabla s) \\
    &=\nabla s-\nabla\alpha\cdot \alpha^{-1}s
.\end{align*}
\end{proof}

As a summary, 
\begin{align*}
    &e=(e_1,\ldots,e_r) && \text{local frame} \\ 
    &\nabla e=e\omega && \omega\text{ connection matrix}
.\end{align*}
If \(s\in C^\infty(M,E)\), \[
    s=(e_1,\ldots,e_r)\begin{pmatrix} f^1\\ \vdots\\ f^r \end{pmatrix}=e\cdot f
.\] \[
    \nabla s=\nabla e\cdot f+e\cdot \dd{f}=e(\dd{f}+\omega f)
.\] If we identify \(s\leftrightarrow f\) then \[
    \nabla s\longleftrightarrow \dd{f}+\omega f
.\] So, \(\omega\) is just a ``correction term'' of \(\dd\).

\section{Riemannian geometry}
For general vector bundle \(E\) there are lots of connections. But in certain
situations, there are canonical, even unique connections. The Levi-Civita connection 
is such a case.

Let \(M\) be a smooth manifold of \(\dim n\).
\begin{definition}
    \(g\in C^\infty(M,T^*M\otimes T^*M)\) is a Riemannian metric on \(M\) if at 
    \(\forall\,p\in M\), \[
        g_p\colon T_p M\times T_p M\longrightarrow \mathbb{R}
    \] is a positive definite symmetric bilinear form, \ie\ \(g_p\) defines an inner
    product on the tangent space.
\end{definition}

\end{document}
