% !TeX program = xelatex
\documentclass[12pt]{article}
\usepackage{standalone}

\usepackage[dvipsnames,svgnames,x11names]{xcolor}
\usepackage[a4paper,margin=1in]{geometry}
\usepackage{microtype}
\usepackage{amsmath}
\usepackage{amsthm}
\usepackage{mathtools}
\usepackage{mathrsfs}
\usepackage{stmaryrd}
\usepackage{extarrows}
\usepackage{enumerate}
\usepackage{tensor}
\usepackage{physics}
\usepackage{graphicx}
\usepackage{subcaption}
\usepackage{tikz}
\usepackage{tikz-3dplot}
% \usepackage{tikz-cd}
\usepackage{quiver}
  \tikzset{every picture/.style={line width=0.75pt}}
\usepackage{pgfplots}
  \pgfplotsset{compat=newest}
\usepackage{tcolorbox}
  \tcbuselibrary{most}
\usepackage[colorlinks=true,linkcolor=blue]{hyperref}
\usepackage{cleveref}
\usepackage[hyperref=true,backend=biber,style=alphabetic,backref=true,url=false]{biblatex}
\usepackage[warnings-off={mathtools-colon,mathtools-overbracket}]{unicode-math}
\usepackage[default]{fontsetup}
\usepackage{fancyhdr}
\usepackage{import}

\newcommand{\Id}{\mathbb{1}}
\newcommand{\lap}{\increment}

\DeclareMathOperator{\sign}{sign}
\DeclareMathOperator{\dom}{dom}
\DeclareMathOperator{\ran}{ran}
\DeclareMathOperator{\ord}{ord}
\DeclareMathOperator{\Span}{span}
\DeclareMathOperator{\img}{Im}
\DeclareMathOperator{\Ric}{Ric}
\newcommand{\card}{\texttt{\#}}
\newcommand{\ie}{\emph{i.e.}}
\newcommand{\st}{\emph{s.t.}}
\newcommand{\eps}{\varepsilon}
\newcommand{\vphi}{\varphi}
\newcommand{\vthe}{\vartheta}
\newcommand{\II}{I\!I}
\renewcommand{\emptyset}{\varnothing}
\newcommand{\acts}{\curvearrowright}
\newcommand{\xrr}{\xlongrightarrow}
\newcommand{\into}{\hookrightarrow}
\newcommand{\pdif}[2]{\frac{\partial #1}{\partial #2}}
\renewcommand{\op}{\operatorname}

\theoremstyle{plain}\newtheorem{theorem}{Theorem}
\theoremstyle{definition}\newtheorem{definition}[theorem]{Definition}
\theoremstyle{definition}\newtheorem{example}[theorem]{Example}
\theoremstyle{plain}\newtheorem{axiom}[theorem]{Axiom}
\theoremstyle{plain}\newtheorem{corollary}[theorem]{Corollary}
\theoremstyle{plain}\newtheorem{lemma}[theorem]{Lemma}
\theoremstyle{plain}\newtheorem{proposition}[theorem]{Proposition}
\theoremstyle{plain}\newtheorem{prop}[theorem]{Proposition}
\theoremstyle{plain}\newtheorem{conjecture}[theorem]{Conjecture}
\theoremstyle{plain}\newtheorem{conj}[theorem]{Conjecture}
\theoremstyle{plain}\newtheorem{problem}[theorem]{Problem}
\theoremstyle{remark}\newtheorem{notation}[theorem]{Notation}
\theoremstyle{definition}\newtheorem*{question}{Question}
\theoremstyle{definition}\newtheorem*{answer}{Answer}
\theoremstyle{definition}\newtheorem*{goal}{Goal}
\theoremstyle{plain}\newtheorem*{application}{Application}
\theoremstyle{plain}\newtheorem*{exercise}{Exercise}
\theoremstyle{remark}\newtheorem*{remark}{Remark}
\theoremstyle{remark}\newtheorem*{note}{\small{Note}}
\numberwithin{equation}{section}
\numberwithin{theorem}{section}
\numberwithin{figure}{section}

\usepackage{xeCJK}
\setCJKmainfont{FZShuSong-Z01}[BoldFont=FZXiaoBiaoSong-B05,ItalicFont=FZKai-Z03]
\setCJKsansfont{FZXiHeiI-Z08}[BoldFont=FZHei-B01]
\setCJKmonofont{FZFangSong-Z02}
\setCJKfamilyfont{zhsong}{FZShuSong-Z01}[BoldFont=FZXiaoBiaoSong-B05]
\setCJKfamilyfont{zhhei}{FZHei-B01}
\setCJKfamilyfont{zhkai}{FZKai-Z03}
\setCJKfamilyfont{zhfs}{FZFangSong-Z02}
\setCJKfamilyfont{zhli}{FZLiShu-S01}
\setCJKfamilyfont{zhyou}{FZXiYuan-M01}[BoldFont=FZZhunYuan-M02]

\geometry{a4paper,margin=1in}
\allowdisplaybreaks{}

\counterwithout{theorem}{section}
\counterwithout{equation}{section}

\begin{document}
Now we move on to the proof of Kodaira embedding theorem. For that purpose,
we need some preparatory results.

Recall that the blow-up of a complex manifold at one point was defined in
homework 6. We describe it here in the following way:

Let \(z=(z^1,\ldots,z^n)\) be a local holomorphic coordinate around \(p\) in
a coordinate neighborhood \(U\) of a complex manifold \(M\). Let \(t=[t^1:\cdots
:t^m\) be the homogeneous coordinate of \(\mathbb{P}^{m-1}(\mathbb{C})\).
Consider the complex submanifold \(W\) in \(U\times\mathbb{P}^{m-1}(\mathbb{C})\)
defined by \[
  W=\{(z,t)\in U\times \mathbb{P}^{m-1}(\mathbb{C}):t^i z^j-t^j z^i=0,
  i,j=1,\ldots,m\}
.\] Some \(t^i\) has to be non-zero, but can be 0. We can define \[
  \pi\colon W\lrr U,\quad(z,t)\longmapsto z
\] and also define \(E:=\pi^{-1}(0)\). Then \[
  E=\pi^{-1}(0)=\{0\}\times \mathbb{P}^{m-1}(\mathbb{C})
  \cong\mathbb{P}^{m-1}(\mathbb{C})
.\] And \(\pi\colon W\setminus E\to U\setminus\{0\}\) is an isomorphism.
Of course, \(W\) is isomorphic to a neighborhood of zero section of
\(\mathcal{O}(-1)\to \mathbb{P}^{m-1}(\mathbb{C})\). The blow-up of \(M\) at
\(p\) is obtained by replacing \(U\) by \(W\). (\(p\) is the point \(z=0\)).
The blow-up has many namings: \emph{Monoidal transformation}, \emph{\(\sigma\)-
process}, \emph{quadratic transformation}.

Because of the last naming, the blow-up of \(M\) at \(p\) is often denoted by
\(Q_p(M)\). But of course \(\tilde{M}\) is simpler and more convenient if the
point \(p\) where blow-up is made is not so important.

Let \(\pi:\tilde{M}=Q_p(M)\to M\) be the blow-up of \(M\) at \(p\),
\(E:=\pi^{-1}(p)\) is called the exceptional divisor.
\begin{theorem}
  \[
    K_{\tilde{M}}=\pi^* K_M \otimes [E]^{m-1}
  \] where \(E\) is the line bundle associated to \(E\). In particular, \[
    c_1(\tilde{M})=\pi^* c_1(M)-(m-1)[E]
  \] where \([E]\) denotes the Poincaré the homology class of \(E\),
  \ie\ \(c_1([E])\). (This way, divisors, line bundles and their first Chern
  classes are often identified and denoted by the same letter).
\end{theorem}
\begin{proof}
  \(\tilde{M}\setminus E\cong M\setminus\{p\}\) and \[
    K_{\tilde{M}}\Big|_{\tilde{M}\setminus E}=\pi^* K_M\Big|_{M\setminus\{p\}}
  .\] Since \([E]\big|_{\tilde{M}\setminus E}\) is trivial, the identity in the
  theorem holds on \(\tilde{M}\setminus E\).
  Thus we have only to show the identity along \(E\). It is sufficient to
  consider the case when \(t^1\neq 0\). Along \(E\), we can take local
  coordinate \[
    u_1^2=\frac{t^2}{t^1},\ldots,u_1^m=\frac{t^m}{t^1}
  .\] In a neighborhood of \(E\), \[
    (z^1,u_1^2,\ldots,u_1^m)\longmapsto (z,t)
  \] with \(E=\{z^1=0\}\) and 
  \begin{align*}
    z&=(z^1,z^1 u_1^2,\ldots,z^1 u_1^m), \\
    t&=[1:u_1^2:\cdots :u_1^m]
  \end{align*}
  gives a local coordinate. Then using \(\frac{t^i}{t^1}=\frac{t^i}{z^1}\), \[
    \dd{z^1}\wedge\dd{\dd{u_1^2}}\wedge\cdots\wedge\dd{u_1^m}
    =\dd{z^1}\wedge\dd{(\frac{z^2}{z^1})}\wedge\cdots\wedge\dd{(\frac{z^m}{z^1})}
    =\frac{1}{(z^1)^{m-1}}\dd{z^1}\wedge\cdots\wedge\dd{z^m}
  .\] When \(t^\alpha\neq 0\), by the similar computation with \[
  u_{\alpha}^1=\frac{t^1}{t^\alpha},\ldots,\widehat{u_{\alpha}^\alpha},
  \cdots,u_\alpha^m=\frac{t^m}{t^\alpha},\text{ and }z^\alpha
  ,\] we have \[
    \dd{\dd{u_\alpha^1}}\wedge\cdots\wedge\dd{z^\alpha}\wedge\cdots\wedge
    \dd{u_\alpha^m}=\frac{1}{(z^\alpha)^{m-1}}\dd{z^1}\wedge\cdots\wedge\dd{z^m}
  .\] Thus the transition function \(f_{\alpha\beta}\) of \(K_{\tilde{M}}
  \otimes \pi^*K_M^{-1}\) is computed by
  \begin{align*}
    \dd{u_\beta^1}\wedge\cdots\wedge\dd{z^\beta}\wedge\cdots\wedge
    \dd{u_\beta^m}&\otimes (\dd{z^1}\wedge\cdots\wedge\dd{z^m})^{-1} \\
    &=f_{\alpha\beta}\dd{u_\alpha^1}\wedge\cdots\wedge\dd{z^\alpha}\wedge\cdots
    \wedge\dd{u_\alpha^m}\otimes (\dd{z^1}\wedge\cdots\wedge\dd{z^m})^{-1}
  .\end{align*}
  Hence \(f_{\alpha\beta}=\left(\frac{z^\alpha}{z^\beta}\right)^{m-1}\).
  This is the transition function of \([E]^{m-1}\).

  On a neighborhood \(V_\alpha\) which does not meet \(E\), we take local
  coordinates \(z^1,\ldots,z^m\) and if
  \begin{align*}
    s_\alpha\dd{u_\alpha^1}\wedge\cdots\wedge\dd{z^\alpha}\wedge\cdots\wedge
    \dd{u_\alpha^m}&\otimes (\dd{z^1}\wedge\cdots\wedge\dd{z^m})^{-1} \\
    &=s_a\dd{z^1}\wedge\cdots\wedge\dd{z^m}
    \otimes (\dd{z^1}\wedge\cdots\wedge\dd{z^m})^{-1}
  .\end{align*}
  Then we have \(s_a=\left(\frac{1}{z^\alpha}\right)^{m-1}s_{\alpha}\).
  \ie\ \[
    f_{a\alpha}=\left(\frac{1}{z^\alpha}\right)^{m-1}
  .\] This is the transition function of \([E]^{m-1}\) again.
\end{proof}

\begin{lemma}\label{lem:20-ample}
  Let \(L\to M\) be an ample line bundle and \(G\to M\) be arbitrary line bundle.
  Let \(k\) be a positive integer. Then there is a larger number \(n_0=
  n_0(L,G,k)\) such that for any \(n\ge n_0\), the line bundle \[
    \pi^* L^n\otimes \pi^* G \otimes [E]^{-k}\to Q_p(M)=:\tilde{M}
  \] is ample. Here \(L^n=L\otimes\cdots\otimes L\) (\(n\)-times).
\end{lemma}
\begin{proof}
  Recall that the neighborhood of the exceptional divisor is isomorphic to \[
    W=\{(z,t)\in U\times \mathbb{P}^{m-1}(\mathbb{C}):z^i t^j-z^j t^i=0,
    i,j=1,\ldots,m\}
  .\] Consider the projection \[
    \tau\colon W\to \mathbb{P}^{m-1}(\mathbb{C})\cong E
  .\] Recall also \[
  [E]\big|_{E}\cong \mathcal{O}(-1)
  \] and \[
    [E]=\tau^* \mathcal{O}(-1)
  .\] Thus pulling back the standard metric on \(\mathcal{O}(-1)\), 
  \(h=1+|u_1^2|^2+\cdots +|u_1^m|^2\), its curvature is \[
    \tau^* \bar{\partial}\partial\log(1+|u_1^2|^2+\cdots+|u_1^m|^2)
    =-\partial\bar{\partial}\log(|z^1|^2+\cdots +|z^m|^2)
  \] using \(u_1^i=\frac{t^i}{t^1}=\frac{z^i}{z^1}\). The curvature is negative
  in \(E\) direction and \(0\) in normal direction to \(E\).

  On the other hand, \([E]\) is trivial outside \(W\). Thus using a partition
  of unity we have a metric \(h\) of \([E]\) whose curvature is \(0\) outside
  \(W\) and coincides with \[
    -\partial\bar{\partial}\log(|z^1|^2+\cdots+|z^m|^2)
  \] on a neighborhood of \(E\) slightly smaller than \(W\).

  On the hand \(L\) is ample and thus \(\pi^*L\) has positive definite curvature
  on \(M\setminus E\), positive along \(E\) in the normal direction. So for
  sufficiently large \(n\ge n_0\), \(\pi^*L^n\otimes [E]^{-k}\) has positive
  definite curvature.

  The same argument applies for \(\pi^*L^n\otimes \pi^*G\otimes [E]^{-k}\).
  One problem is that \(n_0\) may depend on the point \(p\) where the blow-up
  is made. But since the construction of the metric is continuous with respect
  to \(p\). So for small neighborhood of \(p\), we can choose \(n_0\) to be
  independent of \(p\). Further, as we assume \(M\) is compact, we can choose
  finite number of such neighborhoods. Then we can take the maximum of such
  \(n_0\).
\end{proof}

\begin{lemma}
  Let \(S\) be a hypersurface in \(M\), and \(\ell_S\) be the ideal sheaf of
  \(S\), \ie\ the sheaf of germs of holomorphic functions on \(M\) vanishing
  along \(S\). Then there is a natural isomorphism \[
    \ell_S\cong \mathcal{O}([S]^{-1})
  .\] Where \([S]\) is the line bundle associated to the divisor \(S\).

  More generally, there is an isomorphism \[
    \mathcal{O}(L)\otimes \ell_S\cong \mathcal{O}(L\otimes [S]^{-1})
  \] for any line bundle \(L\). (The right hand side is often denoted by
  \(\mathcal{O}(L-S)\) additively).
\end{lemma}
\begin{proof}
  Let \(\{U_i\}\) be an open covering of \(M\) and suppose that  \[
    U_i\cap S=\{f_i=0\}
  \] for holomorphic function \(f_i\) on \(U_i\). Then the line bundle \([S]\)
  is defined by transition functions \[
    f_{ij}=\frac{f_i}{f_j}
  .\] It follows that \[
    f_{i}=f_{ij}f_j
  \] and \(s=\{f_i\}_i\) defines a global section of \([S]\). Then for any
  \(t\in \mathcal{O}([S]^{-1})\), the germ \(st\) is a germ of \(\mathcal{O}(
  [S]\otimes [S]^{-1})\cong \mathcal{O}\), vanishing along \(S\) with at least
  1st order since \(s=\{f_i\}\) is the defining section of \(S\). Thus the
  multiplication \(s\) gives the isomorphism
  \begin{align*}
    \mathcal{O}([S]^{-1}) &\longrightarrow \ell_S\subset \mathcal{O}=
    \mathcal{O}([S]\otimes [S]^{-1})\\
    t &\longmapsto st
  .\end{align*}
  Similarly 
  \begin{align*}
    \mathcal{O}(L-S) &\longrightarrow \mathcal{O}(L)\otimes\ell_S\subset
    \mathcal{O}(L\otimes [S]\otimes [S]^{-1})\\
    t &\longmapsto st
  \end{align*}
  is an isomorphism.
\end{proof}

\begin{remark}
  \(\mathcal{O}(L-kS)=\)``the sheaf of germs of local sections of \(L\) vanishing
  along \(S\) with at least order \(k\)''.
\end{remark}

\end{document}
