% !TeX program = xelatex
\documentclass[12pt]{article}
\usepackage{standalone}

\usepackage[dvipsnames,svgnames,x11names]{xcolor}
\usepackage[a4paper,margin=1in]{geometry}
\usepackage{microtype}
\usepackage{amsmath}
\usepackage{amsthm}
\usepackage{mathtools}
\usepackage{mathrsfs}
\usepackage{stmaryrd}
\usepackage{extarrows}
\usepackage{enumerate}
\usepackage{tensor}
\usepackage{physics2}
  \usephysicsmodule{ab,xmat}
\usepackage{fixdif}
  \newcommand{\dd}{\d}
\usepackage{derivative}
  \newcommand{\dv}{\odv}
  \newcommand{\pd}[1]{\pdv{}{#1}}
  \newcommand{\eval}[1]{#1\big|}
\usepackage{graphicx}
\usepackage{subcaption}
\usepackage{tikz}
\usepackage{tikz-3dplot}
% \usepackage{tikz-cd}
% \usepackage{quiver}
% ========== Block quiver.sty ========== %
\usepackage{tikz-cd}
% \usepackage{amssymb}
\usetikzlibrary{calc}
\usetikzlibrary{decorations.pathmorphing}
\tikzset{curve/.style={settings={#1},to path={(\tikztostart)
    .. controls ($(\tikztostart)!\pv{pos}!(\tikztotarget)!\pv{height}!270:(\tikztotarget)$)
    and ($(\tikztostart)!1-\pv{pos}!(\tikztotarget)!\pv{height}!270:(\tikztotarget)$)
    .. (\tikztotarget)\tikztonodes}},
    settings/.code={\tikzset{quiver/.cd,#1}
        \def\pv##1{\pgfkeysvalueof{/tikz/quiver/##1}}},
    quiver/.cd,pos/.initial=0.35,height/.initial=0}
\tikzset{tail reversed/.code={\pgfsetarrowsstart{tikzcd to}}}
\tikzset{2tail/.code={\pgfsetarrowsstart{Implies[reversed]}}}
\tikzset{2tail reversed/.code={\pgfsetarrowsstart{Implies}}}
\tikzset{no body/.style={/tikz/dash pattern=on 0 off 1mm}}
% =========== End block ========== %
  \tikzset{every picture/.style={line width=0.75pt}}
\usepackage{pgfplots}
  \pgfplotsset{compat=newest}
\usepackage{tcolorbox}
  \tcbuselibrary{most}
\usepackage[colorlinks=true,linkcolor=blue]{hyperref}
\usepackage{cleveref}
% \usepackage[hyperref=true,backend=biber,style=alphabetic,backref=true,url=false]{biblatex}
\usepackage[warnings-off={mathtools-colon,mathtools-overbracket}]{unicode-math}
\usepackage[default,amsbb]{fontsetup}
  \setmathfont[StylisticSet=1,range=\mathscr]{NewCMMath-Book.otf}
\usepackage{fancyhdr}
\usepackage{import}

\newcommand{\Id}{\mathbb{1}}
\newcommand{\lap}{\increment}

\DeclareMathOperator{\sign}{sign}
\DeclareMathOperator{\dom}{dom}
\DeclareMathOperator{\ran}{ran}
\DeclareMathOperator{\ord}{ord}
\DeclareMathOperator{\Span}{span}
\DeclareMathOperator{\img}{Im}
\DeclareMathOperator{\Ric}{Ric}
\newcommand{\card}{\texttt{\#}}
\newcommand{\ie}{\emph{i.e.}}
\newcommand{\st}{\emph{s.t.}}
\newcommand{\eps}{\varepsilon}
\newcommand{\vphi}{\varphi}
\newcommand{\vthe}{\vartheta}
\newcommand{\II}{I\!I}
\renewcommand{\emptyset}{⌀}
\newcommand{\acts}{\curvearrowright}
\newcommand{\xrr}{\xlongrightarrow}
\newcommand{\lrr}{\longrightarrow}
\newcommand{\lmt}{\longmapsto}
\newcommand{\into}{\hookrightarrow}
\newcommand{\op}{\operatorname}

\let\originalleft\left
\let\originalright\right
\renewcommand{\left}{\mathopen{}\mathclose\bgroup\originalleft}
\renewcommand{\right}{\aftergroup\egroup\originalright}

\theoremstyle{plain}\newtheorem{theorem}{Theorem}
\theoremstyle{definition}\newtheorem{definition}[theorem]{Definition}
\theoremstyle{definition}\newtheorem{example}[theorem]{Example}
\theoremstyle{definition}\newtheorem{problem}[theorem]{Problem}
\theoremstyle{plain}\newtheorem{axiom}[theorem]{Axiom}
\theoremstyle{plain}\newtheorem{corollary}[theorem]{Corollary}
\theoremstyle{plain}\newtheorem{lemma}[theorem]{Lemma}
\theoremstyle{plain}\newtheorem{proposition}[theorem]{Proposition}
\theoremstyle{plain}\newtheorem{prop}[theorem]{Proposition}
\theoremstyle{plain}\newtheorem{conjecture}[theorem]{Conjecture}
\theoremstyle{plain}\newtheorem{conj}[theorem]{Conjecture}
\theoremstyle{remark}\newtheorem{notation}[theorem]{Notation}
\theoremstyle{definition}\newtheorem*{question}{Question}
\theoremstyle{definition}\newtheorem*{answer}{Answer}
\theoremstyle{definition}\newtheorem*{goal}{Goal}
\theoremstyle{definition}\newtheorem*{application}{Application}
\theoremstyle{plain}\newtheorem*{exercise}{Exercise}
\theoremstyle{remark}\newtheorem*{remark}{Remark}
\theoremstyle{remark}\newtheorem*{note}{\small{Note}}
\numberwithin{equation}{section}
\numberwithin{theorem}{section}
\numberwithin{figure}{section}

\usepackage{xeCJK}
\setCJKmainfont{FZShuSong-Z01}[BoldFont=FZXiaoBiaoSong-B05,ItalicFont=FZKai-Z03]
\setCJKsansfont{FZXiHeiI-Z08}[BoldFont=FZHei-B01]
\setCJKmonofont{FZFangSong-Z02}
\setCJKfamilyfont{zhsong}{FZShuSong-Z01}[BoldFont=FZXiaoBiaoSong-B05]
\setCJKfamilyfont{zhhei}{FZHei-B01}
\setCJKfamilyfont{zhkai}{FZKai-Z03}
\setCJKfamilyfont{zhfs}{FZFangSong-Z02}
\setCJKfamilyfont{zhli}{FZLiShu-S01}
\setCJKfamilyfont{zhyou}{FZXiYuan-M01}[BoldFont=FZZhunYuan-M02]

\allowdisplaybreaks{}

\newcommand{\isFullBook}[2]{
  \ifnum\pdfstrcmp{\FullBook}{True}=0
    \ifnum\pdfstrcmp{}{#1}=0\unskip\else#1\fi
  \else
    \ifnum\pdfstrcmp{}{#2}=0\unskip\else#2\fi
  \fi\ignorespaces{}
}

\counterwithout{theorem}{section}
\counterwithout{equation}{section}

\begin{document}
This is standard in homological algebra.

\begin{remark}
\[
  \iota\colon C^p(\mathcal{U},S')\lrr C^p(\mathcal{U},S)  
\] is always injective but \[
  \pi\colon C^p(\mathcal{U},S)\lrr C^p(\mathcal{U},S'')
\] may not be surjective. However for any \(c''\in C^p(S'',T)\) we can choose a
refinement \(\mathcal{W}\) of \(\mathcal{U}\) and \(c\in C^p(\mathcal{W},S)\) such
that \(\lambda_{\mathcal{W}}^{\mathcal{U}}(c'')=\pi(c)\).

(This is Lemma 4.6 in \emph{Morrow-Kodaira}, or Lemma 3.3 in \emph{Kodaira - Complex
manifolds and deformation complex structure}).
\end{remark}

Granting this, the connecting homomorphism \(\delta^*\) is defined as follows:

For \(\gamma''=[c'']\in H^p(X,S'')\), \(c''\in Z^p(\mathcal{U},S'')\), by the above
remark we have a refinement \(\mathcal{W}\) of \(\mathcal{U}\) and \(c\in C^p(
\mathcal{W},S)\) such that \(\lambda_{\mathcal{W}}^{\mathcal{U}}(c'')=\pi(c)\). Then \[
  \pi(\delta c)=\delta\pi(c)=\delta(\lambda_{\mathcal{W}}^{\mathcal{U}}(c''))
  =\lambda_{\mathcal{W}}^{\mathcal{U}}(\delta c'')=0
.\] Thus \(\exists\,c'\in Z^{p+1}(\mathcal{W},S')\) such that \[
  \iota(c')=\delta(c)
.\] Finally we define \[
  \delta^* \gamma''=[c']
.\] As a diagram,
\[\begin{tikzcd}[row sep=tiny]
	0 & {S'} & S & {S''} & 0 \\
	&& c & {c''} \\
	\\ \\ 
	& {c'} & {\delta c}
	\arrow[from=1-1, to=1-2]
	\arrow[from=1-2, to=1-3]
	\arrow[from=1-3, to=1-4]
	\arrow[from=1-4, to=1-5]
	\arrow[from=2-3, to=2-4]
	\arrow[from=5-2, to=5-3]
	\arrow[from=2-3, to=5-3]
\end{tikzcd}\]
This is the standard way except that we take refinement \(\mathcal{W}\).

The proof of exactness of the long exact sequence can be checked yourself, or you 
can refer to other textbooks on algebraic topology, or it is expected that you have
encountered it before.

Reference: \emph{Morrow-Kodaira} Theorem 4.1, page 58-60. \emph{Kodaira - Complex
manifolds and deformation of complex structures}, Theorem 3.7, page 126-132.
\emph{Griffiths-Harris}, page 40-41

\noindent\textbf{Appendix} (Exactness at \(H^p(X,S'')\)).

If \(c\in Z^p(\mathcal{W},S)\), \ie\ \([c'']\in \img\pi^*\), then \(\delta c=0\),
so \(c'=0\). Hence \(\delta^*[c'']=0\).

If \(\delta^*[c'']=0\), then \(\exists\,d\in C^{p+1}(\mathcal{W},S)\), \(c'=\delta d\).
So \(\delta(c-\iota(d))=\delta c-\iota(c')=0\) and \(\pi(c-\iota(d))=\pi(c)=c''\).
So we can replace \(c\) by \(c-\iota(d)\). But \(\delta(c-\iota(d))=0\), so
\(c''\in \img\pi^*\).

\section{de Rham theorem and Dolbeault theorem}

Let \(S\) be a sheaf of Abelian groups.
\begin{definition}
  \(S\) is called a \textbf{fine sheaf} if for any locally finite open covering
  \(\mathcal{U}=\{U_i\}\) of the base \(X\), there are homomorphisms
  \(h_i\colon S\to S\) such that
  \begin{enumerate}[(a)]
  \item \(\mathrm{supp}\,h_i:=\overline{\{x\in X:h_i(S_x)\neq 0\}}\subset U_i\).
  \item \(\sum_i h_i=\Id\).
  \end{enumerate}
\end{definition}
\begin{example}
  Let \(X\) be a real smooth manifold, and \(\{\rho_i\}\) a partition of unity
  subordinate to \(\mathcal{U}=\{U_i\}\), that is, set of smooth functions \(\rho_i\)
  such that
  \begin{enumerate}[(i)]
  \item \(0\le \rho_i\le 1\),
  \item \(\mathrm{supp}\,\rho_i\subset U_i\),
  \item \(\sum_i \rho_i=1\).
  \end{enumerate}
  Then the sheaf \(\mathcal{A}^p\) of germs of smooth \(p\)-forms on \(X\) is a fine
  sheaf. In fact, we may take \(h_i\colon S\to S\) as \[
    h_i(\sigma)=\rho_i \sigma
  .\] 
\end{example}
\begin{example}
  Similarly, the sheaf \(\mathcal{A}^{p,q}\) of germs of smooth \((p,q)\)-forms on
  a complex manifold \(X\) is a fine sheaf.
\end{example}
\begin{example}
  However, the sheaf \(\Omega^{p}\) of germs of holomorphic \(p\)-forms is \emph{not}
  a fine sheaf.
\end{example}

\begin{theorem}
  If \(S\) is a fine sheaf, then \[
    H^{p}(X,S)=0\quad \text{for }p>0
  .\] 
\end{theorem}
\begin{proof}
  Take \(\gamma\in H^p(X,S)\), it is expressed as \(\gamma=[c]\), \(c=\{c_{i_0,\ldots,
  i_p}\}\in Z^p(\mathcal{U},S)\). By the cocycle condition we have \[
  0=(\delta c)_{i_0,\ldots,i_p,j}=\sum_{s=0}^{p}(-1)^s c_{i_0,\ldots,\widehat{i_s},
  \ldots,i_p,j}+(-1)^{p+1}c_{i_0,\ldots,i_p}
  \] by taking \(j=i_{p+1}\). Multiply this by \(h_i\) and take sum with respect to
  \(j\) we get \[
    \sum_{s=0}^{p}(-1)^s \sum_i h_i c_{i_0,\ldots,\widehat{i_s},\ldots,i_p,j}
    +(-1)^{p+1}c_{i_0,\ldots,i_p}=0
  .\] Put \[
    e_{i_1,\ldots,i_p}=\sum_j h_j c_{i_1,\ldots,i_p,j}
  \] where \(h_j c_{i_1,\ldots,i_p,j}\in \Gamma(U_{i_1}\cap\cdots\cap U_{i_p}\cap U_j,
  S)\) is extended to be zero outside \(\mathrm{supp}\,h_i\) and considered as a
  section over \(U_{i_1}\cap\cdots\cap U_{i_p}\). Thus we obtain \(e_{i_1,\ldots,i_p}
  \in \Gamma(U_{i_1}\cap\cdots\cap U_{i_p},S)\), \(e=\{e_{i_1,\ldots,i_p}\}\in C^{p-1}
  (\mathcal{U},S)\) with \[
    \sum_{s=0}^{p}(-1)^s e_{i_0,\ldots,\widehat{i_s},\ldots,i_p}+
    (-1)^{p+1}c_{i_0,\ldots,i_p}=0
  .\] \ie\ \(c=(-1)^{p}\delta e\).
\end{proof}

\begin{definition}
  A \textbf{fine resolution} of \(S\) is an exact sequence
  \[\begin{tikzcd}
    0 \arrow[r] & S \arrow[r,"\iota"] & \mathcal{F}_0 \arrow[r,"\dd"]
    & \mathcal{F}_1 \arrow[r,"\dd"] & \mathcal{F}_2 \arrow[r] & \cdots
  \end{tikzcd}\] 
  for fine sheaves \(\mathcal{F}_0,\mathcal{F}_1,\mathcal{F}_2,\ldots\).
\end{definition}

\begin{theorem}\label{thm:7-1:fine}
  If \(S\) is given a fine resolution as above, the cohomology \(H^p(X,S)\) is
  computed by \[
    H^p(X,S)= \frac{\ker\left(\dd\colon\Gamma(X,\mathcal{F}_p)\to
    \Gamma(X,\mathcal{F}_{p+1})\right)}{\img\left(\dd\colon\Gamma(X,\mathcal{F}_{p-1})
    \to\Gamma(X,\mathcal{F}_p)\right)}
  .\] 
\end{theorem}
\begin{proof}
  If we write \(Z_p=\ker(\dd\colon \mathcal{F}_p\to \mathcal{F}_{p+1})\) then \[
    0\lrr Z_{p-1}\lrr \mathcal{F}_{p-1}\lrr Z_p\lrr 0
  \] is an exact sequence. From this we can get a long exact sequence as we discussed
  previously. Using \(H^q(X,\mathcal{F}_{p-1})=0\) for the fine sheaf \(\mathcal{F}_{
  p-1}\) we obtain from the long exact sequence \[
    H^q(X,Z_p)\cong H^{q+1}(X,Z_{p-1})\quad\text{ for }p-1\ge 0,q\ge 1
  .\] Thus \[
    H^p(X,Z_0)\cong H^{p-1}(X,Z_1)\cong \cdots \cong H^1(X,Z_{p-1})
  .\] Since \(Z_0=S\), the left hand side is \(H^p(X,S)\). On the other hand, the
  right hand side appears in the long exact sequence \[
    0\lrr H^0(X,\mathcal{F}_{p-1})\lrr H^0(X,Z_p)\lrr H^1(X,Z_{p-1})\lrr
    \underbrace{H^1(X,\mathcal{F}_{p-1})}_{=0}\lrr\cdots
  .\] Thus \[
    H^1(X,Z_{p-1})\cong \frac{H^0(X,Z_p)}{\img\left(H^0(X,\mathcal{F}_{p-1})
    \to H^0(X,Z_p)\right)}
  .\] Of course
  \begin{gather*}
    H^0(X,Z_p)=\Gamma(X,Z_p)=\ker\left(\dd\colon\Gamma(X,\mathcal{F}_p)\to 
    \Gamma(X,\mathcal{F}_{p+1})\right), \\
    \img\left(H^0(X,\mathcal{F}_{p-1})\to H^0(X,Z_p)\right)=\img\left(\dd\colon\Gamma(
    X,\mathcal{F}_{p-1})\to \Gamma(X,\mathcal{F}_p)\right).
  \end{gather*}
\end{proof}

\begin{remark}
  In this proof we only used \[
    H^q(X,\mathcal{F}_p)=0\quad\text{ for }p\ge 0,q>0
  .\] So \(\mathcal{F}_p\) need not to be fine sheaves, if only these vanishing
  conditions are satisfied.
\end{remark}

\end{document}
