% !TeX program = xelatex
\documentclass[12pt]{article}
\usepackage{standalone}

\usepackage[dvipsnames,svgnames,x11names]{xcolor}
\usepackage[a4paper,margin=1in]{geometry}
\usepackage{microtype}
\usepackage{amsmath}
\usepackage{amsthm}
\usepackage{mathtools}
\usepackage{mathrsfs}
\usepackage{stmaryrd}
\usepackage{extarrows}
\usepackage{enumerate}
\usepackage{tensor}
\usepackage{physics2}
  \usephysicsmodule{ab,xmat}
\usepackage{fixdif}
  \newcommand{\dd}{\d}
\usepackage{derivative}
  \newcommand{\dv}{\odv}
  \newcommand{\pd}[1]{\pdv{}{#1}}
  \newcommand{\eval}[1]{#1\big|}
\usepackage{graphicx}
\usepackage{subcaption}
\usepackage{tikz}
\usepackage{tikz-3dplot}
% \usepackage{tikz-cd}
% \usepackage{quiver}
% ========== Block quiver.sty ========== %
\usepackage{tikz-cd}
% \usepackage{amssymb}
\usetikzlibrary{calc}
\usetikzlibrary{decorations.pathmorphing}
\tikzset{curve/.style={settings={#1},to path={(\tikztostart)
    .. controls ($(\tikztostart)!\pv{pos}!(\tikztotarget)!\pv{height}!270:(\tikztotarget)$)
    and ($(\tikztostart)!1-\pv{pos}!(\tikztotarget)!\pv{height}!270:(\tikztotarget)$)
    .. (\tikztotarget)\tikztonodes}},
    settings/.code={\tikzset{quiver/.cd,#1}
        \def\pv##1{\pgfkeysvalueof{/tikz/quiver/##1}}},
    quiver/.cd,pos/.initial=0.35,height/.initial=0}
\tikzset{tail reversed/.code={\pgfsetarrowsstart{tikzcd to}}}
\tikzset{2tail/.code={\pgfsetarrowsstart{Implies[reversed]}}}
\tikzset{2tail reversed/.code={\pgfsetarrowsstart{Implies}}}
\tikzset{no body/.style={/tikz/dash pattern=on 0 off 1mm}}
% =========== End block ========== %
  \tikzset{every picture/.style={line width=0.75pt}}
\usepackage{pgfplots}
  \pgfplotsset{compat=newest}
\usepackage{tcolorbox}
  \tcbuselibrary{most}
\usepackage[colorlinks=true,linkcolor=blue]{hyperref}
\usepackage{cleveref}
% \usepackage[hyperref=true,backend=biber,style=alphabetic,backref=true,url=false]{biblatex}
\usepackage[warnings-off={mathtools-colon,mathtools-overbracket}]{unicode-math}
\usepackage[default,amsbb]{fontsetup}
  \setmathfont[StylisticSet=1,range=\mathscr]{NewCMMath-Book.otf}
\usepackage{fancyhdr}
\usepackage{import}

\newcommand{\Id}{\mathbb{1}}
\newcommand{\lap}{\increment}

\DeclareMathOperator{\sign}{sign}
\DeclareMathOperator{\dom}{dom}
\DeclareMathOperator{\ran}{ran}
\DeclareMathOperator{\ord}{ord}
\DeclareMathOperator{\Span}{span}
\DeclareMathOperator{\img}{Im}
\DeclareMathOperator{\Ric}{Ric}
\newcommand{\card}{\texttt{\#}}
\newcommand{\ie}{\emph{i.e.}}
\newcommand{\st}{\emph{s.t.}}
\newcommand{\eps}{\varepsilon}
\newcommand{\vphi}{\varphi}
\newcommand{\vthe}{\vartheta}
\newcommand{\II}{I\!I}
\renewcommand{\emptyset}{⌀}
\newcommand{\acts}{\curvearrowright}
\newcommand{\xrr}{\xlongrightarrow}
\newcommand{\lrr}{\longrightarrow}
\newcommand{\lmt}{\longmapsto}
\newcommand{\into}{\hookrightarrow}
\newcommand{\op}{\operatorname}

\let\originalleft\left
\let\originalright\right
\renewcommand{\left}{\mathopen{}\mathclose\bgroup\originalleft}
\renewcommand{\right}{\aftergroup\egroup\originalright}

\theoremstyle{plain}\newtheorem{theorem}{Theorem}
\theoremstyle{definition}\newtheorem{definition}[theorem]{Definition}
\theoremstyle{definition}\newtheorem{example}[theorem]{Example}
\theoremstyle{definition}\newtheorem{problem}[theorem]{Problem}
\theoremstyle{plain}\newtheorem{axiom}[theorem]{Axiom}
\theoremstyle{plain}\newtheorem{corollary}[theorem]{Corollary}
\theoremstyle{plain}\newtheorem{lemma}[theorem]{Lemma}
\theoremstyle{plain}\newtheorem{proposition}[theorem]{Proposition}
\theoremstyle{plain}\newtheorem{prop}[theorem]{Proposition}
\theoremstyle{plain}\newtheorem{conjecture}[theorem]{Conjecture}
\theoremstyle{plain}\newtheorem{conj}[theorem]{Conjecture}
\theoremstyle{remark}\newtheorem{notation}[theorem]{Notation}
\theoremstyle{definition}\newtheorem*{question}{Question}
\theoremstyle{definition}\newtheorem*{answer}{Answer}
\theoremstyle{definition}\newtheorem*{goal}{Goal}
\theoremstyle{definition}\newtheorem*{application}{Application}
\theoremstyle{plain}\newtheorem*{exercise}{Exercise}
\theoremstyle{remark}\newtheorem*{remark}{Remark}
\theoremstyle{remark}\newtheorem*{note}{\small{Note}}
\numberwithin{equation}{section}
\numberwithin{theorem}{section}
\numberwithin{figure}{section}

\usepackage{xeCJK}
\setCJKmainfont{FZShuSong-Z01}[BoldFont=FZXiaoBiaoSong-B05,ItalicFont=FZKai-Z03]
\setCJKsansfont{FZXiHeiI-Z08}[BoldFont=FZHei-B01]
\setCJKmonofont{FZFangSong-Z02}
\setCJKfamilyfont{zhsong}{FZShuSong-Z01}[BoldFont=FZXiaoBiaoSong-B05]
\setCJKfamilyfont{zhhei}{FZHei-B01}
\setCJKfamilyfont{zhkai}{FZKai-Z03}
\setCJKfamilyfont{zhfs}{FZFangSong-Z02}
\setCJKfamilyfont{zhli}{FZLiShu-S01}
\setCJKfamilyfont{zhyou}{FZXiYuan-M01}[BoldFont=FZZhunYuan-M02]

\allowdisplaybreaks{}

\newcommand{\isFullBook}[2]{
  \ifnum\pdfstrcmp{\FullBook}{True}=0
    \ifnum\pdfstrcmp{}{#1}=0\unskip\else#1\fi
  \else
    \ifnum\pdfstrcmp{}{#2}=0\unskip\else#2\fi
  \fi\ignorespaces{}
}

\counterwithout{theorem}{section}
\counterwithout{equation}{section}

\begin{document}

Recall we are considering \(\Omega^{p,q}(M)\), \(E=\) trivial line bundle, \(\lap_g=
\partial^*\partial +\partial\partial^*\) makes sense for \((p,q)\)-forms.
\begin{lemma}
  On K\"ahler manifolds, \[
    \lap_{\dd}=2\lap_{\bar{\partial}}=2\lap_{\partial}
  .\] 
\end{lemma}
\begin{proof}
  \begin{align*}
    \lap_{\dd{}}&=(\partial +\bar{\partial})(\partial +\bar{\partial})^*
    +(\partial +\bar{\partial})^*(\partial +\bar{\partial}) \\
    &=(\partial\partial^*+\partial^*\partial)+(\partial\bar{\partial}^*
    +\bar{\partial}^*\partial)+(\bar{\partial}\partial^*+\partial^*\bar{\partial})
    +(\bar{\partial}\bar{\partial}^*+\bar{\partial}^*\bar{\partial})
  .\end{align*}
  But by Hodge identities,
  \begin{align*}
    \lap_{\partial}&=\partial\partial^*+\partial^*\partial
    =\sqrt{-1}\partial (\Lambda\bar{\partial}-\bar{\partial}\Lambda)+
    \sqrt{-1}(\Lambda \bar{\partial}-\bar{\partial}\Lambda)\partial, \\
    \lap_{\bar{\partial}}&=\bar{\partial}\bar{\partial}^*+\bar{\partial}^*
    \bar{\partial}=-\sqrt{-1}\bar{\partial}(\Lambda\partial -\partial\Lambda)
    -\sqrt{-1}(\Lambda\partial-\partial\Lambda)\bar{\partial}
  .\end{align*}
  Hence \(\lap_{\partial}=\lap_{\bar{\partial}}\). Again by Hodge identities,
  \begin{align*}
    \partial\bar{\partial}^*+\bar{\partial}^*\partial&=-\sqrt{-1}\partial(\Lambda
    \partial-\partial\Lambda)-\sqrt{-1}(\Lambda\partial-\partial\Lambda)\partial=0,\\
    \bar{\partial}\partial^*+\partial^*\bar{\partial}&=\sqrt{-1}\bar{\partial}(\Lambda
    \bar{\partial}-\bar{\partial}\Lambda)+\sqrt{-1}(\Lambda\bar{\partial}
    -\bar{\partial}\Lambda)\bar{\partial}=0
  .\end{align*}
  From these we obtain \[
    \lap_{\dd{}}=2\lap_{\bar{\partial}}=2\lap_{\partial}
  .\] 
\end{proof}

Recall that \[
  \mathcal{H}_{\dd{}}^{r}(M)=\{\alpha\in \Omega^r(M):\lap_{\dd{}}\alpha=0\}
.\] We also apply Hodge theory for trivial holomorphic bundle \(E=M\times \mathbb{C}\),
then \(\Omega^{p,q}(E)=\Omega^{p,q}(M)\), and \[
  \mathcal{H}_{\bar{\partial}}^{p,q}(M)=\{\alpha\in \Omega^{p,q}(M):
  \lap_{\bar{\partial}}\alpha=0\}
.\] 
\begin{corollary}
  If \(M\) is a compact K\"ahler manifold, then \[
    \mathcal{H}_{\dd{}}^{r}(M)\otimes \mathbb{C}
    =\bigoplus_{p+q=r}\mathcal{H}_{\bar{\partial}}^{p,q}(M)
  .\] 
\end{corollary}
Apply de Rham theorem and Dolbeault theorem, we obtain
\begin{corollary}
  If \(M\) is a compact K\"ahler manifold, then \[
    H^{r}(M,\underset{\mathclap{\substack{\uparrow\\\text{constant}\\\text{sheaf}}}}
    {\mathbb{C}})\cong\bigoplus_{p+q=r}H^q(M,\underbrace{\Omega^{p}(M)}_{\mathclap{
    \substack{\text{sheaf of germs of}\\\text{holo }p\text{-forms}}}})
  .\] 
\end{corollary}

One more consequence of \(\lap_{\dd{}}=2\lap_{\bar{\partial}}=2\lap_{\partial}\) on
compact K\"ahler manifolds is that: Since \(\lap_{\dd{}}\) is a real operator, \ie\ \[
  \lap_{\dd{}}\bar{\vphi}=\overline{\lap_{\dd{}}\vphi}
.\] Hence \[
  \lap_{\bar{\partial}}\bar{\vphi}=\overline{\lap_{\bar{\partial}}\vphi}
\] and \[
  \lap_{\bar{\partial}}\vphi=0\iff \lap_{\bar{\partial}}\bar{\vphi}=0
.\] This proves the following:
\begin{theorem}
  On a compact K\"ahler manifold, \[
    H^{q}(M,\Omega^{p}(M))\cong H^{p}(M,\Omega^q(M))
  .\] 
\end{theorem}

Let \(E\to M\) be a holomorphic vector bundle over a compact complex manifold \(M\),
and \(E^*\to M\) be its dual bundle. For \(\alpha\in \mathcal{A}^{p,q}(E)\),
\(E\)-valued smooth \((p,q)\)-form and \(\beta\in \mathcal{A}^{m-p,m-q}(E^*)\), we
define a pairing \[
  \left<\alpha,\beta\right> =\int_{M}\left<\alpha\wedge \beta\right> 
.\] Where \(\left<\alpha\wedge \beta\right> \) is given by the pairing of \(E\)
and \(E^*\) and the wedge product as differential forms. Then this pairing is
non-degenerate and thus \[
  \mathcal{A}^{m-p,m-q}(E^*)=\left(\mathcal{A}^{p,q}(E)\right)^*
.\] Given a Hermitian metric on \(E\), and thus its inverse on \(E^*\), and a
Hermitian metric on \(M\), we can consider the space \(\mathcal{H}_{\bar{\partial}}^{p,
q}(M,E)\) and \(\mathcal{H}_{\bar{\partial}}^{p,q}(M,E^*)\) of harmonic
\((p,q)\)-forms. One can show that the natural pairing \[
  \left<\cdot ,\cdot \right> \colon\mathcal{H}_{\bar{\partial}}^{p,q}(E)\times 
  \mathcal{H}_{\bar{\partial}}^{m-p,m-q}(E^*)\to \mathbb{C}
\] is non-degenerate, and thus combining the Dolbeault theorem and Hodge theory, we
obtain the following:
\begin{theorem}[Kodaira-Serre duality]
  \[
    H^{q}(M,\Omega^p(E))\cong H^{m-q}(M,\Omega^{m-p}(E^*))^*
  .\] For \(p=0\), this is \[
    H^{q}(M,E)\cong H^{m-q}(M,\mathcal{O}(E^*\otimes K_M))^*
  .\]
\end{theorem}
\begin{remark}
  To prove the non-degeneracy of the pairing, one may use the Hodge star operator
  ``\(\star\)''. See page 103 of \emph{Griffiths-Harris}. The proof of this duality
  can be given purely algebraically (without using Hodge theory), see \emph{Hartshorne
  - Algebraic Geometry}, Chapter 3, section 7.
\end{remark}

Now we return to K\"ahler geometry.
Recall under the K\"ahler condition \(\dd{\gamma}=0\), the Levi-Civita connection
and the Chern connection on the tangent bundle coincides. We know the connection form
\(\omega\) is given by \[
  \omega=g^{-1}\partial g=(g^{i\bar{q}}\partial g_{j\bar{q}}),\quad
  g_{i\bar{j}}=g\left(\pd{z^i},\pd{\bar{z}^j}\right)
.\] So the Christoffel  symbol \[
  \Gamma_{kj}^i=g^{i\bar{q}}\pdv{g_{j\bar{q}}}{z^k}
.\] Also \[
  \Gamma_{\bar{k}j}^i=0
\] since the connection form is type (1,0).

Also, since the covariant derivative of Chern connection of \(T'M\) gives a section
of \(T'M\), we must have \[
  \Gamma_{kj}^{\bar{i}}=0,\quad\Gamma_{\bar{k}j}^{\bar{i}}=0
.\] Taking complex conjugate of these we have \[
  \Gamma_{k\bar{j}}^{\bar{i}}=\Gamma_{\bar{k}\bar{j}}^{i}=\Gamma_{k\bar{j}}^i=0
.\] As a conclusion, the Christoffel symbols are zero except for \[
  \Gamma_{kj}^i\text{ and }\Gamma_{\bar{k}\bar{j}}^{\bar{i}}
.\] The curvature is expressed by \[
  \Omega=\dd{\omega}+\omega\wedge \omega=\bar{\partial}\omega
  =\bar{\partial}(g^{-1}\partial g),\quad \Omega\indices{^i_j}=\bar{\partial}
  (g^{i\bar{q}}\partial g_{\bar{q}j})
.\] Thus \[
  R(X,Y)\pd{z^j}=\Omega\indices{^i_j}(X,Y)\pd{z^i}
.\] Since Riemannian curvature tensor is given by \[
  R(X,Y,Z,W)=g(Z,R(X,Y)W)
,\] we obtain
\begin{align*}
  R_{\bar{k}l\bar{p}j}&=g\left(\pd{\bar{z}^p},R\left(\pd{\bar{z}^k},\pd{z^l}\right)
  \pd{z^j}\right)=g\left(\pd{\bar{z}^p},\Omega\indices{^i_j}\left(\pd{\bar{z}^k},
  \pd{z^l}\right)\pd{z^i}\right) \\
  &=g_{\bar{p}i}\pd{\bar{z}^k}\left(g^{i\bar{q}}\pdv{g_{j\bar{q}}}{z^l}\right) \\
  &=\pdv{g_{j\bar{p}}}{z^l,\bar{z}^k}-g_{\bar{p}i}g^{i\bar{s}}g^{r\bar{q}}
  \pdv{g_{r\bar{s}}}{\bar{z}^k}\pdv{g_{j\bar{q}}}{z^l} \\
  &=\pdv{g_{j\bar{p}}}{z^l,\bar{z}^k}-g^{r\bar{q}}\pdv{g_{r\bar{p}}}{\bar{z}^k}
  \pdv{g_{j\bar{q}}}{z^l}
.\end{align*} 
Note that Riemannian metric symmetric, \ie\ \(g_{\bar{q}j}=g_{j\bar{q}}\).
Using the symmetries of Riemannian curvature tensor (See previous notes), \[
  R_{i\bar{j}k\bar{l}}=R_{\bar{j}i\bar{l}k}=R_{\bar{l}k\bar{j}i}
  =\pdv{g_{i\bar{j}}}{z^k,\bar{z}^l}-g^{p\bar{q}}\pdv{g_{i\bar{q}}}{z^k}
  \pdv{g_{p\bar{j}}}{\bar{z}^k}
.\] 
Let us define ``K\"ahler-Ricci'' curvature as follows (This terminology is not
standard. This is only in this lecture notes):

Suppose local coordinates \(z^1,\ldots,z^m\) so that \(\pd{z^1},\ldots,\pd{z^m}\) are
orthonormal basis of \(T'M\). Define for real vector field \(X\), \[
  \op{Kähler-Ric}(X,Y)=\sum_{i=1}^{m}R(X,\pd{\bar{z}^i},Y,\pd{\bar{z}^i})
.\] Since \(\pd{z}=\frac{1}{2}\left(\pd{x}-\sqrt{-1}\pd{y}\right)\) and \[
  \left|\pd{z}\right|=1\iff \left|\pd{x}\right|=\left|\pd{y}\right|=\sqrt{2}
,\] \[
  \op{Kähler-Ric}(X,Y)=\frac{1}{4}\sum_{i}R(X,\pd{x^i},Y,\pd{x^i})
  +R(X,\pd{y^i},Y,\pd{y^i})=\frac{1}{2}\Ric(X,Y)
.\] 

\begin{remark}
  Since the curvature form is given by \(\bar{\partial}(g^{-1}\partial g)\), \[
    R\indices{^\alpha_{kij}}=R\indices{^\alpha_{k\bar{i}\bar{j}}}=0,\quad
    \alpha=1,\ldots,m,\bar{1},\ldots,\bar{m}
  .\] By the symmetry of Riemannian curvature tensor, also \[
    R_{ij\alpha k}=R_{\bar{i}\bar{j}\alpha k}=0,\quad
    \alpha=1,\ldots,m,\bar{1},\ldots,\bar{m}
  .\] Thus the only possibly non-zero components of the curvature tensor on a Kähler
  manifold are \[
    R_{i\bar{j}k\bar{l}}=-R_{\bar{j}ik\bar{l}}
    =-R_{i\bar{j}\bar{l}k}=R_{\bar{j}i\bar{l}k}
  .\] The relation is obtained by Bianchi identities.
  So we define the tensor \(R_{i\bar{j}}\) by
  \begin{align*}
    R_{i\bar{j}}&=\op{Kähler-Ric}\left(\pd{z^i},\pd{\bar{z}^j}\right) \\
    &=g^{k\bar{l}}R_{i\bar{l}\bar{j}k}=-g^{k\bar{l}}R_{i\bar{j}k\bar{l}} \\
    &=R\indices{_{i\bar{j}}^k_k}=R\indices{^k_{ki\bar{j}}} \\
    &=(\op{tr}R)_{i\bar{j}}
  .\end{align*} 
  In local coordinates,
  \begin{align*}
    R_{i\bar{j}}&=-g^{k\bar{l}}R_{i\bar{j}k\bar{l}}=-g^{k\bar{l}}R_{k\bar{l}i\bar{j}}\\
    &=-g^{k\bar{l}}\pdv{g_{k\bar{l}}}{z^i,\bar{z}^j}+g^{k\bar{l}}g^{p\bar{q}}
    \pdv{g_{k\bar{q}}}{z^i}\pdv{g_{p\bar{l}}}{\bar{z}^j} \\
    &=-\pd{z^i,\bar{z}^j}\log\det(g_{k\bar{l}})
  .\end{align*}
\end{remark}

\end{document}
