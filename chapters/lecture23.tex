% !TeX program = xelatex
\documentclass[12pt]{article}
\usepackage{standalone}

\usepackage[dvipsnames,svgnames,x11names]{xcolor}
\usepackage[a4paper,margin=1in]{geometry}
\usepackage{microtype}
\usepackage{amsmath}
\usepackage{amsthm}
\usepackage{mathtools}
\usepackage{mathrsfs}
\usepackage{stmaryrd}
\usepackage{extarrows}
\usepackage{enumerate}
\usepackage{tensor}
\usepackage{physics}
\usepackage{graphicx}
\usepackage{subcaption}
\usepackage{tikz}
\usepackage{tikz-3dplot}
% \usepackage{tikz-cd}
\usepackage{quiver}
  \tikzset{every picture/.style={line width=0.75pt}}
\usepackage{pgfplots}
  \pgfplotsset{compat=newest}
\usepackage{tcolorbox}
  \tcbuselibrary{most}
\usepackage[colorlinks=true,linkcolor=blue]{hyperref}
\usepackage{cleveref}
\usepackage[hyperref=true,backend=biber,style=alphabetic,backref=true,url=false]{biblatex}
\usepackage[warnings-off={mathtools-colon,mathtools-overbracket}]{unicode-math}
\usepackage[default]{fontsetup}
\usepackage{fancyhdr}
\usepackage{import}

\newcommand{\Id}{\mathbb{1}}
\newcommand{\lap}{\increment}

\DeclareMathOperator{\sign}{sign}
\DeclareMathOperator{\dom}{dom}
\DeclareMathOperator{\ran}{ran}
\DeclareMathOperator{\ord}{ord}
\DeclareMathOperator{\Span}{span}
\DeclareMathOperator{\img}{Im}
\DeclareMathOperator{\Ric}{Ric}
\newcommand{\card}{\texttt{\#}}
\newcommand{\ie}{\emph{i.e.}}
\newcommand{\st}{\emph{s.t.}}
\newcommand{\eps}{\varepsilon}
\newcommand{\vphi}{\varphi}
\newcommand{\vthe}{\vartheta}
\newcommand{\II}{I\!I}
\renewcommand{\emptyset}{\varnothing}
\newcommand{\acts}{\curvearrowright}
\newcommand{\xrr}{\xlongrightarrow}
\newcommand{\into}{\hookrightarrow}
\newcommand{\pdif}[2]{\frac{\partial #1}{\partial #2}}
\renewcommand{\op}{\operatorname}

\theoremstyle{plain}\newtheorem{theorem}{Theorem}
\theoremstyle{definition}\newtheorem{definition}[theorem]{Definition}
\theoremstyle{definition}\newtheorem{example}[theorem]{Example}
\theoremstyle{plain}\newtheorem{axiom}[theorem]{Axiom}
\theoremstyle{plain}\newtheorem{corollary}[theorem]{Corollary}
\theoremstyle{plain}\newtheorem{lemma}[theorem]{Lemma}
\theoremstyle{plain}\newtheorem{proposition}[theorem]{Proposition}
\theoremstyle{plain}\newtheorem{prop}[theorem]{Proposition}
\theoremstyle{plain}\newtheorem{conjecture}[theorem]{Conjecture}
\theoremstyle{plain}\newtheorem{conj}[theorem]{Conjecture}
\theoremstyle{plain}\newtheorem{problem}[theorem]{Problem}
\theoremstyle{remark}\newtheorem{notation}[theorem]{Notation}
\theoremstyle{definition}\newtheorem*{question}{Question}
\theoremstyle{definition}\newtheorem*{answer}{Answer}
\theoremstyle{definition}\newtheorem*{goal}{Goal}
\theoremstyle{plain}\newtheorem*{application}{Application}
\theoremstyle{plain}\newtheorem*{exercise}{Exercise}
\theoremstyle{remark}\newtheorem*{remark}{Remark}
\theoremstyle{remark}\newtheorem*{note}{\small{Note}}
\numberwithin{equation}{section}
\numberwithin{theorem}{section}
\numberwithin{figure}{section}

\usepackage{xeCJK}
\setCJKmainfont{FZShuSong-Z01}[BoldFont=FZXiaoBiaoSong-B05,ItalicFont=FZKai-Z03]
\setCJKsansfont{FZXiHeiI-Z08}[BoldFont=FZHei-B01]
\setCJKmonofont{FZFangSong-Z02}
\setCJKfamilyfont{zhsong}{FZShuSong-Z01}[BoldFont=FZXiaoBiaoSong-B05]
\setCJKfamilyfont{zhhei}{FZHei-B01}
\setCJKfamilyfont{zhkai}{FZKai-Z03}
\setCJKfamilyfont{zhfs}{FZFangSong-Z02}
\setCJKfamilyfont{zhli}{FZLiShu-S01}
\setCJKfamilyfont{zhyou}{FZXiYuan-M01}[BoldFont=FZZhunYuan-M02]

\geometry{a4paper,margin=1in}
\allowdisplaybreaks{}

\counterwithout{theorem}{section}
\counterwithout{equation}{section}

\begin{document}
So we may believe that \(\pdv{M_t}{t}\) should be expressed in terms of \[
  \pdv{f_{\alpha\beta}^i(z_\beta,t)}{t_\lambda}
.\] Kodaira-Spencer showed this idea works. We put \[
  \theta_{\alpha\beta}(t)=\sum_i\pdv{f_{\alpha\beta}^i(z_\beta,t)}{t_\lambda}
  \pd{z_\alpha^i}
.\] For a complex manifold \(M\) we denote by \(\Theta=\mathcal{O}(T'M)\), and
call it the tangent sheaf. (This is a standard notation in deformation theory).

\begin{lemma}
\(\{\theta_{\alpha\beta}(t)\}\) is a 1-cocycle of \(\Theta_t\), \ie\ \[
  \theta_{\alpha\beta}+\theta_{\beta\gamma}+\theta_{\gamma\alpha}=0\quad
  \text{ on }U_\alpha\cap U_\beta\cap U_\gamma
\] and \(\theta_{\beta\alpha}=-\theta_{\alpha\beta}\). Here \(\Theta_t=
\mathcal{O}(T'M_t)\).
\end{lemma}
\begin{proof}
  We have \[
    z_\alpha^i=f_{\alpha\gamma}^i(z_\alpha,t)=f_{\alpha\beta}^i(z_\beta,t)
    =f_{\alpha\beta}^i(f_{\beta\gamma}(z_\gamma,t),t)
  .\] So \[
    \pd{z_\beta^j}=\pdv{f_{\alpha\beta}^i}{z_\beta^j}\pd{z_\alpha^i}
  .\] Thus
  \begin{align*}
    \theta_{\alpha\gamma}&=\pdv{f_{\alpha\gamma}^i}{t_\lambda} \\
    &=\pdv{f_{\alpha\beta}^i}{t_\lambda}\pd{z_\alpha^i}+\pdv{f_{\alpha\beta}^i}
    {z_\beta^j}\pdv{f_{\beta\gamma}^j}{t_\lambda}\pd{z_\alpha^i} \\
    &=\pdv{f_{\alpha\beta}^i}{t_\lambda}\pd{z_\alpha^i}
    +\pdv{f_{\alpha\beta}^i}{t_\lambda}\pd{z_\alpha^i}
    +\pdv{f_{\beta\gamma}^j}{t_\lambda}\pd{z_\beta^j} \\
    &=\theta_{\alpha\beta}+\theta_{\beta\gamma}
  .\end{align*}
  \(f_{\alpha\beta}(f_{\beta\alpha}(z,t),t)=\Id\), so \[
    \pdv{f_{\alpha\beta}^i}{t_\lambda}\pd{z_\alpha^i}+\pdv{z_\alpha^i}{z_\beta^j}
    \pdv{f_{\beta\alpha}^j}{t_\lambda}\pd{z_\alpha^i}=0
  .\] Hence \(\theta_{\alpha\beta}+\theta_{\beta\alpha}=0\).
\end{proof}

\begin{definition}
  The cohomology class \(\{\theta_{\alpha\beta}\}\) in \(H^1(\Theta_t)\) is
  defined to be \(\pdv{M_t}{t_\lambda}\), and called the \textbf{infinitesimal
  deformation}.

  Of course, in general \[
    \pd{t}=\sum_{\lambda=1}^{n}c_\lambda\pd{t_\lambda}
  ,\] and we define \[
    \pdv{M_t}{t}=\sum_{\lambda}c_\lambda\pdv{M_t}{t_\lambda}
  .\] 
\end{definition}

\begin{prop}
  The cohomology class \(\pdv{M_t}{t}\) does not depend on the choice of the 
  covering \(\bigcup_{\alpha}\mathcal{U}_\alpha\) and local coordinates
  \(z_\alpha^i\).
\end{prop}
\begin{proof}
  If we choose another coordinate charts \((w_\alpha^i,t)\) on the same open
  covering \(\bigcup_{\alpha}\mathcal{U}_\alpha\), we have two coordinate
  change \[
    z_\alpha^i=f_{\alpha\beta}^i(z_\beta,t),\quad
    w_\alpha^i=g_{\alpha\beta}^i(w_\beta,t),\quad
  \] and the coordinate change from \(w\) to \(z\), \[
    z_\alpha^i=h_\alpha^i(w_\alpha,t)
  .\] We put \[
    \theta_{\alpha\beta}=\pdv{f_{\alpha\beta}^i}{t_\lambda}\pd{z_\alpha^i},\quad
    \hat{\theta}_{\alpha\beta}=\pdv{g_{\alpha\beta}^i}{t_\lambda}\pd{w_\alpha^i}
  .\] But since \[
    f_{\alpha\beta}^i(h_\beta(w_\beta,t),t)=z_\alpha^i
    =h_\alpha^i(w_\alpha,t)=h_\alpha^i(g_{\alpha\beta}(w_\beta,t),t)
  ,\] we obtain \[
    \pdv{f_{\alpha\beta}^i}{t_\lambda}+\pdv{z_\alpha^i}{z_\beta^j}
    \pdv{h_\beta^j}{t_\lambda}=\pdv{h_\alpha^i}{t_\lambda}
    +\pdv{z_\alpha^i}{w_\alpha^j}\pdv{g_{\alpha\beta}^j}{t_\lambda}
  .\] Multiplying \(\pd{z_\alpha^i}\) and taking sum w.r.t. \(i\), we get \[
    \theta_{\alpha\beta}+\pdv{h_\beta^i}{t_\lambda}\pd{z_\beta^i}
    =\pdv{h_{\alpha}^i}{t_\lambda}\pd{z_\alpha^i}+\hat{\theta}_{\alpha\beta}
  .\] Putting  \[
    \eta_\alpha=\pdv{h_\alpha^i}{t_\lambda}\pd{z_\alpha^i} 
  ,\] we obtain \[
    \theta_{\alpha\beta}-\hat{\theta}_{\alpha\beta}=\eta_\alpha-\eta_\beta
  .\] Hence \(\hat{\theta}-\theta=\delta\{\eta_\alpha\}\).

  For general two open coverings \(\{U_\alpha\},\{V_\sigma\}\) with respective
  coordinates, we consider a common refinement. On this refinement we have two
  coordinates, but by the above arguments they define the same cohomology class.
  Since cohomology class does not change through refinement, two coverings with
  respective coordinate give the same cohomology class.
\end{proof}

One of main results of Kodaira-Spencer is the following:
\begin{theorem}
  Let  \(M\) be a compact complex manifold. If \(H^2(M,\Theta)=0\) then for any
  \(\beta\in H^1(M,\Theta)\), there is a deformation \(M_t\), \(t\) small,
  such that \[
    \pdv{M_t}{t}\bigg|_{t=0}=\beta
  .\] (c.f. Theorem 2.1 in \emph{Morrow-Kodaiara}, page 155).
\end{theorem}
This result was extended by Kuranishi in the case of \(H^2(M,\Theta)\neq 0\).
His theorem is stated roughly as follows:
\begin{theorem}
  There is a map \(\psi\colon H^1(M,\Theta)\to H^2(M,\Theta)\) such that
  \(\psi^{-1}(0)\) parametrizes the deformations of complex structures of \(M\).
  (c.f. Theorem 3.1 in \emph{Morrow-Kodaira}, page 192).
\end{theorem}

The method of Kuranishi was applied in many other non-linear problems involving
moduli spaces, notably Donaldson theory on self-duality equation on 4-manifolds.

See Freed-Uhlenbeck's book.

\hfill

\noindent\textbf{\large More on deformation of complex structures}
\[\begin{tikzcd}[row sep=tiny]
	\displaystyle H^1(M,\mathcal{O}(T'M)) & \displaystyle H_{\bar{\partial}}^{0,1}
  (M,T'M) \\
	\displaystyle \pdv{M_t}{t}\bigg|_{t=0} & \displaystyle \eta=
  \eta_{\bar{\alpha}}^{\beta}\pd{z^\beta}\otimes
  \dd{\bar{z}^{\alpha}}
	\arrow["{\displaystyle\cong}"{description}, draw=none, from=1-1, to=1-2]
	\arrow[tail reversed, from=2-1, to=2-2]
\end{tikzcd}\]
\begin{prop}
  \(\exists\,\vphi(t)\in \mathcal{A}^{0,1}(T'M)\) such that
  \begin{enumerate}[(1)]
  \item \(f\) holomorphic with respect to \(\bar{\partial}_t\) if and only if \[
      \pdv{f}{\bar{z}^{\alpha}}-\vphi_{\bar{\alpha}}^{\beta}\pdv{f}{z^\beta}=0
    .\] 
  \item \(\vphi(0)=0\).
  \item \(\bar{\partial}\vphi(t)-\frac{1}{2}[\vphi(t),\vphi(t)]=0\).
  \item \(\dot{\vphi}(0)=-\eta\).
  \end{enumerate}
  Here for \(\vphi\in \mathcal{A}^{p}(T'M),\psi\in \mathcal{A}^q(T'M)\), \[
    [\vphi,\psi]:=\vphi\psi-(-1)^{pq}\psi\vphi
  .\] So \[
    [\vphi,\vphi]=2\vphi\vphi=2\phi^\beta\partial_\beta\vphi^\gamma
    \partial_\gamma
  .\] More detail will be given later.
\end{prop}

\section{Chern-Weil theory}
Let \(E\to M\) be a smooth complex vector bundle of rank \(r\) over a compact
real smooth manifold \(M\). Let \(\nabla\) be a connection and \[
  \dd^\nabla\colon C^\infty(M,E\otimes \wedge^p M)\lrr
  C^\infty(M,E\otimes\wedge^{p+1}(M))
\] be the exterior covariant derivative.

Recall that \[
  R=\dd^\nabla\circ \dd^\nabla\in C^\infty(M,\op{End}(E)\otimes\wedge^2 M)
\] is the curvature 2-form. As a homework we studied the Bianchi identity \[
  \dd^\nabla R=0
.\] 

Probably the best answer for the homework to prove it would be as follows:
\begin{proof}
  For a section \(s\in C^\infty(M,E)\), \[
    Rs=\dd^\nabla (\dd^\nabla s)
  .\] Then \[
    (\dd^\nabla R)s=\dd^\nabla (Rs)-R(\dd^\nabla S)
    =\dd^\nabla (\dd^\nabla \dd^\nabla s)-(\dd^\nabla\dd^\nabla)(\dd^\nabla s)
    =0
  .\] Write \(\det E:=\wedge^r E\), which is a complex line bundle and
  \(\op{End}(\det E)\cong (\det E)\otimes(\det E)^{-1}\) is a trivial line
  bundle.
\end{proof}

\end{document}
