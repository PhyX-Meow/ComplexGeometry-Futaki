% !TeX program = xelatex
\documentclass[12pt]{article}
\usepackage{standalone}

\usepackage[dvipsnames,svgnames,x11names]{xcolor}
\usepackage[a4paper,margin=1in]{geometry}
\usepackage{microtype}
\usepackage{amsmath}
\usepackage{amsthm}
\usepackage{mathtools}
\usepackage{mathrsfs}
\usepackage{stmaryrd}
\usepackage{extarrows}
\usepackage{enumerate}
\usepackage{tensor}
\usepackage{physics2}
  \usephysicsmodule{ab,xmat}
\usepackage{fixdif}
  \newcommand{\dd}{\d}
\usepackage{derivative}
  \newcommand{\dv}{\odv}
  \newcommand{\pd}[1]{\pdv{}{#1}}
  \newcommand{\eval}[1]{#1\big|}
\usepackage{graphicx}
\usepackage{subcaption}
\usepackage{tikz}
\usepackage{tikz-3dplot}
% \usepackage{tikz-cd}
% \usepackage{quiver}
% ========== Block quiver.sty ========== %
\usepackage{tikz-cd}
% \usepackage{amssymb}
\usetikzlibrary{calc}
\usetikzlibrary{decorations.pathmorphing}
\tikzset{curve/.style={settings={#1},to path={(\tikztostart)
    .. controls ($(\tikztostart)!\pv{pos}!(\tikztotarget)!\pv{height}!270:(\tikztotarget)$)
    and ($(\tikztostart)!1-\pv{pos}!(\tikztotarget)!\pv{height}!270:(\tikztotarget)$)
    .. (\tikztotarget)\tikztonodes}},
    settings/.code={\tikzset{quiver/.cd,#1}
        \def\pv##1{\pgfkeysvalueof{/tikz/quiver/##1}}},
    quiver/.cd,pos/.initial=0.35,height/.initial=0}
\tikzset{tail reversed/.code={\pgfsetarrowsstart{tikzcd to}}}
\tikzset{2tail/.code={\pgfsetarrowsstart{Implies[reversed]}}}
\tikzset{2tail reversed/.code={\pgfsetarrowsstart{Implies}}}
\tikzset{no body/.style={/tikz/dash pattern=on 0 off 1mm}}
% =========== End block ========== %
  \tikzset{every picture/.style={line width=0.75pt}}
\usepackage{pgfplots}
  \pgfplotsset{compat=newest}
\usepackage{tcolorbox}
  \tcbuselibrary{most}
\usepackage[colorlinks=true,linkcolor=blue]{hyperref}
\usepackage{cleveref}
% \usepackage[hyperref=true,backend=biber,style=alphabetic,backref=true,url=false]{biblatex}
\usepackage[warnings-off={mathtools-colon,mathtools-overbracket}]{unicode-math}
\usepackage[default,amsbb]{fontsetup}
  \setmathfont[StylisticSet=1,range=\mathscr]{NewCMMath-Book.otf}
\usepackage{fancyhdr}
\usepackage{import}

\newcommand{\Id}{\mathbb{1}}
\newcommand{\lap}{\increment}

\DeclareMathOperator{\sign}{sign}
\DeclareMathOperator{\dom}{dom}
\DeclareMathOperator{\ran}{ran}
\DeclareMathOperator{\ord}{ord}
\DeclareMathOperator{\Span}{span}
\DeclareMathOperator{\img}{Im}
\DeclareMathOperator{\Ric}{Ric}
\newcommand{\card}{\texttt{\#}}
\newcommand{\ie}{\emph{i.e.}}
\newcommand{\st}{\emph{s.t.}}
\newcommand{\eps}{\varepsilon}
\newcommand{\vphi}{\varphi}
\newcommand{\vthe}{\vartheta}
\newcommand{\II}{I\!I}
\renewcommand{\emptyset}{⌀}
\newcommand{\acts}{\curvearrowright}
\newcommand{\xrr}{\xlongrightarrow}
\newcommand{\lrr}{\longrightarrow}
\newcommand{\lmt}{\longmapsto}
\newcommand{\into}{\hookrightarrow}
\newcommand{\op}{\operatorname}

\let\originalleft\left
\let\originalright\right
\renewcommand{\left}{\mathopen{}\mathclose\bgroup\originalleft}
\renewcommand{\right}{\aftergroup\egroup\originalright}

\theoremstyle{plain}\newtheorem{theorem}{Theorem}
\theoremstyle{definition}\newtheorem{definition}[theorem]{Definition}
\theoremstyle{definition}\newtheorem{example}[theorem]{Example}
\theoremstyle{definition}\newtheorem{problem}[theorem]{Problem}
\theoremstyle{plain}\newtheorem{axiom}[theorem]{Axiom}
\theoremstyle{plain}\newtheorem{corollary}[theorem]{Corollary}
\theoremstyle{plain}\newtheorem{lemma}[theorem]{Lemma}
\theoremstyle{plain}\newtheorem{proposition}[theorem]{Proposition}
\theoremstyle{plain}\newtheorem{prop}[theorem]{Proposition}
\theoremstyle{plain}\newtheorem{conjecture}[theorem]{Conjecture}
\theoremstyle{plain}\newtheorem{conj}[theorem]{Conjecture}
\theoremstyle{remark}\newtheorem{notation}[theorem]{Notation}
\theoremstyle{definition}\newtheorem*{question}{Question}
\theoremstyle{definition}\newtheorem*{answer}{Answer}
\theoremstyle{definition}\newtheorem*{goal}{Goal}
\theoremstyle{definition}\newtheorem*{application}{Application}
\theoremstyle{plain}\newtheorem*{exercise}{Exercise}
\theoremstyle{remark}\newtheorem*{remark}{Remark}
\theoremstyle{remark}\newtheorem*{note}{\small{Note}}
\numberwithin{equation}{section}
\numberwithin{theorem}{section}
\numberwithin{figure}{section}

\usepackage{xeCJK}
\setCJKmainfont{FZShuSong-Z01}[BoldFont=FZXiaoBiaoSong-B05,ItalicFont=FZKai-Z03]
\setCJKsansfont{FZXiHeiI-Z08}[BoldFont=FZHei-B01]
\setCJKmonofont{FZFangSong-Z02}
\setCJKfamilyfont{zhsong}{FZShuSong-Z01}[BoldFont=FZXiaoBiaoSong-B05]
\setCJKfamilyfont{zhhei}{FZHei-B01}
\setCJKfamilyfont{zhkai}{FZKai-Z03}
\setCJKfamilyfont{zhfs}{FZFangSong-Z02}
\setCJKfamilyfont{zhli}{FZLiShu-S01}
\setCJKfamilyfont{zhyou}{FZXiYuan-M01}[BoldFont=FZZhunYuan-M02]

\allowdisplaybreaks{}

\newcommand{\isFullBook}[2]{
  \ifnum\pdfstrcmp{\FullBook}{True}=0
    \ifnum\pdfstrcmp{}{#1}=0\unskip\else#1\fi
  \else
    \ifnum\pdfstrcmp{}{#2}=0\unskip\else#2\fi
  \fi\ignorespaces{}
}

\counterwithout{theorem}{section}
\counterwithout{equation}{section}

\begin{document}
Let \(M\) be a compact complex manifold. Recall that from the short exact sequence \[
  0\lrr \mathbb{Z}\xrr{j}\mathcal{O}_M\xrr{e}\mathcal{O}_M^*\lrr 0,\quad
  e(a)=\exp(2\pi\sqrt{-1}a)
,\] we obtain an exact sequence \[
  H^1(M,\mathcal{O}_M^*)\xrr{\delta^*}H^2(M,\mathbb{Z})\xrr{j^*}H^1(M,\mathcal{O}_M)
.\] On the other hand, from the inclusion \(\mathbb{Z}\to \mathbb{C}\) and the
de Rham theorem we also have \[
  H^2(M,\mathbb{Z})\lrr H^2(M,\mathbb{C})\xrr{\sim} H_{\mathrm{dR}}^2(M,\mathbb{C})
.\] 

\begin{definition}
  We write \(H^{1,1}(M,\mathbb{Z})\) for the set of all elements \(\gamma\) in
  \(H^2(M,\mathbb{Z})\) such that \(\gamma\) is represented by a (1,1)-form in
  \(H_{\mathrm{dR}}^2(M,\mathbb{C})\) by the above map \[
    H^2(M,\mathbb{Z})\lrr H_{\mathrm{dR}}^2(M,\mathbb{C})
  .\] 
\end{definition}
As we saw in the last lecture there is a map \[
  \delta^*\colon H^1(M,\mathcal{O}_M^*)\lrr H^2(M,\mathbb{Z}),\quad
  [L]\longmapsto c_1(L)
.\] If we choose a Hermitian metric of \(L\) then its curvature is type (1,1) and
\(c_1(L)\in H^{1,1}(M,\mathbb{Z})\).

\begin{theorem}
  Let \(M\) be a compact complex manifold, then \[
    H^{1,1}(M,\mathbb{Z})=\delta^*(H^1(M,\mathcal{O}_M^*))
  .\] 
\end{theorem}
\begin{proof}
  This follows from the exact sequence \[
    H^1(M,\mathcal{O}_M^*)\xrr{\delta^*}H^2(M,\mathbb{Z})\xrr{j^*}H^2(M,\mathcal{O}_M)
  \] and the next lemma.
\end{proof}

\begin{lemma}
  \(j^*(H^{1,1}(M,\mathbb{Z}))=0\).
\end{lemma}
\begin{proof}
  By de Rham theorem and Dolbeault theorem we have
  \[\begin{tikzcd}[row sep=tiny]
    && {H^{2}(M,\mathbb{C})} & {H_{\mathrm{dR}}^{2}(M,\mathbb{C})} \\
    {H^{1,1}(M,\mathbb{Z})} & {H^{2}(M,\mathbb{Z})} \\
    && {H^{2}(M,\mathcal{O}_M)} & {H_{\mathrm{Dol}}^{0,2}(M,\mathbb{C})}
    \arrow[hook, from=2-1, to=2-2]
    \arrow["{j^*}", from=2-2, to=3-3]
    \arrow["\sim", from=3-3, to=3-4]
    \arrow["\sim", from=1-3, to=1-4]
    \arrow[from=2-2, to=1-3]
  \end{tikzcd}\]
  We wish to express \(c\in H^{1,1}(M,\mathbb{Z})\) both as a de Rham class and
  Dolbeault class. First, from \[
    H^2(M,\mathbb{Z})\lrr H^2(M,\mathbb{C})\lrr H^2(M,\mathscr{A})=0
  ,\] for any \(c=\{c_{\lambda\mu\nu}\}\in H^2(M,Z)\), \[
    c_{\lambda\mu\nu}=\delta\{\alpha\}_{\lambda\mu\nu}=\alpha_{\mu\nu}
    -\alpha_{\lambda\mu}+\alpha_{\lambda\mu}
  \] for some \(\{\alpha_{\lambda\mu}\}\in C^1(\mathcal{U},\mathscr{A})\). Since 
  \(c_{\lambda\mu\nu}\) are constant, \[
    \{\dd{\alpha_{\lambda\mu}}\}\in Z^1(\mathcal{U},\mathcal{A}^1)
  .\] (Recall that \(\mathcal{A}^1\) is the sheaf of complex valued \(C^\infty\)
  1-forms). But \(H^1(M,\mathcal{A}^1)=0\) as \(\mathcal{A}^1\) is a fine sheaf.
  Thus \[
    \dd{\alpha_{\lambda\mu}}=\beta_\mu-\beta_\lambda
  \] for some \(\beta=\{\beta_\lambda\}\in C^0(\mathcal{U},\mathcal{A}^1)\). Then \[
    \gamma=\dd{\beta_\lambda}=\dd{\beta_\mu}
  \] is a global 2-form. This is the class \(c\) in de Rham cohomology. Let us repeat
  the same computation in Dolbeault cohomology. For that purpose, decompose previous
  \(\alpha_{\lambda\mu}\) by \[
    \dd{\alpha_{\lambda\mu}}=\partial\alpha_{\lambda\mu}
    +\bar{\partial}\alpha_{\lambda\mu}
  .\] Then \(\{\bar{\partial}\alpha_{\lambda\mu}\}\in Z^1(\mathcal{U},\mathcal{A}^{0,1}
  )\). But \(H^1(M,\mathcal{A}^{0,1})=0\) since \(\mathcal{A}^{0,1}\) is a fine sheaf.
  In fact if we decompose previous \(\beta_\lambda\) as \[
    \beta_\lambda=\beta_\lambda'+\beta_\lambda''\in \mathcal{A}^{1,0}
    \oplus \mathcal{A}^{0,1}
  ,\] then \[
    \bar{\partial}\alpha_{\lambda\mu}=\beta_\mu''-\beta_\lambda''
  .\] If we decompose previous \(\gamma\) as \[
    \gamma=\gamma^{2,0}+\gamma^{1,1}+\gamma^{0,2}
  ,\] then \[
    \gamma^{0,2}=\bar{\partial}\beta_\lambda''=\bar{\partial}\beta_\mu''
  ,\] and \(j^*c=[\gamma^{0,2}]\). Our assumption is \(c=[\gamma]\) is represented by
  type (1,1)-form, so \[
    \gamma=\omega+\dd{\xi}
  \] for some closed (1,1)-form \(\omega\) and a 1-form \(\xi\). This shows \[
    [\gamma^{0,2}]=[\bar{\partial}\xi^{0,1}]=0
  \] in \(H^2(M,\mathcal{O})\cong H^{0,2}_{\mathrm{Dol}}(M)\).
\end{proof}

We also have the following:
\begin{theorem}
  For a compact complex manifold \(M\), \[
    \op{Pic}(M)/\op{Pic}^0(M)\cong H^{1,1}(M,\mathbb{Z})
  .\] 
\end{theorem}
\begin{proof}
  By the previous result, we have a surjective map \[
    \op{Pic}(M)\cong H^1(M,\mathcal{O}^*)\lrr H^{1,1}(M,\mathbb{Z})
  .\] The kernel of this map is exactly \(\op{Pic}^0(M)\) so we obtain an
  isomorphism \[
    \op{Pic}(M)/\op{Pic}^0(M)\xrr{\sim}H^{1,1}(M,\mathbb{Z})
  .\] We put \[
    \rho:=\op{rank}_{\mathbb{Z}}\op{Pic}(M)/\op{Pic}^0(M)
  ,\] where the rank is as an Abelian group.
\end{proof}
\begin{definition}
  This \(\rho\) is called the \textbf{Picard number}.
\end{definition}
\begin{definition}
  If a Kähler form of a Kähler metric represents \(H^{1,1}(M,\mathbb{Z})\) then we
  call the Kähler metric the \textbf{Hodge metric}. If a compact complex manifold 
  admits a Hodge metric then it is called a \textbf{Hodge manifold}.
\end{definition}
\begin{theorem}
  A compact manifold \(M\) is a Hodge manifold if and only if there exists an ample
  line bundle. Thus \(M\) is Hodge if and only if \(M\) is algebraic.
\end{theorem}
\begin{proof}
  If \(L\) is an ample line bundle, its first Chern class is a positive form, thus
  a Kähler form representing a class in \(H^{1,1}(M,\mathbb{Z})\). Conversely, if
  \(M\) is Hodge then the Kähler class represents \(H^{1,1}(M,\mathbb{Z})\). But by
  the previous theorem, there is a line bundle \(L\) such that \(c_1(L)\) is the
  Kähler class. Hence \(L\) is the desired ample line bundle.

  The last statement follows from Kodaira embedding.
\end{proof}

\begin{corollary}
  If \(M\) is a compact Kähler manifold and \(H^2(M,\mathcal{O}_M)=0\), then \(M\)
  is algebraic.
\end{corollary}
\begin{proof}
  Since \(H^{2,0}(M,\mathbb{C})\cong H^{0,2}(M,\mathbb{C})=0\), we have
  \(H^2(M,\mathbb{R})=H^{1,1}(M,\mathbb{R})\) and the image of \(H^2(M,\mathbb{Z})\) is
  \(H^{1,1}(M,\mathbb{Z})\). Here we send it by \(\mathbb{Z}\to \mathbb{R}\) and
  denote by the same letter \(H^{1,1}(M,\mathbb{Z})\). \(H^{1,1}(M,\mathbb{Q})\) is
  dense in \(H^{1,1}(M,\mathbb{R})\) and \(H^{1,1}(M,\mathbb{Q})=H^2(M,\mathbb{Q})\)
  because \(H^{1,1}(M,\mathbb{Z})=H^2(M,\mathbb{Z})\). The given Kähler form is
  in \(H^2(M,\mathbb{R})\) and is approximated by a sequence of closed 2-forms
  in \(H^2(M,\mathbb{Q})=H^{1,1}(M,\mathbb{Q})\). These 2-forms are positive forms
  since they are converging to the Kähler form. Choose a member of such sequence,
  and multiply an appropriate positive integer to get a positive form representing
  \(H^{1,1}(M,\mathbb{Z})\).
\end{proof}
As an immediate corollary:
\begin{corollary}
  Any compact Riemann surface is an algebraic curve.
\end{corollary}

Using previous terminology, \((M,L)\) is a polarized manifold for some line bundle
\(L\) if and only if \(M\) is a Hodge manifold. In this case, \(L\) is called the
polarization. (The term ``polarization'' is used in many other contexts with differnet
meanings).


\end{document}
