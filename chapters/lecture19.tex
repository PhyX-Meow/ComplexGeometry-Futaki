% !TeX program = xelatex
\documentclass[12pt]{article}
\usepackage{standalone}

\usepackage[dvipsnames,svgnames,x11names]{xcolor}
\usepackage[a4paper,margin=1in]{geometry}
\usepackage{microtype}
\usepackage{amsmath}
\usepackage{amsthm}
\usepackage{mathtools}
\usepackage{mathrsfs}
\usepackage{stmaryrd}
\usepackage{extarrows}
\usepackage{enumerate}
\usepackage{tensor}
\usepackage{physics2}
  \usephysicsmodule{ab,xmat}
\usepackage{fixdif}
  \newcommand{\dd}{\d}
\usepackage{derivative}
  \newcommand{\dv}{\odv}
  \newcommand{\pd}[1]{\pdv{}{#1}}
  \newcommand{\eval}[1]{#1\big|}
\usepackage{graphicx}
\usepackage{subcaption}
\usepackage{tikz}
\usepackage{tikz-3dplot}
% \usepackage{tikz-cd}
% \usepackage{quiver}
% ========== Block quiver.sty ========== %
\usepackage{tikz-cd}
% \usepackage{amssymb}
\usetikzlibrary{calc}
\usetikzlibrary{decorations.pathmorphing}
\tikzset{curve/.style={settings={#1},to path={(\tikztostart)
    .. controls ($(\tikztostart)!\pv{pos}!(\tikztotarget)!\pv{height}!270:(\tikztotarget)$)
    and ($(\tikztostart)!1-\pv{pos}!(\tikztotarget)!\pv{height}!270:(\tikztotarget)$)
    .. (\tikztotarget)\tikztonodes}},
    settings/.code={\tikzset{quiver/.cd,#1}
        \def\pv##1{\pgfkeysvalueof{/tikz/quiver/##1}}},
    quiver/.cd,pos/.initial=0.35,height/.initial=0}
\tikzset{tail reversed/.code={\pgfsetarrowsstart{tikzcd to}}}
\tikzset{2tail/.code={\pgfsetarrowsstart{Implies[reversed]}}}
\tikzset{2tail reversed/.code={\pgfsetarrowsstart{Implies}}}
\tikzset{no body/.style={/tikz/dash pattern=on 0 off 1mm}}
% =========== End block ========== %
  \tikzset{every picture/.style={line width=0.75pt}}
\usepackage{pgfplots}
  \pgfplotsset{compat=newest}
\usepackage{tcolorbox}
  \tcbuselibrary{most}
\usepackage[colorlinks=true,linkcolor=blue]{hyperref}
\usepackage{cleveref}
% \usepackage[hyperref=true,backend=biber,style=alphabetic,backref=true,url=false]{biblatex}
\usepackage[warnings-off={mathtools-colon,mathtools-overbracket}]{unicode-math}
\usepackage[default,amsbb]{fontsetup}
  \setmathfont[StylisticSet=1,range=\mathscr]{NewCMMath-Book.otf}
\usepackage{fancyhdr}
\usepackage{import}

\newcommand{\Id}{\mathbb{1}}
\newcommand{\lap}{\increment}

\DeclareMathOperator{\sign}{sign}
\DeclareMathOperator{\dom}{dom}
\DeclareMathOperator{\ran}{ran}
\DeclareMathOperator{\ord}{ord}
\DeclareMathOperator{\Span}{span}
\DeclareMathOperator{\img}{Im}
\DeclareMathOperator{\Ric}{Ric}
\newcommand{\card}{\texttt{\#}}
\newcommand{\ie}{\emph{i.e.}}
\newcommand{\st}{\emph{s.t.}}
\newcommand{\eps}{\varepsilon}
\newcommand{\vphi}{\varphi}
\newcommand{\vthe}{\vartheta}
\newcommand{\II}{I\!I}
\renewcommand{\emptyset}{⌀}
\newcommand{\acts}{\curvearrowright}
\newcommand{\xrr}{\xlongrightarrow}
\newcommand{\lrr}{\longrightarrow}
\newcommand{\lmt}{\longmapsto}
\newcommand{\into}{\hookrightarrow}
\newcommand{\op}{\operatorname}

\let\originalleft\left
\let\originalright\right
\renewcommand{\left}{\mathopen{}\mathclose\bgroup\originalleft}
\renewcommand{\right}{\aftergroup\egroup\originalright}

\theoremstyle{plain}\newtheorem{theorem}{Theorem}
\theoremstyle{definition}\newtheorem{definition}[theorem]{Definition}
\theoremstyle{definition}\newtheorem{example}[theorem]{Example}
\theoremstyle{definition}\newtheorem{problem}[theorem]{Problem}
\theoremstyle{plain}\newtheorem{axiom}[theorem]{Axiom}
\theoremstyle{plain}\newtheorem{corollary}[theorem]{Corollary}
\theoremstyle{plain}\newtheorem{lemma}[theorem]{Lemma}
\theoremstyle{plain}\newtheorem{proposition}[theorem]{Proposition}
\theoremstyle{plain}\newtheorem{prop}[theorem]{Proposition}
\theoremstyle{plain}\newtheorem{conjecture}[theorem]{Conjecture}
\theoremstyle{plain}\newtheorem{conj}[theorem]{Conjecture}
\theoremstyle{remark}\newtheorem{notation}[theorem]{Notation}
\theoremstyle{definition}\newtheorem*{question}{Question}
\theoremstyle{definition}\newtheorem*{answer}{Answer}
\theoremstyle{definition}\newtheorem*{goal}{Goal}
\theoremstyle{definition}\newtheorem*{application}{Application}
\theoremstyle{plain}\newtheorem*{exercise}{Exercise}
\theoremstyle{remark}\newtheorem*{remark}{Remark}
\theoremstyle{remark}\newtheorem*{note}{\small{Note}}
\numberwithin{equation}{section}
\numberwithin{theorem}{section}
\numberwithin{figure}{section}

\usepackage{xeCJK}
\setCJKmainfont{FZShuSong-Z01}[BoldFont=FZXiaoBiaoSong-B05,ItalicFont=FZKai-Z03]
\setCJKsansfont{FZXiHeiI-Z08}[BoldFont=FZHei-B01]
\setCJKmonofont{FZFangSong-Z02}
\setCJKfamilyfont{zhsong}{FZShuSong-Z01}[BoldFont=FZXiaoBiaoSong-B05]
\setCJKfamilyfont{zhhei}{FZHei-B01}
\setCJKfamilyfont{zhkai}{FZKai-Z03}
\setCJKfamilyfont{zhfs}{FZFangSong-Z02}
\setCJKfamilyfont{zhli}{FZLiShu-S01}
\setCJKfamilyfont{zhyou}{FZXiYuan-M01}[BoldFont=FZZhunYuan-M02]

\allowdisplaybreaks{}

\newcommand{\isFullBook}[2]{
  \ifnum\pdfstrcmp{\FullBook}{True}=0
    \ifnum\pdfstrcmp{}{#1}=0\unskip\else#1\fi
  \else
    \ifnum\pdfstrcmp{}{#2}=0\unskip\else#2\fi
  \fi\ignorespaces{}
}

\counterwithout{theorem}{section}
\counterwithout{equation}{section}

\begin{document}
Recall a line bundle is said to be \textbf{ample} or \textbf{positive} if its
first Chern class is represented by a positive form (\ie\ a Kähler form).

The assumption of last theorem is rephrased as ``If \(L\otimes K^{-1}\) is
ample'' since \(c_1(K^{-1})\) is represented by the Ricci form. Since divisors
and line bundle are identified, tensor products of line bundle are expressed
additively. So \(L\otimes K^{-1}=L-K\). Thus we obtain
\begin{theorem}\label{thm:kodaira-vanishing}
  If \(L-K\) is ample, then \[
    H^q(M,\mathcal{O}(L))=0\quad\text{ for }q\ge 1
  .\] 
\end{theorem}
\begin{theorem}
  If \(-L\) is positive (or \(L\) is negative), then \[
    H^q(M,\mathcal{O}(L))=0\quad\text{ for }q\le n-1
  .\] (We have proved for \(q=0\) using Bochner's formula).
\end{theorem}
\begin{proof}
  By Serre duality, \[
    H^q(M,\mathcal{O}(L))\cong H^{m-q}(M,\Omega^m(-L))
    \cong H^{m-q}(M,\mathcal{O}(K-L))
  .\] Note that \(L^*\cong -L\), and the last term is \(0\) if \((K-L)-K\) is
  positive. \ie\ \(-L\) is positive by applying \cref{thm:kodaira-vanishing}.
\end{proof}

\section{Kodaira embedding theorem}
The following is used in Kodaira embedding theorem.
\begin{theorem}
  Let \(L\to M\) be an ample line bundle. Then there is a sufficiently large
  number \(N>0\) such that for any \(n\ge N\), \[
    H^q(M,\mathcal{O}(L^m))=0\quad\text{ for }q\ge 1
  .\] (\(L^m=L\otimes\cdots\otimes L\), \(m\)-times tensor product).
\end{theorem}
\begin{proof}
  For a harmonic \((0,q)\)-form \(\vphi\) with value in \(L\), we have \[
    0=(\bar{\partial}\vphi,\bar{\partial}\vphi)_{L^2}
    +\int_{M}\sum_{\beta=1}^{q}g^{i_1\bar{j}_1}\cdots (n\psi_{l\bar{j}_p}
    +R_{l\bar{j}_p})\cdots g^{i_q\bar{j}_q}
    \vphi\indices{_{\bar{j}_1\cdots}^l_{\cdots \bar{j}_q}}
    \overline{\vphi\indices{_{\bar{j}_1\cdots}^{j_p}_{\cdots \bar{j}_q}}}
  .\] Since \((\psi_{i\bar{j}})\) is positive definite, \((n\psi_{i\bar{j}}
  +R_{i\bar{j}})\) is positive definite for sufficiently large \(n\). Thus if
  \(\vphi\neq 0\) then the right hand side is strictly positive. This is a
  contradiction.
\end{proof}

Theorem 7.7 in \emph{Morrow-Kodaira} states the vanishing for \(H^q(M,
\Omega^p(L))\) for ``sufficiently large'' \(L\). In this case it is not easy
to state neatly as we stated above for \(p=0\), because what matters is the
comparison of \(\psi_{i\bar{j}}\) with \(R_{i\bar{j}k\bar{l}}\).

These vanishing theorems are due to Kodaira, but \cref{thm:kodaira-vanishing}
was letter extended by Nakano in the following way.

\begin{theorem}\label{thm:nakano-vanishing}
  If \(L\) is a negative line bundle, then \(H^q(M,\Omega^p(L))=0\) for \(p+q
  <n\).
\end{theorem}
We need a lemma for this. Write \(e(\psi)\alpha=\psi\wedge\alpha\) where \(\psi\)
is the curvature of \(L\) as before.
\begin{lemma}
  For \(L\)-valued \(\bar{\partial}\)-harmonic \(\alpha\), \[
    \sqrt{-1}((\Lambda e(\psi)-e(\psi)\Lambda)\alpha,\alpha)
    =\|\nabla'\alpha\|_{L^2}^2+\|\nabla^{\prime*}\alpha\|_{L^2}^2
  .\] Where we decompose \[
    \dd^\nabla =\nabla'+\nabla''=\nabla'+\bar{\partial}
  .\] 
\end{lemma}
\begin{proof}
  For \(L\)-valued \((1,0)\)-form \(\alpha\), \[
    \nabla^{\prime*}=-g^{i\bar{j}}\nabla_{\bar{j}}\alpha_i
  .\] For bundle part, \(\nabla_{\bar{j}}=\bar{\partial}_j\) and this is the 
  same for ordinary \((1,0)\)-form. Thus \[
    \nabla^{\prime*}=\partial^*
  .\] And then
  \begin{align*}
    (\nabla'\alpha,\nabla'\alpha)_{L^2}&=(\partial^*\nabla'\alpha,\alpha) \\
    &=-\sqrt{-1}([\bar{\partial},\Lambda]\nabla'\alpha,\alpha) \\
    &=-\sqrt{-1}(\Lambda \nabla'\alpha,\bar{\partial}^*\alpha)+\sqrt{-1}
    (\Lambda(\bar{\partial}\nabla'+\nabla'\bar{\partial})\alpha,\alpha) \\
    &=\sqrt{-1}(\Lambda e(\psi)\alpha,\alpha)
  .\end{align*}
  Note that we used the fact that curvature is type \((1,1)\), so \[
    \psi=\dd^\nabla\circ \dd^\nabla =(\nabla'+\bar{\partial})\circ 
    (\nabla'+\bar{\partial})=\nabla'\bar{\partial}+\bar{\partial}\nabla'
  .\] Similarly,
  \begin{align*}
    (\nabla^{\prime*}\alpha,\nabla^{\prime*}\alpha)&=(\nabla'\partial^*\alpha,
    \alpha) \\
    &=-\sqrt{-1}(\nabla'(\bar{\partial}\Lambda-\Lambda\bar{\partial})\alpha,
    \alpha) \\
    &=-\sqrt{-1}(\nabla'\bar{\partial}\Lambda \alpha,\alpha) \\
    &=-\sqrt{-1}((\nabla'\bar{\partial}+\bar{\partial}\nabla')\Lambda\alpha,
    \alpha) \\
    &=-\sqrt{-1}(e(\psi)\Lambda\alpha,\alpha)
  .\end{align*}
\end{proof}
\begin{proof}[Proof of \cref{thm:nakano-vanishing}]
  We use \(-\sqrt{-1}\psi_{i\bar{j}}\dd{z^i}\wedge \dd{\bar{z}^j}\) as a Kähler
  form, then \[
    L=-\sqrt{-1}e(\psi)\quad \leftarrow\text{this is the operator, not the line
      bundle}
  .\] Then the right hand side in lemma is \[
    -((\Lambda L-L\Lambda)\alpha,\alpha)
  .\] The by lemma, \[
    ((\Lambda L-L \Lambda)\alpha,\alpha)\le 0
  .\] But \[
    \Lambda L-L\Lambda=n-p-q\quad\text{(Exercise)}
  .\]  Thus if \(n-p-q>0\) then \(\alpha=0\).
\end{proof}

\end{document}
