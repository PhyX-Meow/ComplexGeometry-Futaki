% !TeX program = xelatex
\documentclass[12pt]{article}
\usepackage{standalone}

\usepackage[dvipsnames,svgnames,x11names]{xcolor}
\usepackage[a4paper,margin=1in]{geometry}
\usepackage{microtype}
\usepackage{amsmath}
\usepackage{amsthm}
\usepackage{mathtools}
\usepackage{mathrsfs}
\usepackage{stmaryrd}
\usepackage{extarrows}
\usepackage{enumerate}
\usepackage{tensor}
\usepackage{physics2}
  \usephysicsmodule{ab,xmat}
\usepackage{fixdif}
  \newcommand{\dd}{\d}
\usepackage{derivative}
  \newcommand{\dv}{\odv}
  \newcommand{\pd}[1]{\pdv{}{#1}}
  \newcommand{\eval}[1]{#1\big|}
\usepackage{graphicx}
\usepackage{subcaption}
\usepackage{tikz}
\usepackage{tikz-3dplot}
% \usepackage{tikz-cd}
% \usepackage{quiver}
% ========== Block quiver.sty ========== %
\usepackage{tikz-cd}
% \usepackage{amssymb}
\usetikzlibrary{calc}
\usetikzlibrary{decorations.pathmorphing}
\tikzset{curve/.style={settings={#1},to path={(\tikztostart)
    .. controls ($(\tikztostart)!\pv{pos}!(\tikztotarget)!\pv{height}!270:(\tikztotarget)$)
    and ($(\tikztostart)!1-\pv{pos}!(\tikztotarget)!\pv{height}!270:(\tikztotarget)$)
    .. (\tikztotarget)\tikztonodes}},
    settings/.code={\tikzset{quiver/.cd,#1}
        \def\pv##1{\pgfkeysvalueof{/tikz/quiver/##1}}},
    quiver/.cd,pos/.initial=0.35,height/.initial=0}
\tikzset{tail reversed/.code={\pgfsetarrowsstart{tikzcd to}}}
\tikzset{2tail/.code={\pgfsetarrowsstart{Implies[reversed]}}}
\tikzset{2tail reversed/.code={\pgfsetarrowsstart{Implies}}}
\tikzset{no body/.style={/tikz/dash pattern=on 0 off 1mm}}
% =========== End block ========== %
  \tikzset{every picture/.style={line width=0.75pt}}
\usepackage{pgfplots}
  \pgfplotsset{compat=newest}
\usepackage{tcolorbox}
  \tcbuselibrary{most}
\usepackage[colorlinks=true,linkcolor=blue]{hyperref}
\usepackage{cleveref}
% \usepackage[hyperref=true,backend=biber,style=alphabetic,backref=true,url=false]{biblatex}
\usepackage[warnings-off={mathtools-colon,mathtools-overbracket}]{unicode-math}
\usepackage[default,amsbb]{fontsetup}
  \setmathfont[StylisticSet=1,range=\mathscr]{NewCMMath-Book.otf}
\usepackage{fancyhdr}
\usepackage{import}

\newcommand{\Id}{\mathbb{1}}
\newcommand{\lap}{\increment}

\DeclareMathOperator{\sign}{sign}
\DeclareMathOperator{\dom}{dom}
\DeclareMathOperator{\ran}{ran}
\DeclareMathOperator{\ord}{ord}
\DeclareMathOperator{\Span}{span}
\DeclareMathOperator{\img}{Im}
\DeclareMathOperator{\Ric}{Ric}
\newcommand{\card}{\texttt{\#}}
\newcommand{\ie}{\emph{i.e.}}
\newcommand{\st}{\emph{s.t.}}
\newcommand{\eps}{\varepsilon}
\newcommand{\vphi}{\varphi}
\newcommand{\vthe}{\vartheta}
\newcommand{\II}{I\!I}
\renewcommand{\emptyset}{⌀}
\newcommand{\acts}{\curvearrowright}
\newcommand{\xrr}{\xlongrightarrow}
\newcommand{\lrr}{\longrightarrow}
\newcommand{\lmt}{\longmapsto}
\newcommand{\into}{\hookrightarrow}
\newcommand{\op}{\operatorname}

\let\originalleft\left
\let\originalright\right
\renewcommand{\left}{\mathopen{}\mathclose\bgroup\originalleft}
\renewcommand{\right}{\aftergroup\egroup\originalright}

\theoremstyle{plain}\newtheorem{theorem}{Theorem}
\theoremstyle{definition}\newtheorem{definition}[theorem]{Definition}
\theoremstyle{definition}\newtheorem{example}[theorem]{Example}
\theoremstyle{definition}\newtheorem{problem}[theorem]{Problem}
\theoremstyle{plain}\newtheorem{axiom}[theorem]{Axiom}
\theoremstyle{plain}\newtheorem{corollary}[theorem]{Corollary}
\theoremstyle{plain}\newtheorem{lemma}[theorem]{Lemma}
\theoremstyle{plain}\newtheorem{proposition}[theorem]{Proposition}
\theoremstyle{plain}\newtheorem{prop}[theorem]{Proposition}
\theoremstyle{plain}\newtheorem{conjecture}[theorem]{Conjecture}
\theoremstyle{plain}\newtheorem{conj}[theorem]{Conjecture}
\theoremstyle{remark}\newtheorem{notation}[theorem]{Notation}
\theoremstyle{definition}\newtheorem*{question}{Question}
\theoremstyle{definition}\newtheorem*{answer}{Answer}
\theoremstyle{definition}\newtheorem*{goal}{Goal}
\theoremstyle{definition}\newtheorem*{application}{Application}
\theoremstyle{plain}\newtheorem*{exercise}{Exercise}
\theoremstyle{remark}\newtheorem*{remark}{Remark}
\theoremstyle{remark}\newtheorem*{note}{\small{Note}}
\numberwithin{equation}{section}
\numberwithin{theorem}{section}
\numberwithin{figure}{section}

\usepackage{xeCJK}
\setCJKmainfont{FZShuSong-Z01}[BoldFont=FZXiaoBiaoSong-B05,ItalicFont=FZKai-Z03]
\setCJKsansfont{FZXiHeiI-Z08}[BoldFont=FZHei-B01]
\setCJKmonofont{FZFangSong-Z02}
\setCJKfamilyfont{zhsong}{FZShuSong-Z01}[BoldFont=FZXiaoBiaoSong-B05]
\setCJKfamilyfont{zhhei}{FZHei-B01}
\setCJKfamilyfont{zhkai}{FZKai-Z03}
\setCJKfamilyfont{zhfs}{FZFangSong-Z02}
\setCJKfamilyfont{zhli}{FZLiShu-S01}
\setCJKfamilyfont{zhyou}{FZXiYuan-M01}[BoldFont=FZZhunYuan-M02]

\allowdisplaybreaks{}

\newcommand{\isFullBook}[2]{
  \ifnum\pdfstrcmp{\FullBook}{True}=0
    \ifnum\pdfstrcmp{}{#1}=0\unskip\else#1\fi
  \else
    \ifnum\pdfstrcmp{}{#2}=0\unskip\else#2\fi
  \fi\ignorespaces{}
}

\counterwithout{theorem}{section}
\counterwithout{equation}{section}

\begin{document}
\begin{theorem}
  If there exists an ample line bundle \(L\) on a compact complex manifold \(M\) then
  \[
    \op{Div}(M)/\op{Div}_\ell(M)\lrr \op{Pic}(M)
  \] is an isomorphism.
\end{theorem}
\begin{proof}
  Following the previous theorem, we only need to show the map is surjective.
  Let \(G\) be any line bundle on \(M\). By 
  \ifdefined\FullBook \cref{lem:21-immersion},
  \else lemma 1 in lecture 21,
  \fi there exist non-zero sections both on \(L^\mu\) and \(L^\mu\otimes G\) for
  sufficiently large \(\mu\). Let \(s\in \Gamma(\mathcal{O}(L^m))\) and
  \(t\in \Gamma(\mathcal{O}(L^m\otimes G))\) be non-zero sections.

  Then \(\frac{t}{s}\) is a meromorphic section of \(G\). So \(\frac{t}{s}\) defines
  meromorphic functions \(u_\alpha\) on \(U_\alpha\) such that \[
    u_\alpha=f_{\alpha\beta}u_\beta
  \] where \(f_{\alpha\beta}\) are the transition functions of \(G\). Thus \[
    f_{\alpha\beta}=\frac{u_\alpha}{u_\beta}
  .\] Let \(D=(\frac{t}{s})\) denote the divisor defined by the zeros of \(t\) and
  the poles of \(s\). This divisor is the same as the divisor defined by
  \(\{u_\alpha\}\). Thus \([D]=G\).
\end{proof}

It follows from the above theorem that if \(M\) is an algebraic manifold then \[
  \op{Div}(M)/\op{Div}_{\ell}(M)\cong \op{Pic}(M)
.\] Next we define \[
  \op{Div}_h(M)=\{D\in \op{Div}_M:D\text{ is homologous to }D\}
.\]
\begin{prop}
  \(\op{Div}_\ell(M)\subset \op{Div}_h(M)\).
\end{prop}
To prove this we recall the relation of \(c_1([D])\) and the Poincaré dual of the
homology class of \(D\). We already saw it for divisors in Riemann surface in the
lecture and smooth divisors in higher dimensional complex manifolds in the homework.

For general possibly singular divisors one can use the following 
\textbf{Poincaré-Lelong equation} (homework):

Let \(f\) be a holomorphic function on domain \(U\subset \mathbb{C}^m\), and
\((f)\) be the divisor defined by the zero locus of \(f\). Then as a current, \[
  (f)=\frac{1}{2\pi}\dd\dd^c \log|f|=\frac{\sqrt{-1}}{\pi}\partial\bar{\partial}
  \log|f|,\quad \dd^c=\sqrt{-1}(\bar{\partial}-\partial)
.\] Here we say a current in the following sense: For any smooth \((m-1,m-1)\)-form
\(\vphi\) with compact support in \(U\), we have \[
  \int_{(f)}\vphi=\frac{1}{2\pi}\int_{U}\dd\dd^c \log|f|\wedge\vphi
.\] We can use this to show the following (homework):

Let \(D\) be a divisor on a compact complex manifold \(M\), and \(s\) be a meromorphic
section of \([D]\). Then for a Hermitian metric \(h\) of \([D]\) we have \[
  \frac{1}{2\pi}\dd \dd^c \log|s|_h=-c_h+(s)
.\] Where
\begin{align*}
  |s|^2_h&=h(\bar{s},s) \\
  c_h&=\text{the first Chern form w.r.t. }h \\
  &=-\frac{\sqrt{-1}}{2\pi}\partial\bar{\partial}\log h \\
  (s)&=\text{the divisor defined by zeros and poles of }s\text{ with their
  multiplicity}
.\end{align*}
From this for any closed \((2m-2)\)-form \(\alpha\), we have \[
  \int_{(s)}\alpha=\int \alpha\wedge c_h
.\] This implies \(c_1([D])\) is the Poincaré dual of the homology class of \(D\).

\begin{proof}[Proof of the Proposition]
  Let \(D=(f)\) be a principal divisor. As we saw in the last lecture, \([D]\) is
  isomorphic to the trivial bundle, and \(c_1([D])=0\). Thus the homology class
  of \(D\) is 0.
\end{proof}

Note also the above proof implies \[
  \op{Div}_h(M)=\{D\in \op{Div}(M):c_1([D])=0\}
.\] Recall the Picard variety \(\op{Pic}^0(M)\) is defined by \[
  \op{Pic}^0(M)=\{L\in \op{Pic}(M):c_1(L)=0\}
.\] Thus the injective map \[
  \op{Div}(M)/\op{Div}_\ell(M)\lrr \op{Pic}(M)
\] induces injective map \[
  \op{Div}_h(M)/\op{Div}_\ell(M)\lrr \op{Pic}^0(M)
.\] And by the first theorem of this lecture we have the following:
\begin{theorem}
  If \(M\) is an algebraic manifold then \[
    \op{Div}_h(M)/\op{Div}_\ell(M)\lrr \op{Pic}^0(M)
  \] is an isomorphism.
\end{theorem}
We further define the \textbf{Néron-Severi group} \[
  \op{NS}(M):=\op{Div}(M)/\op{Div}_h(M)
.\] Then the natural map \[
  \op{NS}(M)\lrr \op{Pic}(M)/\op{Pic}^0(M)
\] is injective. But we also as above:
\begin{theorem}
  If \(M\) is an algebraic manifold then \[
    \op{NS}(M)\cong\op{Pic}(M)/\op{Pic}^0(M)
  .\]
\end{theorem}

\begin{remark}
  In the literature there are other different definitions of Néron-Severi group.
  But basically they are same as ours at least when \(M\) is algebraic.
\end{remark}

In summary,
\begin{align*}
  \op{Pic}(M)&=H^1(M,\mathcal{O}^*)\text{ isomorphism classes of line bundles}, \\
  \op{Pic}^0(M)&=\{[L]\in H^1(M,\mathcal{O}^*):c_1(L)=0\}, \\
  \op{cl}(M)&=\op{div}(M)/\op{Div}_\ell(M)\text{ divisor class group}, \\
  \op{cl}(M)&\lrr \op{Pic}(M)\text{ injective}, \\
  \op{Div}_h(M)&=\{D\in \op{Div}(M):c_1([D])=0\}, \\
  \op{Div}_h(M)&/\op{Div}_\ell(M)\lrr \op{Pic}^0(M)\text{ injective}, \\
  \op{NS}(M)&=\op{Div}(M)/\op{Div}_h(M)\xrr{\text{inj.}}
  \op{Pic}(M)/\op{Pic}^0(M)
.\end{align*}
\(M\) algebraic \(\implies\op{NS}(M)\cong\op{Pic}(M)/\op{Pic}^0(M)\cong
H^{1,1}(M,\mathbb{Z})\).

\end{document}
