% !TeX program = xelatex
\documentclass[12pt]{article}
\usepackage{standalone}

\usepackage[dvipsnames,svgnames,x11names]{xcolor}
\usepackage[a4paper,margin=1in]{geometry}
\usepackage{microtype}
\usepackage{amsmath}
\usepackage{amsthm}
\usepackage{mathtools}
\usepackage{mathrsfs}
\usepackage{stmaryrd}
\usepackage{extarrows}
\usepackage{enumerate}
\usepackage{tensor}
\usepackage{physics2}
  \usephysicsmodule{ab,xmat}
\usepackage{fixdif}
  \newcommand{\dd}{\d}
\usepackage{derivative}
  \newcommand{\dv}{\odv}
  \newcommand{\pd}[1]{\pdv{}{#1}}
  \newcommand{\eval}[1]{#1\big|}
\usepackage{graphicx}
\usepackage{subcaption}
\usepackage{tikz}
\usepackage{tikz-3dplot}
% \usepackage{tikz-cd}
% \usepackage{quiver}
% ========== Block quiver.sty ========== %
\usepackage{tikz-cd}
% \usepackage{amssymb}
\usetikzlibrary{calc}
\usetikzlibrary{decorations.pathmorphing}
\tikzset{curve/.style={settings={#1},to path={(\tikztostart)
    .. controls ($(\tikztostart)!\pv{pos}!(\tikztotarget)!\pv{height}!270:(\tikztotarget)$)
    and ($(\tikztostart)!1-\pv{pos}!(\tikztotarget)!\pv{height}!270:(\tikztotarget)$)
    .. (\tikztotarget)\tikztonodes}},
    settings/.code={\tikzset{quiver/.cd,#1}
        \def\pv##1{\pgfkeysvalueof{/tikz/quiver/##1}}},
    quiver/.cd,pos/.initial=0.35,height/.initial=0}
\tikzset{tail reversed/.code={\pgfsetarrowsstart{tikzcd to}}}
\tikzset{2tail/.code={\pgfsetarrowsstart{Implies[reversed]}}}
\tikzset{2tail reversed/.code={\pgfsetarrowsstart{Implies}}}
\tikzset{no body/.style={/tikz/dash pattern=on 0 off 1mm}}
% =========== End block ========== %
  \tikzset{every picture/.style={line width=0.75pt}}
\usepackage{pgfplots}
  \pgfplotsset{compat=newest}
\usepackage{tcolorbox}
  \tcbuselibrary{most}
\usepackage[colorlinks=true,linkcolor=blue]{hyperref}
\usepackage{cleveref}
% \usepackage[hyperref=true,backend=biber,style=alphabetic,backref=true,url=false]{biblatex}
\usepackage[warnings-off={mathtools-colon,mathtools-overbracket}]{unicode-math}
\usepackage[default,amsbb]{fontsetup}
  \setmathfont[StylisticSet=1,range=\mathscr]{NewCMMath-Book.otf}
\usepackage{fancyhdr}
\usepackage{import}

\newcommand{\Id}{\mathbb{1}}
\newcommand{\lap}{\increment}

\DeclareMathOperator{\sign}{sign}
\DeclareMathOperator{\dom}{dom}
\DeclareMathOperator{\ran}{ran}
\DeclareMathOperator{\ord}{ord}
\DeclareMathOperator{\Span}{span}
\DeclareMathOperator{\img}{Im}
\DeclareMathOperator{\Ric}{Ric}
\newcommand{\card}{\texttt{\#}}
\newcommand{\ie}{\emph{i.e.}}
\newcommand{\st}{\emph{s.t.}}
\newcommand{\eps}{\varepsilon}
\newcommand{\vphi}{\varphi}
\newcommand{\vthe}{\vartheta}
\newcommand{\II}{I\!I}
\renewcommand{\emptyset}{⌀}
\newcommand{\acts}{\curvearrowright}
\newcommand{\xrr}{\xlongrightarrow}
\newcommand{\lrr}{\longrightarrow}
\newcommand{\lmt}{\longmapsto}
\newcommand{\into}{\hookrightarrow}
\newcommand{\op}{\operatorname}

\let\originalleft\left
\let\originalright\right
\renewcommand{\left}{\mathopen{}\mathclose\bgroup\originalleft}
\renewcommand{\right}{\aftergroup\egroup\originalright}

\theoremstyle{plain}\newtheorem{theorem}{Theorem}
\theoremstyle{definition}\newtheorem{definition}[theorem]{Definition}
\theoremstyle{definition}\newtheorem{example}[theorem]{Example}
\theoremstyle{definition}\newtheorem{problem}[theorem]{Problem}
\theoremstyle{plain}\newtheorem{axiom}[theorem]{Axiom}
\theoremstyle{plain}\newtheorem{corollary}[theorem]{Corollary}
\theoremstyle{plain}\newtheorem{lemma}[theorem]{Lemma}
\theoremstyle{plain}\newtheorem{proposition}[theorem]{Proposition}
\theoremstyle{plain}\newtheorem{prop}[theorem]{Proposition}
\theoremstyle{plain}\newtheorem{conjecture}[theorem]{Conjecture}
\theoremstyle{plain}\newtheorem{conj}[theorem]{Conjecture}
\theoremstyle{remark}\newtheorem{notation}[theorem]{Notation}
\theoremstyle{definition}\newtheorem*{question}{Question}
\theoremstyle{definition}\newtheorem*{answer}{Answer}
\theoremstyle{definition}\newtheorem*{goal}{Goal}
\theoremstyle{definition}\newtheorem*{application}{Application}
\theoremstyle{plain}\newtheorem*{exercise}{Exercise}
\theoremstyle{remark}\newtheorem*{remark}{Remark}
\theoremstyle{remark}\newtheorem*{note}{\small{Note}}
\numberwithin{equation}{section}
\numberwithin{theorem}{section}
\numberwithin{figure}{section}

\usepackage{xeCJK}
\setCJKmainfont{FZShuSong-Z01}[BoldFont=FZXiaoBiaoSong-B05,ItalicFont=FZKai-Z03]
\setCJKsansfont{FZXiHeiI-Z08}[BoldFont=FZHei-B01]
\setCJKmonofont{FZFangSong-Z02}
\setCJKfamilyfont{zhsong}{FZShuSong-Z01}[BoldFont=FZXiaoBiaoSong-B05]
\setCJKfamilyfont{zhhei}{FZHei-B01}
\setCJKfamilyfont{zhkai}{FZKai-Z03}
\setCJKfamilyfont{zhfs}{FZFangSong-Z02}
\setCJKfamilyfont{zhli}{FZLiShu-S01}
\setCJKfamilyfont{zhyou}{FZXiYuan-M01}[BoldFont=FZZhunYuan-M02]

\allowdisplaybreaks{}

\newcommand{\isFullBook}[2]{
  \ifnum\pdfstrcmp{\FullBook}{True}=0
    \ifnum\pdfstrcmp{}{#1}=0\unskip\else#1\fi
  \else
    \ifnum\pdfstrcmp{}{#2}=0\unskip\else#2\fi
  \fi\ignorespaces{}
}

\counterwithout{theorem}{section}
\counterwithout{equation}{section}

\begin{document}
Let \(E\to M\) be a \(C^\infty\) complex vector bundle over a compact manifold \(M\),
and \(\nabla\) be a connection and \(R\) its curvature.

\begin{definition}
  The cohomology class \[
    \op{ch}(E)=\left[\op{tr}\exp\biggl(\frac{\sqrt{-1}}{2\pi}R\biggr)\right]
    =\left[\op{tr}\biggl(I+\frac{\sqrt{-1}}{2\pi}R+\frac{1}{2!}\biggl(
    \frac{\sqrt{-1}}{2\pi}R\biggr)^2+\cdots\biggr)\right]
  \] is called the \textbf{Chern character}.
  According to the degree, we write \[
    \op{ch}_k(E)=\frac{1}{k!}\left[\op{tr}\biggl(\frac{\sqrt{-1}}{2\pi}R\biggr)^k
    \right]
  .\] Because of \(\frac{1}{k!}\), this class belongs to \(\mathbb{Q}\)-coefficient
  cohomology \(H^{2k}(M,\mathbb{Q})\).
\end{definition}

Chern character and the total Chern class is related by ``Newton formula'' and one 
is recovered from the other. We will not go into the detail here, but just refer
to \emph{Milnor-Stasheff}.

Chern character has the following two nice properties: 
\begin{lemma}\label{lem:25-chern-character}
For \(E\to M\) and \(F\to M\),
\begin{enumerate}[(1)]
  \item \(\op{ch}(E\oplus F)=\op{ch}(E)+\op{ch}(F)\).
  \item \(\op{ch}(E\otimes F)=\op{ch}(E)\cup\op{ch}(F)\).
\end{enumerate}
\end{lemma}
\begin{proof}
To prove (1) and (2), let \(\nabla_E,\nabla_F\) be connections of \(E\) and \(F\),
and \(R_E,R_F\) their curvatures. Then \[
  \sum_{k\ge 0}\frac{1}{k!}\op{tr}\begin{pmatrix}
    \frac{\sqrt{-1}}{2\pi}R_E & 0 \\
    0 & \frac{\sqrt{-1}}{2\pi}R_F
  \end{pmatrix}^k
  =\sum_{k\ge 0}\frac{1}{k!}\op{tr}\biggl(\frac{\sqrt{-1}}{2\pi}R_E\biggr)^k
  +\sum_{k\ge 0}\frac{1}{k!}\op{tr}\biggl(\frac{\sqrt{-1}}{2\pi}R_F\biggr)^k
.\] This proves (1). The connection on \(E\otimes F\) is given by \[
  \nabla_E\otimes\Id_F+\Id_E\otimes\nabla_F
.\] Its curvature is then \[
  R_E\otimes I_F+I_E\otimes R_F
.\] Then
\begin{align*}
  &\exp\biggl(\frac{\sqrt{-1}}{2\pi}(R_E\otimes I_F+I_E\otimes R_F)\biggr) \\
  &=\exp\biggl(\frac{\sqrt{-1}}{2\pi}(R_E\otimes I_F)\biggr)\wedge
  \exp\biggl(\frac{\sqrt{-1}}{2\pi}(I_E\otimes R_F)\biggr) \\
  &=\left(\sum_{k\ge 0}\frac{1}{k!}\biggl(\frac{\sqrt{-1}}{2\pi}R_E\otimes I_F\biggr)^k
  \right)\wedge\sum_{l\ge 0}\frac{1}{l!}\biggl(\frac{\sqrt{-1}}{2\pi}I_E\otimes R_F
  \biggr)^l \\
  &=\sum_{k,l\ge 0}\frac{1}{k!l!}\biggl(\frac{\sqrt{-1}}{2\pi}R_E\biggr)^k\otimes 
  \biggl(\frac{\sqrt{-1}}{2\pi}R_F\biggr)^l
.\end{align*} 
Since \(\op{tr}(A\otimes B)=\op{tr}A\cdot \op{tr}B\), we obtain the proof of (2).
\end{proof}

The ring \(K(M)\) consists of stable equivalence classes of vector bundles with
addition defined by Whitney sum and multiplication by tensor product. See for example
\emph{Atiyah, K-theory}. \cref{lem:25-chern-character} show that \[
  \op{ch}\colon K(M)\lrr H^*(M,\mathbb{Q})
\] is a ring homomorphism.

Next we wish to define \textbf{Todd class} of \(M\). Consider the formal
decomposition \[
  \sum_{k}c_k(M)t^k=\prod_{i=1}^m(1+\xi_i t)
.\] This is based on the splitting principle. (See e.g. \emph{Bott-Tu, Differential
forms in algebraic topology} to justify it). We formally consider \(T'M\) splits
into line bundles \[
  T'M=L_1\oplus \cdots\oplus L_m
.\] Then putting \(\xi_i=c_1(L_i)\) we obtain from the Axiom of Whitney sum formula \[
  c(T'M)=c(L_1)\cup \cdots\cup c(L_m)
.\] Then the decomposition follows.

With this preparation we define \[
  \op{Todd}(M)=\prod_{i=1}^{m}\frac{\xi_i}{1-e^{-\xi_i}}
.\] Since \(\op{Todd}(M)\) is a symmetric function of \(\xi_1,\ldots,\xi_m\), it is
expressed in terms of the elementary symmetric functions of \(\xi_i,\ldots,\xi_m\)
which are \(c_1(M),\ldots,c_m(M)\). One can check \[
  \op{Todd}(M)=1+\frac{1}{2}c_1(M)+\frac{1}{12}(c_1(M)^2+c_2(M))+\cdots
.\] Let \[
  \chi(M,E)=\sum_{k=0}^{m}(-1)^k \dim H^k(M,\mathcal{O}(E))
.\] 

\begin{theorem}[Riemann-Roch-Hirzebruch formula]
  \[
    \chi(M,E)=\int_{M}\op{Todd}(M)\op{ch}(E)
  .\] Here on the right hand side we take the degree \(2m\) term and integrate it
  over \(M\).
\end{theorem}

This formula has a long history, but Hirzebruch proved it on compact complex manifolds.
Later, Atiyah-Singer extended it to more general elliptic complexes, e.g. Dirac
operators. The book by Lawson-Michelsohn: ``Spin geometry'', may be good to read.
But there are many other books about the Atiyah-Singer theorem.

Let \(L\) be an ample line bundle. Then by Kodaira vanishing, for large \(n\),
\(L^n\otimes K_M^{-1}\) is positive and \[
  H^q(M,\mathcal{O}(L^n))=0\quad\text{ for }q>0
.\] Then \[
  \chi(M,L^n)=\dim H^0(M,L^n)
.\] But by Riemannian-Roch, 
\begin{align*}
  \chi(M,L^n)&=\int_{M}\op{Todd}(M)e^{c_1(L^n)}=\int_{M}\op{Todd}(M)e^{nc_1(L)} \\
  &=\int_{M}(1+\tau_1(M)+\cdots +\tau_m(M))(1+nc_1(L)+\cdots +\frac{n^m}{m!}c_1(L)^m)\\
  &=\int_{M}\frac{c_1(L)^m}{m!}n^m+\frac{\tau_1(M)\cdot c_1(L)^{m-1}}{(m-1)!}+\cdots
.\end{align*}
This is a polynomial in \(n\) of degree \(m\) with positive top term
\(\frac{n^m}{m!}\int_{M}c_1(L)^m\). Thus \(\dim H^0(M,L^n)\) is positive for large
\(m\). We may put \[
  N_n+1=\dim H^0(M,L^n),\quad \Phi_{L^n}\colon M\lrr \mathbb{P}^{N_n}(\mathbb{C})
.\] This has been implicitly used in Kodaira embedding theorem.

Next we will discuss the relation of complex line bundle, its Chern class and
divisors in more detail.

Consider smooth category first. Let \(L\to M\) and \(L'\to M\) be complex line
bundles over a real smooth manifold \(M\). Then \(L\otimes L'\to M\) is also a line
bundle. Obviously the set of all complex line bundles form a group by tensor product.
The trivial line bundle is the identity. The dual line bundle is the inverse.

If the transition functions of \(L\) and \(L'\) are given by \(\{a_{\lambda\mu}\}\) and
\(\{a'_{\lambda\mu}\}\) in terms of an open covering \(\{U_{\lambda}\}_{\lambda\in
\Lambda}\). Then the transition functions of \(L\otimes L'\) are given by 
\(\{a_{\lambda\mu}a'_{\lambda\mu}\}\). On the other hand, \(\{a_{\lambda\mu}\}\)
satisfies the cocycle condition \[
  a_{\lambda\mu}a_{\mu\nu}a_{\nu\lambda}=1,\quad
  a_{\mu\lambda}=a_{\lambda\mu}^{-1}
.\] Thus it defines an element of \(H^1(M,\mathscr{A}^*)\), where \(\mathscr{A}^*\)
is the sheaf of germs of non-vanishing complex valued functions.

\begin{prop}
  \(H^1(M,\mathscr{A}^*)\) is isomorphic to the group of all isomorphism classes 
  of complex line bundle over \(M\).
\end{prop}

\end{document}
