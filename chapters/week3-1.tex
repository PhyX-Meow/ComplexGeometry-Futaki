% !TeX program = xelatex
\documentclass[12pt]{article}
\usepackage{standalone}

\usepackage[dvipsnames,svgnames,x11names]{xcolor}
\usepackage[a4paper,margin=1in]{geometry}
\usepackage{microtype}
\usepackage{amsmath}
\usepackage{amsthm}
\usepackage{mathtools}
\usepackage{mathrsfs}
\usepackage{stmaryrd}
\usepackage{extarrows}
\usepackage{enumerate}
\usepackage{tensor}
\usepackage{physics2}
  \usephysicsmodule{ab,xmat}
\usepackage{fixdif}
  \newcommand{\dd}{\d}
\usepackage{derivative}
  \newcommand{\dv}{\odv}
  \newcommand{\pd}[1]{\pdv{}{#1}}
  \newcommand{\eval}[1]{#1\big|}
\usepackage{graphicx}
\usepackage{subcaption}
\usepackage{tikz}
\usepackage{tikz-3dplot}
% \usepackage{tikz-cd}
% \usepackage{quiver}
% ========== Block quiver.sty ========== %
\usepackage{tikz-cd}
% \usepackage{amssymb}
\usetikzlibrary{calc}
\usetikzlibrary{decorations.pathmorphing}
\tikzset{curve/.style={settings={#1},to path={(\tikztostart)
    .. controls ($(\tikztostart)!\pv{pos}!(\tikztotarget)!\pv{height}!270:(\tikztotarget)$)
    and ($(\tikztostart)!1-\pv{pos}!(\tikztotarget)!\pv{height}!270:(\tikztotarget)$)
    .. (\tikztotarget)\tikztonodes}},
    settings/.code={\tikzset{quiver/.cd,#1}
        \def\pv##1{\pgfkeysvalueof{/tikz/quiver/##1}}},
    quiver/.cd,pos/.initial=0.35,height/.initial=0}
\tikzset{tail reversed/.code={\pgfsetarrowsstart{tikzcd to}}}
\tikzset{2tail/.code={\pgfsetarrowsstart{Implies[reversed]}}}
\tikzset{2tail reversed/.code={\pgfsetarrowsstart{Implies}}}
\tikzset{no body/.style={/tikz/dash pattern=on 0 off 1mm}}
% =========== End block ========== %
  \tikzset{every picture/.style={line width=0.75pt}}
\usepackage{pgfplots}
  \pgfplotsset{compat=newest}
\usepackage{tcolorbox}
  \tcbuselibrary{most}
\usepackage[colorlinks=true,linkcolor=blue]{hyperref}
\usepackage{cleveref}
% \usepackage[hyperref=true,backend=biber,style=alphabetic,backref=true,url=false]{biblatex}
\usepackage[warnings-off={mathtools-colon,mathtools-overbracket}]{unicode-math}
\usepackage[default,amsbb]{fontsetup}
  \setmathfont[StylisticSet=1,range=\mathscr]{NewCMMath-Book.otf}
\usepackage{fancyhdr}
\usepackage{import}

\newcommand{\Id}{\mathbb{1}}
\newcommand{\lap}{\increment}

\DeclareMathOperator{\sign}{sign}
\DeclareMathOperator{\dom}{dom}
\DeclareMathOperator{\ran}{ran}
\DeclareMathOperator{\ord}{ord}
\DeclareMathOperator{\Span}{span}
\DeclareMathOperator{\img}{Im}
\DeclareMathOperator{\Ric}{Ric}
\newcommand{\card}{\texttt{\#}}
\newcommand{\ie}{\emph{i.e.}}
\newcommand{\st}{\emph{s.t.}}
\newcommand{\eps}{\varepsilon}
\newcommand{\vphi}{\varphi}
\newcommand{\vthe}{\vartheta}
\newcommand{\II}{I\!I}
\renewcommand{\emptyset}{⌀}
\newcommand{\acts}{\curvearrowright}
\newcommand{\xrr}{\xlongrightarrow}
\newcommand{\lrr}{\longrightarrow}
\newcommand{\lmt}{\longmapsto}
\newcommand{\into}{\hookrightarrow}
\newcommand{\op}{\operatorname}

\let\originalleft\left
\let\originalright\right
\renewcommand{\left}{\mathopen{}\mathclose\bgroup\originalleft}
\renewcommand{\right}{\aftergroup\egroup\originalright}

\theoremstyle{plain}\newtheorem{theorem}{Theorem}
\theoremstyle{definition}\newtheorem{definition}[theorem]{Definition}
\theoremstyle{definition}\newtheorem{example}[theorem]{Example}
\theoremstyle{definition}\newtheorem{problem}[theorem]{Problem}
\theoremstyle{plain}\newtheorem{axiom}[theorem]{Axiom}
\theoremstyle{plain}\newtheorem{corollary}[theorem]{Corollary}
\theoremstyle{plain}\newtheorem{lemma}[theorem]{Lemma}
\theoremstyle{plain}\newtheorem{proposition}[theorem]{Proposition}
\theoremstyle{plain}\newtheorem{prop}[theorem]{Proposition}
\theoremstyle{plain}\newtheorem{conjecture}[theorem]{Conjecture}
\theoremstyle{plain}\newtheorem{conj}[theorem]{Conjecture}
\theoremstyle{remark}\newtheorem{notation}[theorem]{Notation}
\theoremstyle{definition}\newtheorem*{question}{Question}
\theoremstyle{definition}\newtheorem*{answer}{Answer}
\theoremstyle{definition}\newtheorem*{goal}{Goal}
\theoremstyle{definition}\newtheorem*{application}{Application}
\theoremstyle{plain}\newtheorem*{exercise}{Exercise}
\theoremstyle{remark}\newtheorem*{remark}{Remark}
\theoremstyle{remark}\newtheorem*{note}{\small{Note}}
\numberwithin{equation}{section}
\numberwithin{theorem}{section}
\numberwithin{figure}{section}

\usepackage{xeCJK}
\setCJKmainfont{FZShuSong-Z01}[BoldFont=FZXiaoBiaoSong-B05,ItalicFont=FZKai-Z03]
\setCJKsansfont{FZXiHeiI-Z08}[BoldFont=FZHei-B01]
\setCJKmonofont{FZFangSong-Z02}
\setCJKfamilyfont{zhsong}{FZShuSong-Z01}[BoldFont=FZXiaoBiaoSong-B05]
\setCJKfamilyfont{zhhei}{FZHei-B01}
\setCJKfamilyfont{zhkai}{FZKai-Z03}
\setCJKfamilyfont{zhfs}{FZFangSong-Z02}
\setCJKfamilyfont{zhli}{FZLiShu-S01}
\setCJKfamilyfont{zhyou}{FZXiYuan-M01}[BoldFont=FZZhunYuan-M02]

\allowdisplaybreaks{}

\newcommand{\isFullBook}[2]{
  \ifnum\pdfstrcmp{\FullBook}{True}=0
    \ifnum\pdfstrcmp{}{#1}=0\unskip\else#1\fi
  \else
    \ifnum\pdfstrcmp{}{#2}=0\unskip\else#2\fi
  \fi\ignorespaces{}
}

\counterwithout{theorem}{section}
\counterwithout{equation}{section}

\begin{document}

Further remarks on Riemannian curvature:
\begin{enumerate}[1.]
\item If there is a constant  \(\lambda\in \mathbb{R}\) such that
  \begin{equation*}
    R_{ij}=\lambda g_{ij} \tag{\ast}
  .\end{equation*}
  Then the Riemannian metric \(g\) is called an Einstein metric.

  Notice that the curvature tensor \(R\) and Ricci tensor \(\Ric\) are expressed by
  second derivative of the metric \(g\).\ (\ast) is a differential equation of \(g\),
  called the Einstein (field) equation. This is an equation satisfied by the
    gravity (with Lorentzian metric).
\item Ricci curvature \(R_{ij}\) and the Riemannian metric are both symmetric tensors.
  Let \(g(t)\) be a smooth family of Riemannian metric with parameter \(t\in
  \mathbb{R}\). \(\{g(t)\}_{t\in \mathbb{R}}\) is called the Ricci flow if it
  satisfies \[
    \pdv{g}{t}=-2\Ric(g(t))
  ,\] where \(\Ric(g(t))\) is the Ricci curvature of the metric \(g(t)\).

  Ricci flow was used to prove Poincaré conjecture.
\item For compact surface \(M\subset \mathbb{R}^3\), the Gaussian curvature coincide
  with \(R(e_1,e_2,e_1,e_2)\). I leave this fundamental fact to other lectures, or
  your own study.

  A very important theorem on the Gaussian curvature is that its integral over \(M\)
  is equal to \(2\pi\) times the Euler characteristic \(\chi(M)\). \[
    \frac{1}{2\pi}\int_{M}K\dd{V_g}=\chi(M).
    \quad\text{\underline{Gauss-Bonnet theorem}}
  \] Where \(K\) is the Gaussian curvature and \(\dd{V_g}\) is the Riemannian volume
  form \[
    \dd{V_g}=\sqrt{\det(g_{ij})}\dd{x^1}\wedge \dd{x^2}
  .\] The de Rham class \(\left[\frac{1}{2\pi}K\dd{V_g}\right]\) is an invariant
  of \(M\), called the Euler class.
\item For complex vector bundle \(E\) there are Chern classes \(c_1(E),\ldots,
  c_r(E)\). For the tangent bundle \(TM\) of a compact complex manifold \(M\), the 
  first Chern class \(c_1(TM)\) is expressed in terms of the Ricci curvature.
  So, the Ricci curvature knows the topology of \(M\).
\end{enumerate}
We will study 1\(\sim\)4 in more detail in following chapters.

\ifdefined\FullBook{}
  \chapter{Holomorphic Vector Bundles}
  \section{Canonical connection on Hermitian bundles}
\else
  \section{Canonical connection for holomorphic vector bundle with Hermitian metrics}
\fi

Let \(M\) be a complex manifold of complex dimension \(m\). Let \(z^1,\ldots,z^m\)
be local holomorphic coordinates on \(U\subset M\). Set \(z^i=x^i+\sqrt{-1}y^i\) with
real part \(x^i\) and imaginary part \(y^i\). Then \[
  (x^1,y^1,\ldots,x^m,y^m)
\] are local coordinates of \(M\) as a smooth manifold. Since \[
  \pd{z^i}=\frac{1}{2}\left(\pd{x^i}-\sqrt{-1}\pd{y^i}\right),\quad
  \pd{\bar{z}^i}=\frac{1}{2}\left(\pd{x^i}+\sqrt{-1}\pd{y^i}\right),
\] \[
  \pd{x^i}=\pd{z^i}+\pd{\bar{z}^i},\quad
  \pd{y^i}=\sqrt{-1}(\pd{z^i}-\pd{\bar{z}^i}),
\] we have a decomposition \[
  TM\otimes \mathbb{C}=T'M\oplus T''M
,\] where
\begin{align*}
  T'M&=\Span\{\pd{z^1},\ldots,\pd{z^m}\}, \\
  T''M&=\Span\{\pd{\bar{z}^1},\ldots,\pd{\bar{z}^m}\}
.\end{align*}
On the coordinate neighborhood \((U;z^1,\ldots,z^m)\),
\begin{align*}
  T'M\Big|_U&=\{\xi^1\pd{z^1}+\cdots +\xi^m\pd{z^m}:
  \xi^1,\ldots,\xi^m\in \mathbb{C}\} \\
  &\xrightarrow[\vphi_U]{\sim}\{(z^1,\ldots,z^m,\xi^1,\ldots,\xi^m)\in
  U\times \mathbb{C}^n\}
\end{align*}
gives a local trivialization.

On another coordinate neighbourhood \((V,w^1,\ldots,w^n)\) with \(U\cap V\neq
\emptyset\),
\begin{align*}
  \vphi_U\circ \vphi_V^{-1}(w^1,\ldots,w^m,\eta^1,\ldots,\eta^m)
  &=\vphi_U\left(\eta^1\pd{w^1}+\cdots+\eta^m\pd{w^m}\right) \\
  &=\vphi_U\left(\sum \eta^j\pdv{z^i}{w^j}\pd{z^i}\right) \\
  &=\left(z^1,\ldots,z^m,\pdv{z^1}{w^j}\eta^j,\ldots,\pdv{z^m}{w^j}\eta^j\right)
.\end{align*}
Thus the transition function \(\vphi_{UV}(p)\) is given by \[
  \vphi_{UV}(p)=\left(\pdv{z^i}{w^j}(p)\right)
,\] which is holomorphic. So \(T'M\) is a holomorphic vector bundle.

The transition function of \(T''M\) is \(\left(\bar{\pdv{z^i}{w^j}}(p)\right)\),
which is not holomorphic. So \(T''M\) is not a holomorphic vector bundle.

Similarly, the cotangent bundle has the decomposition \[
  T^*M\otimes \mathbb{C}=T^{*\prime}M\oplus T^{*\prime\prime}M
.\] Where
\begin{align*}
  T^{*\prime}M&=\Span\{\dd{z^1},\ldots,\dd{z^m}\}, \\
  T^{*\prime\prime}M&=\Span\{\dd{\bar{z}^1},\ldots,\dd{\bar{z}^m}\}
.\end{align*}
The transition function of \(T^{*\prime}M\) is \(\tensor[^t]{\left(\pdv{z^i}{w^j}
\right)}{^{-1}}\), which is holomorphic.

\(T'M\) is called the holomorphic tangent bundle or tangent bundle of type \((1,0)\),
and \(T^{*\prime}M\) is called the holomorphic cotangent bundle.

\begin{example}
  Tensor product \[
    (\otimes^p T'M)\otimes (\otimes^q T^{*\prime}M)
  \] is also a holomorphic vector bundle.
\end{example}

\begin{definition}
  \(K_M=\wedge^m T^{*\prime}M\), \(m=\dim_{\mathbb{C}}M\) is a line bundle, called the
  \textbf{canonical line bundle} of \(M\).
  
  This line bundle plays an important role in the classification theory of complex 
  manifolds.
\end{definition}

Let \(D\) be a complex submanifold of codimension 1, which is also called a
non-singular \textbf{divisor}. For a divisor \(D\) there is a corresponding line
bundle \([D]\) defined as follows:

Let \(\{U_\lambda\}\) be an open covering by coordinate neighbourhoods such that there 
exist holomorphic functions \(f_\lambda\) vanishing along \(D\) with first order: \[
  D\cap U_\lambda=\{f_\lambda=0\}
.\] When \(D\cap U_\lambda=\emptyset\) we take \(f_\lambda\equiv 1\) (for example).
Then \[
  f_{\lambda\mu}=\frac{f_\lambda}{f_\mu}
\] is a nowhere vanishing holomorphic function on \(U_\lambda\cap U_\mu\). So \[
  f_{\lambda\mu}\colon U_\lambda\cap U_\mu\longrightarrow GL(1,\mathbb{C})\cong
  \mathbb{C}^*
,\] and \(f_{\lambda\mu}\cdot f_{\mu\nu}\cdot f_{\nu\lambda}=1\).

So \(\{f_{\lambda\mu}\}_{\lambda,\mu\in \Lambda}\) defines a holomorphic line bundle,
denoted by \([D]\), called the line bundle associated with the divisor \(D\).

\begin{remark}
  It is possible to consider singular divisors and also their formal sums.
  Assume \(D\rightsquigarrow L_D=[D]\), \(D'\rightsquigarrow L_{D'}=[D']\), then
  \begin{align*}
    D+D'&\rightsquigarrow L_{D+D'}=[D]\otimes [D']=L_D\otimes L_{D'} \\
    -D&\rightsquigarrow L_{-D}=L_D^{-1}
  .\end{align*}
  Where the transition function of \(L_D^{-1}\) is the inverse of the transition
  functions of \(L_D\). In particular, \[
    2D\rightsquigarrow L_{2D}=L_D\otimes L_D
  ,\] denoted by \(L_D^{\otimes 2}\) or \(L_D^2\).
\end{remark}
\begin{example}
  Let \[
    H=\{[0:z^1:\cdots :z^n]\in \mathbb{P}^m(\mathbb{C}):(z^1,\ldots,z^n)\neq 0\}
  \] be the hyperplane in \(\mathbb{P}^m(\mathbb{C})\). The associated line bundle
  \([H]\) is called the hyperplane bundle, denoted by \(\mathcal{O}(1)\).
  \(L_{-H}\) is denoted by \(\mathcal{O}(-1)\), and is isomorphic to the tautological
  line bundle (see homework 5).
\end{example}
\begin{example}
  Consider the case \(m=1\), \[
    \mathbb{P}^1(\mathbb{C})=\{[z^0:z^1]\},\quad H=\{[0:1]\}
  .\] The charts are
  \begin{gather*}
    U_0=\{z^0\neq 0\},\quad s=\frac{z^1}{z^0} \\
    U_1=\{z^1\neq 0\},\quad t=\frac{z^0}{z^1}
  \end{gather*}
  \(U_0\cap H=\emptyset\), so \(f_0\equiv 1\), \(U_1\cap H=\{t=0\}\), so \(f_1=t\).
  Then \[
    f_{01}=\frac{1}{t}=s,\quad f_{10}=t.
  \] We will use this example later.
\end{example}

\end{document}
