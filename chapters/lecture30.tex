% !TeX program = xelatex
\documentclass[12pt]{article}
\usepackage{standalone}

\usepackage[dvipsnames,svgnames,x11names]{xcolor}
\usepackage[a4paper,margin=1in]{geometry}
\usepackage{microtype}
\usepackage{amsmath}
\usepackage{amsthm}
\usepackage{mathtools}
\usepackage{mathrsfs}
\usepackage{stmaryrd}
\usepackage{extarrows}
\usepackage{enumerate}
\usepackage{tensor}
\usepackage{physics2}
  \usephysicsmodule{ab,xmat}
\usepackage{fixdif}
  \newcommand{\dd}{\d}
\usepackage{derivative}
  \newcommand{\dv}{\odv}
  \newcommand{\pd}[1]{\pdv{}{#1}}
  \newcommand{\eval}[1]{#1\big|}
\usepackage{graphicx}
\usepackage{subcaption}
\usepackage{tikz}
\usepackage{tikz-3dplot}
% \usepackage{tikz-cd}
% \usepackage{quiver}
% ========== Block quiver.sty ========== %
\usepackage{tikz-cd}
% \usepackage{amssymb}
\usetikzlibrary{calc}
\usetikzlibrary{decorations.pathmorphing}
\tikzset{curve/.style={settings={#1},to path={(\tikztostart)
    .. controls ($(\tikztostart)!\pv{pos}!(\tikztotarget)!\pv{height}!270:(\tikztotarget)$)
    and ($(\tikztostart)!1-\pv{pos}!(\tikztotarget)!\pv{height}!270:(\tikztotarget)$)
    .. (\tikztotarget)\tikztonodes}},
    settings/.code={\tikzset{quiver/.cd,#1}
        \def\pv##1{\pgfkeysvalueof{/tikz/quiver/##1}}},
    quiver/.cd,pos/.initial=0.35,height/.initial=0}
\tikzset{tail reversed/.code={\pgfsetarrowsstart{tikzcd to}}}
\tikzset{2tail/.code={\pgfsetarrowsstart{Implies[reversed]}}}
\tikzset{2tail reversed/.code={\pgfsetarrowsstart{Implies}}}
\tikzset{no body/.style={/tikz/dash pattern=on 0 off 1mm}}
% =========== End block ========== %
  \tikzset{every picture/.style={line width=0.75pt}}
\usepackage{pgfplots}
  \pgfplotsset{compat=newest}
\usepackage{tcolorbox}
  \tcbuselibrary{most}
\usepackage[colorlinks=true,linkcolor=blue]{hyperref}
\usepackage{cleveref}
% \usepackage[hyperref=true,backend=biber,style=alphabetic,backref=true,url=false]{biblatex}
\usepackage[warnings-off={mathtools-colon,mathtools-overbracket}]{unicode-math}
\usepackage[default,amsbb]{fontsetup}
  \setmathfont[StylisticSet=1,range=\mathscr]{NewCMMath-Book.otf}
\usepackage{fancyhdr}
\usepackage{import}

\newcommand{\Id}{\mathbb{1}}
\newcommand{\lap}{\increment}

\DeclareMathOperator{\sign}{sign}
\DeclareMathOperator{\dom}{dom}
\DeclareMathOperator{\ran}{ran}
\DeclareMathOperator{\ord}{ord}
\DeclareMathOperator{\Span}{span}
\DeclareMathOperator{\img}{Im}
\DeclareMathOperator{\Ric}{Ric}
\newcommand{\card}{\texttt{\#}}
\newcommand{\ie}{\emph{i.e.}}
\newcommand{\st}{\emph{s.t.}}
\newcommand{\eps}{\varepsilon}
\newcommand{\vphi}{\varphi}
\newcommand{\vthe}{\vartheta}
\newcommand{\II}{I\!I}
\renewcommand{\emptyset}{⌀}
\newcommand{\acts}{\curvearrowright}
\newcommand{\xrr}{\xlongrightarrow}
\newcommand{\lrr}{\longrightarrow}
\newcommand{\lmt}{\longmapsto}
\newcommand{\into}{\hookrightarrow}
\newcommand{\op}{\operatorname}

\let\originalleft\left
\let\originalright\right
\renewcommand{\left}{\mathopen{}\mathclose\bgroup\originalleft}
\renewcommand{\right}{\aftergroup\egroup\originalright}

\theoremstyle{plain}\newtheorem{theorem}{Theorem}
\theoremstyle{definition}\newtheorem{definition}[theorem]{Definition}
\theoremstyle{definition}\newtheorem{example}[theorem]{Example}
\theoremstyle{definition}\newtheorem{problem}[theorem]{Problem}
\theoremstyle{plain}\newtheorem{axiom}[theorem]{Axiom}
\theoremstyle{plain}\newtheorem{corollary}[theorem]{Corollary}
\theoremstyle{plain}\newtheorem{lemma}[theorem]{Lemma}
\theoremstyle{plain}\newtheorem{proposition}[theorem]{Proposition}
\theoremstyle{plain}\newtheorem{prop}[theorem]{Proposition}
\theoremstyle{plain}\newtheorem{conjecture}[theorem]{Conjecture}
\theoremstyle{plain}\newtheorem{conj}[theorem]{Conjecture}
\theoremstyle{remark}\newtheorem{notation}[theorem]{Notation}
\theoremstyle{definition}\newtheorem*{question}{Question}
\theoremstyle{definition}\newtheorem*{answer}{Answer}
\theoremstyle{definition}\newtheorem*{goal}{Goal}
\theoremstyle{definition}\newtheorem*{application}{Application}
\theoremstyle{plain}\newtheorem*{exercise}{Exercise}
\theoremstyle{remark}\newtheorem*{remark}{Remark}
\theoremstyle{remark}\newtheorem*{note}{\small{Note}}
\numberwithin{equation}{section}
\numberwithin{theorem}{section}
\numberwithin{figure}{section}

\usepackage{xeCJK}
\setCJKmainfont{FZShuSong-Z01}[BoldFont=FZXiaoBiaoSong-B05,ItalicFont=FZKai-Z03]
\setCJKsansfont{FZXiHeiI-Z08}[BoldFont=FZHei-B01]
\setCJKmonofont{FZFangSong-Z02}
\setCJKfamilyfont{zhsong}{FZShuSong-Z01}[BoldFont=FZXiaoBiaoSong-B05]
\setCJKfamilyfont{zhhei}{FZHei-B01}
\setCJKfamilyfont{zhkai}{FZKai-Z03}
\setCJKfamilyfont{zhfs}{FZFangSong-Z02}
\setCJKfamilyfont{zhli}{FZLiShu-S01}
\setCJKfamilyfont{zhyou}{FZXiYuan-M01}[BoldFont=FZZhunYuan-M02]

\allowdisplaybreaks{}

\newcommand{\isFullBook}[2]{
  \ifnum\pdfstrcmp{\FullBook}{True}=0
    \ifnum\pdfstrcmp{}{#1}=0\unskip\else#1\fi
  \else
    \ifnum\pdfstrcmp{}{#2}=0\unskip\else#2\fi
  \fi\ignorespaces{}
}

\counterwithout{theorem}{section}
\counterwithout{equation}{section}

\begin{document}
\section{Temporary title}
Recall:
\[\begin{tikzcd}[row sep=small]
	{\mathscr{M}} & B \\
	{\varpi^{-1}(N_\eps)} & {N_\eps}
	\arrow["{\varpi}", from=1-1, to=1-2]
	\arrow["\subset"{marking, allow upside down}, draw=none, from=2-1, to=1-1]
	\arrow[from=2-1, to=2-2]
	\arrow["\subset"{marking, allow upside down}, draw=none, from=2-2, to=1-2]
\end{tikzcd}\]
a complex analytic family of compact complex manifolds. Where
\(\varpi^{-1}(N_\eps)=\bigcup_{\alpha\in A}\mathcal{U}_\alpha\), \[
  \mathcal{U}_\alpha=\{(z_\alpha,t):z_\alpha=(z^1_\alpha,\ldots,z^m_\alpha),
  |z_\alpha^i|<1,t\in N_\eps\}
.\] And when \(\mathcal{U}_\alpha\cap \mathcal{U}_\beta\neq \emptyset\),
\begin{align*}
  &z_\alpha^i=f_{\alpha\beta}^i(z_\beta,t),\quad i=1,\ldots,m; \\
  &\theta_{\alpha\beta}(t)=\sum \pdv{f_{\alpha\beta}^i}{t}\pd{z^i_\alpha},\quad
  \text{1-cocycle of }\Theta_t; \\
  &\pdv{M_t}{t}\bigg|_{t=0}=[\{\theta_{\alpha\beta}(0)\}]\in H^1(M,\Theta), \\
  &\qquad\qquad\qquad\text{``infinitesimal deformation'' of }M=M_0=\varpi^{-1}(0)
.\end{align*}
But by Dolbeault theorem, \[
  H^1(M,\Theta)\cong H^{0,1}_{\bar{\partial}}(M,T')
\] where \(T'\) is the holomorphic tangent bundle and \(\Theta=\mathcal{O}(T')\).
Recall in the Dolbeault isomorphism, \[
  H^1(M,\Theta)=H^1(M,\mathcal{O}(T'))=H^{0,1}_{\bar{\partial}}(M,T')
.\] \([\{\theta_{\alpha\beta}\}]\) corresponds to \([\eta]\in H^{0,1}_{\bar{\partial}}
(M,T')\) as follows: \[
  0\lrr \Theta\lrr \mathcal{A}^0(T')\xrr{\bar{\partial}}\mathcal{A}^{0,1}(T')
  \xrr{\bar{\partial}}\mathcal{A}^{0,2}(T')\lrr \cdots
\] is a fine resolution. We have
\begin{align*}
  Z_i&=\ker(\bar{\partial}\colon \mathcal{A}^{0,i}(T')\lrr \mathcal{A}^{0,i+1}(T')) \\
  \Theta&=Z_0=\ker(\bar{\partial}\colon\mathcal{A}^{0}(T')
    \lrr\mathcal{A}^{0,1}(T'))\\
  &\mathrel{\phantom{=}}Z_1=\ker(\bar{\partial}\colon\mathcal{A}^{0,1}(T')
    \lrr\mathcal{A}^{0,2}(T'))
.\end{align*}
\[
  H^1(M,\Theta)\cong \frac{\Gamma(Z_1)}{\img(\bar{\partial}\colon \Gamma(\mathcal{A}^0)
    \lrr \Gamma(Z_1))}=H^{0,1}_{\bar{\partial}}(M,T')
.\] 
\[\begin{tikzcd}[row sep=tiny]
	0 & \Theta & {\mathcal{A}^0(T')} & {Z_1} & 0 \\
	&& {\{\xi_\alpha\}} & {\{\bar{\partial}\xi_\alpha\}=\eta} \\
	\\
	\\
	& {\{\theta_{\alpha\beta}\}} & {\{\theta_{\alpha\beta}\}}
	\arrow[from=2-3, to=2-4]
	\arrow[from=1-2, to=1-3]
	\arrow[from=1-1, to=1-2]
	\arrow[from=1-4, to=1-5]
	\arrow["{\bar{\partial}}", from=1-3, to=1-4]
	\arrow["\delta", maps to, from=2-3, to=5-3]
	\arrow[maps to, from=5-2, to=5-3]
\end{tikzcd}\]
Here \([\{\theta_{\alpha\beta}\}]\in H^1(M,\Theta)\) defines \([\{\theta_{\alpha\beta}
\}]\in H^1(M,\mathcal{A}^0(T'))=0\). Thus there exists \(\{\xi_\alpha\}\in C^1(M,
\mathcal{A}^{0}(T'))\) such that \(\delta\{\xi_{\alpha}\}=\xi_\beta-\xi_\alpha=
\theta_{\alpha\beta}\). But since \(\bar{\partial}\theta_{\alpha\beta}=0\), \[
  \bar{\partial}\xi_\alpha=\bar{\partial}\xi_\beta,\quad \forall\,\alpha,\beta
.\] Thus \(\eta:=\bar{\partial}\xi_\alpha\) defines a \(\bar{\partial}\)-closed
global \((0,1)\)-form.

As a matter of notations for complex analytic family \(\{M_t\}_{t\in B}\), let
\(z_0=(z^1_0,\ldots,z^m_0)\) be local holomorphic coordinate for \(M_0=\varpi^{-1}(0)\)
on \((U_\alpha,0)\). (Strictly speaking, \(z_0\) and \(z^i_0\) should be written as 
\(z_{0\alpha}\) and \(z_{0\alpha}^i\). But we omit \(\alpha\) just for notational
convenience).

We consider local holomorphic coordinate on \(M_t\), \[
  z_\alpha=z_\alpha(z^0,t)=(z^1_\alpha(z_0,t),\ldots,z_\alpha^m(z^0,t))
\] as smooth functions w.r.t.\ \(z_0\) and \(t\). We write \[
  \partial_i=\pd{z^i_0},\quad \bar{\partial_i}=\pd{\bar{z}^i_0},\quad 
  \bar{\partial}f=\sum \pdv{f}{\bar{z}^i_0}\dd{\bar{z}^i_0}
.\] 
\begin{prop}\label{prop:30-1-1}
  \[
    \eta=-\sum_{i=1}^{m}\bar{\partial}\left(\pdv{z^i_\alpha}{t}\bigg|_{t=0}\right)
    \pd{z^i_0}
  .\] 
\end{prop}
\begin{proof}
  We write \[
    \dot{z}^i_\alpha=\pdv{z^i_\alpha}{t}\bigg|_{t=0}
  .\] Then, from \(z^i_\alpha=f_{\alpha\beta}^i(z_\beta(z^0,t),t)\), \[
    \dot{z}^i_\alpha=\sum \pdv{f_{\alpha\beta}^i}{z^j_\beta}\dot{z^j_\beta}
    +\pdv{f_{\alpha\beta}^i}{t}\bigg|_{t=0}
  .\] Hence \[
    \theta_{\alpha\beta}=\sum \pdv{f_{\alpha\beta}^i}{t}\pd{z^i_\alpha}\bigg|_{t=0}
    =\sum\dot{z}^i_\alpha\pd{z^i_\alpha}-\sum\dot{z}^j_\beta\pd{z^j_\beta}\bigg|_{t=0}
  .\] Thus, we can take the above \(\xi_\alpha\) to be \[
    \xi_\alpha=-\dot{z}^i_\alpha\pd{z^i_0},\quad
    \text{ and }\eta=\bar{\partial}\xi_\alpha=-(\bar{\partial}\dot{z}^i_\alpha)
    \pd{z^i_0}
  .\]
\end{proof}
Next, we take a different view point. For \(M_0=\varpi^{-1}(0)\), we have a splitting
\[
  T^*M\otimes\mathbb{C}=T^{\prime*}M_0\oplus T^{\prime\prime*}M_0,\quad
  T^{\prime\prime*}M_0=\overline{T^{\prime*}M_0}
\] of the complexified cotangent bundle \(T^*M_0\otimes\mathbb{C}\) into the
Whitney sum of the holomorphic cotangent bundle and the anti-holomorphic cotangent
bundle.

If \(t\) is close to 0, then \(T^{\prime*}M_t\) and \(T^{\prime\prime}M_t\) is
close to \(T^{\prime*}M_0\) and \(T^{\prime\prime}M_0\). In particular,
\(T^{\prime*}M_t\) should be expressed as a graph over \(T^{\prime*}M_0\) in the form
\[
  \mathbb{x}_t=\mathbb{x}_0+\vphi(\mathbb{x}_0),\quad
  \mathbb{x}_t\in T^{\prime*}M_t,\ \mathbb{x}_0\in T^{\prime*}M_0,\ 
  \vphi(\mathbb{x}_0)\in T^{\prime\prime*}M_0
.\] So \(\vphi_t\in \op{End}(T^{\prime*},T^{\prime\prime*}M_0)=T'M_0\otimes 
T^{\prime\prime*}M_0\), \[
  \vphi_t=\vphi\indices{^i_{\bar{j}}}(t)\pd{z^i_0}\otimes\dd{\bar{z}^j_0}
.\] Thus \(T^{\prime*}M_t\) is spanned by \[
  \dd{z^i_0}+\vphi\indices{^i_{\bar{j}}}\dd{\bar{z}^j_0},\quad i=1,\ldots,m
.\] Equivalently, \(T''M_t\) is spanned by \[
  \pd{\bar{z}^j_0}-\vphi\indices{^i_{\bar{j}}}\pd{z^i_0},\quad j=1,\ldots,m
.\] Hence \[
  \left<\dd{z}^i_0+\vphi\indices{^i_{\bar{k}}}\dd{\bar{z}^k_0},
  \pd{\bar{z}^j_0}-\vphi\indices{^l_{\bar{j}}}\pd{z^l_0}\right> 
  =\vphi\indices{^i_{\bar{j}}}-\vphi\indices{^i_{\bar{j}}}=0
.\] 

\begin{theorem}\hfill
  \begin{enumerate}[(1)]
  \item \(\vphi(0)=0\);
  \item \(\bar{\partial}\vphi_t-\frac{1}{2}[\vphi_t,\vphi_t]=0\). ``Maurer-Cartan
    equation'';
  \item \[
      H^1(M,\Theta)\ni\pdv{M_t}{t}\bigg|_{t=0}\longleftrightarrow
      \eta=-\pdv{\vphi_t}{t}\bigg|_{t=0}\in H^{0,1}_{\bar{\partial}}(T'M)
    \]
  \end{enumerate}
\end{theorem}
\begin{proof}
  We write \(\vphi_t=\vphi^i\pd{z^i_0}\) with \(\vphi^i=\vphi\indices{^i_{\bar{j}}}(t)
  \dd{\bar{z}^j_0}\).
  \begin{enumerate}[(1)]
  \item Obvious since \(M_t\big|_{t=0}=M\).
  \item \(z^i_\alpha\) are holomorphic coordinate on \(M_t\). Thus \[
      (\bar{\partial}-\vphi^j\partial_j)z^i_\alpha=\left(\left(\pd{\bar{z}^k_0}
      -\vphi\indices{^j_{\bar{l}}}\pd{z^j_0}\right)z^i_\alpha\right)\dd{\bar{z}^k_0}=0
    .\] Hence \[
      \bar{\partial}(\vphi^j\partial_j z^i_\alpha)=\bar{\partial}\vphi^j\cdot
      \partial_j z^i_\alpha-\vphi^j\wedge\bar{\partial}\partial_j z^i_\alpha
      =\bar{\partial}\bar{\partial}z^i_0=0
    .\] Then
    \begin{align*}
      (\bar{\partial}\vphi)z^i_\alpha=\bar{\partial}\vphi^j\cdot \partial_j z^i_\alpha
      &=\vphi^j\wedge\partial_j \bar{\partial}z^i_\alpha=\vphi^j\wedge\partial_j
      (\vphi^k\partial_k z^i_\alpha) \\
      &=\vphi^j\wedge\partial_j\vphi^k\cdot \partial_k z^i_\alpha+\underbrace{
      \vphi^j\wedge\vphi^k\cdot \partial_j\partial_k z^i_\alpha}_{=0} \\
      &=\frac{1}{2}[\vphi,\vphi]z^i_\alpha
    .\end{align*} 
    (Note that \([\vphi,\psi]=\vphi\psi-(-1)^{pq}\psi\vphi\), and \(\vphi\wedge\vphi
    =\frac{1}{2}[\vphi,\vphi]\)).
  \item By \cref{prop:30-1-1}, \[
      \eta=-(\bar{\partial}\dot{z}^i_\alpha)\pd{z^i_0}
    .\] But since \(\bar{\partial}z^i_\alpha=\vphi^j\partial_j z^i_\alpha\), we have \[
      \bar{\partial}\dot{z}^i_\alpha=\dot{\vphi}^j\big|_{t=0}\partial_j z_\alpha^i
      +\underbrace{\vphi^j\big|_{t=0}}_{=0}\partial_j \dot{z}^i_\alpha
    .\] Hence \[
      \eta=-(\dot{\vphi}^j\partial_j z^i_\alpha)\Big|_{t=0}\pd{z^i_0}
      =-\dot{\vphi}^j\pd{z^j_0}=-\pdv{\vphi}{t}\bigg|_{t=0}
    .\] 
  \end{enumerate}
\end{proof}
For \(\vphi\in \mathcal{A}^p(T'),\psi\in \mathcal{A}^q(T')\) with
\(\bar{\partial}\vphi=0\), \(\bar{\partial}\psi=0\), we have \[
  [\vphi,\psi]\in \mathcal{A}^{p+q}(T')\quad\text{ with }\bar{\partial}[\vphi,\psi]=0
.\] This give rise to
\begin{gather*}
  H^{0,p}_{\bar{\partial}}(T')\times H^{0,q}_{\bar{\partial}}(T')
  \lrr H^{0,p+q}_{\bar{\partial}}(T'). \\
  H^{p}(M,\Theta)\times H^q(M,\Theta)\lrr H^{p+q}(M,\Theta).
\end{gather*}
\begin{theorem}
  If \(\rho=\dv{M_t}{t}\Big|_{t=0}\in H^1(M,\Theta)\), then \([\rho,\rho]=0\).
\end{theorem}
\begin{proof}
  \begin{align*}
    &\bar{\partial}\vphi_t=\frac{1}{2}[\vphi_t,\vphi_t]=0 \\
    &\bar{\partial}\dot{\vphi}_t=\frac{1}{2}[\dot{\vphi}_t,\vphi_t]+\frac{1}{2}
    [\vphi_t,\dot{\vphi}_t] \\
    &\bar{\partial}\ddot{\vphi}_t\big|_{t=0}=[\dot{\vphi}_0,\dot{\vphi}_0]
    =[\eta,\eta]\quad(\text{since }\vphi_0=0)
  .\end{align*}
  Hence \([\eta,\eta]\) is cohomologous to 0. Thus \([\rho,\rho]=0\) in
  \(H^2(M,\Theta)\).
\end{proof}

\begin{definition}
  \[
    \psi\colon H^1(M,\Theta)\lrr H^2(M,\Theta),\quad \psi(\rho)=[\rho,\rho]
  .\] This is the map in Kuranishi's theorem.
\end{definition}
\begin{definition}
  The deformations of \(M\) are \textbf{unobstructed} if \(\psi\equiv 0\).
\end{definition}

\begin{prop}
  \begin{align*}
    &\bar{\partial}\vphi-\frac{1}{2}[\vphi,\vphi]=0 \\
    \iff & [\Gamma(T''M_t),\Gamma(T''M_t)]\subset \Gamma(T''M_t) \\
    \iff & [\Gamma(T'M_t),\Gamma(T'M_t)]\subset \Gamma(T'M_t) \\
    \iff & \text{Nijenhuis tensor vanishes} \\
    \iff & \text{The almost complex structure defined by }\vphi\text{ is integrable}
  .\end{align*}
\end{prop}
\begin{proof}
  For \(X,Y\in T''M_0\), \(X-\vphi(X),Y-\vphi(Y)\in T''M_t\). \[
    [X-\vphi(X),Y-\vphi(Y)]=[X,Y]-\vphi(X)Y+\vphi(Y)X+[\vphi(X),\vphi(Y)]
    -X(\vphi(Y))+Y(\vphi(X))
  .\] This lies in \(\Gamma(T''M_t)\iff\) \[
    [\vphi(X),\vphi(Y)]-X(\vphi(Y))+Y(\vphi(X))=-\vphi([X,Y]-\vphi(X)Y+\vphi(Y)X)
  .\] \(\iff \)
  \begin{multline*}
    X(\vphi(Y))-Y(\vphi(X))-\vphi([X,Y]) \\
    -(\vphi(X))(\vphi(Y))+(\vphi(Y))(\vphi(X))+\vphi(\vphi(X)Y)-\vphi(\vphi(Y)X)
    =0
  .\end{multline*} 
  \(\iff \) \[
    (\bar{\partial}\vphi)(X,Y)-(\vphi(X)\vphi)(Y)+(\vphi(Y)\vphi)(X)=
    (\bar{\partial}\vphi-\frac{1}{2}[\vphi,\vphi])(X,Y)=0
  .\] 
\end{proof}

\end{document}
